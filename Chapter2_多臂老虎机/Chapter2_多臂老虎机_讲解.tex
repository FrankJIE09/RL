\documentclass[12pt,a4paper]{article}
\usepackage[UTF8]{ctex}
\usepackage{amsmath}
\usepackage{amssymb}
\usepackage{amsthm}
\usepackage{geometry}
\usepackage{hyperref}
\usepackage{enumitem}
\usepackage{xcolor}

\geometry{margin=2.5cm}

\title{第2章:多臂老虎机(Multi-armed Bandits)讲解}
\author{}
\date{}

\begin{document}

\maketitle

\section{章节概述}

第2章介绍了强化学习中最基本的探索与利用问题。多臂老虎机问题是一个简化的强化学习场景,它不涉及多个状态,只关注如何在不同动作之间进行选择以最大化累积奖励。

\section{核心概念}

\subsection{评估性反馈 vs 指导性反馈}

\begin{itemize}
    \item \textbf{评估性反馈(Evaluative Feedback)}:告诉我们采取的动作有多好,但不告诉我们这是否是最佳或最差动作
    \item \textbf{指导性反馈(Instructive Feedback)}:告诉我们应该采取的正确动作,与实际上采取的动作无关
\end{itemize}

强化学习使用评估性反馈,这创造了\textbf{探索与利用的冲突}。

\subsection{k-臂老虎机问题}

\textbf{问题定义}:
\begin{itemize}
    \item 有 $k$ 个不同的动作可以选择
    \item 每次选择动作后,会收到一个数值奖励
    \item 奖励来自一个依赖于所选动作的平稳概率分布
    \item 目标是在一段时间内最大化期望总奖励
\end{itemize}

\textbf{关键符号}:
\begin{itemize}
    \item $q_*(a)$:动作 $a$ 的真实价值(期望奖励)
    \item $Q_t(a)$:在时间步 $t$ 对动作 $a$ 的估计价值
    \item $A_t$:在时间步 $t$ 选择的动作
    \item $R_t$:在时间步 $t$ 收到的奖励
\end{itemize}

\subsection{探索与利用}

\begin{itemize}
    \item \textbf{利用(Exploitation)}:选择当前估计价值最高的动作(贪婪动作)
    \item \textbf{探索(Exploration)}:选择非贪婪动作,以改进对它们价值的估计
\end{itemize}

\textbf{核心冲突}:不可能同时探索和利用,必须在两者之间取得平衡。

\section{主要方法}

\subsection{动作-价值方法(Action-value Methods)}

\textbf{样本平均方法}:
\begin{equation}
Q_t(a) = \frac{\text{所有在时间 } t \text{ 之前选择动作 } a \text{ 的奖励之和}}{\text{在时间 } t \text{ 之前选择动作 } a \text{ 的次数}}
\end{equation}

\textbf{贪婪动作选择}:
\begin{equation}
A_t = \arg\max_a Q_t(a)
\end{equation}

\textbf{$\varepsilon$-贪婪方法}:
\begin{itemize}
    \item 以概率 $1-\varepsilon$ 选择贪婪动作
    \item 以概率 $\varepsilon$ 随机选择动作
\end{itemize}

\textbf{优点}:
\begin{itemize}
    \item 简单易实现
    \item 在极限情况下,所有动作都会被无限次采样
    \item 所有 $Q_t(a)$ 都会收敛到 $q_*(a)$
\end{itemize}

\subsection{10-臂测试平台}

\textbf{实验设置}:
\begin{itemize}
    \item 2000 个随机生成的 10-臂老虎机问题
    \item 每个动作的真实价值 $q_*(a)$ 从均值为 0、方差为 1 的正态分布中选择
    \item 实际奖励从均值为 $q_*(A_t)$、方差为 1 的正态分布中选择
\end{itemize}

\textbf{实验结果}:
\begin{itemize}
    \item 贪婪方法在长期表现更差,因为它经常陷入次优动作
    \item $\varepsilon$-贪婪方法最终表现更好,因为它们继续探索
    \item $\varepsilon = 0.1$ 的方法探索更多,通常更早找到最优动作
    \item $\varepsilon = 0.01$ 的方法改进更慢,但最终可能表现更好
\end{itemize}

\subsection{增量实现}

\textbf{增量更新公式}:
\begin{equation}
Q_{n+1} = Q_n + \frac{1}{n}[R_n - Q_n]
\end{equation}

\textbf{通用形式}:
\begin{equation}
\text{新估计} \leftarrow \text{旧估计} + \text{步长}[\text{目标} - \text{旧估计}]
\end{equation}

这种方法只需要常数内存和常数计算量。

\subsection{跟踪非平稳问题}

对于非平稳问题,使用\textbf{常数步长参数}:
\begin{equation}
Q_{n+1} = Q_n + \alpha[R_n - Q_n]
\end{equation}

其中 $\alpha \in (0, 1]$ 是常数。

\textbf{指数近因加权平均}:
\begin{equation}
Q_{n+1} = (1-\alpha)^n Q_1 + \sum_{i=1}^n \alpha(1-\alpha)^{n-i} R_i
\end{equation}

\textbf{收敛条件}(随机逼近理论):
\begin{enumerate}
    \item $\sum_{n=1}^{\infty} \alpha_n(a) = \infty$
    \item $\sum_{n=1}^{\infty} \alpha_n^2(a) < \infty$
\end{enumerate}

\subsection{乐观初始值}

\textbf{思想}:将初始动作价值估计设置为乐观值(如 +5),鼓励探索。

\textbf{优点}:
\begin{itemize}
    \item 即使总是选择贪婪动作,系统也会进行大量探索
    \item 所有动作在价值估计收敛前都会被尝试多次
\end{itemize}

\textbf{缺点}:
\begin{itemize}
    \item 不适合非平稳问题
    \item 探索驱动力是暂时的
\end{itemize}

\subsection{上置信界动作选择(UCB)}

\textbf{UCB 动作选择}:
\begin{equation}
A_t = \arg\max_a \left[ Q_t(a) + c\sqrt{\frac{\ln t}{N_t(a)}} \right]
\end{equation}

其中:
\begin{itemize}
    \item $\ln t$:$t$ 的自然对数
    \item $N_t(a)$:在时间 $t$ 之前动作 $a$ 被选择的次数
    \item $c > 0$:控制探索程度的参数
\end{itemize}

\textbf{思想}:
\begin{itemize}
    \item 平方根项是对动作价值估计不确定性的度量
    \item 被最大化的量是动作真实价值的上界
    \item 每次选择动作 $a$,不确定性减少
    \item 每次选择其他动作,不确定性增加
\end{itemize}

\textbf{优点}:在 10-臂测试平台上通常表现良好

\textbf{缺点}:
\begin{itemize}
    \item 比 $\varepsilon$-贪婪更难扩展到更一般的强化学习设置
    \item 处理非平稳问题需要更复杂的方法
    \item 处理大状态空间时不实用
\end{itemize}

\subsection{梯度老虎机算法}

\textbf{思想}:学习每个动作的数值偏好 $H_t(a)$,而不是估计动作价值。

\textbf{动作选择}(soft-max 分布):
\begin{equation}
\Pr\{A_t = a\} = \frac{e^{H_t(a)}}{\sum_{b=1}^k e^{H_t(b)}} = \pi_t(a)
\end{equation}

\textbf{更新规则}:
\begin{align}
H_{t+1}(A_t) &= H_t(A_t) + \alpha[R_t - \bar{R}_t][1 - \pi_t(A_t)] \\
H_{t+1}(a) &= H_t(a) - \alpha[R_t - \bar{R}_t]\pi_t(a), \quad \text{对所有 } a \neq A_t
\end{align}

其中:
\begin{itemize}
    \item $\alpha > 0$:步长参数
    \item $\bar{R}_t$:到时间 $t$ 为止所有奖励的平均值(基线)
\end{itemize}

\textbf{理论保证}:梯度老虎机算法是随机梯度上升的一个实例。

\subsection{关联搜索(上下文老虎机)}

\textbf{关联搜索任务}:
\begin{itemize}
    \item 涉及试错学习以搜索最佳动作
    \item 将这些动作与它们最佳的情况关联起来
    \item 也称为\textbf{上下文老虎机(Contextual Bandits)}
\end{itemize}

\textbf{特点}:
\begin{itemize}
    \item 介于 $k$-臂老虎机问题和完整强化学习问题之间
    \item 像完整强化学习问题一样涉及学习策略
    \item 像 $k$-臂老虎机问题一样,每个动作只影响即时奖励
\end{itemize}

\subsection{总结}

本章介绍了几种平衡探索与利用的简单方法:

\begin{enumerate}
    \item \textbf{$\varepsilon$-贪婪方法}:随机选择一小部分时间
    \item \textbf{UCB 方法}:确定性选择,但通过微妙地偏向采样较少的动作来实现探索
    \item \textbf{梯度老虎机算法}:估计动作偏好而非动作价值,使用 soft-max 分布
    \item \textbf{乐观初始值}:即使贪婪方法也会显著探索
\end{enumerate}

\textbf{参数研究}:在 10-臂测试平台上,UCB 似乎表现最好。

\section{关键要点}

\begin{enumerate}
    \item \textbf{探索与利用的平衡}是强化学习的核心挑战
    \item \textbf{$\varepsilon$-贪婪方法}简单有效,是实践中常用的方法
    \item \textbf{UCB 方法}在理论上有更好的性能保证
    \item \textbf{梯度方法}提供了另一种学习动作偏好的视角
    \item 这些简单方法为后续章节的完整强化学习问题奠定了基础
\end{enumerate}

\section{练习题要点}

\begin{itemize}
    \item \textbf{练习 2.1}:理解 $\varepsilon$-贪婪动作选择的概率计算
    \item \textbf{练习 2.2}:分析具体序列中哪些时间步发生了随机选择
    \item \textbf{练习 2.3}:比较不同方法的长期性能
    \item \textbf{练习 2.5}:编程实现非平稳问题的实验
    \item \textbf{练习 2.8}:理解 UCB 算法中的性能尖峰现象
\end{itemize}

\section{与后续章节的联系}

本章介绍的方法将在后续章节中扩展到:
\begin{itemize}
    \item 完整强化学习问题(第3章及以后)
    \item 多状态、多动作的决策问题
    \item 延迟奖励和长期规划
\end{itemize}

\vspace{1cm}

\textbf{参考文献}:Reinforcement Learning: An Introduction (2nd Edition), Chapter 2

\end{document}


