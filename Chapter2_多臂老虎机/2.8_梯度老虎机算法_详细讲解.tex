\documentclass[12pt,a4paper]{article}
\usepackage[UTF8]{ctex}
\usepackage{amsmath}
\usepackage{amssymb}
\usepackage{amsthm}
\usepackage{geometry}
\usepackage{hyperref}
\usepackage{enumitem}
\usepackage{xcolor}
\usepackage{bm}

\geometry{margin=2.5cm}

\title{2.8 梯度老虎机算法(Gradient Bandit Algorithms)详细讲解}
\author{}
\date{}

\newtheorem{theorem}{定理}
\newtheorem{proof}{证明}

\begin{document}

\maketitle

\section{章节概述}

本节介绍了一种不同于之前方法的新思路:不是估计动作价值,而是学习每个动作的\textbf{数值偏好(numerical preference)}。这种方法基于\textbf{随机梯度上升(stochastic gradient ascent)}的思想,为后续的策略梯度方法奠定了基础。

\section{核心思想}

\subsection{从动作价值到动作偏好}

\textbf{之前的方法}:
\begin{itemize}
    \item 估计每个动作的价值 $Q_t(a)$
    \item 根据价值估计选择动作(如 $\varepsilon$-贪婪、UCB)
\end{itemize}

\textbf{梯度方法}:
\begin{itemize}
    \item 学习每个动作的数值偏好 $H_t(a)$
    \item 偏好值越大,该动作被选择的概率越大
    \item \textbf{关键点}:偏好值本身没有奖励意义的解释,只有相对偏好才重要
\end{itemize}

\subsection{为什么使用偏好而不是价值?}

\textbf{优势}:
\begin{itemize}
    \item 偏好值可以任意平移(加常数)而不影响动作概率
    \item 更适合表示动作之间的相对关系
    \item 为后续的策略梯度方法提供理论基础
\end{itemize}

\section{Soft-Max 分布(Gibbs/Boltzmann 分布)}

\subsection{动作选择概率}

根据 soft-max 分布,动作 $a$ 被选择的概率为:
\begin{equation}
\pi_t(a) = \Pr\{A_t = a\} = \frac{e^{H_t(a)}}{\sum_{b=1}^k e^{H_t(b)}}
\end{equation}

\textbf{关键特性}:
\begin{itemize}
    \item 所有动作的概率和为 1:$\sum_a \pi_t(a) = 1$
    \item 偏好值越大,被选择的概率越大
    \item 如果所有偏好值都加上常数 $C$,概率分布不变
    \item 初始时,通常设置 $H_1(a) = 0$(对所有 $a$),使得所有动作等概率
\end{itemize}

\subsection{Soft-Max 的直观理解}

\textbf{温度参数视角}(虽然这里没有显式温度):
\begin{itemize}
    \item 当偏好值差异很大时,分布接近确定性(一个动作概率接近1)
    \item 当偏好值差异很小时,分布接近均匀分布
\end{itemize}

\textbf{与 sigmoid 函数的关系}:
\begin{itemize}
    \item 对于两个动作的情况,soft-max 等价于 sigmoid 函数
    \item 这是练习 2.9 的内容
\end{itemize}

\subsection{数学性质}

\textbf{归一化}:
\begin{equation}
\sum_{a=1}^k \pi_t(a) = \frac{\sum_{a=1}^k e^{H_t(a)}}{\sum_{b=1}^k e^{H_t(b)}} = 1
\end{equation}

\textbf{平移不变性}:
如果对所有 $a$,$H_t(a) \leftarrow H_t(a) + C$,则:
\begin{align}
\pi_t(a) &= \frac{e^{H_t(a) + C}}{\sum_b e^{H_t(b) + C}} \\
         &= \frac{e^C \cdot e^{H_t(a)}}{e^C \cdot \sum_b e^{H_t(b)}} \\
         &= \frac{e^{H_t(a)}}{\sum_b e^{H_t(b)}}
\end{align}
概率分布保持不变!

\section{更新规则详解}

\subsection{基本更新公式}

在时间步 $t$,选择动作 $A_t$ 并收到奖励 $R_t$ 后,更新规则为:

\textbf{对于被选择的动作}:
\begin{equation}
H_{t+1}(A_t) = H_t(A_t) + \alpha[R_t - \bar{R}_t][1 - \pi_t(A_t)]
\end{equation}

\textbf{对于未被选择的动作}(所有 $a \neq A_t$):
\begin{equation}
H_{t+1}(a) = H_t(a) - \alpha[R_t - \bar{R}_t]\pi_t(a)
\end{equation}

\textbf{参数说明}:
\begin{itemize}
    \item $\alpha > 0$:步长参数
    \item $\bar{R}_t$:到时间 $t$ 为止所有奖励的平均值(基线)
    \item $\pi_t(a)$:在时间 $t$ 选择动作 $a$ 的概率
\end{itemize}

\subsection{更新规则的直观理解}

\textbf{被选择的动作 $A_t$}:
\begin{itemize}
    \item 如果 $R_t > \bar{R}_t$(奖励高于基线):
    \begin{itemize}
        \item $H_{t+1}(A_t) = H_t(A_t) + \text{正数} \times [1 - \pi_t(A_t)]$
        \item 偏好值增加,未来被选择的概率增加
    \end{itemize}
    \item 如果 $R_t < \bar{R}_t$(奖励低于基线):
    \begin{itemize}
        \item $H_{t+1}(A_t) = H_t(A_t) + \text{负数} \times [1 - \pi_t(A_t)]$
        \item 偏好值减少,未来被选择的概率减少
    \end{itemize}
\end{itemize}

\textbf{未被选择的动作 $a$}:
\begin{itemize}
    \item 如果 $R_t > \bar{R}_t$:
    \begin{itemize}
        \item $H_{t+1}(a) = H_t(a) - \text{正数} \times \pi_t(a)$
        \item 偏好值减少(相对于被选择的动作)
    \end{itemize}
    \item 如果 $R_t < \bar{R}_t$:
    \begin{itemize}
        \item $H_{t+1}(a) = H_t(a) - \text{负数} \times \pi_t(a)$
        \item 偏好值增加(相对于被选择的动作)
    \end{itemize}
\end{itemize}

\textbf{关键洞察}:
\begin{itemize}
    \item 被选择的动作和未被选择的动作向相反方向更新
    \item 这保证了概率分布始终归一化
\end{itemize}

\subsection{基线(Baseline)的作用}

\textbf{基线 $\bar{R}_t$ 的重要性}:

\begin{enumerate}
    \item \textbf{减少方差}:
    \begin{itemize}
        \item 基线不改变期望更新(理论上)
        \item 但显著减少更新的方差
        \item 提高收敛速度
    \end{itemize}
    
    \item \textbf{适应奖励尺度}:
    \begin{itemize}
        \item 如果所有奖励都加上常数 $C$,基线自动适应
        \item 算法性能不受影响
        \item 如图 2.5 所示,当奖励均值从 0 变为 +4 时,有基线的算法性能不变
    \end{itemize}
    
    \item \textbf{无基线的情况}:
    \begin{itemize}
        \item 如果 $\bar{R}_t = 0$(常数),性能会显著下降
        \item 特别是在奖励均值不为 0 的情况下
    \end{itemize}
\end{enumerate}

\textbf{基线的计算}:
\begin{itemize}
    \item 可以使用增量更新(如 2.4 节)
    \item 对于非平稳问题,可以使用常数步长(如 2.5 节)
\end{itemize}

\section{理论证明:作为随机梯度上升}

\subsection{性能度量}

\textbf{期望奖励}:
\begin{equation}
\mathbb{E}[R_t] = \sum_x \pi_t(x) q_*(x)
\end{equation}

其中:
\begin{itemize}
    \item $\pi_t(x)$ 是选择动作 $x$ 的概率
    \item $q_*(x)$ 是动作 $x$ 的真实价值
\end{itemize}

\subsection{精确梯度上升}

如果我们可以精确计算梯度,更新规则应该是:
\begin{equation}
H_{t+1}(a) = H_t(a) + \alpha \cdot \frac{\partial \mathbb{E}[R_t]}{\partial H_t(a)}
\end{equation}

\textbf{问题}:我们不知道 $q_*(x)$,无法精确计算梯度!

\subsection{性能梯度的推导}

\textbf{第一步:写出性能梯度}
\begin{align}
\frac{\partial \mathbb{E}[R_t]}{\partial H_t(a)} &= \frac{\partial}{\partial H_t(a)} \left[\sum_x \pi_t(x) q_*(x)\right] \\
                                                   &= \sum_x q_*(x) \cdot \frac{\partial \pi_t(x)}{\partial H_t(a)}
\end{align}

\textbf{第二步:引入基线}

由于 $\sum_x \frac{\partial \pi_t(x)}{\partial H_t(a)} = 0$(概率和恒为1),我们可以添加任何不依赖于 $x$ 的基线 $B_t$:
\begin{equation}
\frac{\partial \mathbb{E}[R_t]}{\partial H_t(a)} = \sum_x [q_*(x) - B_t] \cdot \frac{\partial \pi_t(x)}{\partial H_t(a)}
\end{equation}

\textbf{第三步:转换为期望形式}

乘以 $\pi_t(x)/\pi_t(x)$:
\begin{align}
\frac{\partial \mathbb{E}[R_t]}{\partial H_t(a)} &= \sum_x \pi_t(x)[q_*(x) - B_t] \cdot \frac{\frac{\partial \pi_t(x)}{\partial H_t(a)}}{\pi_t(x)} \\
                                                   &= \mathbb{E}\left[[q_*(A_t) - B_t] \cdot \frac{\frac{\partial \pi_t(A_t)}{\partial H_t(a)}}{\pi_t(A_t)}\right]
\end{align}

\textbf{第四步:用样本奖励替代}

由于 $\mathbb{E}[R_t | A_t] = q_*(A_t)$,我们可以用 $R_t$ 替代 $q_*(A_t)$:
\begin{equation}
\frac{\partial \mathbb{E}[R_t]}{\partial H_t(a)} = \mathbb{E}\left[[R_t - \bar{R}_t] \cdot \frac{\frac{\partial \pi_t(A_t)}{\partial H_t(a)}}{\pi_t(A_t)}\right]
\end{equation}

其中我们选择 $B_t = \bar{R}_t$(奖励平均值作为基线)。

\subsection{Soft-Max 的梯度计算}

\textbf{关键结果}:
\begin{equation}
\frac{\partial \pi_t(x)}{\partial H_t(a)} = \pi_t(x)[\mathbf{1}(a=x) - \pi_t(a)]
\end{equation}

其中 $\mathbf{1}(a=x)$ 是指示函数($a=x$ 时为1,否则为0)。

\textbf{证明}(使用商的求导法则):

\begin{align}
\frac{\partial \pi_t(x)}{\partial H_t(a)} &= \frac{\partial}{\partial H_t(a)} \left[\frac{e^{H_t(x)}}{\sum_y e^{H_t(y)}}\right]
\end{align}

使用商的求导法则:
\begin{equation}
\frac{\partial [f/g]}{\partial x} = \frac{f'g - fg'}{g^2}
\end{equation}

得到:
\begin{align}
\frac{\partial \pi_t(x)}{\partial H_t(a)} &= \frac{\mathbf{1}(a=x) e^{H_t(x)} \cdot \sum_y e^{H_t(y)} - e^{H_t(x)} \cdot e^{H_t(a)}}{\left[\sum_y e^{H_t(y)}\right]^2} \\
&= \mathbf{1}(a=x) \pi_t(x) - \pi_t(x) \pi_t(a) \\
&= \pi_t(x)[\mathbf{1}(a=x) - \pi_t(a)]
\end{align}

\textbf{Q.E.D.}

\subsection{最终更新规则}

将梯度表达式代入:
\begin{align}
\frac{\partial \mathbb{E}[R_t]}{\partial H_t(a)} &= \mathbb{E}\left[[R_t - \bar{R}_t] \cdot \frac{\pi_t(A_t)[\mathbf{1}(a=A_t) - \pi_t(a)]}{\pi_t(A_t)}\right] \\
&= \mathbb{E}\left[[R_t - \bar{R}_t][\mathbf{1}(a=A_t) - \pi_t(a)]\right]
\end{align}

用样本替代期望,得到更新规则:
\begin{equation}
H_{t+1}(a) = H_t(a) + \alpha[R_t - \bar{R}_t][\mathbf{1}(a=A_t) - \pi_t(a)]
\end{equation}

\textbf{展开}:
\begin{itemize}
    \item 如果 $a = A_t$:$H_{t+1}(A_t) = H_t(A_t) + \alpha[R_t - \bar{R}_t][1 - \pi_t(A_t)]$
    \item 如果 $a \neq A_t$:$H_{t+1}(a) = H_t(a) + \alpha[R_t - \bar{R}_t][0 - \pi_t(a)] = H_t(a) - \alpha[R_t - \bar{R}_t]\pi_t(a)$
\end{itemize}

这正是我们原始的更新规则(2.12)!

\subsection{理论保证}

\textbf{关键结论}:
\begin{itemize}
    \item 梯度老虎机算法的\textbf{期望更新}等于性能函数的梯度
    \item 因此,算法是\textbf{随机梯度上升}的一个实例
    \item 这保证了算法具有\textbf{稳健的收敛性质}
\end{itemize}

\section{实验性能分析}

\subsection{实验设置}

\textbf{10-臂测试平台的变体}:
\begin{itemize}
    \item 真实期望奖励从 $\mathcal{N}(0, 1)$ 改为 $\mathcal{N}(+4, 1)$
    \item 所有奖励都向上平移了 4 个单位
\end{itemize}

\subsection{实验结果(图 2.5)}

\textbf{有基线的情况}:
\begin{itemize}
    \item 性能不受奖励平移影响
    \item 基线自动适应新的奖励水平
    \item 收敛到接近最优性能
\end{itemize}

\textbf{无基线的情况}($\bar{R}_t = 0$):
\begin{itemize}
    \item 性能显著下降
    \item 无法适应奖励尺度的变化
    \item 收敛速度慢
\end{itemize}

\textbf{步长参数的影响}:
\begin{itemize}
    \item $\alpha = 0.1$:较慢但稳定
    \item $\alpha = 0.4$:较快但可能不稳定
    \item 需要根据问题调整
\end{itemize}

\section{深入理解}

\subsection{为什么基线不影响期望更新?}

\textbf{数学原因}:
\begin{itemize}
    \item 基线 $B_t$ 不依赖于动作 $x$
    \item 梯度 $\sum_x \frac{\partial \pi_t(x)}{\partial H_t(a)} = 0$
    \item 因此 $\sum_x B_t \cdot \frac{\partial \pi_t(x)}{\partial H_t(a)} = B_t \cdot 0 = 0$
\end{itemize}

\textbf{直观理解}:
\begin{itemize}
    \item 基线只是改变了更新的"参考点"
    \item 期望方向不变,但方差会改变
\end{itemize}

\subsection{为什么基线减少方差?}

\textbf{方差来源}:
\begin{itemize}
    \item 更新依赖于实际奖励 $R_t$
    \item 如果奖励方差大,更新方差也大
\end{itemize}

\textbf{基线的作用}:
\begin{itemize}
    \item $\text{Var}[R_t - \bar{R}_t] < \text{Var}[R_t]$
    \item 因为减去了平均值,减少了波动
\end{itemize}

\subsection{与其他方法的关系}

\textbf{与动作价值方法}:
\begin{itemize}
    \item 动作价值方法:直接估计 $q_*(a)$
    \item 梯度方法:学习偏好 $H_t(a)$,间接影响选择
\end{itemize}

\textbf{与策略梯度方法}(第13章):
\begin{itemize}
    \item 梯度老虎机算法是策略梯度方法的特例
    \item 为理解 Actor-Critic 方法奠定基础
\end{itemize}

\section{算法实现要点}

\subsection{初始化}

\begin{verbatim}
# 初始化动作偏好
H = np.zeros(k)  # k 个动作,初始偏好都为 0

# 初始化基线(奖励平均值)
R_bar = 0.0
count = 0
\end{verbatim}

\subsection{动作选择}

\begin{verbatim}
# 计算 soft-max 概率
exp_H = np.exp(H)
pi = exp_H / np.sum(exp_H)

# 根据概率分布采样
action = np.random.choice(k, p=pi)
\end{verbatim}

\subsection{更新规则}

\begin{verbatim}
# 收到奖励后
reward = get_reward(action)

# 更新基线(增量更新)
count += 1
R_bar = R_bar + (1/count) * (reward - R_bar)
# 或使用常数步长:R_bar = R_bar + alpha_baseline * (reward - R_bar)

# 更新偏好
for a in range(k):
    if a == action:
        H[a] = H[a] + alpha * (reward - R_bar) * (1 - pi[a])
    else:
        H[a] = H[a] - alpha * (reward - R_bar) * pi[a]
\end{verbatim}

\subsection{数值稳定性}

\textbf{问题}:$e^{H_t(a)}$ 可能溢出

\textbf{解决方案}:
\begin{verbatim}
# 减去最大值(不改变概率分布)
H_shifted = H - np.max(H)
exp_H = np.exp(H_shifted)
pi = exp_H / np.sum(exp_H)
\end{verbatim}

\section{关键要点总结}

\begin{enumerate}
    \item \textbf{核心思想}:
    \begin{itemize}
        \item 学习动作偏好而非动作价值
        \item 使用 soft-max 分布将偏好转换为概率
    \end{itemize}
    
    \item \textbf{更新规则}:
    \begin{itemize}
        \item 被选择的动作:$H_{t+1}(A_t) = H_t(A_t) + \alpha[R_t - \bar{R}_t][1 - \pi_t(A_t)]$
        \item 未被选择的动作:$H_{t+1}(a) = H_t(a) - \alpha[R_t - \bar{R}_t]\pi_t(a)$
    \end{itemize}
    
    \item \textbf{基线的重要性}:
    \begin{itemize}
        \item 不改变期望更新
        \item 但显著减少方差
        \item 提高收敛速度
    \end{itemize}
    
    \item \textbf{理论保证}:
    \begin{itemize}
        \item 是随机梯度上升的实例
        \item 具有稳健的收敛性质
    \end{itemize}
    
    \item \textbf{实际应用}:
    \begin{itemize}
        \item 对奖励尺度变化鲁棒
        \item 需要适当选择步长参数
        \item 为策略梯度方法提供基础
    \end{itemize}
\end{enumerate}

\section{与后续内容的联系}

\begin{enumerate}
    \item \textbf{第13章:策略梯度方法}:
    \begin{itemize}
        \item 梯度老虎机算法是策略梯度方法的简化版本
        \item 理解本节有助于理解 Actor-Critic 方法
    \end{itemize}
    
    \item \textbf{REINFORCE 算法}:
    \begin{itemize}
        \item 使用类似的梯度上升思想
        \item 基线在 REINFORCE 中也很重要
    \end{itemize}
    
    \item \textbf{深度强化学习}:
    \begin{itemize}
        \item 策略网络使用类似的 soft-max 输出
        \item 梯度更新规则类似
    \end{itemize}
\end{enumerate}

\section{思考题}

\begin{enumerate}
    \item \textbf{为什么偏好值可以任意平移而不影响概率?}
    \begin{itemize}
        \item 因为 soft-max 只依赖于相对值
    \end{itemize}
    
    \item \textbf{基线为什么可以减少方差?}
    \begin{itemize}
        \item 因为 $\text{Var}[R_t - \bar{R}_t] < \text{Var}[R_t]$
    \end{itemize}
    
    \item \textbf{如何选择基线?}
    \begin{itemize}
        \item 奖励平均值是简单有效的选择
        \item 理论上任何不依赖于动作的值都可以
    \end{itemize}
    
    \item \textbf{梯度方法比动作价值方法有什么优势?}
    \begin{itemize}
        \item 更适合表示相对偏好
        \item 为策略梯度方法提供理论基础
        \item 对奖励尺度变化更鲁棒
    \end{itemize}
    
    \item \textbf{如何理解更新规则中的 $[1 - \pi_t(A_t)]$ 和 $\pi_t(a)$?}
    \begin{itemize}
        \item 这来自于 soft-max 的梯度
        \item 保证了概率分布始终归一化
    \end{itemize}
\end{enumerate}

\vspace{1cm}

\textbf{参考文献}:
\begin{itemize}
    \item Reinforcement Learning: An Introduction (2nd Edition), Section 2.8
    \item 相关数学推导和实验分析
\end{itemize}

\end{document}


