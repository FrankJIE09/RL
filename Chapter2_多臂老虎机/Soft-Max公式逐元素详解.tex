\documentclass[12pt,a4paper]{article}
\usepackage[UTF8]{ctex}
\usepackage{amsmath}
\usepackage{amssymb}
\usepackage{amsthm}
\usepackage{geometry}
\usepackage{hyperref}
\usepackage{enumitem}
\usepackage{xcolor}
\usepackage{bm}
\usepackage{tikz}

\geometry{margin=2.5cm}

\title{Soft-Max 公式逐元素详解}
\subtitle{$\pi_t(a) = \Pr\{A_t = a\} = \frac{e^{H_t(a)}}{\sum_{b=1}^k e^{H_t(b)}}$}
\author{}
\date{}

\newtheorem{definition}{定义}
\newtheorem{example}{示例}

\begin{document}

\maketitle

\section{公式概览}

本文档详细解释以下公式中的每一个元素:
\begin{equation}
\pi_t(a) = \Pr\{A_t = a\} = \frac{e^{H_t(a)}}{\sum_{b=1}^k e^{H_t(b)}}
\label{eq:softmax}
\end{equation}

这个公式将动作偏好转换为动作选择概率,是梯度老虎机算法和策略梯度方法的核心。

\section{公式结构}

公式 \eqref{eq:softmax} 由三个部分组成,用等号连接:
\begin{align}
\pi_t(a) &= \Pr\{A_t = a\} \\
         &= \frac{e^{H_t(a)}}{\sum_{b=1}^k e^{H_t(b)}}
\end{align}

这三个部分表示同一个概念的不同表示方式。

\section{左侧:$\pi_t(a)$}

\subsection{符号 $\pi$}

\begin{itemize}
    \item \textbf{字母选择}:$\pi$ 是希腊字母"pi",在强化学习中传统上用于表示\textbf{策略(policy)}
    \item \textbf{含义}:策略是从状态(或情况)到动作选择概率的映射
    \item \textbf{历史}:这个符号的选择可能源于"policy"的首字母,或者是数学中常用的函数符号
\end{itemize}

\subsection{下标 $t$}

\begin{itemize}
    \item \textbf{含义}:表示\textbf{时间步(time step)}
    \item \textbf{取值范围}:$t = 1, 2, 3, \ldots$(离散时间)
    \item \textbf{重要性}:策略是时变的,随着学习过程,策略会不断更新
    \item \textbf{示例}:
    \begin{itemize}
        \item $\pi_1(a)$:第1个时间步的策略
        \item $\pi_{100}(a)$:第100个时间步的策略
        \item $\pi_t(a)$:第$t$个时间步的策略(一般形式)
    \end{itemize}
\end{itemize}

\subsection{参数 $a$}

\begin{itemize}
    \item \textbf{含义}:表示\textbf{动作(action)}
    \item \textbf{取值范围}:$a \in \{1, 2, \ldots, k\}$,其中 $k$ 是动作总数
    \item \textbf{在公式中的作用}:$\pi_t(a)$ 表示在时间 $t$ 选择动作 $a$ 的概率
    \item \textbf{示例}:
    \begin{itemize}
        \item 在10臂老虎机问题中:$k = 10$,$a \in \{1, 2, \ldots, 10\}$
        \item $\pi_t(1)$:选择第1个动作的概率
        \item $\pi_t(5)$:选择第5个动作的概率
    \end{itemize}
\end{itemize}

\subsection{整体含义 $\pi_t(a)$}

\begin{definition}[策略函数]
$\pi_t(a)$ 表示在时间步 $t$,根据当前策略选择动作 $a$ 的概率。
\end{definition}

\textbf{关键性质}:
\begin{enumerate}
    \item \textbf{概率性}:$0 \leq \pi_t(a) \leq 1$ 对所有 $a$
    \item \textbf{归一化}:$\sum_{a=1}^k \pi_t(a) = 1$
    \item \textbf{时变性}:$\pi_t(a)$ 可能随 $t$ 变化
\end{enumerate}

\section{中间:$\Pr\{A_t = a\}$}

\subsection{符号 $\Pr$}

\begin{itemize}
    \item \textbf{含义}:\textbf{概率(Probability)}的缩写
    \item \textbf{标准记号}:在概率论中,$\Pr\{\cdot\}$ 表示事件发生的概率
    \item \textbf{等价记号}:有时也写作 $P(\cdot)$ 或 $\mathbb{P}(\cdot)$
\end{itemize}

\subsection{花括号 $\{\}$}

\begin{itemize}
    \item \textbf{含义}:表示一个\textbf{事件(event)}
    \item \textbf{在概率论中}:花括号内描述的是随机事件
    \item \textbf{这里的事件}:$\{A_t = a\}$ 表示"在时间 $t$ 选择的动作等于 $a$"这个事件
\end{itemize}

\subsection{随机变量 $A_t$}

\begin{itemize}
    \item \textbf{大写字母}:$A$ 是\textbf{随机变量(random variable)},用大写字母表示
    \item \textbf{含义}:表示在时间步 $t$ 选择的动作
    \item \textbf{随机性}:$A_t$ 的值是随机的,取决于策略 $\pi_t$
    \item \textbf{可能取值}:$A_t \in \{1, 2, \ldots, k\}$
    \item \textbf{示例}:
    \begin{itemize}
        \item $A_1 = 3$:在第1个时间步选择了动作3
        \item $A_5 = 7$:在第5个时间步选择了动作7
        \item $A_t = a$:在第$t$个时间步选择了动作$a$(一般形式)
    \end{itemize}
\end{itemize}

\subsection{等号 $=$}

\begin{itemize}
    \item \textbf{含义}:表示"等于"或"取值"
    \item \textbf{在事件中}:$\{A_t = a\}$ 表示"随机变量 $A_t$ 的取值等于 $a$"这个事件
    \item \textbf{概率意义}:$\Pr\{A_t = a\}$ 表示这个事件发生的概率
\end{itemize}

\subsection{整体含义 $\Pr\{A_t = a\}$}

\begin{definition}[动作选择概率]
$\Pr\{A_t = a\}$ 表示在时间步 $t$,智能体选择动作 $a$ 的概率。
\end{definition}

\textbf{与 $\pi_t(a)$ 的关系}:
\begin{equation}
\pi_t(a) = \Pr\{A_t = a\}
\end{equation}
这两个记号表示同一个概念:在时间 $t$ 选择动作 $a$ 的概率。

\section{右侧:分数形式}

右侧是一个分数(有理函数),形式为:
\begin{equation}
\frac{\text{分子}}{\text{分母}} = \frac{e^{H_t(a)}}{\sum_{b=1}^k e^{H_t(b)}}
\end{equation}

\subsection{分子:$e^{H_t(a)}$}

\subsubsection{常数 $e$}

\begin{itemize}
    \item \textbf{定义}:$e$ 是\textbf{自然常数(Euler's number)},也称为\textbf{纳皮尔常数}
    \item \textbf{数值}:$e \approx 2.718281828459045\ldots$
    \item \textbf{定义方式}:
    \begin{align}
    e &= \lim_{n \to \infty} \left(1 + \frac{1}{n}\right)^n \\
      &= \sum_{n=0}^{\infty} \frac{1}{n!} = 1 + 1 + \frac{1}{2!} + \frac{1}{3!} + \cdots
    \end{align}
    \item \textbf{重要性}:$e$ 是指数函数 $e^x$ 的底数,具有特殊的数学性质
    \item \textbf{在公式中的作用}:作为指数函数的底数,将偏好值转换为正数
\end{itemize}

\subsubsection{指数运算 $e^{H_t(a)}$}

\begin{itemize}
    \item \textbf{含义}:$e$ 的 $H_t(a)$ 次方
    \item \textbf{函数}:这是\textbf{指数函数(exponential function)}
    \item \textbf{性质}:
    \begin{enumerate}
        \item 对于任意实数 $x$,$e^x > 0$(总是正数)
        \item 如果 $x > y$,则 $e^x > e^y$(单调递增)
        \item $e^0 = 1$
        \item $e^1 = e \approx 2.718$
        \item 当 $x \to \infty$ 时,$e^x \to \infty$(增长非常快)
        \item 当 $x \to -\infty$ 时,$e^x \to 0$
    \end{enumerate}
    \item \textbf{在公式中的作用}:
    \begin{itemize}
        \item 将偏好值 $H_t(a)$ 转换为正数
        \item 保持单调性:如果 $H_t(a) > H_t(b)$,则 $e^{H_t(a)} > e^{H_t(b)}$
        \item 放大差异:较大的偏好值会产生更大的指数值
    \end{itemize}
\end{itemize}

\subsubsection{偏好值 $H_t(a)$}

\begin{itemize}
    \item \textbf{字母选择}:$H$ 可能代表"preference"(偏好)或"heat"(热度,来自统计物理)
    \item \textbf{含义}:表示在时间步 $t$,对动作 $a$ 的\textbf{数值偏好(numerical preference)}
    \item \textbf{取值范围}:$H_t(a) \in \mathbb{R}$(可以是任意实数)
    \item \textbf{关键特性}:
    \begin{enumerate}
        \item \textbf{相对性}:只有相对大小重要,绝对值不重要
        \item \textbf{可学习}:$H_t(a)$ 会随着学习过程更新
        \item \textbf{无单位}:偏好值本身没有物理意义,只是数值
    \end{enumerate}
    \item \textbf{更新规则}(在梯度老虎机算法中):
    \begin{align}
    H_{t+1}(A_t) &= H_t(A_t) + \alpha[R_t - \bar{R}_t][1 - \pi_t(A_t)] \\
    H_{t+1}(a) &= H_t(a) - \alpha[R_t - \bar{R}_t]\pi_t(a), \quad \forall a \neq A_t
    \end{align}
    \item \textbf{初始值}:通常 $H_1(a) = 0$ 对所有 $a$,使得初始时所有动作等概率
\end{itemize}

\subsubsection{整体含义 $e^{H_t(a)}$}

\begin{itemize}
    \item \textbf{作用}:将动作 $a$ 的偏好值转换为一个正数
    \item \textbf{性质}:
    \begin{itemize}
        \item 总是正数:$e^{H_t(a)} > 0$
        \item 单调性:如果 $H_t(a) > H_t(b)$,则 $e^{H_t(a)} > e^{H_t(b)}$
        \item 放大效应:偏好值的差异会被指数函数放大
    \end{itemize}
    \item \textbf{示例}:
    \begin{center}
    \begin{tabular}{|c|c|}
    \hline
    $H_t(a)$ & $e^{H_t(a)}$ \\
    \hline
    $-2$ & $0.135$ \\
    $-1$ & $0.368$ \\
    $0$ & $1.000$ \\
    $1$ & $2.718$ \\
    $2$ & $7.389$ \\
    $3$ & $20.086$ \\
    \hline
    \end{tabular}
    \end{center}
\end{itemize}

\subsection{分母:$\sum_{b=1}^k e^{H_t(b)}$}

\subsubsection{求和符号 $\sum$}

\begin{itemize}
    \item \textbf{含义}:\textbf{求和(summation)}符号,表示将一系列数相加
    \item \textbf{读法}:"sigma"(希腊字母)
    \item \textbf{标准形式}:$\sum_{i=m}^n f(i)$ 表示从 $i=m$ 到 $i=n$ 的 $f(i)$ 的和
\end{itemize}

\subsubsection{求和变量 $b$}

\begin{itemize}
    \item \textbf{含义}:\textbf{求和索引(summation index)},也称为"哑变量(dummy variable)"
    \item \textbf{取值范围}:$b = 1, 2, \ldots, k$
    \item \textbf{为什么用 $b$?}:
    \begin{itemize}
        \item 在分子中已经用了 $a$ 表示特定动作
        \item 为了避免混淆,分母中用 $b$ 表示求和变量
        \item 这是数学中的常见做法:用不同字母区分固定变量和求和变量
    \end{itemize}
    \item \textbf{可替换性}:$b$ 可以用任何其他字母替换,如 $i, j, c$ 等,结果不变:
    \begin{equation}
    \sum_{b=1}^k e^{H_t(b)} = \sum_{i=1}^k e^{H_t(i)} = \sum_{j=1}^k e^{H_t(j)}
    \end{equation}
\end{itemize}

\subsubsection{下界 $b=1$}

\begin{itemize}
    \item \textbf{含义}:求和的起始值
    \item \textbf{假设}:动作编号从 1 开始
    \item \textbf{一般性}:如果动作编号从 0 开始,则下界为 $b=0$
    \item \textbf{在集合中}:也可以写作 $\sum_{b \in \mathcal{A}} e^{H_t(b)}$,其中 $\mathcal{A} = \{1, 2, \ldots, k\}$ 是动作集合
\end{itemize}

\subsubsection{上界 $k$}

\begin{itemize}
    \item \textbf{含义}:\textbf{动作总数(number of actions)}
    \item \textbf{在 $k$-臂老虎机中}:$k$ 就是"臂"的数量
    \item \textbf{示例}:
    \begin{itemize}
        \item 10-臂老虎机:$k = 10$
        \item 2-臂老虎机:$k = 2$(二分类问题)
        \item 一般情况:$k$ 可以是任意正整数
    \end{itemize}
    \item \textbf{在公式中的作用}:确定求和的范围
\end{itemize}

\subsubsection{被求和项 $e^{H_t(b)}$}

\begin{itemize}
    \item \textbf{含义}:对每个动作 $b$,计算 $e^{H_t(b)}$
    \item \textbf{与分子的关系}:
    \begin{itemize}
        \item 分子:$e^{H_t(a)}$(特定动作 $a$)
        \item 分母:$\sum_{b=1}^k e^{H_t(b)}$(所有动作的指数和)
    \end{itemize}
    \item \textbf{展开形式}:
    \begin{align}
    \sum_{b=1}^k e^{H_t(b)} &= e^{H_t(1)} + e^{H_t(2)} + e^{H_t(3)} + \cdots + e^{H_t(k)} \\
                            &= \sum_{i=1}^k e^{H_t(i)}
    \end{align}
\end{itemize}

\subsubsection{整体含义 $\sum_{b=1}^k e^{H_t(b)}$}

\begin{itemize}
    \item \textbf{作用}:\textbf{归一化因子(normalization factor)}
    \item \textbf{目的}:确保所有动作的概率和为 1
    \item \textbf{计算}:将所有动作的指数值相加
    \item \textbf{性质}:
    \begin{itemize}
        \item 总是正数:$\sum_{b=1}^k e^{H_t(b)} > 0$(因为每个 $e^{H_t(b)} > 0$)
        \item 依赖于所有动作的偏好值
        \item 随着偏好值的变化而变化
    \end{itemize}
\end{itemize}

\section{公式整体理解}

\subsection{完整公式的含义}

\begin{equation}
\pi_t(a) = \Pr\{A_t = a\} = \frac{e^{H_t(a)}}{\sum_{b=1}^k e^{H_t(b)}}
\end{equation}

\textbf{文字描述}:
\begin{quote}
在时间步 $t$,选择动作 $a$ 的概率等于:动作 $a$ 的偏好值的指数,除以所有动作的偏好值的指数之和。
\end{quote}

\subsection{工作流程}

\begin{enumerate}
    \item \textbf{输入}:每个动作的偏好值 $H_t(1), H_t(2), \ldots, H_t(k)$
    \item \textbf{指数化}:计算每个动作的指数 $e^{H_t(1)}, e^{H_t(2)}, \ldots, e^{H_t(k)}$
    \item \textbf{求和}:计算所有指数的和 $\sum_{b=1}^k e^{H_t(b)}$
    \item \textbf{归一化}:对每个动作,用其指数除以总和
    \item \textbf{输出}:得到每个动作的选择概率 $\pi_t(1), \pi_t(2), \ldots, \pi_t(k)$
\end{enumerate}

\subsection{数值示例}

假设在时间步 $t$,三个动作的偏好值为:
\begin{align}
H_t(1) &= 1 \\
H_t(2) &= 2 \\
H_t(3) &= 3
\end{align}

\textbf{计算过程}:

\begin{enumerate}
    \item 计算指数:
    \begin{align}
    e^{H_t(1)} &= e^1 \approx 2.718 \\
    e^{H_t(2)} &= e^2 \approx 7.389 \\
    e^{H_t(3)} &= e^3 \approx 20.086
    \end{align}
    
    \item 计算分母(归一化因子):
    \begin{equation}
    \sum_{b=1}^3 e^{H_t(b)} = 2.718 + 7.389 + 20.086 = 30.193
    \end{equation}
    
    \item 计算每个动作的概率:
    \begin{align}
    \pi_t(1) &= \frac{2.718}{30.193} \approx 0.090 \\
    \pi_t(2) &= \frac{7.389}{30.193} \approx 0.245 \\
    \pi_t(3) &= \frac{20.086}{30.193} \approx 0.665
    \end{align}
    
    \item 验证归一化:
    \begin{equation}
    \pi_t(1) + \pi_t(2) + \pi_t(3) = 0.090 + 0.245 + 0.665 = 1.000 \quad \checkmark
    \end{equation}
\end{enumerate}

\section{关键性质总结}

\subsection{概率性质}

\begin{enumerate}
    \item \textbf{非负性}:$\pi_t(a) \geq 0$ 对所有 $a$(因为分子和分母都是正数)
    \item \textbf{归一化}:$\sum_{a=1}^k \pi_t(a) = 1$(由分母的构造保证)
    \item \textbf{概率分布}:$\{\pi_t(1), \pi_t(2), \ldots, \pi_t(k)\}$ 构成一个有效的概率分布
\end{enumerate}

\subsection{单调性}

\begin{proposition}[单调性]
如果 $H_t(a) > H_t(b)$,则 $\pi_t(a) > \pi_t(b)$。
\end{proposition}

\textbf{证明}:
\begin{itemize}
    \item 因为指数函数单调递增,$H_t(a) > H_t(b)$ 意味着 $e^{H_t(a)} > e^{H_t(b)}$
    \item 分母对所有动作都相同,所以 $\frac{e^{H_t(a)}}{\sum_{b=1}^k e^{H_t(b)}} > \frac{e^{H_t(b)}}{\sum_{b=1}^k e^{H_t(b)}}$
    \item 因此 $\pi_t(a) > \pi_t(b)$
\end{itemize}

\subsection{平移不变性}

\begin{proposition}[平移不变性]
如果对所有动作 $a$,$H_t(a) \leftarrow H_t(a) + c$($c$ 为常数),则 $\pi_t(a)$ 不变。
\end{proposition}

\textbf{证明}:
\begin{align}
\pi_t(a) &= \frac{e^{H_t(a) + c}}{\sum_{b=1}^k e^{H_t(b) + c}} \\
         &= \frac{e^c \cdot e^{H_t(a)}}{e^c \cdot \sum_{b=1}^k e^{H_t(b)}} \\
         &= \frac{e^{H_t(a)}}{\sum_{b=1}^k e^{H_t(b)}}
\end{align}

\textbf{意义}:只有偏好值的\textbf{相对大小}重要,绝对值不重要。

\section{常见问题}

\subsection{为什么使用 $e$ 而不是其他底数?}

\begin{itemize}
    \item \textbf{数学便利性}:$e$ 的导数等于自身,便于梯度计算
    \item \textbf{传统}:在统计物理和信息论中,$e$ 是自然选择
    \item \textbf{可替换性}:理论上可以用任何正数作为底数,但 $e$ 最方便
\end{itemize}

\subsection{为什么分母要对所有动作求和?}

\begin{itemize}
    \item \textbf{归一化}:确保概率和为 1
    \item \textbf{相对性}:概率是相对于所有可能动作的
    \item \textbf{竞争性}:一个动作概率的增加必然导致其他动作概率的减少
\end{itemize}

\subsection{如果某个 $H_t(a)$ 非常大怎么办?}

\begin{itemize}
    \item \textbf{数值溢出}:$e^{H_t(a)}$ 可能溢出
    \item \textbf{解决方案}:利用平移不变性,减去最大值:
    \begin{equation}
    \pi_t(a) = \frac{e^{H_t(a) - \max_b H_t(b)}}{\sum_{b=1}^k e^{H_t(b) - \max_b H_t(b)}}
    \end{equation}
    \item \textbf{效果}:最大偏好值变为 0,其他为负数,避免溢出
\end{itemize}

\section{总结}

公式 $\pi_t(a) = \frac{e^{H_t(a)}}{\sum_{b=1}^k e^{H_t(b)}}$ 中的每个元素都有明确的含义:

\begin{center}
\begin{tabular}{|c|l|}
\hline
\textbf{元素} & \textbf{含义} \\
\hline
$\pi_t(a)$ & 在时间 $t$ 选择动作 $a$ 的概率 \\
$\Pr\{A_t = a\}$ & 随机变量 $A_t$ 等于 $a$ 的概率 \\
$e$ & 自然常数(约 2.718) \\
$H_t(a)$ & 在时间 $t$ 对动作 $a$ 的偏好值 \\
$e^{H_t(a)}$ & 偏好值的指数(正数) \\
$\sum$ & 求和符号 \\
$b$ & 求和变量(遍历所有动作) \\
$1$ & 求和的起始值(动作编号起点) \\
$k$ & 动作总数 \\
$\sum_{b=1}^k e^{H_t(b)}$ & 归一化因子(所有动作指数的和) \\
\hline
\end{tabular}
\end{center}

这个公式将任意实数偏好值转换为有效的概率分布,是强化学习中策略表示的基础。

\vspace{1cm}

\textbf{参考文献}:
\begin{itemize}
    \item Sutton, R. S., \& Barto, A. G. (2018). Reinforcement Learning: An Introduction (2nd Edition). MIT Press.
    \item Bishop, C. M. (2006). Pattern Recognition and Machine Learning. Springer.
\end{itemize}

\end{document}


