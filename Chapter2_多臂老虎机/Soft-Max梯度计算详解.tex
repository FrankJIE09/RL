\documentclass[12pt,a4paper]{article}
\usepackage[UTF8]{ctex}
\usepackage{amsmath}
\usepackage{amssymb}
\usepackage{amsthm}
\usepackage{geometry}
\usepackage{hyperref}
\usepackage{enumitem}
\usepackage{xcolor}
\usepackage{bm}
\usepackage{tikz}
\usepackage{pgfplots}

\geometry{margin=2.5cm}

\title{Soft-Max 梯度计算详解}
\subtitle{重点:梯度推导与 Kronecker Delta 符号}
\author{}
\date{}

\newtheorem{definition}{定义}
\newtheorem{theorem}{定理}
\newtheorem{proposition}{命题}
\newtheorem{example}{示例}

\begin{document}

\maketitle

\tableofcontents
\newpage

\section{引言}

Soft-Max 函数的梯度计算是理解梯度老虎机算法和策略梯度方法的关键。本文档详细推导 soft-max 的梯度,并深入解释 Kronecker delta 符号 $\delta_{ij}$ 的含义和应用。

\section{Kronecker Delta 符号详解}

\subsection{定义}

\begin{definition}[Kronecker Delta]
Kronecker delta 是一个二元函数,定义为:
\begin{equation}
\delta_{ij} = \begin{cases}
1 & \text{如果 } i = j \\
0 & \text{如果 } i \neq j
\end{cases}
\end{equation}
其中 $i$ 和 $j$ 是整数索引。
\end{definition}

\subsection{符号来源}

\begin{itemize}
    \item \textbf{命名}:以德国数学家 Leopold Kronecker (1823-1891) 命名
    \item \textbf{符号}:$\delta$ 是希腊字母"delta"
    \item \textbf{历史}:Kronecker 在数论和代数中广泛使用这个符号
\end{itemize}

\subsection{基本性质}

\subsubsection{对称性}
\begin{equation}
\delta_{ij} = \delta_{ji}
\end{equation}
因为等式关系是对称的。

\subsubsection{选择性质}
\begin{equation}
\sum_{j=1}^n a_j \delta_{ij} = a_i
\end{equation}
\textbf{解释}:在求和中,只有当 $j = i$ 时,$\delta_{ij} = 1$,其他项都是 0,所以只保留 $a_i$。

\textbf{示例}:
\begin{align}
\sum_{j=1}^3 a_j \delta_{2j} &= a_1 \delta_{21} + a_2 \delta_{22} + a_3 \delta_{23} \\
                              &= a_1 \cdot 0 + a_2 \cdot 1 + a_3 \cdot 0 \\
                              &= a_2
\end{align}

\subsubsection{单位矩阵表示}
Kronecker delta 可以表示单位矩阵的元素:
\begin{equation}
\mathbf{I}_{ij} = \delta_{ij}
\end{equation}
其中 $\mathbf{I}$ 是单位矩阵。

\textbf{示例}(3×3 单位矩阵):
\begin{equation}
\mathbf{I} = \begin{pmatrix}
\delta_{11} & \delta_{12} & \delta_{13} \\
\delta_{21} & \delta_{22} & \delta_{23} \\
\delta_{31} & \delta_{32} & \delta_{33}
\end{pmatrix} = \begin{pmatrix}
1 & 0 & 0 \\
0 & 1 & 0 \\
0 & 0 & 1
\end{pmatrix}
\end{equation}

\subsection{在导数中的应用}

Kronecker delta 在计算偏导数时非常有用,特别是当我们需要区分"对自身求导"和"对其他变量求导"的情况。

\textbf{一般规则}:
\begin{equation}
\frac{\partial x_i}{\partial x_j} = \delta_{ij}
\end{equation}

\textbf{解释}:
\begin{itemize}
    \item 如果 $i = j$:$\frac{\partial x_i}{\partial x_i} = 1$(对自身求导)
    \item 如果 $i \neq j$:$\frac{\partial x_i}{\partial x_j} = 0$(对其他变量求导,结果为 0)
\end{itemize}

\textbf{示例}:
\begin{align}
\frac{\partial x_1}{\partial x_1} &= \delta_{11} = 1 \\
\frac{\partial x_1}{\partial x_2} &= \delta_{12} = 0 \\
\frac{\partial x_2}{\partial x_1} &= \delta_{21} = 0 \\
\frac{\partial x_2}{\partial x_2} &= \delta_{22} = 1
\end{align}

\subsection{在 Soft-Max 梯度中的作用}

在 soft-max 梯度计算中,Kronecker delta 用于区分两种情况:
\begin{enumerate}
    \item \textbf{对角情况}($i = j$):对自身求导
    \item \textbf{非对角情况}($i \neq j$):对其他元素求导
\end{enumerate}

\section{Soft-Max 函数回顾}

\subsection{定义}

Soft-Max 函数将 $k$ 个实数 $z_1, z_2, \ldots, z_k$ 转换为概率分布:
\begin{equation}
\sigma_i = \text{softmax}(\bm{z})_i = \frac{e^{z_i}}{\sum_{j=1}^k e^{z_j}}, \quad i = 1, 2, \ldots, k
\end{equation}

其中 $\bm{z} = (z_1, z_2, \ldots, z_k)^T$ 是输入向量。

\subsection{关键性质}

\begin{enumerate}
    \item \textbf{归一化}:$\sum_{i=1}^k \sigma_i = 1$
    \item \textbf{非负性}:$\sigma_i \geq 0$ 对所有 $i$
    \item \textbf{可微性}:$\sigma_i$ 对所有 $z_j$ 可微
\end{enumerate}

\section{Soft-Max 梯度推导}

\subsection{目标}

计算 soft-max 函数关于输入 $z_j$ 的偏导数:
\begin{equation}
\frac{\partial \sigma_i}{\partial z_j} = \frac{\partial}{\partial z_j} \left[\frac{e^{z_i}}{\sum_{l=1}^k e^{z_l}}\right]
\end{equation}

\subsection{商的求导法则}

我们需要使用\textbf{商的求导法则}:
\begin{equation}
\frac{d}{dx}\left[\frac{f(x)}{g(x)}\right] = \frac{f'(x)g(x) - f(x)g'(x)}{[g(x)]^2}
\end{equation}

在我们的情况下:
\begin{align}
f(z_j) &= e^{z_i} \\
g(z_j) &= \sum_{l=1}^k e^{z_l}
\end{align}

\subsection{分情况讨论}

由于 $z_i$ 和 $z_j$ 的关系,我们需要分两种情况:

\subsubsection{情况 1:$i = j$(对角情况)}

当 $i = j$ 时,我们对 $\sigma_i$ 关于 $z_i$ 求导。

\textbf{步骤 1}:应用商的求导法则
\begin{align}
\frac{\partial \sigma_i}{\partial z_i} &= \frac{\frac{\partial e^{z_i}}{\partial z_i} \cdot \sum_{l=1}^k e^{z_l} - e^{z_i} \cdot \frac{\partial \sum_{l=1}^k e^{z_l}}{\partial z_i}}{\left(\sum_{l=1}^k e^{z_l}\right)^2}
\end{align}

\textbf{步骤 2}:计算各项导数
\begin{align}
\frac{\partial e^{z_i}}{\partial z_i} &= e^{z_i} \\
\frac{\partial \sum_{l=1}^k e^{z_l}}{\partial z_i} &= \frac{\partial}{\partial z_i}\left[e^{z_1} + e^{z_2} + \cdots + e^{z_i} + \cdots + e^{z_k}\right] \\
                                                    &= 0 + 0 + \cdots + e^{z_i} + \cdots + 0 \\
                                                    &= e^{z_i}
\end{align}

\textbf{步骤 3}:代入并化简
\begin{align}
\frac{\partial \sigma_i}{\partial z_i} &= \frac{e^{z_i} \cdot \sum_{l=1}^k e^{z_l} - e^{z_i} \cdot e^{z_i}}{\left(\sum_{l=1}^k e^{z_l}\right)^2} \\
                                       &= \frac{e^{z_i}\left(\sum_{l=1}^k e^{z_l} - e^{z_i}\right)}{\left(\sum_{l=1}^k e^{z_l}\right)^2} \\
                                       &= \frac{e^{z_i}}{\sum_{l=1}^k e^{z_l}} \cdot \frac{\sum_{l=1}^k e^{z_l} - e^{z_i}}{\sum_{l=1}^k e^{z_l}} \\
                                       &= \sigma_i \cdot \frac{\sum_{l=1}^k e^{z_l} - e^{z_i}}{\sum_{l=1}^k e^{z_l}} \\
                                       &= \sigma_i \left(1 - \frac{e^{z_i}}{\sum_{l=1}^k e^{z_l}}\right) \\
                                       &= \sigma_i (1 - \sigma_i)
\end{align}

\subsubsection{情况 2:$i \neq j$(非对角情况)}

当 $i \neq j$ 时,我们对 $\sigma_i$ 关于 $z_j$ 求导。

\textbf{步骤 1}:应用商的求导法则
\begin{align}
\frac{\partial \sigma_i}{\partial z_j} &= \frac{\frac{\partial e^{z_i}}{\partial z_j} \cdot \sum_{l=1}^k e^{z_l} - e^{z_i} \cdot \frac{\partial \sum_{l=1}^k e^{z_l}}{\partial z_j}}{\left(\sum_{l=1}^k e^{z_l}\right)^2}
\end{align}

\textbf{步骤 2}:计算各项导数
\begin{align}
\frac{\partial e^{z_i}}{\partial z_j} &= 0 \quad \text{(因为 } i \neq j \text{,} z_i \text{ 不依赖于 } z_j\text{)} \\
\frac{\partial \sum_{l=1}^k e^{z_l}}{\partial z_j} &= e^{z_j} \quad \text{(只有 } e^{z_j} \text{ 项对 } z_j \text{ 求导非零)}
\end{align}

\textbf{步骤 3}:代入并化简
\begin{align}
\frac{\partial \sigma_i}{\partial z_j} &= \frac{0 \cdot \sum_{l=1}^k e^{z_l} - e^{z_i} \cdot e^{z_j}}{\left(\sum_{l=1}^k e^{z_l}\right)^2} \\
                                       &= \frac{-e^{z_i} e^{z_j}}{\left(\sum_{l=1}^k e^{z_l}\right)^2} \\
                                       &= -\frac{e^{z_i}}{\sum_{l=1}^k e^{z_l}} \cdot \frac{e^{z_j}}{\sum_{l=1}^k e^{z_l}} \\
                                       &= -\sigma_i \sigma_j
\end{align}

\subsection{统一公式}

使用 Kronecker delta,我们可以将两种情况统一为一个公式:

\begin{theorem}[Soft-Max 梯度]
Soft-Max 函数关于输入 $z_j$ 的偏导数为:
\begin{equation}
\frac{\partial \sigma_i}{\partial z_j} = \sigma_i (\delta_{ij} - \sigma_j)
\label{eq:softmax_gradient}
\end{equation}
\end{theorem}

\textbf{验证}:

\textbf{情况 1}:$i = j$($\delta_{ij} = 1$)
\begin{align}
\frac{\partial \sigma_i}{\partial z_i} &= \sigma_i (1 - \sigma_i) = \sigma_i - \sigma_i^2 \quad \checkmark
\end{align}

\textbf{情况 2}:$i \neq j$($\delta_{ij} = 0$)
\begin{align}
\frac{\partial \sigma_i}{\partial z_j} &= \sigma_i (0 - \sigma_j) = -\sigma_i \sigma_j \quad \checkmark
\end{align}

\subsection{矩阵形式}

如果我们计算所有偏导数,可以得到\textbf{雅可比矩阵(Jacobian matrix)}:
\begin{equation}
\mathbf{J}_{ij} = \frac{\partial \sigma_i}{\partial z_j} = \sigma_i (\delta_{ij} - \sigma_j)
\end{equation}

\textbf{展开形式}(以 $k=3$ 为例):
\begin{equation}
\mathbf{J} = \begin{pmatrix}
\sigma_1(1-\sigma_1) & -\sigma_1\sigma_2 & -\sigma_1\sigma_3 \\
-\sigma_2\sigma_1 & \sigma_2(1-\sigma_2) & -\sigma_2\sigma_3 \\
-\sigma_3\sigma_1 & -\sigma_3\sigma_2 & \sigma_3(1-\sigma_3)
\end{pmatrix}
\end{equation}

\section{梯度公式的深入理解}

\subsection{对角元素:$\frac{\partial \sigma_i}{\partial z_i} = \sigma_i(1-\sigma_i)$}

\textbf{含义}:
\begin{itemize}
    \item 这是 $\sigma_i$ 关于自身输入 $z_i$ 的敏感度
    \item 形式类似于\textbf{逻辑函数的导数}
    \item 当 $\sigma_i$ 接近 0 或 1 时,梯度很小(饱和)
    \item 当 $\sigma_i = 0.5$ 时,梯度最大($0.5 \times 0.5 = 0.25$)
\end{itemize}

\textbf{图形理解}:
\begin{center}
\begin{tikzpicture}
\begin{axis}[
    xlabel={$\sigma_i$},
    ylabel={$\frac{\partial \sigma_i}{\partial z_i}$},
    domain=0:1,
    samples=100,
    width=10cm,
    height=6cm
]
\addplot[blue, thick] {x*(1-x)};
\end{axis}
\end{tikzpicture}
\end{center}

这是一个倒置的抛物线,在 $\sigma_i = 0.5$ 处达到最大值。

\subsection{非对角元素:$\frac{\partial \sigma_i}{\partial z_j} = -\sigma_i\sigma_j$}

\textbf{含义}:
\begin{itemize}
    \item 这是 $\sigma_i$ 关于其他输入 $z_j$($j \neq i$)的敏感度
    \item 总是\textbf{负数}(因为概率是竞争性的)
    \item 当 $\sigma_i$ 或 $\sigma_j$ 接近 0 时,梯度接近 0
    \item 当两者都较大时,梯度(绝对值)较大
\end{itemize}

\textbf{直观理解}:
\begin{itemize}
    \item 如果增加 $z_j$,则 $\sigma_j$ 增加
    \item 由于归一化约束,$\sigma_i$ 必须减少
    \item 因此梯度为负
\end{itemize}

\subsection{归一化约束的影响}

\textbf{关键观察}:
\begin{equation}
\sum_{j=1}^k \frac{\partial \sigma_i}{\partial z_j} = \sum_{j=1}^k \sigma_i (\delta_{ij} - \sigma_j) = \sigma_i \left(\sum_{j=1}^k \delta_{ij} - \sum_{j=1}^k \sigma_j\right) = \sigma_i (1 - 1) = 0
\end{equation}

\textbf{解释}:
\begin{itemize}
    \item 所有偏导数的和为 0
    \item 这反映了归一化约束:$\sum_{i=1}^k \sigma_i = 1$
    \item 如果所有 $z_j$ 同时增加相同量,$\sigma_i$ 不变(平移不变性)
\end{itemize}

\section{数值示例}

\subsection{示例 1:简单情况}

设 $k=3$,输入为 $\bm{z} = (1, 2, 3)^T$。

\textbf{步骤 1}:计算 soft-max 输出
\begin{align}
e^{z_1} &= e^1 \approx 2.718 \\
e^{z_2} &= e^2 \approx 7.389 \\
e^{z_3} &= e^3 \approx 20.086 \\
\sum e^{z_i} &\approx 30.193
\end{align}

\begin{align}
\sigma_1 &= \frac{2.718}{30.193} \approx 0.090 \\
\sigma_2 &= \frac{7.389}{30.193} \approx 0.245 \\
\sigma_3 &= \frac{20.086}{30.193} \approx 0.665
\end{align}

\textbf{步骤 2}:计算梯度矩阵

\textbf{对角元素}:
\begin{align}
\frac{\partial \sigma_1}{\partial z_1} &= 0.090 \times (1 - 0.090) = 0.090 \times 0.910 = 0.082 \\
\frac{\partial \sigma_2}{\partial z_2} &= 0.245 \times (1 - 0.245) = 0.245 \times 0.755 = 0.185 \\
\frac{\partial \sigma_3}{\partial z_3} &= 0.665 \times (1 - 0.665) = 0.665 \times 0.335 = 0.223
\end{align}

\textbf{非对角元素}:
\begin{align}
\frac{\partial \sigma_1}{\partial z_2} &= -0.090 \times 0.245 = -0.022 \\
\frac{\partial \sigma_1}{\partial z_3} &= -0.090 \times 0.665 = -0.060 \\
\frac{\partial \sigma_2}{\partial z_1} &= -0.245 \times 0.090 = -0.022 \\
\frac{\partial \sigma_2}{\partial z_3} &= -0.245 \times 0.665 = -0.163 \\
\frac{\partial \sigma_3}{\partial z_1} &= -0.665 \times 0.090 = -0.060 \\
\frac{\partial \sigma_3}{\partial z_2} &= -0.665 \times 0.245 = -0.163
\end{align}

\textbf{完整雅可比矩阵}:
\begin{equation}
\mathbf{J} \approx \begin{pmatrix}
0.082 & -0.022 & -0.060 \\
-0.022 & 0.185 & -0.163 \\
-0.060 & -0.163 & 0.223
\end{pmatrix}
\end{equation}

\textbf{验证}:每行和应该为 0(归一化约束)
\begin{align}
\text{第1行和} &= 0.082 - 0.022 - 0.060 = 0.000 \quad \checkmark \\
\text{第2行和} &= -0.022 + 0.185 - 0.163 = 0.000 \quad \checkmark \\
\text{第3行和} &= -0.060 - 0.163 + 0.223 = 0.000 \quad \checkmark
\end{align}

\section{在梯度老虎机算法中的应用}

\subsection{性能梯度}

在梯度老虎机算法中,我们需要计算期望奖励关于偏好 $H_t(a)$ 的梯度:
\begin{equation}
\frac{\partial \mathbb{E}[R_t]}{\partial H_t(a)} = \frac{\partial}{\partial H_t(a)} \left[\sum_x \pi_t(x) q_*(x)\right]
\end{equation}

\subsection{使用 Soft-Max 梯度}

由于 $\pi_t(x) = \text{softmax}(\bm{H}_t)_x$,我们可以使用 soft-max 梯度:
\begin{align}
\frac{\partial \mathbb{E}[R_t]}{\partial H_t(a)} &= \sum_x q_*(x) \frac{\partial \pi_t(x)}{\partial H_t(a)} \\
                                                  &= \sum_x q_*(x) \pi_t(x) (\delta_{ax} - \pi_t(a)) \\
                                                  &= \sum_x q_*(x) \pi_t(x) \delta_{ax} - \pi_t(a) \sum_x q_*(x) \pi_t(x) \\
                                                  &= q_*(a) \pi_t(a) - \pi_t(a) \mathbb{E}[R_t] \\
                                                  &= \pi_t(a) [q_*(a) - \mathbb{E}[R_t]]
\end{align}

\textbf{关键步骤}:使用了 Kronecker delta 的选择性质:
\begin{equation}
\sum_x q_*(x) \pi_t(x) \delta_{ax} = q_*(a) \pi_t(a)
\end{equation}

\subsection{引入基线}

为了减少方差,我们引入基线 $B_t$(不依赖于 $x$):
\begin{align}
\frac{\partial \mathbb{E}[R_t]}{\partial H_t(a)} &= \sum_x [q_*(x) - B_t] \pi_t(x) (\delta_{ax} - \pi_t(a)) \\
                                                  &= \sum_x [q_*(x) - B_t] \pi_t(x) \delta_{ax} - \pi_t(a) \sum_x [q_*(x) - B_t] \pi_t(x) \\
                                                  &= [q_*(a) - B_t] \pi_t(a) - \pi_t(a) \sum_x [q_*(x) - B_t] \pi_t(x)
\end{align}

由于 $\sum_x \pi_t(x) = 1$,第二项为 0:
\begin{equation}
\sum_x [q_*(x) - B_t] \pi_t(x) = \sum_x q_*(x) \pi_t(x) - B_t \sum_x \pi_t(x) = \mathbb{E}[R_t] - B_t
\end{equation}

如果选择 $B_t = \mathbb{E}[R_t]$,则第二项为 0。

\section{常见错误与注意事项}

\subsection{错误 1:忘记 Kronecker Delta}

\textbf{错误写法}:
\begin{equation}
\frac{\partial \sigma_i}{\partial z_j} = \sigma_i (1 - \sigma_j) \quad \text{❌ 错误!}
\end{equation}

\textbf{正确写法}:
\begin{equation}
\frac{\partial \sigma_i}{\partial z_j} = \sigma_i (\delta_{ij} - \sigma_j) \quad \text{✓ 正确}
\end{equation}

\textbf{区别}:错误公式在 $i \neq j$ 时给出错误结果。

\subsection{错误 2:混淆索引}

\textbf{常见混淆}:
\begin{itemize}
    \item $\frac{\partial \sigma_i}{\partial z_j}$:对哪个变量求导?
    \item 记住:分母的下标 $j$ 表示对 $z_j$ 求导
    \item 分子的下标 $i$ 表示哪个 $\sigma$ 的导数
\end{itemize}

\subsection{数值稳定性}

在计算梯度时,也要注意数值稳定性:
\begin{verbatim}
# 数值稳定的 soft-max 梯度计算
def softmax_gradient_stable(z):
    z_shifted = z - np.max(z)
    exp_z = np.exp(z_shifted)
    sigma = exp_z / np.sum(exp_z)
    
    # 计算梯度矩阵
    k = len(z)
    J = np.zeros((k, k))
    for i in range(k):
        for j in range(k):
            delta_ij = 1 if i == j else 0
            J[i, j] = sigma[i] * (delta_ij - sigma[j])
    
    return J
\end{verbatim}

\section{总结}

\subsection{关键公式}

\begin{enumerate}
    \item \textbf{Soft-Max 梯度}:
    \begin{equation}
    \frac{\partial \sigma_i}{\partial z_j} = \sigma_i (\delta_{ij} - \sigma_j)
    \end{equation}
    
    \item \textbf{Kronecker Delta}:
    \begin{equation}
    \delta_{ij} = \begin{cases}
    1 & \text{如果 } i = j \\
    0 & \text{如果 } i \neq j
    \end{cases}
    \end{equation}
    
    \item \textbf{选择性质}:
    \begin{equation}
    \sum_j a_j \delta_{ij} = a_i
    \end{equation}
\end{enumerate}

\subsection{重要性质}

\begin{enumerate}
    \item \textbf{对角元素}:$\frac{\partial \sigma_i}{\partial z_i} = \sigma_i(1-\sigma_i)$(总是正数)
    \item \textbf{非对角元素}:$\frac{\partial \sigma_i}{\partial z_j} = -\sigma_i\sigma_j$(总是负数)
    \item \textbf{归一化约束}:$\sum_j \frac{\partial \sigma_i}{\partial z_j} = 0$
    \item \textbf{对称性}:$\frac{\partial \sigma_i}{\partial z_j} \neq \frac{\partial \sigma_j}{\partial z_i}$(一般不对称)
\end{enumerate}

\subsection{应用场景}

\begin{enumerate}
    \item 梯度老虎机算法
    \item 策略梯度方法
    \item 神经网络中的分类层
    \item 任何需要 soft-max 梯度的优化问题
\end{enumerate}

\vspace{1cm}

\textbf{参考文献}:
\begin{itemize}
    \item Sutton, R. S., \& Barto, A. G. (2018). Reinforcement Learning: An Introduction (2nd Edition). MIT Press.
    \item Bishop, C. M. (2006). Pattern Recognition and Machine Learning. Springer.
    \item Goodfellow, I., Bengio, Y., \& Courville, A. (2016). Deep Learning. MIT Press.
\end{itemize}

\end{document}


