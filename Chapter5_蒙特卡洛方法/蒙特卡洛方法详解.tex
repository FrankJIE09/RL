\documentclass[12pt,a4paper]{article}
\usepackage[UTF8]{ctex}
\usepackage{amsmath}
\usepackage{amssymb}
\usepackage{amsthm}
\usepackage{geometry}
\usepackage{hyperref}
\usepackage{enumitem}
\usepackage{xcolor}
\usepackage{bm}
\usepackage{tikz}
\usepackage{booktabs}
\usepackage{array}
\usepackage{algorithm}
\usepackage{algorithmic}

\geometry{margin=2.5cm}

\title{蒙特卡洛方法详解}
\subtitle{第5章:从经验中学习价值函数和最优策略}
\author{}
\date{}

\newtheorem{definition}{定义}
\newtheorem{theorem}{定理}
\newtheorem{proposition}{命题}
\newtheorem{example}{示例}
\newtheorem{remark}{注记}

\begin{document}

\maketitle

\tableofcontents
\newpage

\section{引言}

\subsection{什么是蒙特卡洛方法?}

\begin{definition}[蒙特卡洛方法]
\textbf{蒙特卡洛方法}(Monte Carlo Methods)是一类基于平均样本回报来解决强化学习问题的方法。它们只需要经验——从与环境实际或模拟交互中获得的样本序列(状态、动作和奖励)。
\end{definition}

\textbf{关键特征}:
\begin{itemize}
    \item \textbf{不需要完整的环境模型}:只需要经验样本
    \item \textbf{基于平均样本回报}:通过平均观察到的回报来估计价值函数
    \item \textbf{只适用于回合制任务}:需要完整的回合才能计算回报
    \item \textbf{不使用自举法}:不使用其他状态的估计值
\end{itemize}

\subsection{蒙特卡洛方法的优势}

\textbf{1. 不需要完整的环境模型}:
\begin{itemize}
    \item 动态规划需要完整的环境动态 $p(s', r | s, a)$
    \item 蒙特卡洛方法只需要样本经验
    \item 可以从实际交互中学习
\end{itemize}

\textbf{2. 只需要样本生成}:
\begin{itemize}
    \item 即使有模型,也只需要生成样本转移
    \item 不需要完整的概率分布
    \item 在很多情况下,生成样本比计算分布更容易
\end{itemize}

\textbf{3. 可以学习最优行为}:
\begin{itemize}
    \item 从实际经验中学习,无需先验知识
    \item 仍然可以达到最优行为
    \item 适用于未知环境
\end{itemize}

\subsection{蒙特卡洛方法的限制}

\textbf{1. 只适用于回合制任务}:
\begin{itemize}
    \item 需要完整的回合才能计算回报
    \item 回合必须终止
    \item 不适用于持续任务
\end{itemize}

\textbf{2. 回合结束后才能更新}:
\begin{itemize}
    \item 不能在线学习(步进式)
    \item 只能回合式增量学习
    \item 需要等待回合结束
\end{itemize}

\section{蒙特卡洛方法 vs 动态规划}

\subsection{对比表格}

\begin{center}
\begin{tabular}{|l|c|c|}
\hline
\textbf{特性} & \textbf{动态规划} & \textbf{蒙特卡洛方法} \\
\hline
\textbf{需要模型} & 是(完整模型) & 否(只需要经验) \\
\hline
\textbf{更新方式} & 期望更新 & 采样更新 \\
\hline
\textbf{自举法} & 是 & 否 \\
\hline
\textbf{适用任务} & 所有MDP & 只适用于回合制任务 \\
\hline
\textbf{更新时机} & 可以立即更新 & 需要等待回合结束 \\
\hline
\textbf{计算复杂度} & $O(|\mathcal{S}|^2 \times |\mathcal{A}|)$ & $O(|\mathcal{S}|)$(每个状态独立) \\
\hline
\textbf{方差} & 无(精确) & 有(估计) \\
\hline
\textbf{偏差} & 无 & 无(无偏估计) \\
\hline
\end{tabular}
\end{center}

\subsection{关键区别}

\textbf{1. 更新方式}:

\textbf{动态规划}(期望更新):
\begin{equation}
v_\pi^{k+1}(s) = \sum_{a} \pi(a | s) \sum_{s', r} p(s', r | s, a) [r + \gamma v_\pi^k(s')]
\end{equation}

\textbf{蒙特卡洛方法}(采样更新):
\begin{equation}
v(s) \gets v(s) + \alpha [G_t - v(s)]
\end{equation}

其中 $G_t$ 是完整的真实回报。

\textbf{2. 自举法}:

\textbf{动态规划}:
\begin{itemize}
    \item 使用其他状态的估计值 $v_k(s')$
    \item 基于估计值更新估计值
    \item 使用自举法
\end{itemize}

\textbf{蒙特卡洛方法}:
\begin{itemize}
    \item 使用完整的真实回报 $G_t$
    \item 不依赖其他状态的估计值
    \item 不使用自举法
\end{itemize}

\textbf{3. 状态独立性}:

\textbf{动态规划}:
\begin{itemize}
    \item 状态之间相互依赖
    \item 一个状态的更新依赖于其他状态
    \item 需要扫描整个状态空间
\end{itemize}

\textbf{蒙特卡洛方法}:
\begin{itemize}
    \item 每个状态的估计是独立的
    \item 一个状态的估计不依赖于其他状态
    \item 只需要访问过的状态
\end{itemize}

\section{蒙特卡洛预测(策略评估)}

\subsection{基本思想}

\textbf{目标}:估计给定策略 $\pi$ 的状态价值函数 $v_\pi(s)$。

\textbf{方法}:
\begin{quote}
\textbf{平均观察到的回报}:从状态 $s$ 开始,遵循策略 $\pi$,观察回报,然后平均这些回报。
\end{quote}

\textbf{数学基础}:
\begin{equation}
v_\pi(s) = \mathbb{E}_\pi[G_t | S_t = s]
\end{equation}

根据大数定律,如果我们观察足够多的回报,平均值会收敛到期望值。

\subsection{首次访问 vs 每次访问}

\textbf{问题}:一个状态可能在同一个回合中被访问多次。

\textbf{两种方法}:

\textbf{1. 首次访问蒙特卡洛方法}(First-Visit MC):
\begin{itemize}
    \item 只计算\textbf{首次访问}状态 $s$ 后的回报
    \item 忽略同一回合中的后续访问
    \item 更广泛研究,理论性质更清楚
\end{itemize}

\textbf{2. 每次访问蒙特卡洛方法}(Every-Visit MC):
\begin{itemize}
    \item 计算\textbf{所有访问}状态 $s$ 后的回报
    \item 包括同一回合中的多次访问
    \item 更自然地扩展到函数逼近和资格迹
\end{itemize}

\subsection{首次访问蒙特卡洛预测算法}

\begin{algorithm}[H]
\caption{首次访问蒙特卡洛预测(估计 $v_\pi$)}
\begin{algorithmic}[1]
\REQUIRE 要评估的策略 $\pi$
\ENSURE 状态价值函数 $v_\pi$ 的估计 $V$
\STATE \textbf{初始化}:
\STATE $V(s) \in \mathbb{R}$ 任意初始化,对所有 $s \in \mathcal{S}$
\STATE $\text{Returns}(s) \gets$ 空列表,对所有 $s \in \mathcal{S}$
\STATE
\REPEAT
    \STATE \textbf{生成回合}:遵循策略 $\pi$ 生成回合 $S_0, A_0, R_1, S_1, A_1, R_2, \ldots, S_{T-1}, A_{T-1}, R_T$
    \STATE $G \gets 0$
    \STATE \textbf{对回合的每一步,$t = T-1, T-2, \ldots, 0$}:
    \STATE $G \gets \gamma G + R_{t+1}$ \COMMENT{计算回报}
    \IF{$S_t$ 不在 $S_0, S_1, \ldots, S_{t-1}$ 中出现}
        \STATE 将 $G$ 追加到 $\text{Returns}(S_t)$
        \STATE $V(S_t) \gets \text{average}(\text{Returns}(S_t))$
    \ENDIF
\UNTIL{收敛}
\RETURN $V$
\end{algorithmic}
\end{algorithm}

\textbf{关键步骤}:
\begin{enumerate}
    \item \textbf{生成回合}:遵循策略 $\pi$ 生成一个完整的回合
    \item \textbf{反向计算回报}:从回合结束向前计算每个状态的回报
    \item \textbf{首次访问检查}:只对首次访问的状态更新
    \item \textbf{平均回报}:将回报加入列表,计算平均值
\end{enumerate}

\subsection{回报的计算}

\textbf{回报的定义}:
\begin{equation}
G_t = R_{t+1} + \gamma R_{t+2} + \gamma^2 R_{t+3} + \cdots + \gamma^{T-t-1} R_T
\end{equation}

\textbf{反向计算}:
\begin{align}
G_{T-1} &= R_T \\
G_{T-2} &= R_{T-1} + \gamma R_T = R_{T-1} + \gamma G_{T-1} \\
G_{T-3} &= R_{T-2} + \gamma R_{T-1} + \gamma^2 R_T = R_{T-2} + \gamma G_{T-2} \\
&\vdots \\
G_t &= R_{t+1} + \gamma G_{t+1}
\end{align}

\textbf{算法中的实现}:
\begin{verbatim}
G = 0
for t = T-1, T-2, ..., 0:
    G = γ * G + R_{t+1}  # 反向计算
\end{verbatim}

\subsection{收敛性}

\begin{theorem}[首次访问蒙特卡洛收敛性]
首次访问蒙特卡洛方法在访问次数(或首次访问次数)趋于无穷时,收敛到 $v_\pi(s)$。
\end{theorem}

\textbf{证明思路}:
\begin{itemize}
    \item 每个回报是 $v_\pi(s)$ 的独立同分布估计
    \item 回报有有限方差
    \item 根据大数定律,平均值收敛到期望值
    \item 每个平均值都是无偏估计
    \item 误差的标准差以 $1/\sqrt{n}$ 的速度下降,其中 $n$ 是平均的回报数量
\end{itemize}

\textbf{每次访问蒙特卡洛}:
\begin{itemize}
    \item 也收敛到 $v_\pi(s)$
    \item 理论分析更复杂
    \item 估计也以二次速度收敛
\end{itemize}

\section{蒙特卡洛估计动作价值函数}

\subsection{为什么需要动作价值函数?}

\textbf{问题}:如果只有状态价值函数 $v_\pi(s)$,如何改进策略?

\textbf{困难}:
\begin{itemize}
    \item 策略改进需要比较动作价值:$q_\pi(s, a)$
    \item 如果只有 $v_\pi(s)$,需要知道环境模型 $p(s', r | s, a)$
    \item 但蒙特卡洛方法的目标是\textbf{不需要模型}
\end{itemize}

\textbf{解决方案}:
\begin{quote}
\textbf{直接估计动作价值函数 $q_\pi(s, a)$},而不是状态价值函数 $v_\pi(s)$。
\end{quote}

\subsection{动作价值函数的估计}

\textbf{方法}:
\begin{quote}
\textbf{平均观察到的回报}:在状态 $s$ 采取动作 $a$,然后遵循策略 $\pi$,观察回报,然后平均这些回报。
\end{quote}

\textbf{数学基础}:
\begin{equation}
q_\pi(s, a) = \mathbb{E}_\pi[G_t | S_t = s, A_t = a]
\end{equation}

\subsection{首次访问蒙特卡洛动作价值估计算法}

\begin{algorithm}[H]
\caption{首次访问蒙特卡洛动作价值估计(估计 $q_\pi$)}
\begin{algorithmic}[1]
\REQUIRE 要评估的策略 $\pi$
\ENSURE 动作价值函数 $q_\pi$ 的估计 $Q$
\STATE \textbf{初始化}:
\STATE $Q(s, a) \in \mathbb{R}$ 任意初始化,对所有 $s \in \mathcal{S}$,$a \in \mathcal{A}(s)$
\STATE $\text{Returns}(s, a) \gets$ 空列表,对所有 $s \in \mathcal{S}$,$a \in \mathcal{A}(s)$
\STATE
\REPEAT
    \STATE \textbf{生成回合}:遵循策略 $\pi$ 生成回合 $S_0, A_0, R_1, S_1, A_1, R_2, \ldots, S_{T-1}, A_{T-1}, R_T$
    \STATE $G \gets 0$
    \STATE \textbf{对回合的每一步,$t = T-1, T-2, \ldots, 0$}:
    \STATE $G \gets \gamma G + R_{t+1}$ \COMMENT{计算回报}
    \IF{$(S_t, A_t)$ 不在 $(S_0, A_0), (S_1, A_1), \ldots, (S_{t-1}, A_{t-1})$ 中出现}
        \STATE 将 $G$ 追加到 $\text{Returns}(S_t, A_t)$
        \STATE $Q(S_t, A_t) \gets \text{average}(\text{Returns}(S_t, A_t))$
    \ENDIF
\UNTIL{收敛}
\RETURN $Q$
\end{algorithmic}
\end{algorithm}

\subsection{探索问题}

\textbf{问题}:
\begin{itemize}
    \item 如果策略 $\pi$ 是确定性的,某些状态-动作对可能永远不会被访问
    \item 如果 $(s, a)$ 从未被访问,$Q(s, a)$ 无法估计
    \item 无法进行策略改进
\end{itemize}

\textbf{解决方案}:
\begin{enumerate}
    \item \textbf{探索性开始}(Exploring Starts):确保所有状态-动作对都有非零概率被访问
    \item \textbf{随机策略}:使用随机策略,确保所有动作都有被选择的概率
    \item \textbf{$\varepsilon$-贪婪策略}:以 $\varepsilon$ 的概率随机选择动作,以 $1-\varepsilon$ 的概率选择最优动作
\end{enumerate}

\section{蒙特卡洛控制}

\subsection{广义策略迭代(GPI)}

\textbf{蒙特卡洛控制也使用GPI框架}:

\begin{enumerate}
    \item \textbf{策略评估}:估计动作价值函数 $q_\pi$
    \item \textbf{策略改进}:基于 $q_\pi$ 改进策略
    \item \textbf{重复}:直到收敛到最优策略
\end{enumerate}

\textbf{与动态规划的区别}:
\begin{itemize}
    \item 动态规划:从模型计算价值函数
    \item 蒙特卡洛:从样本回报学习价值函数
    \item 但GPI框架相同
\end{itemize}

\subsection{蒙特卡洛控制算法(探索性开始)}

\begin{algorithm}[H]
\caption{蒙特卡洛控制(探索性开始)}
\begin{algorithmic}[1]
\REQUIRE 所有状态-动作对都有非零概率被访问
\ENSURE 最优动作价值函数 $q_*$ 和最优策略 $\pi_*$
\STATE \textbf{初始化}:
\STATE $Q(s, a) \in \mathbb{R}$ 任意初始化,对所有 $s \in \mathcal{S}$,$a \in \mathcal{A}(s)$
\STATE $\pi(s) \in \mathcal{A}(s)$ 任意初始化,对所有 $s \in \mathcal{S}$
\STATE $\text{Returns}(s, a) \gets$ 空列表,对所有 $s \in \mathcal{S}$,$a \in \mathcal{A}(s)$
\STATE
\REPEAT
    \STATE \textbf{生成回合}(探索性开始):随机选择 $S_0 \in \mathcal{S}$,$A_0 \in \mathcal{A}(S_0)$,然后遵循策略 $\pi$ 生成回合
    \STATE $G \gets 0$
    \STATE \textbf{对回合的每一步,$t = T-1, T-2, \ldots, 0$}:
    \STATE $G \gets \gamma G + R_{t+1}$
    \IF{$(S_t, A_t)$ 不在 $(S_0, A_0), (S_1, A_1), \ldots, (S_{t-1}, A_{t-1})$ 中出现}
        \STATE 将 $G$ 追加到 $\text{Returns}(S_t, A_t)$
        \STATE $Q(S_t, A_t) \gets \text{average}(\text{Returns}(S_t, A_t))$
        \STATE $\pi(S_t) \gets \arg\max_{a} Q(S_t, a)$ \COMMENT{策略改进}
    \ENDIF
\UNTIL{策略稳定}
\RETURN $q_* = Q$,$\pi_* = \pi$
\end{algorithmic}
\end{algorithm}

\textbf{关键特征}:
\begin{itemize}
    \item \textbf{探索性开始}:每个回合从随机状态-动作对开始
    \item \textbf{策略评估}:估计动作价值函数
    \item \textbf{策略改进}:选择使动作价值最大的动作
    \item \textbf{交替进行}:评估和改进交替进行
\end{itemize}

\section{无探索性开始的蒙特卡洛控制}

\subsection{问题}

\textbf{探索性开始的限制}:
\begin{itemize}
    \item 在实际应用中,可能无法控制起始状态-动作对
    \item 需要确保所有状态-动作对都被访问
    \item 探索性开始可能不现实
\end{itemize}

\textbf{解决方案}:
\begin{quote}
\textbf{使用随机策略},确保所有动作都有被选择的概率。
\end{quote}

\subsection{$\varepsilon$-贪婪策略}

\begin{definition}[$\varepsilon$-贪婪策略]
\textbf{$\varepsilon$-贪婪策略}以 $1-\varepsilon$ 的概率选择最优动作,以 $\varepsilon$ 的概率随机选择动作。
\end{definition}

\textbf{数学表达}:
\begin{equation}
\pi(a | s) = \begin{cases}
1 - \varepsilon + \frac{\varepsilon}{|\mathcal{A}(s)|} & \text{如果 } a = \arg\max_{a'} Q(s, a') \\
\frac{\varepsilon}{|\mathcal{A}(s)|} & \text{其他}
\end{cases}
\end{equation}

\textbf{特点}:
\begin{itemize}
    \item 保证所有动作都有非零概率
    \item 仍然主要选择最优动作
    \item 平衡探索和利用
\end{itemize}

\subsection{无探索性开始的蒙特卡洛控制算法}

\begin{algorithm}[H]
\caption{蒙特卡洛控制(无探索性开始,$\varepsilon$-贪婪)}
\begin{algorithmic}[1]
\REQUIRE 所有状态-动作对都有非零概率被访问(通过 $\varepsilon$-贪婪策略)
\ENSURE 最优动作价值函数 $q_*$ 和最优策略 $\pi_*$
\STATE \textbf{初始化}:
\STATE $Q(s, a) \in \mathbb{R}$ 任意初始化,对所有 $s \in \mathcal{S}$,$a \in \mathcal{A}(s)$
\STATE $\pi$ 为 $\varepsilon$-贪婪策略,基于 $Q$
\STATE $\text{Returns}(s, a) \gets$ 空列表,对所有 $s \in \mathcal{S}$,$a \in \mathcal{A}(s)$
\STATE
\REPEAT
    \STATE \textbf{生成回合}:遵循策略 $\pi$ 生成回合
    \STATE $G \gets 0$
    \STATE \textbf{对回合的每一步,$t = T-1, T-2, \ldots, 0$}:
    \STATE $G \gets \gamma G + R_{t+1}$
    \IF{$(S_t, A_t)$ 不在 $(S_0, A_0), (S_1, A_1), \ldots, (S_{t-1}, A_{t-1})$ 中出现}
        \STATE 将 $G$ 追加到 $\text{Returns}(S_t, A_t)$
        \STATE $Q(S_t, A_t) \gets \text{average}(\text{Returns}(S_t, A_t))$
        \STATE $\pi(S_t) \gets \varepsilon$-贪婪策略,基于 $Q(S_t, \cdot)$
    \ENDIF
\UNTIL{策略稳定}
\RETURN $q_* = Q$,$\pi_* = \pi$
\end{algorithmic}
\end{algorithm}

\section{离策略蒙特卡洛控制}

\subsection{问题}

\textbf{在策略 vs 离策略}:
\begin{itemize}
    \item \textbf{在策略}(On-Policy):评估和改进的是同一个策略
    \item \textbf{离策略}(Off-Policy):评估一个策略,但改进另一个策略
\end{itemize}

\textbf{为什么需要离策略?}:
\begin{itemize}
    \item 学习最优策略,同时保持探索
    \item 从其他智能体的经验中学习
    \item 重用旧策略的数据
\end{itemize}

\subsection{重要性采样}

\textbf{基本思想}:
\begin{quote}
使用一个策略(行为策略 $b$)生成数据,但评估另一个策略(目标策略 $\pi$)。
\end{quote}

\textbf{重要性采样比率}:
\begin{equation}
\rho_{t:T-1} = \prod_{k=t}^{T-1} \frac{\pi(A_k | S_k)}{b(A_k | S_k)} \frac{p(S_{k+1} | S_k, A_k)}{p(S_{k+1} | S_k, A_k)} = \prod_{k=t}^{T-1} \frac{\pi(A_k | S_k)}{b(A_k | S_k)}
\end{equation}

\textbf{加权回报}:
\begin{equation}
V(s) \gets V(s) + \alpha [\rho_{t:T-1} G_t - V(s)]
\end{equation}

\subsection{离策略蒙特卡洛控制算法}

\begin{algorithm}[H]
\caption{离策略蒙特卡洛控制}
\begin{algorithmic}[1]
\REQUIRE 行为策略 $b$(探索性),目标策略 $\pi$(确定性,贪婪)
\ENSURE 最优动作价值函数 $q_*$ 和最优策略 $\pi_*$
\STATE \textbf{初始化}:
\STATE $Q(s, a) \in \mathbb{R}$ 任意初始化,对所有 $s \in \mathcal{S}$,$a \in \mathcal{A}(s)$
\STATE $C(s, a) \gets 0$,对所有 $s \in \mathcal{S}$,$a \in \mathcal{A}(s)$
\STATE $\pi$ 为确定性贪婪策略,基于 $Q$
\STATE
\REPEAT
    \STATE \textbf{生成回合}:遵循行为策略 $b$ 生成回合
    \STATE $G \gets 0$
    \STATE $W \gets 1$ \COMMENT{重要性采样权重}
    \STATE \textbf{对回合的每一步,$t = T-1, T-2, \ldots, 0$}:
    \STATE $G \gets \gamma G + R_{t+1}$
    \STATE $C(S_t, A_t) \gets C(S_t, A_t) + W$
    \STATE $Q(S_t, A_t) \gets Q(S_t, A_t) + \frac{W}{C(S_t, A_t)} [G - Q(S_t, A_t)]$
    \STATE $\pi(S_t) \gets \arg\max_{a} Q(S_t, a)$
    \IF{$A_t \neq \pi(S_t)$}
        \STATE \textbf{退出内层循环} \COMMENT{如果动作不匹配,停止更新}
    \ENDIF
    \STATE $W \gets W \frac{1}{b(A_t | S_t)}$ \COMMENT{更新重要性采样权重}
\UNTIL{策略稳定}
\RETURN $q_* = Q$,$\pi_* = \pi$
\end{algorithmic}
\end{algorithm}

\section{蒙特卡洛方法的优缺点}

\subsection{优点}

\textbf{1. 不需要环境模型}:
\begin{itemize}
    \item 只需要经验样本
    \item 可以从实际交互中学习
    \item 适用于未知环境
\end{itemize}

\textbf{2. 可以只使用样本生成}:
\begin{itemize}
    \item 即使有模型,也只需要生成样本
    \item 不需要完整的概率分布
    \item 在很多情况下更容易
\end{itemize}

\textbf{3. 状态独立性}:
\begin{itemize}
    \item 每个状态的估计是独立的
    \item 只需要访问过的状态
    \item 计算复杂度与状态数量无关(对单个状态)
\end{itemize}

\textbf{4. 无偏估计}:
\begin{itemize}
    \item 估计是无偏的
    \item 收敛到真实值
    \item 不需要自举法
\end{itemize}

\subsection{缺点}

\textbf{1. 只适用于回合制任务}:
\begin{itemize}
    \item 需要完整的回合
    \item 回合必须终止
    \item 不适用于持续任务
\end{itemize}

\textbf{2. 需要等待回合结束}:
\begin{itemize}
    \item 不能在线学习
    \item 只能回合式学习
    \item 学习速度可能较慢
\end{itemize}

\textbf{3. 高方差}:
\begin{itemize}
    \item 使用单个样本回报
    \item 方差可能很大
    \item 需要很多样本才能收敛
\end{itemize}

\textbf{4. 探索问题}:
\begin{itemize}
    \item 需要确保所有状态-动作对被访问
    \item 可能需要探索性开始或随机策略
    \item 探索策略可能不是最优的
\end{itemize}

\section{总结}

\subsection{核心要点}

\begin{enumerate}
    \item \textbf{定义}:蒙特卡洛方法基于平均样本回报来估计价值函数
    
    \item \textbf{关键特征}:
    \begin{itemize}
        \item 不需要完整的环境模型
        \item 只需要经验样本
        \item 只适用于回合制任务
        \item 不使用自举法
    \end{itemize}
    
    \item \textbf{主要算法}:
    \begin{itemize}
        \item 首次访问蒙特卡洛预测
        \item 每次访问蒙特卡洛预测
        \item 蒙特卡洛控制(探索性开始)
        \item 蒙特卡洛控制($\varepsilon$-贪婪)
        \item 离策略蒙特卡洛控制
    \end{itemize}
    
    \item \textbf{与动态规划的区别}:
    \begin{itemize}
        \item 动态规划:期望更新,需要模型,使用自举法
        \item 蒙特卡洛:采样更新,不需要模型,不使用自举法
    \end{itemize}
    
    \item \textbf{优势}:
    \begin{itemize}
        \item 不需要环境模型
        \item 状态独立性
        \item 无偏估计
    \end{itemize}
    
    \item \textbf{局限性}:
    \begin{itemize}
        \item 只适用于回合制任务
        \item 需要等待回合结束
        \item 高方差
        \item 探索问题
    \end{itemize}
\end{enumerate}

\subsection{关键洞察}

\begin{quote}
\textbf{蒙特卡洛方法通过平均观察到的回报来估计价值函数,不需要完整的环境模型,只需要经验样本。虽然只适用于回合制任务,但它们提供了从实际交互中学习的能力,这对于许多实际应用非常重要。}
\end{quote}

\vspace{1cm}

\textbf{参考文献}:
\begin{itemize}
    \item Sutton, R. S., \& Barto, A. G. (2018). \textit{Reinforcement Learning: An Introduction} (2nd Edition). MIT Press, Chapter 5.
\end{itemize}

\end{document}

