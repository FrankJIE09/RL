\documentclass[12pt,a4paper]{article}
\usepackage[UTF8]{ctex}
\usepackage{amsmath}
\usepackage{amssymb}
\usepackage{amsthm}
\usepackage{geometry}
\usepackage{hyperref}
\usepackage{enumitem}
\usepackage{booktabs}

\geometry{left=2.5cm,right=2.5cm,top=2.5cm,bottom=2.5cm}

\title{REINFORCE推导详解}
\subtitle{从期望形式到REINFORCE更新规则的详细推导}
\author{强化学习笔记}
\date{\today}

\newtheorem{definition}{定义}
\newtheorem{theorem}{定理}
\newtheorem{example}{示例}
\newtheorem{remark}{注记}

\begin{document}

\maketitle

\tableofcontents
\newpage

\section{问题}

\textbf{问题}:如何从策略梯度定理的期望形式推导出REINFORCE更新规则?

\textbf{推导链条}:
\begin{align}
\nabla_\theta J(\theta) &= \mathbb{E}_\pi\left[\sum_{a} \pi(a|S_t, \theta) q_\pi(S_t, a) \frac{\nabla_\theta \pi(a|S_t, \theta)}{\pi(a|S_t, \theta)}\right] \label{eq:11} \\
                         &= \mathbb{E}_\pi\left[q_\pi(S_t, A_t) \frac{\nabla_\theta \pi(A_t|S_t, \theta)}{\pi(A_t|S_t, \theta)}\right] \label{eq:12} \\
                         &= \mathbb{E}_\pi\left[G_t \frac{\nabla_\theta \pi(A_t|S_t, \theta)}{\pi(A_t|S_t, \theta)}\right] \label{eq:13}
\end{align}

\section{公式(11):引入动作概率}

\subsection{公式(11)}

\begin{equation}
\nabla_\theta J(\theta) = \mathbb{E}_\pi\left[\sum_{a} \pi(a|S_t, \theta) q_\pi(S_t, a) \frac{\nabla_\theta \pi(a|S_t, \theta)}{\pi(a|S_t, \theta)}\right]
\label{eq:formula11}
\end{equation}

\subsection{从期望形式到公式(11)}

\textbf{策略梯度定理(期望形式)}:
\begin{equation}
\nabla_\theta J(\theta) \propto \mathbb{E}_\pi\left[\sum_{a} q_\pi(S_t, a) \nabla_\theta \pi(a|S_t, \theta)\right]
\end{equation}

\textbf{关键技巧}:乘以并除以 $\pi(a|S_t, \theta)$
\begin{align}
\nabla_\theta J(\theta) &\propto \mathbb{E}_\pi\left[\sum_{a} q_\pi(S_t, a) \nabla_\theta \pi(a|S_t, \theta)\right] \\
                         &= \mathbb{E}_\pi\left[\sum_{a} \pi(a|S_t, \theta) q_\pi(S_t, a) \frac{\nabla_\theta \pi(a|S_t, \theta)}{\pi(a|S_t, \theta)}\right]
\end{align}

\textbf{为什么这样做?}
\begin{itemize}
    \item 为了引入动作概率 $\pi(a|S_t, \theta)$
    \item 这样可以将求和转换为期望(关于动作的期望)
    \item 为下一步引入实际选择的动作做准备
\end{itemize}

\textbf{验证}:
\begin{align}
\sum_{a} \pi(a|S_t, \theta) q_\pi(S_t, a) \frac{\nabla_\theta \pi(a|S_t, \theta)}{\pi(a|S_t, \theta)} &= \sum_{a} \pi(a|S_t, \theta) \times q_\pi(S_t, a) \times \frac{\nabla_\theta \pi(a|S_t, \theta)}{\pi(a|S_t, \theta)} \\
                                                                                                    &= \sum_{a} q_\pi(S_t, a) \nabla_\theta \pi(a|S_t, \theta)
\end{align}

确实相等!

\section{公式(11)到公式(12):引入实际选择的动作}

\subsection{公式(12)}

\begin{equation}
\nabla_\theta J(\theta) = \mathbb{E}_\pi\left[q_\pi(S_t, A_t) \frac{\nabla_\theta \pi(A_t|S_t, \theta)}{\pi(A_t|S_t, \theta)}\right]
\label{eq:formula12}
\end{equation}

\subsection{推导过程}

\textbf{步骤1:识别期望形式}

公式(11)中的表达式:
\begin{equation}
\sum_{a} \pi(a|S_t, \theta) q_\pi(S_t, a) \frac{\nabla_\theta \pi(a|S_t, \theta)}{\pi(a|S_t, \theta)}
\end{equation}

\textbf{关键观察}:
\begin{itemize}
    \item 这是一个关于动作 $a$ 的加权求和
    \item 权重是 $\pi(a|S_t, \theta)$
    \item 这可以写成关于动作的期望
\end{itemize}

\textbf{步骤2:转换为期望}

\textbf{定义}:
\begin{itemize}
    \item 设 $A_t$ 是随机变量,取值于动作空间 $\mathcal{A}(S_t)$
    \item $A_t$ 的分布是 $\pi(\cdot|S_t, \theta)$:$\Pr(A_t = a | S_t) = \pi(a|S_t, \theta)$
\end{itemize}

\textbf{期望的定义}:
\begin{equation}
\mathbb{E}[f(A_t) | S_t] = \sum_{a} \Pr(A_t = a | S_t) f(a) = \sum_{a} \pi(a|S_t, \theta) f(a)
\end{equation}

\textbf{应用}:
\begin{align}
\sum_{a} \pi(a|S_t, \theta) q_\pi(S_t, a) \frac{\nabla_\theta \pi(a|S_t, \theta)}{\pi(a|S_t, \theta)} &= \sum_{a} \pi(a|S_t, \theta) f(a) \\
                                                                                                        &= \mathbb{E}[f(A_t) | S_t]
\end{align}

其中 $f(a) = q_\pi(S_t, a) \frac{\nabla_\theta \pi(a|S_t, \theta)}{\pi(a|S_t, \theta)}$。

\textbf{展开}:
\begin{align}
\mathbb{E}[f(A_t) | S_t] &= \mathbb{E}\left[q_\pi(S_t, A_t) \frac{\nabla_\theta \pi(A_t|S_t, \theta)}{\pi(A_t|S_t, \theta)} \middle| S_t\right] \\
                          &= \sum_{a} \pi(a|S_t, \theta) q_\pi(S_t, a) \frac{\nabla_\theta \pi(a|S_t, \theta)}{\pi(a|S_t, \theta)}
\end{align}

\textbf{步骤3:全期望}

\textbf{全期望公式}:
\begin{equation}
\mathbb{E}_\pi[X] = \mathbb{E}_\pi[\mathbb{E}[X | S_t]]
\end{equation}

\textbf{应用}:
\begin{align}
\nabla_\theta J(\theta) &= \mathbb{E}_\pi\left[\sum_{a} \pi(a|S_t, \theta) q_\pi(S_t, a) \frac{\nabla_\theta \pi(a|S_t, \theta)}{\pi(a|S_t, \theta)}\right] \\
                         &= \mathbb{E}_\pi\left[\mathbb{E}\left[q_\pi(S_t, A_t) \frac{\nabla_\theta \pi(A_t|S_t, \theta)}{\pi(A_t|S_t, \theta)} \middle| S_t\right]\right] \\
                         &= \mathbb{E}_\pi\left[q_\pi(S_t, A_t) \frac{\nabla_\theta \pi(A_t|S_t, \theta)}{\pi(A_t|S_t, \theta)}\right]
\end{align}

\textbf{关键}:
\begin{itemize}
    \item 从对所有动作的求和,转换为对实际选择的动作 $A_t$ 的期望
    \item $A_t \sim \pi(\cdot|S_t, \theta)$(按照策略选择)
    \item 这使我们能够使用样本(实际选择的动作)来估计梯度
\end{itemize}

\subsection{具体例子}

\textbf{场景}:在状态 $S_t$,有3个动作 $a_1, a_2, a_3$

\textbf{策略}:
\begin{align}
\pi(a_1|S_t, \theta) &= 0.5 \\
\pi(a_2|S_t, \theta) &= 0.3 \\
\pi(a_3|S_t, \theta) &= 0.2
\end{align}

\textbf{公式(11)的计算}:
\begin{align}
&\sum_{a} \pi(a|S_t, \theta) q_\pi(S_t, a) \frac{\nabla_\theta \pi(a|S_t, \theta)}{\pi(a|S_t, \theta)} \\
&= 0.5 \times q_\pi(S_t, a_1) \times \frac{\nabla_\theta \pi(a_1|S_t, \theta)}{0.5} \\
&\quad + 0.3 \times q_\pi(S_t, a_2) \times \frac{\nabla_\theta \pi(a_2|S_t, \theta)}{0.3} \\
&\quad + 0.2 \times q_\pi(S_t, a_3) \times \frac{\nabla_\theta \pi(a_3|S_t, \theta)}{0.2} \\
&= q_\pi(S_t, a_1) \nabla_\theta \pi(a_1|S_t, \theta) \\
&\quad + q_\pi(S_t, a_2) \nabla_\theta \pi(a_2|S_t, \theta) \\
&\quad + q_\pi(S_t, a_3) \nabla_\theta \pi(a_3|S_t, \theta)
\end{align}

\textbf{公式(12)的计算}:
\begin{itemize}
    \item 按照策略选择动作:$A_t \sim \pi(\cdot|S_t, \theta)$
    \item 如果 $A_t = a_1$(概率 $0.5$):
    \begin{equation}
    q_\pi(S_t, a_1) \frac{\nabla_\theta \pi(a_1|S_t, \theta)}{\pi(a_1|S_t, \theta)} = q_\pi(S_t, a_1) \frac{\nabla_\theta \pi(a_1|S_t, \theta)}{0.5}
    \end{equation}
    \item 如果 $A_t = a_2$(概率 $0.3$):
    \begin{equation}
    q_\pi(S_t, a_2) \frac{\nabla_\theta \pi(a_2|S_t, \theta)}{\pi(a_2|S_t, \theta)} = q_\pi(S_t, a_2) \frac{\nabla_\theta \pi(a_2|S_t, \theta)}{0.3}
    \end{equation}
    \item 如果 $A_t = a_3$(概率 $0.2$):
    \begin{equation}
    q_\pi(S_t, a_3) \frac{\nabla_\theta \pi(a_3|S_t, \theta)}{\pi(a_3|S_t, \theta)} = q_\pi(S_t, a_3) \frac{\nabla_\theta \pi(a_3|S_t, \theta)}{0.2}
    \end{equation}
    \item 期望值:
    \begin{align}
    &\mathbb{E}_\pi\left[q_\pi(S_t, A_t) \frac{\nabla_\theta \pi(A_t|S_t, \theta)}{\pi(A_t|S_t, \theta)}\right] \\
    &= 0.5 \times q_\pi(S_t, a_1) \frac{\nabla_\theta \pi(a_1|S_t, \theta)}{0.5} \\
    &\quad + 0.3 \times q_\pi(S_t, a_2) \frac{\nabla_\theta \pi(a_2|S_t, \theta)}{0.3} \\
    &\quad + 0.2 \times q_\pi(S_t, a_3) \frac{\nabla_\theta \pi(a_3|S_t, \theta)}{0.2} \\
    &= q_\pi(S_t, a_1) \nabla_\theta \pi(a_1|S_t, \theta) \\
    &\quad + q_\pi(S_t, a_2) \nabla_\theta \pi(a_2|S_t, \theta) \\
    &\quad + q_\pi(S_t, a_3) \nabla_\theta \pi(a_3|S_t, \theta)
    \end{align}
\end{itemize}

\textbf{验证}:公式(11)和公式(12)得到相同的结果!

\section{公式(12)到公式(13):使用回报}

\subsection{公式(13)}

\begin{equation}
\nabla_\theta J(\theta) = \mathbb{E}_\pi\left[G_t \frac{\nabla_\theta \pi(A_t|S_t, \theta)}{\pi(A_t|S_t, \theta)}\right]
\label{eq:formula13}
\end{equation}

\subsection{推导过程}

\textbf{关键关系}:
\begin{equation}
\mathbb{E}_\pi[G_t | S_t, A_t] = q_\pi(S_t, A_t)
\label{eq:key_relation}
\end{equation}

\textbf{含义}:
\begin{itemize}
    \item 给定状态 $S_t$ 和动作 $A_t$,回报 $G_t$ 的期望等于动作价值 $q_\pi(S_t, A_t)$
    \item 这是动作价值函数的定义
\end{itemize}

\textbf{推导}:

\textbf{步骤1:条件期望}
\begin{align}
\mathbb{E}_\pi\left[q_\pi(S_t, A_t) \frac{\nabla_\theta \pi(A_t|S_t, \theta)}{\pi(A_t|S_t, \theta)}\right] &= \mathbb{E}_\pi\left[\mathbb{E}_\pi\left[q_\pi(S_t, A_t) \frac{\nabla_\theta \pi(A_t|S_t, \theta)}{\pi(A_t|S_t, \theta)} \middle| S_t, A_t\right]\right] \\
                                                                                                            &= \mathbb{E}_\pi\left[q_\pi(S_t, A_t) \frac{\nabla_\theta \pi(A_t|S_t, \theta)}{\pi(A_t|S_t, \theta)}\right]
\end{align}

\textbf{步骤2:使用关键关系}

因为 $\mathbb{E}_\pi[G_t | S_t, A_t] = q_\pi(S_t, A_t)$,我们可以替换:
\begin{align}
\mathbb{E}_\pi\left[q_\pi(S_t, A_t) \frac{\nabla_\theta \pi(A_t|S_t, \theta)}{\pi(A_t|S_t, \theta)}\right] &= \mathbb{E}_\pi\left[\mathbb{E}_\pi[G_t | S_t, A_t] \frac{\nabla_\theta \pi(A_t|S_t, \theta)}{\pi(A_t|S_t, \theta)}\right] \\
                                                                                                            &= \mathbb{E}_\pi\left[G_t \frac{\nabla_\theta \pi(A_t|S_t, \theta)}{\pi(A_t|S_t, \theta)}\right]
\end{align}

\textbf{关键步骤}:
\begin{itemize}
    \item 因为 $\frac{\nabla_\theta \pi(A_t|S_t, \theta)}{\pi(A_t|S_t, \theta)}$ 只依赖于 $S_t$ 和 $A_t$(给定 $S_t, A_t$ 后是确定的)
    \item 所以可以提到条件期望外面:
    \begin{equation}
    \mathbb{E}_\pi[G_t | S_t, A_t] \times \frac{\nabla_\theta \pi(A_t|S_t, \theta)}{\pi(A_t|S_t, \theta)} = \mathbb{E}_\pi\left[G_t \frac{\nabla_\theta \pi(A_t|S_t, \theta)}{\pi(A_t|S_t, \theta)} \middle| S_t, A_t\right]
    \end{equation}
    \item 然后使用全期望公式
\end{itemize}

\textbf{严格推导}:

\textbf{全期望公式}:
\begin{equation}
\mathbb{E}_\pi[X] = \mathbb{E}_\pi[\mathbb{E}_\pi[X | S_t, A_t]]
\end{equation}

\textbf{应用}:
\begin{align}
\mathbb{E}_\pi\left[q_\pi(S_t, A_t) \frac{\nabla_\theta \pi(A_t|S_t, \theta)}{\pi(A_t|S_t, \theta)}\right] &= \mathbb{E}_\pi\left[\mathbb{E}_\pi\left[q_\pi(S_t, A_t) \frac{\nabla_\theta \pi(A_t|S_t, \theta)}{\pi(A_t|S_t, \theta)} \middle| S_t, A_t\right]\right]
\end{align}

\textbf{关键}:给定 $S_t, A_t$,$q_\pi(S_t, A_t)$ 和 $\frac{\nabla_\theta \pi(A_t|S_t, \theta)}{\pi(A_t|S_t, \theta)}$ 都是确定的,所以:
\begin{align}
\mathbb{E}_\pi\left[q_\pi(S_t, A_t) \frac{\nabla_\theta \pi(A_t|S_t, \theta)}{\pi(A_t|S_t, \theta)} \middle| S_t, A_t\right] &= q_\pi(S_t, A_t) \frac{\nabla_\theta \pi(A_t|S_t, \theta)}{\pi(A_t|S_t, \theta)} \\
                                                                                                                              &= \mathbb{E}_\pi[G_t | S_t, A_t] \frac{\nabla_\theta \pi(A_t|S_t, \theta)}{\pi(A_t|S_t, \theta)} \\
                                                                                                                              &= \mathbb{E}_\pi\left[G_t \frac{\nabla_\theta \pi(A_t|S_t, \theta)}{\pi(A_t|S_t, \theta)} \middle| S_t, A_t\right]
\end{align}

\textbf{最后一步}:
\begin{align}
\mathbb{E}_\pi\left[\mathbb{E}_\pi\left[G_t \frac{\nabla_\theta \pi(A_t|S_t, \theta)}{\pi(A_t|S_t, \theta)} \middle| S_t, A_t\right]\right] &= \mathbb{E}_\pi\left[G_t \frac{\nabla_\theta \pi(A_t|S_t, \theta)}{\pi(A_t|S_t, \theta)}\right]
\end{align}

\subsection{为什么可以这样替换?}

\textbf{关键关系}:
\begin{equation}
\mathbb{E}_\pi[G_t | S_t, A_t] = q_\pi(S_t, A_t)
\end{equation}

\textbf{证明}(动作价值函数的定义):
\begin{align}
q_\pi(S_t, A_t) &= \mathbb{E}_\pi[G_t | S_t, A_t] \\
                 &= \mathbb{E}_\pi\left[\sum_{k=0}^{\infty} \gamma^k R_{t+k+1} \middle| S_t, A_t\right]
\end{align}

\textbf{含义}:
\begin{itemize}
    \item 动作价值函数 $q_\pi(S_t, A_t)$ 的定义就是从状态 $S_t$ 采取动作 $A_t$ 的期望回报
    \item 因此,$q_\pi(S_t, A_t) = \mathbb{E}_\pi[G_t | S_t, A_t]$
    \item 我们可以用 $G_t$ 的期望来替换 $q_\pi(S_t, A_t)$
\end{itemize}

\textbf{替换的合理性}:
\begin{itemize}
    \item 在期望中,我们可以用条件期望来替换
    \item $\mathbb{E}_\pi[q_\pi(S_t, A_t) \cdots | S_t, A_t] = \mathbb{E}_\pi[\mathbb{E}_\pi[G_t | S_t, A_t] \cdots | S_t, A_t]$
    \item 然后可以简化
\end{itemize}

\section{完整推导链条}

\subsection{从策略梯度定理到REINFORCE}

\textbf{步骤1:策略梯度定理(期望形式)}
\begin{equation}
\nabla_\theta J(\theta) \propto \mathbb{E}_\pi\left[\sum_{a} q_\pi(S_t, a) \nabla_\theta \pi(a|S_t, \theta)\right]
\end{equation}

\textbf{步骤2:引入动作概率(公式11)}
\begin{align}
\nabla_\theta J(\theta) &= \mathbb{E}_\pi\left[\sum_{a} \pi(a|S_t, \theta) q_\pi(S_t, a) \frac{\nabla_\theta \pi(a|S_t, \theta)}{\pi(a|S_t, \theta)}\right] \\
                         &= \mathbb{E}_\pi\left[\sum_{a} \pi(a|S_t, \theta) q_\pi(S_t, a) \nabla_\theta \ln \pi(a|S_t, \theta)\right]
\end{align}

\textbf{步骤3:引入实际选择的动作(公式12)}
\begin{align}
\nabla_\theta J(\theta) &= \mathbb{E}_\pi\left[\sum_{a} \pi(a|S_t, \theta) q_\pi(S_t, a) \nabla_\theta \ln \pi(a|S_t, \theta)\right] \\
                         &= \mathbb{E}_\pi\left[\mathbb{E}\left[q_\pi(S_t, A_t) \nabla_\theta \ln \pi(A_t|S_t, \theta) \middle| S_t\right]\right] \\
                         &= \mathbb{E}_\pi\left[q_\pi(S_t, A_t) \nabla_\theta \ln \pi(A_t|S_t, \theta)\right]
\end{align}

\textbf{步骤4:使用回报(公式13)}
\begin{align}
\nabla_\theta J(\theta) &= \mathbb{E}_\pi\left[q_\pi(S_t, A_t) \nabla_\theta \ln \pi(A_t|S_t, \theta)\right] \\
                         &= \mathbb{E}_\pi\left[\mathbb{E}_\pi[G_t | S_t, A_t] \nabla_\theta \ln \pi(A_t|S_t, \theta)\right] \\
                         &= \mathbb{E}_\pi\left[G_t \nabla_\theta \ln \pi(A_t|S_t, \theta)\right]
\end{align}

\textbf{步骤5:REINFORCE更新}
\begin{equation}
\theta_{t+1} = \theta_t + \alpha G_t \nabla_\theta \ln \pi(A_t|S_t, \theta_t)
\end{equation}

\section{具体计算例子}

\subsection{例子:完整的计算过程}

\textbf{场景}:在状态 $S_t$,有2个动作 $a_1, a_2$

\textbf{策略}:
\begin{align}
\pi(a_1|S_t, \theta) &= 0.6 \\
\pi(a_2|S_t, \theta) &= 0.4
\end{align}

\textbf{动作价值}:
\begin{align}
q_\pi(S_t, a_1) &= 5 \\
q_\pi(S_t, a_2) &= 3
\end{align}

\textbf{策略梯度}(假设):
\begin{align}
\nabla_\theta \pi(a_1|S_t, \theta) &= [0.3, -0.3]^T \\
\nabla_\theta \pi(a_2|S_t, \theta) &= [-0.3, 0.3]^T
\end{align}

\textbf{公式(11)的计算}:
\begin{align}
&\sum_{a} \pi(a|S_t, \theta) q_\pi(S_t, a) \frac{\nabla_\theta \pi(a|S_t, \theta)}{\pi(a|S_t, \theta)} \\
&= 0.6 \times 5 \times \frac{[0.3, -0.3]^T}{0.6} + 0.4 \times 3 \times \frac{[-0.3, 0.3]^T}{0.4} \\
&= 5 \times [0.3, -0.3]^T + 3 \times [-0.3, 0.3]^T \\
&= [1.5, -1.5]^T + [-0.9, 0.9]^T \\
&= [0.6, -0.6]^T
\end{align}

\textbf{公式(12)的计算}:
\begin{itemize}
    \item 按照策略选择动作:$A_t \sim \pi(\cdot|S_t, \theta)$
    \item 如果 $A_t = a_1$(概率 $0.6$):
    \begin{align}
    q_\pi(S_t, a_1) \frac{\nabla_\theta \pi(a_1|S_t, \theta)}{\pi(a_1|S_t, \theta)} &= 5 \times \frac{[0.3, -0.3]^T}{0.6} \\
                                                                                    &= 5 \times [0.5, -0.5]^T \\
                                                                                    &= [2.5, -2.5]^T
    \end{align}
    \item 如果 $A_t = a_2$(概率 $0.4$):
    \begin{align}
    q_\pi(S_t, a_2) \frac{\nabla_\theta \pi(a_2|S_t, \theta)}{\pi(a_2|S_t, \theta)} &= 3 \times \frac{[-0.3, 0.3]^T}{0.4} \\
                                                                                    &= 3 \times [-0.75, 0.75]^T \\
                                                                                    &= [-2.25, 2.25]^T
    \end{align}
    \item 期望值:
    \begin{align}
    &\mathbb{E}_\pi\left[q_\pi(S_t, A_t) \frac{\nabla_\theta \pi(A_t|S_t, \theta)}{\pi(A_t|S_t, \theta)}\right] \\
    &= 0.6 \times [2.5, -2.5]^T + 0.4 \times [-2.25, 2.25]^T \\
    &= [1.5, -1.5]^T + [-0.9, 0.9]^T \\
    &= [0.6, -0.6]^T
    \end{align}
\end{itemize}

\textbf{验证}:公式(11)和公式(12)得到相同的结果 $[0.6, -0.6]^T$!

\textbf{公式(13)的计算}:
\begin{itemize}
    \item 假设实际选择的动作是 $A_t = a_1$,获得的回报是 $G_t = 5$
    \item 计算:
    \begin{align}
    G_t \frac{\nabla_\theta \pi(A_t|S_t, \theta)}{\pi(A_t|S_t, \theta)} &= 5 \times \frac{[0.3, -0.3]^T}{0.6} \\
                                                                          &= 5 \times [0.5, -0.5]^T \\
                                                                          &= [2.5, -2.5]^T
    \end{align}
    \item 这是单个样本的值
    \item 期望值(多次采样平均):
    \begin{align}
    &\mathbb{E}_\pi\left[G_t \frac{\nabla_\theta \pi(A_t|S_t, \theta)}{\pi(A_t|S_t, \theta)}\right] \\
    &= 0.6 \times \mathbb{E}[G_t | S_t, A_t = a_1] \times \frac{[0.3, -0.3]^T}{0.6} \\
    &\quad + 0.4 \times \mathbb{E}[G_t | S_t, A_t = a_2] \times \frac{[-0.3, 0.3]^T}{0.4} \\
    &= 0.6 \times 5 \times [0.5, -0.5]^T + 0.4 \times 3 \times [-0.75, 0.75]^T \\
    &= [1.5, -1.5]^T + [-0.9, 0.9]^T \\
    &= [0.6, -0.6]^T
    \end{align}
\end{itemize}

\textbf{验证}:三种形式都得到相同的结果!

\section{关键技巧总结}

\subsection{技巧1:乘以并除以 $\pi(a|S_t, \theta)$}

\textbf{目的}:引入动作概率,为转换为期望做准备

\textbf{方法}:
\begin{equation}
\nabla_\theta \pi(a|S_t, \theta) = \pi(a|S_t, \theta) \times \frac{\nabla_\theta \pi(a|S_t, \theta)}{\pi(a|S_t, \theta)}
\end{equation}

\subsection{技巧2:从求和到期望}

\textbf{目的}:引入实际选择的动作

\textbf{方法}:
\begin{equation}
\sum_{a} \pi(a|S_t, \theta) f(a) = \mathbb{E}[f(A_t) | S_t]
\end{equation}

其中 $A_t \sim \pi(\cdot|S_t, \theta)$。

\subsection{技巧3:使用条件期望}

\textbf{目的}:用回报 $G_t$ 替换动作价值 $q_\pi(S_t, A_t)$

\textbf{方法}:
\begin{equation}
q_\pi(S_t, A_t) = \mathbb{E}_\pi[G_t | S_t, A_t]
\end{equation}

\section{总结}

\subsection{推导链条}

\begin{enumerate}
    \item \textbf{策略梯度定理(期望形式)}:
    \begin{equation}
    \nabla_\theta J(\theta) \propto \mathbb{E}_\pi\left[\sum_{a} q_\pi(S_t, a) \nabla_\theta \pi(a|S_t, \theta)\right]
    \end{equation}
    
    \item \textbf{引入动作概率(公式11)}:
    \begin{equation}
    \nabla_\theta J(\theta) = \mathbb{E}_\pi\left[\sum_{a} \pi(a|S_t, \theta) q_\pi(S_t, a) \frac{\nabla_\theta \pi(a|S_t, \theta)}{\pi(a|S_t, \theta)}\right]
    \end{equation}
    
    \item \textbf{引入实际选择的动作(公式12)}:
    \begin{equation}
    \nabla_\theta J(\theta) = \mathbb{E}_\pi\left[q_\pi(S_t, A_t) \frac{\nabla_\theta \pi(A_t|S_t, \theta)}{\pi(A_t|S_t, \theta)}\right]
    \end{equation}
    
    \item \textbf{使用回报(公式13)}:
    \begin{equation}
    \nabla_\theta J(\theta) = \mathbb{E}_\pi\left[G_t \frac{\nabla_\theta \pi(A_t|S_t, \theta)}{\pi(A_t|S_t, \theta)}\right]
    \end{equation}
    
    \item \textbf{REINFORCE更新}:
    \begin{equation}
    \theta_{t+1} = \theta_t + \alpha G_t \nabla_\theta \ln \pi(A_t|S_t, \theta_t)
    \end{equation}
\end{enumerate}

\subsection{关键洞察}

\begin{quote}
\textbf{从对所有动作的求和,到对实际选择的动作的期望,再到使用实际回报}。这个推导过程使我们能够使用样本(实际选择的动作和实际获得的回报)来估计梯度,而不需要知道所有动作的价值。
\end{quote}

\end{document}

