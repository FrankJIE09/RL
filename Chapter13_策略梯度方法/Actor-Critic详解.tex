\documentclass[12pt,a4paper]{article}
\usepackage[UTF8]{ctex}
\usepackage{amsmath}
\usepackage{amssymb}
\usepackage{amsthm}
\usepackage{geometry}
\usepackage{hyperref}
\usepackage{enumitem}
\usepackage{booktabs}

\geometry{left=2.5cm,right=2.5cm,top=2.5cm,bottom=2.5cm}

\title{Actor-Critic详解}
\subtitle{Actor和Critic的定义及其在TD误差中的体现}
\author{强化学习笔记}
\date{\today}

\newtheorem{definition}{定义}
\newtheorem{theorem}{定理}
\newtheorem{example}{示例}
\newtheorem{remark}{注记}

\begin{document}

\maketitle

\tableofcontents
\newpage

\section{问题}

\textbf{问题1}:Actor-Critic方法中,谁是Actor?谁是Critic?

\textbf{问题2}:在TD误差公式中:
\begin{equation}
\delta_t = R_{t+1} + \gamma \hat{v}(S_{t+1}, w) - \hat{v}(S_t, w)
\label{eq:td_error}
\end{equation}

如何体现Actor和Critic的作用?

\section{Actor和Critic的定义}

\subsection{Actor(演员)}

\textbf{定义}:
\begin{itemize}
    \item \textbf{Actor}是策略(policy)$\pi(a|s, \theta)$
    \item 它负责选择动作(acting)
    \item 参数是 $\theta$(策略参数)
    \item 根据当前状态 $s$,输出动作概率分布或直接输出动作
\end{itemize}

\textbf{作用}:
\begin{itemize}
    \item 与环境交互,选择动作
    \item 根据Critic的反馈(TD误差)更新策略参数 $\theta$
    \item 目标是最大化期望回报
\end{itemize}

\textbf{更新规则}:
\begin{equation}
\theta_{t+1} = \theta_t + \alpha \delta_t \nabla_\theta \ln \pi(A_t|S_t, \theta_t)
\label{eq:actor_update}
\end{equation}

其中 $\delta_t$ 是TD误差,由Critic提供。

\subsection{Critic(评论家)}

\textbf{定义}:
\begin{itemize}
    \item \textbf{Critic}是价值函数(value function)$\hat{v}(s, w)$
    \item 它负责评估状态的价值(criticizing)
    \item 参数是 $w$(价值函数参数)
    \item 根据当前状态 $s$,输出状态价值估计
\end{itemize}

\textbf{作用}:
\begin{itemize}
    \item 评估当前状态的价值
    \item 计算TD误差 $\delta_t$,作为Actor的反馈信号
    \item 根据TD误差更新价值函数参数 $w$
\end{itemize}

\textbf{更新规则}:
\begin{equation}
w_{t+1} = w_t + \beta \delta_t \nabla_w \hat{v}(S_t, w_t)
\label{eq:critic_update}
\end{equation}

其中 $\delta_t$ 是TD误差。

\subsection{TD误差}

\textbf{定义}:
\begin{equation}
\delta_t = R_{t+1} + \gamma \hat{v}(S_{t+1}, w) - \hat{v}(S_t, w)
\label{eq:td_error_def}
\end{equation}

\textbf{含义}:
\begin{itemize}
    \item $R_{t+1}$:即时奖励
    \item $\gamma \hat{v}(S_{t+1}, w)$:下一状态的折扣价值
    \item $\hat{v}(S_t, w)$:当前状态的价值估计
    \item $\delta_t$:TD误差,表示实际回报与预期回报的差异
\end{itemize}

\textbf{作用}:
\begin{itemize}
    \item 作为Actor的反馈信号:$\delta_t > 0$ 表示动作好,$\delta_t < 0$ 表示动作差
    \item 作为Critic的更新信号:用于更新价值函数参数 $w$
\end{itemize}

\section{在TD误差中体现Actor和Critic}

\subsection{TD误差公式}

\begin{equation}
\delta_t = R_{t+1} + \gamma \hat{v}(S_{t+1}, w) - \hat{v}(S_t, w)
\end{equation}

\subsection{Critic的体现}

\textbf{Critic的作用}:
\begin{itemize}
    \item $\hat{v}(S_t, w)$:Critic评估当前状态 $S_t$ 的价值
    \item $\hat{v}(S_{t+1}, w)$:Critic评估下一状态 $S_{t+1}$ 的价值
    \item 这两个值都是由Critic(价值函数)提供的
\end{itemize}

\textbf{具体体现}:
\begin{itemize}
    \item \textbf{当前状态价值}:$\hat{v}(S_t, w)$ 是Critic对当前状态的评估
    \item \textbf{下一状态价值}:$\hat{v}(S_{t+1}, w)$ 是Critic对下一状态的评估
    \item \textbf{价值函数参数}:$w$ 是Critic的参数,通过TD误差更新
\end{itemize}

\subsection{Actor的体现}

\textbf{Actor的作用}:
\begin{itemize}
    \item Actor选择动作 $A_t$,导致状态从 $S_t$ 转移到 $S_{t+1}$
    \item Actor获得奖励 $R_{t+1}$
    \item Actor使用TD误差 $\delta_t$ 更新策略参数 $\theta$
\end{itemize}

\textbf{具体体现}:
\begin{itemize}
    \item \textbf{状态转移}:$S_t \to S_{t+1}$ 是由Actor选择的动作 $A_t$ 导致的
    \item \textbf{即时奖励}:$R_{t+1}$ 是Actor执行动作 $A_t$ 后获得的奖励
    \item \textbf{反馈信号}:$\delta_t$ 作为Actor的反馈,用于更新策略
\end{itemize}

\textbf{间接体现}:
\begin{itemize}
    \item TD误差 $\delta_t$ 反映了Actor选择的动作的质量
    \item 如果 $\delta_t > 0$,说明实际回报比预期好,应该增加选择该动作的概率
    \item 如果 $\delta_t < 0$,说明实际回报比预期差,应该减少选择该动作的概率
\end{itemize}

\subsection{完整的交互过程}

\textbf{步骤1:Actor选择动作}
\begin{itemize}
    \item 在状态 $S_t$,Actor根据策略 $\pi(\cdot|S_t, \theta)$ 选择动作 $A_t$
    \item 执行动作 $A_t$,环境转移到状态 $S_{t+1}$,获得奖励 $R_{t+1}$
\end{itemize}

\textbf{步骤2:Critic评估状态}
\begin{itemize}
    \item Critic评估当前状态:$\hat{v}(S_t, w)$
    \item Critic评估下一状态:$\hat{v}(S_{t+1}, w)$
\end{itemize}

\textbf{步骤3:计算TD误差}
\begin{equation}
\delta_t = R_{t+1} + \gamma \hat{v}(S_{t+1}, w) - \hat{v}(S_t, w)
\end{equation}

\textbf{步骤4:更新Actor}
\begin{equation}
\theta_{t+1} = \theta_t + \alpha \delta_t \nabla_\theta \ln \pi(A_t|S_t, \theta_t)
\end{equation}

\textbf{步骤5:更新Critic}
\begin{equation}
w_{t+1} = w_t + \beta \delta_t \nabla_w \hat{v}(S_t, w_t)
\end{equation}

\section{详细例子}

\subsection{例子:Gridworld中的Actor-Critic}

\textbf{场景}:
\begin{itemize}
    \item 状态:$S_t = (2, 3)$(Gridworld中的位置)
    \item 动作:$A_t = \text{右}$(向右移动)
    \item 奖励:$R_{t+1} = -1$(每步的代价)
    \item 下一状态:$S_{t+1} = (2, 4)$
    \item 折扣因子:$\gamma = 0.9$
\end{itemize}

\textbf{步骤1:Actor选择动作}
\begin{itemize}
    \item 在状态 $S_t = (2, 3)$,Actor根据策略 $\pi(\cdot|S_t, \theta)$ 选择动作 $A_t = \text{右}$
    \item 执行动作后,环境转移到 $S_{t+1} = (2, 4)$,获得奖励 $R_{t+1} = -1$
\end{itemize}

\textbf{步骤2:Critic评估状态}
\begin{itemize}
    \item Critic评估当前状态:$\hat{v}(S_t, w) = \hat{v}((2, 3), w) = 5.0$
    \item Critic评估下一状态:$\hat{v}(S_{t+1}, w) = \hat{v}((2, 4), w) = 6.0$
\end{itemize}

\textbf{步骤3:计算TD误差}
\begin{align}
\delta_t &= R_{t+1} + \gamma \hat{v}(S_{t+1}, w) - \hat{v}(S_t, w) \\
         &= -1 + 0.9 \times 6.0 - 5.0 \\
         &= -1 + 5.4 - 5.0 \\
         &= -0.6
\end{align}

\textbf{解释}:
\begin{itemize}
    \item TD误差 $\delta_t = -0.6 < 0$,说明实际回报比预期差
    \item 这意味着选择动作"右"的效果不如预期
    \item Actor应该减少选择该动作的概率
\end{itemize}

\textbf{步骤4:更新Actor}
\begin{equation}
\theta_{t+1} = \theta_t + \alpha \times (-0.6) \times \nabla_\theta \ln \pi(\text{右}|(2, 3), \theta_t)
\end{equation}

\textbf{解释}:
\begin{itemize}
    \item 因为 $\delta_t < 0$,所以更新是负的
    \item 这会减少选择动作"右"的概率
\end{itemize}

\textbf{步骤5:更新Critic}
\begin{equation}
w_{t+1} = w_t + \beta \times (-0.6) \times \nabla_w \hat{v}((2, 3), w_t)
\end{equation}

\textbf{解释}:
\begin{itemize}
    \item 因为 $\delta_t < 0$,所以更新是负的
    \item 这会降低状态 $(2, 3)$ 的价值估计
\end{itemize}

\subsection{例子:TD误差为正的情况}

\textbf{场景}:
\begin{itemize}
    \item 状态:$S_t = (1, 1)$
    \item 动作:$A_t = \text{下}$(向下移动)
    \item 奖励:$R_{t+1} = 10$(到达目标)
    \item 下一状态:$S_{t+1} = (2, 1)$(目标状态)
    \item 折扣因子:$\gamma = 0.9$
\end{itemize}

\textbf{Critic评估}:
\begin{itemize}
    \item $\hat{v}(S_t, w) = \hat{v}((1, 1), w) = 8.0$
    \item $\hat{v}(S_{t+1}, w) = \hat{v}((2, 1), w) = 0.0$(终止状态)
\end{itemize}

\textbf{计算TD误差}:
\begin{align}
\delta_t &= R_{t+1} + \gamma \hat{v}(S_{t+1}, w) - \hat{v}(S_t, w) \\
         &= 10 + 0.9 \times 0.0 - 8.0 \\
         &= 10 - 8.0 \\
         &= 2.0
\end{align}

\textbf{解释}:
\begin{itemize}
    \item TD误差 $\delta_t = 2.0 > 0$,说明实际回报比预期好
    \item 这意味着选择动作"下"的效果比预期好
    \item Actor应该增加选择该动作的概率
\end{itemize}

\textbf{更新Actor}:
\begin{equation}
\theta_{t+1} = \theta_t + \alpha \times 2.0 \times \nabla_\theta \ln \pi(\text{下}|(1, 1), \theta_t)
\end{equation}

\textbf{解释}:
\begin{itemize}
    \item 因为 $\delta_t > 0$,所以更新是正的
    \item 这会增加选择动作"下"的概率
\end{itemize}

\section{Actor和Critic的协作}

\subsection{分工明确}

\textbf{Actor(演员)}:
\begin{itemize}
    \item \textbf{职责}:选择动作,与环境交互
    \item \textbf{参数}:$\theta$(策略参数)
    \item \textbf{更新}:根据TD误差 $\delta_t$ 更新策略
    \item \textbf{目标}:最大化期望回报
\end{itemize}

\textbf{Critic(评论家)}:
\begin{itemize}
    \item \textbf{职责}:评估状态价值,提供反馈
    \item \textbf{参数}:$w$(价值函数参数)
    \item \textbf{更新}:根据TD误差 $\delta_t$ 更新价值函数
    \item \textbf{目标}:准确估计状态价值
\end{itemize}

\subsection{相互依赖}

\textbf{Actor依赖Critic}:
\begin{itemize}
    \item Actor需要Critic提供的TD误差 $\delta_t$ 来更新策略
    \item 没有Critic,Actor不知道动作的好坏
    \item TD误差 $\delta_t$ 是Actor的"老师",告诉它动作的质量
\end{itemize}

\textbf{Critic依赖Actor}:
\begin{itemize}
    \item Critic需要Actor选择的动作来产生状态转移和奖励
    \item 没有Actor,Critic无法获得数据(状态转移、奖励)
    \item Actor是Critic的"数据源"
\end{itemize}

\subsection{协同工作}

\textbf{完整流程}:
\begin{enumerate}
    \item Actor选择动作 $A_t$,与环境交互
    \item 环境返回状态 $S_{t+1}$ 和奖励 $R_{t+1}$
    \item Critic评估状态 $S_t$ 和 $S_{t+1}$ 的价值
    \item 计算TD误差 $\delta_t = R_{t+1} + \gamma \hat{v}(S_{t+1}, w) - \hat{v}(S_t, w)$
    \item Actor根据 $\delta_t$ 更新策略参数 $\theta$
    \item Critic根据 $\delta_t$ 更新价值函数参数 $w$
    \item 重复上述过程
\end{enumerate}

\section{TD误差公式的详细分析}

\subsection{公式分解}

\begin{equation}
\delta_t = R_{t+1} + \gamma \hat{v}(S_{t+1}, w) - \hat{v}(S_t, w)
\end{equation}

\textbf{组成部分}:
\begin{itemize}
    \item $R_{t+1}$:即时奖励,由Actor执行动作 $A_t$ 后获得
    \item $\gamma \hat{v}(S_{t+1}, w)$:下一状态的折扣价值,由Critic评估
    \item $\hat{v}(S_t, w)$:当前状态的价值,由Critic评估
    \item $\delta_t$:TD误差,综合了Actor和Critic的信息
\end{itemize}

\subsection{Actor的贡献}

\textbf{直接贡献}:
\begin{itemize}
    \item $R_{t+1}$:Actor执行动作 $A_t$ 后获得的即时奖励
    \item $S_{t+1}$:Actor选择的动作导致的状态转移
\end{itemize}

\textbf{间接贡献}:
\begin{itemize}
    \item TD误差 $\delta_t$ 反映了Actor选择的动作的质量
    \item 如果 $\delta_t > 0$,说明动作好,应该增加选择该动作的概率
    \item 如果 $\delta_t < 0$,说明动作差,应该减少选择该动作的概率
\end{itemize}

\subsection{Critic的贡献}

\textbf{直接贡献}:
\begin{itemize}
    \item $\hat{v}(S_t, w)$:Critic对当前状态的价值估计
    \item $\hat{v}(S_{t+1}, w)$:Critic对下一状态的价值估计
\end{itemize}

\textbf{间接贡献}:
\begin{itemize}
    \item TD误差 $\delta_t$ 反映了Critic的价值估计的准确性
    \item 如果 $\delta_t \neq 0$,说明价值估计不准确,需要更新
    \item Critic通过更新参数 $w$ 来改进价值估计
\end{itemize}

\section{与其他方法的对比}

\subsection{REINFORCE(只有Actor)}

\textbf{特点}:
\begin{itemize}
    \item 只有Actor,没有Critic
    \item 使用完整episode的回报 $G_t$ 来更新策略
    \item 更新规则:$\theta_{t+1} = \theta_t + \alpha G_t \nabla_\theta \ln \pi(A_t|S_t, \theta_t)$
\end{itemize}

\textbf{问题}:
\begin{itemize}
    \item 需要等待episode结束才能更新
    \item 方差大,学习慢
    \item 没有价值函数来评估状态
\end{itemize}

\subsection{Actor-Critic(Actor + Critic)}

\textbf{特点}:
\begin{itemize}
    \item 有Actor和Critic两个组件
    \item 使用TD误差 $\delta_t$ 来更新策略(不需要等待episode结束)
    \item 更新规则:$\theta_{t+1} = \theta_t + \alpha \delta_t \nabla_\theta \ln \pi(A_t|S_t, \theta_t)$
\end{itemize}

\textbf{优势}:
\begin{itemize}
    \item 可以在线学习(不需要等待episode结束)
    \item 方差小,学习快(使用价值函数作为基线)
    \item 有价值函数来评估状态
\end{itemize}

\subsection{价值函数方法(只有Critic)}

\textbf{特点}:
\begin{itemize}
    \item 只有Critic,没有Actor
    \item 使用价值函数来选择动作(例如:$\varepsilon$-greedy)
    \item 不直接学习策略
\end{itemize}

\textbf{问题}:
\begin{itemize}
    \item 只能处理离散动作空间
    \item 不能直接学习随机策略
    \item 动作选择依赖于价值函数
\end{itemize}

\section{总结}

\subsection{Actor和Critic的定义}

\textbf{Actor(演员)}:
\begin{itemize}
    \item 是策略 $\pi(a|s, \theta)$
    \item 负责选择动作
    \item 参数是 $\theta$
    \item 根据TD误差更新策略
\end{itemize}

\textbf{Critic(评论家)}:
\begin{itemize}
    \item 是价值函数 $\hat{v}(s, w)$
    \item 负责评估状态价值
    \item 参数是 $w$
    \item 根据TD误差更新价值函数
\end{itemize}

\subsection{在TD误差中的体现}

\textbf{TD误差公式}:
\begin{equation}
\delta_t = R_{t+1} + \gamma \hat{v}(S_{t+1}, w) - \hat{v}(S_t, w)
\end{equation}

\textbf{Critic的体现}:
\begin{itemize}
    \item $\hat{v}(S_t, w)$:Critic评估当前状态
    \item $\hat{v}(S_{t+1}, w)$:Critic评估下一状态
    \item 这两个值都是由Critic提供的
\end{itemize}

\textbf{Actor的体现}:
\begin{itemize}
    \item $R_{t+1}$:Actor执行动作后获得的奖励
    \item $S_{t+1}$:Actor选择的动作导致的状态转移
    \item $\delta_t$:作为Actor的反馈信号,用于更新策略
\end{itemize}

\subsection{协作关系}

\begin{quote}
\textbf{Actor和Critic相互依赖、协同工作}:
\begin{itemize}
    \item Actor需要Critic提供的TD误差来更新策略
    \item Critic需要Actor产生的数据(状态转移、奖励)来更新价值函数
    \item TD误差 $\delta_t$ 是两者之间的桥梁,综合了Actor和Critic的信息
\end{itemize}
\end{quote}

\subsection{关键公式}

\textbf{TD误差}:
\begin{equation}
\delta_t = R_{t+1} + \gamma \hat{v}(S_{t+1}, w) - \hat{v}(S_t, w)
\end{equation}

\textbf{Actor更新}:
\begin{equation}
\theta_{t+1} = \theta_t + \alpha \delta_t \nabla_\theta \ln \pi(A_t|S_t, \theta_t)
\end{equation}

\textbf{Critic更新}:
\begin{equation}
w_{t+1} = w_t + \beta \delta_t \nabla_w \hat{v}(S_t, w_t)
\end{equation}

\end{document}

