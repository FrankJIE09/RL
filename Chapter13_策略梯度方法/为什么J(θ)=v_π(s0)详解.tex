\documentclass[12pt,a4paper]{article}
\usepackage[UTF8]{ctex}
\usepackage{amsmath}
\usepackage{amssymb}
\usepackage{amsthm}
\usepackage{geometry}
\usepackage{hyperref}
\usepackage{enumitem}
\usepackage{booktabs}

\geometry{left=2.5cm,right=2.5cm,top=2.5cm,bottom=2.5cm}

\title{为什么回合制任务:$J(\theta) = v_{\pi_\theta}(s_0)$?}
\subtitle{性能指标定义的合理性解释}
\author{强化学习笔记}
\date{\today}

\newtheorem{definition}{定义}
\newtheorem{theorem}{定理}
\newtheorem{proposition}{命题}
\newtheorem{example}{示例}
\newtheorem{remark}{注记}

\begin{document}

\maketitle

\tableofcontents
\newpage

\section{问题}

\textbf{问题}:为什么回合制任务的性能指标定义为:
\begin{equation}
J(\theta) = v_{\pi_\theta}(s_0)
\label{eq:performance}
\end{equation}

其中 $s_0$ 是起始状态,$v_{\pi_\theta}$ 是策略 $\pi_\theta$ 的真实价值函数?

\section{回合制任务的特点}

\subsection{什么是回合制任务?}

\textbf{回合制任务(Episodic Task)}:
\begin{itemize}
    \item 任务有明确的开始和结束
    \item 每个episode从起始状态 $s_0$ 开始
    \item 在某个终止状态结束
    \item 每个episode是独立的
\end{itemize}

\textbf{例子}:
\begin{itemize}
    \item 游戏:每局游戏从开始到结束
    \item 导航任务:从起点到终点
    \item 控制任务:完成一个操作序列
\end{itemize}

\subsection{回合制任务的目标}

\textbf{目标}:
\begin{itemize}
    \item 最大化从起始状态 $s_0$ 开始的期望回报
    \item 我们希望策略在从 $s_0$ 开始时表现最好
    \item 这是最自然的性能度量
\end{itemize}

\section{为什么选择 $v_{\pi_\theta}(s_0)$?}

\subsection{价值函数的定义}

\textbf{状态价值函数}:
\begin{equation}
v_\pi(s) = \mathbb{E}_\pi[G_t | S_t = s] = \mathbb{E}_\pi\left[\sum_{k=0}^{\infty} \gamma^k R_{t+k+1} \middle| S_t = s\right]
\end{equation}

\textbf{含义}:
\begin{itemize}
    \item $v_\pi(s)$ 是从状态 $s$ 开始,遵循策略 $\pi$ 的期望回报
    \item 它量化了"从状态 $s$ 开始,使用策略 $\pi$ 能获得多少回报"
\end{itemize}

\subsection{起始状态的价值函数}

\textbf{起始状态的价值函数}:
\begin{equation}
v_{\pi_\theta}(s_0) = \mathbb{E}_{\pi_\theta}[G_0 | S_0 = s_0]
\end{equation}

\textbf{含义}:
\begin{itemize}
    \item $v_{\pi_\theta}(s_0)$ 是从起始状态 $s_0$ 开始,遵循策略 $\pi_\theta$ 的期望回报
    \item 它量化了"从起始状态开始,使用策略 $\pi_\theta$ 能获得多少回报"
    \item 这正是我们想要最大化的量!
\end{itemize}

\subsection{为什么这是合理的性能指标?}

\textbf{理由1:直接对应目标}
\begin{itemize}
    \item 回合制任务的目标是:从起始状态 $s_0$ 开始,获得尽可能高的回报
    \item $v_{\pi_\theta}(s_0)$ 正是从 $s_0$ 开始的期望回报
    \item 最大化 $v_{\pi_\theta}(s_0)$ 就是最大化从起始状态开始的期望回报
\end{itemize}

\textbf{理由2:自然的性能度量}
\begin{itemize}
    \item 在回合制任务中,我们总是从 $s_0$ 开始
    \item 我们关心的是"从 $s_0$ 开始的表现"
    \item $v_{\pi_\theta}(s_0)$ 直接度量了这个表现
\end{itemize}

\textbf{理由3:数学上的简洁性}
\begin{itemize}
    \item $v_{\pi_\theta}(s_0)$ 是一个标量,易于优化
    \item 可以计算梯度 $\nabla_\theta v_{\pi_\theta}(s_0)$
    \item 策略梯度定理提供了计算这个梯度的方法
\end{itemize}

\section{具体例子}

\subsection{例子1:Gridworld导航}

\textbf{场景}:从起点 $s_0$ 导航到终点 $G$。

\textbf{性能指标}:
\begin{equation}
J(\theta) = v_{\pi_\theta}(s_0)
\end{equation}

\textbf{含义}:
\begin{itemize}
    \item $v_{\pi_\theta}(s_0)$ 是从起点 $s_0$ 开始,使用策略 $\pi_\theta$ 到达终点的期望回报
    \item 如果回报是负的步数(每步 $-1$),则 $v_{\pi_\theta}(s_0)$ 是期望步数的负值
    \item 最大化 $v_{\pi_\theta}(s_0)$ 就是最小化期望步数
    \item 这正是我们想要的!
\end{itemize}

\textbf{计算例子}:
\begin{itemize}
    \item 如果最优策略从 $s_0$ 到 $G$ 需要 $5$ 步,则 $v_{\pi_*}(s_0) = -5$
    \item 如果次优策略需要 $10$ 步,则 $v_{\pi}(s_0) = -10$
    \item 显然 $-5 > -10$,所以最优策略更好
    \item 最大化 $v_{\pi_\theta}(s_0)$ 确实能找到最优策略
\end{itemize}

\subsection{例子2:游戏}

\textbf{场景}:玩一个游戏,每局从开始状态 $s_0$ 开始。

\textbf{性能指标}:
\begin{equation}
J(\theta) = v_{\pi_\theta}(s_0)
\end{equation}

\textbf{含义}:
\begin{itemize}
    \item $v_{\pi_\theta}(s_0)$ 是从游戏开始,使用策略 $\pi_\theta$ 的期望得分
    \item 最大化 $v_{\pi_\theta}(s_0)$ 就是最大化期望得分
    \item 这正是我们想要的!
\end{itemize}

\section{与其他可能的定义对比}

\subsection{为什么不使用其他状态的价值?}

\textbf{选项1:所有状态的平均价值}
\begin{equation}
J(\theta) = \frac{1}{|\mathcal{S}|} \sum_{s} v_{\pi_\theta}(s)
\end{equation}

\textbf{问题}:
\begin{itemize}
    \item 在回合制任务中,不是所有状态都会被访问
    \item 某些状态可能永远不会被访问到
    \item 这些状态的价值对性能没有影响
    \item 不应该在性能指标中考虑它们
\end{itemize}

\textbf{选项2:加权平均}
\begin{equation}
J(\theta) = \sum_{s} \mu(s) v_{\pi_\theta}(s)
\end{equation}

其中 $\mu(s)$ 是状态分布。

\textbf{问题}:
\begin{itemize}
    \item 状态分布 $\mu(s)$ 依赖于策略 $\pi_\theta$
    \item 这会使性能指标的计算变得复杂
    \item 而且,在回合制任务中,我们主要关心从 $s_0$ 开始的表现
\end{itemize}

\textbf{选项3:终止状态的价值}
\begin{equation}
J(\theta) = v_{\pi_\theta}(s_{\text{terminal}})
\end{equation}

\textbf{问题}:
\begin{itemize}
    \item 终止状态的价值通常是 $0$(定义)
    \item 这不能区分不同策略的好坏
    \item 没有意义
\end{itemize}

\subsection{为什么选择起始状态?}

\textbf{优势}:
\begin{itemize}
    \item \textbf{直接对应目标}:从 $s_0$ 开始的表现正是我们关心的
    \item \textbf{数学简洁}:只需要计算一个状态的价值
    \item \textbf{自然定义}:符合回合制任务的特点
    \item \textbf{易于优化}:可以计算梯度并优化
\end{itemize}

\section{数学上的合理性}

\subsection{价值函数的性质}

\textbf{性质1:单调性}
\begin{itemize}
    \item 如果策略 $\pi_1$ 在所有状态的价值都不小于策略 $\pi_2$,则 $\pi_1$ 至少和 $\pi_2$ 一样好
    \item 特别地,如果 $v_{\pi_1}(s_0) \geq v_{\pi_2}(s_0)$,则从 $s_0$ 开始,$\pi_1$ 至少和 $\pi_2$ 一样好
\end{itemize}

\textbf{性质2:最优性}
\begin{itemize}
    \item 最优策略 $\pi_*$ 在所有状态的价值都不小于任何其他策略
    \item 特别地,$v_{\pi_*}(s_0) \geq v_{\pi}(s_0)$ 对所有策略 $\pi$ 成立
    \item 最大化 $v_{\pi_\theta}(s_0)$ 可以找到最优策略
\end{itemize}

\subsection{策略梯度定理的适用性}

\textbf{策略梯度定理}:
\begin{equation}
\nabla_\theta J(\theta) = \nabla_\theta v_{\pi_\theta}(s_0) \propto \sum_{s} \mu(s) \sum_{a} q_\pi(s, a) \nabla_\theta \pi(a|s, \theta)
\end{equation}

\textbf{关键}:
\begin{itemize}
    \item 策略梯度定理提供了计算 $\nabla_\theta v_{\pi_\theta}(s_0)$ 的方法
    \item 不涉及状态分布的导数
    \item 可以估计和优化
\end{itemize}

\section{与持续任务的对比}

\subsection{持续任务}

\textbf{持续任务(Continuing Task)}:
\begin{itemize}
    \item 没有明确的起始状态
    \item 任务持续进行,没有终止
    \item 性能定义为平均奖励率
\end{itemize}

\textbf{性能指标}:
\begin{equation}
J(\theta) = r(\pi) = \lim_{h \to \infty} \frac{1}{h} \sum_{t=1}^{h} \mathbb{E}[R_t | S_0, A_{0:t-1} \sim \pi]
\end{equation}

\textbf{区别}:
\begin{itemize}
    \item 持续任务:使用平均奖励率
    \item 回合制任务:使用起始状态的价值函数
    \item 这是因为任务性质不同
\end{itemize}

\section{具体计算例子}

\subsection{例子:简单Gridworld}

\textbf{环境}:
\begin{itemize}
    \item 状态:$s_0$(起始)→ $s_1$ → $s_2$ → $G$(终止)
    \item 奖励:每步 $-1$
    \item 折扣因子:$\gamma = 1$(无折扣)
\end{itemize}

\textbf{策略 $\pi_1$}:总是向右
\begin{itemize}
    \item 路径:$s_0 \to s_1 \to s_2 \to G$
    \item 步数:$3$ 步
    \item 回报:$G_0 = -1 - 1 - 1 = -3$
    \item 价值:$v_{\pi_1}(s_0) = -3$
\end{itemize}

\textbf{策略 $\pi_2$}:总是向左(然后可能随机)
\begin{itemize}
    \item 路径:可能很长,甚至无法到达 $G$
    \item 期望步数:很大(如 $20$ 步)
    \item 价值:$v_{\pi_2}(s_0) = -20$
\end{itemize}

\textbf{比较}:
\begin{itemize}
    \item $v_{\pi_1}(s_0) = -3 > -20 = v_{\pi_2}(s_0)$
    \item 策略 $\pi_1$ 更好
    \item 最大化 $v_{\pi_\theta}(s_0)$ 确实能找到更好的策略
\end{itemize}

\subsection{性能指标的作用}

\textbf{优化过程}:
\begin{enumerate}
    \item 初始策略:$\pi_{\theta_0}$,$v_{\pi_{\theta_0}}(s_0) = -20$
    \item 计算梯度:$\nabla_\theta v_{\pi_{\theta_0}}(s_0)$
    \item 更新参数:$\theta_1 = \theta_0 + \alpha \nabla_\theta v_{\pi_{\theta_0}}(s_0)$
    \item 新策略:$\pi_{\theta_1}$,$v_{\pi_{\theta_1}}(s_0) = -15$(改进)
    \item 继续优化,最终:$v_{\pi_{\theta_*}}(s_0) = -3$(最优)
\end{enumerate}

\textbf{关键}:
\begin{itemize}
    \item 性能指标 $v_{\pi_\theta}(s_0)$ 量化了策略的好坏
    \item 通过最大化这个指标,我们找到了更好的策略
    \item 这正是我们想要的!
\end{itemize}

\section{为什么可以简化假设?}

\subsection{假设:每个episode从 $s_0$ 开始}

\textbf{原书中的简化}:
\begin{itemize}
    \item 假设每个episode从某个特定的(非随机的)状态 $s_0$ 开始
    \item 这简化了符号和推导
    \item 不失一般性
\end{itemize}

\textbf{如果起始状态是随机的}:
\begin{itemize}
    \item 如果起始状态是从分布 $\mu_0(s)$ 中随机选择的
    \item 性能指标应该是:$J(\theta) = \sum_{s} \mu_0(s) v_{\pi_\theta}(s)$
    \item 但可以简化为:$J(\theta) = v_{\pi_\theta}(s_0)$,其中 $s_0$ 是某个代表性的起始状态
    \item 或者,可以重新定义 $s_0$ 为起始状态的分布
\end{itemize}

\subsection{不失一般性}

\textbf{原因}:
\begin{itemize}
    \item 如果起始状态是随机的,我们可以引入一个虚拟的起始状态
    \item 这个虚拟状态以概率 $\mu_0(s)$ 转移到实际起始状态
    \item 然后从虚拟状态开始,问题就变成了从固定状态 $s_0$ 开始
    \item 因此,假设从 $s_0$ 开始不失一般性
\end{itemize}

\section{总结}

\subsection{核心原因}

\begin{enumerate}
    \item \textbf{直接对应目标}:
    \begin{itemize}
        \item 回合制任务的目标是从起始状态 $s_0$ 开始获得高回报
        \item $v_{\pi_\theta}(s_0)$ 正是从 $s_0$ 开始的期望回报
        \item 最大化 $v_{\pi_\theta}(s_0)$ 就是最大化目标
    \end{itemize}
    
    \item \textbf{自然的性能度量}:
    \begin{itemize}
        \item 在回合制任务中,我们总是从 $s_0$ 开始
        \item 我们关心的是"从 $s_0$ 开始的表现"
        \item $v_{\pi_\theta}(s_0)$ 直接度量了这个表现
    \end{itemize}
    
    \item \textbf{数学上的合理性}:
    \begin{itemize}
        \item $v_{\pi_\theta}(s_0)$ 是一个标量,易于优化
        \item 策略梯度定理提供了计算梯度的方法
        \item 可以有效地优化
    \end{itemize}
    
    \item \textbf{与其他定义的对比}:
    \begin{itemize}
        \item 使用所有状态的平均价值:不合理(不是所有状态都会被访问)
        \item 使用终止状态的价值:无意义(终止状态价值为0)
        \item 使用起始状态的价值:最合理(直接对应目标)
    \end{itemize}
\end{enumerate}

\subsection{关键公式}

\textbf{性能指标定义}:
\begin{equation}
J(\theta) = v_{\pi_\theta}(s_0)
\end{equation}

\textbf{价值函数定义}:
\begin{equation}
v_{\pi_\theta}(s_0) = \mathbb{E}_{\pi_\theta}[G_0 | S_0 = s_0] = \mathbb{E}_{\pi_\theta}\left[\sum_{k=0}^{\infty} \gamma^k R_{k+1} \middle| S_0 = s_0\right]
\end{equation}

\textbf{优化目标}:
\begin{equation}
\theta^* = \arg\max_\theta J(\theta) = \arg\max_\theta v_{\pi_\theta}(s_0)
\end{equation}

\subsection{直观理解}

\begin{quote}
\textbf{为什么 $J(\theta) = v_{\pi_\theta}(s_0)$?}

因为回合制任务总是从起始状态 $s_0$ 开始,我们关心的是"从 $s_0$ 开始,使用策略 $\pi_\theta$ 能获得多少回报"。$v_{\pi_\theta}(s_0)$ 正是这个期望回报,所以它是性能指标的自然选择。最大化 $v_{\pi_\theta}(s_0)$ 就是最大化从起始状态开始的期望回报,这正是我们想要的目标。
\end{quote}

\end{document}

