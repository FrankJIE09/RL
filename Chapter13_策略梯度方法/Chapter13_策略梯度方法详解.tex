\documentclass[12pt,a4paper]{article}
\usepackage[UTF8]{ctex}
\usepackage{amsmath}
\usepackage{amssymb}
\usepackage{amsthm}
\usepackage{algorithm}
\usepackage{algorithmic}
\usepackage{geometry}
\usepackage{hyperref}
\usepackage{booktabs}
\usepackage{enumitem}

\geometry{left=2.5cm,right=2.5cm,top=2.5cm,bottom=2.5cm}

\title{第13章:策略梯度方法详解}
\author{强化学习笔记}
\date{\today}

\newtheorem{definition}{定义}
\newtheorem{theorem}{定理}
\newtheorem{proposition}{命题}
\newtheorem{example}{示例}
\newtheorem{remark}{注记}

\begin{document}

\maketitle

\tableofcontents
\newpage

\section{引言}

本章介绍\textbf{策略梯度方法}(Policy Gradient Methods),这是强化学习中的一类重要方法。与之前章节中基于动作价值函数的方法不同,策略梯度方法直接学习参数化的策略,可以无需价值函数就能选择动作。

\textbf{核心思想}:
\begin{itemize}
    \item 直接学习参数化策略 $\pi(a|s, \theta)$
    \item 通过梯度上升优化性能指标 $J(\theta)$
    \item 使用策略梯度定理计算梯度
    \item 可以学习随机最优策略
\end{itemize}

\section{策略近似及其优势}

\subsection{策略参数化}

\textbf{参数化策略}:
\begin{equation}
\pi(a|s, \theta) = \Pr\{A_t = a | S_t = s, \theta_t = \theta\}
\end{equation}

其中 $\theta \in \mathbb{R}^{d'}$ 是策略的参数向量。

\textbf{要求}:
\begin{itemize}
    \item 策略必须关于参数可微:$\nabla_\theta \pi(a|s, \theta)$ 存在且有限
    \item 为保证探索,策略通常不能变成确定性的:$\pi(a|s, \theta) \in (0, 1)$
\end{itemize}

\subsection{Soft-max动作偏好参数化}

对于离散动作空间,最常用的参数化方式是\textbf{soft-max动作偏好}:

\begin{equation}
\pi(a|s, \theta) = \frac{e^{h(s, a, \theta)}}{\sum_{b} e^{h(s, b, \theta)}}
\label{eq:softmax_policy}
\end{equation}

其中 $h(s, a, \theta) \in \mathbb{R}$ 是状态-动作对的数值偏好。

\textbf{动作偏好的参数化}:
\begin{itemize}
    \item \textbf{线性偏好}:$h(s, a, \theta) = \theta^T x(s, a)$
    \item \textbf{神经网络}:$h(s, a, \theta)$ 由深度神经网络计算(如 AlphaGo)
\end{itemize}

\subsection{策略参数化的优势}

\textbf{1. 可以接近确定性策略}:
\begin{itemize}
    \item 策略梯度方法可以学习接近确定性的策略
    \item 而 $\varepsilon$-贪婪方法总是有 $\varepsilon$ 概率选择随机动作
    \item 最优动作的偏好会被驱动到无穷大(相对于次优动作)
\end{itemize}

\textbf{2. 可以学习随机最优策略}:
\begin{itemize}
    \item 在某些问题中,最优策略可能是随机的(如扑克中的虚张声势)
    \item 动作价值方法无法自然地找到随机最优策略
    \item 策略近似方法可以学习任意概率分布
\end{itemize}

\textbf{3. 策略可能更简单}:
\begin{itemize}
    \item 对于某些问题,策略函数比动作价值函数更简单
    \item 策略方法可能学习更快,得到更好的渐近策略
\end{itemize}

\textbf{4. 注入先验知识}:
\begin{itemize}
    \item 策略参数化是向强化学习系统注入先验知识的好方法
    \item 这是使用基于策略的学习方法的最重要原因之一
\end{itemize}

\section{策略梯度定理}

\subsection{性能指标}

\textbf{回合制任务}:
\begin{equation}
J(\theta) = v_{\pi_\theta}(s_0)
\label{eq:episodic_performance}
\end{equation}

其中 $s_0$ 是起始状态,$v_{\pi_\theta}$ 是策略 $\pi_\theta$ 的真实价值函数。

\textbf{持续任务}:
\begin{equation}
J(\theta) = r(\pi) = \lim_{h \to \infty} \frac{1}{h} \sum_{t=1}^{h} \mathbb{E}[R_t | S_0, A_{0:t-1} \sim \pi]
\end{equation}

\subsection{策略梯度定理(回合制)}

\begin{theorem}[策略梯度定理]
对于回合制任务,性能指标的梯度为:
\begin{equation}
\nabla_\theta J(\theta) \propto \sum_{s} \mu(s) \sum_{a} q_\pi(s, a) \nabla_\theta \pi(a|s, \theta)
\label{eq:policy_gradient_theorem}
\end{equation}

其中:
\begin{itemize}
    \item $\mu(s)$ 是在策略 $\pi$ 下的状态分布(on-policy distribution)
    \item $q_\pi(s, a)$ 是动作价值函数
    \item $\propto$ 表示"正比于"
\end{itemize}
\end{theorem}

\textbf{关键洞察}:
\begin{itemize}
    \item 梯度不涉及状态分布的导数
    \item 只需要知道策略参数化 $\pi(a|s, \theta)$ 和动作价值函数 $q_\pi(s, a)$
    \item 这使得我们可以估计性能梯度,即使不知道策略变化对状态分布的影响
\end{itemize}

\subsection{策略梯度定理的证明思路}

\textbf{步骤1}:状态价值函数的梯度
\begin{equation}
\nabla v_\pi(s) = \nabla \left[\sum_{a} \pi(a|s) q_\pi(s, a)\right]
\end{equation}

\textbf{步骤2}:使用乘积法则
\begin{equation}
\nabla v_\pi(s) = \sum_{a} \left[\nabla \pi(a|s) q_\pi(s, a) + \pi(a|s) \nabla q_\pi(s, a)\right]
\end{equation}

\textbf{步骤3}:展开 $q_\pi(s, a)$ 并递归展开
\begin{equation}
\nabla v_\pi(s) = \sum_{x \in \mathcal{S}} \sum_{k=0}^{\infty} \Pr(s \to x, k, \pi) \sum_{a} \nabla \pi(a|x) q_\pi(x, a)
\end{equation}

\textbf{步骤4}:应用到性能指标
\begin{equation}
\nabla J(\theta) = \nabla v_\pi(s_0) = \sum_{s} \mu(s) \sum_{a} \nabla \pi(a|s) q_\pi(s, a)
\end{equation}

\subsection{策略梯度定理的期望形式}

\begin{equation}
\nabla_\theta J(\theta) \propto \mathbb{E}_\pi \left[\sum_{a} q_\pi(S_t, a) \nabla_\theta \pi(a|S_t, \theta)\right]
\label{eq:policy_gradient_expectation}
\end{equation}

\section{REINFORCE:蒙特卡洛策略梯度}

\subsection{REINFORCE算法推导}

从策略梯度定理的期望形式出发,引入实际选择的动作 $A_t$:

\begin{align}
\nabla_\theta J(\theta) &= \mathbb{E}_\pi \left[\sum_{a} \pi(a|S_t, \theta) q_\pi(S_t, a) \frac{\nabla_\theta \pi(a|S_t, \theta)}{\pi(a|S_t, \theta)}\right] \\
&= \mathbb{E}_\pi \left[q_\pi(S_t, A_t) \frac{\nabla_\theta \pi(A_t|S_t, \theta)}{\pi(A_t|S_t, \theta)}\right] \\
&= \mathbb{E}_\pi \left[G_t \frac{\nabla_\theta \pi(A_t|S_t, \theta)}{\pi(A_t|S_t, \theta)}\right]
\label{eq:reinforce_gradient}
\end{align}

其中最后一步使用了 $\mathbb{E}_\pi[G_t | S_t, A_t] = q_\pi(S_t, A_t)$。

\subsection{REINFORCE更新规则}

使用样本梯度进行随机梯度上升:

\begin{equation}
\theta_{t+1} = \theta_t + \alpha G_t \frac{\nabla_\theta \pi(A_t|S_t, \theta_t)}{\pi(A_t|S_t, \theta_t)}
\label{eq:reinforce_update}
\end{equation}

\textbf{等价形式}(使用对数梯度):
\begin{equation}
\theta_{t+1} = \theta_t + \alpha G_t \nabla_\theta \ln \pi(A_t|S_t, \theta_t)
\label{eq:reinforce_log}
\end{equation}

其中使用了恒等式:$\nabla \ln x = \frac{\nabla x}{x}$。

\textbf{资格向量}(Eligibility Vector):
\begin{equation}
\nabla_\theta \ln \pi(A_t|S_t, \theta_t)
\end{equation}

这是策略参数化在算法中唯一出现的地方。

\subsection{REINFORCE算法伪代码}

\begin{algorithm}[H]
\caption{REINFORCE:蒙特卡洛策略梯度控制(回合制)}
\begin{algorithmic}[1]
\REQUIRE 可微策略参数化 $\pi(a|s, \theta)$
\ENSURE 最优策略 $\pi_*$
\STATE \textbf{算法参数}:步长 $\alpha > 0$
\STATE \textbf{初始化}:策略参数 $\theta \in \mathbb{R}^{d'}$(如设为0)
\REPEAT
    \STATE 生成回合:$S_0, A_0, R_1, \ldots, S_{T-1}, A_{T-1}, R_T$,遵循 $\pi(\cdot|\cdot, \theta)$
    \FOR{回合的每一步 $t = 0, 1, \ldots, T-1$}
        \STATE $G \gets \sum_{k=t+1}^{T} \gamma^{k-t-1} R_k$ \COMMENT{计算回报 $G_t$}
        \STATE $\theta \gets \theta + \alpha \gamma^t G \nabla_\theta \ln \pi(A_t|S_t, \theta)$
    \ENDFOR
\UNTIL{收敛}
\RETURN $\pi_* = \pi(\cdot|\cdot, \theta)$
\end{algorithmic}
\end{algorithm}

\subsection{REINFORCE的特点}

\textbf{优点}:
\begin{itemize}
    \item \textbf{理论收敛性}:期望更新方向与性能梯度方向相同
    \item \textbf{无偏估计}:使用完整回报 $G_t$,不引入偏差
    \item \textbf{简单直观}:更新与回报成正比,与动作概率成反比
\end{itemize}

\textbf{缺点}:
\begin{itemize}
    \item \textbf{高方差}:作为蒙特卡洛方法,方差可能很大
    \item \textbf{学习慢}:高方差导致学习速度慢
    \item \textbf{仅适用于回合制}:需要等待回合结束
\end{itemize}

\subsection{REINFORCE的直观理解}

\textbf{更新规则}:
\begin{equation}
\theta_{t+1} = \theta_t + \alpha G_t \nabla_\theta \ln \pi(A_t|S_t, \theta_t)
\end{equation}

\textbf{解释}:
\begin{itemize}
    \item $G_t$:回报越大,参数更新越大(鼓励高回报动作)
    \item $\nabla_\theta \ln \pi(A_t|S_t, \theta_t)$:增加在状态 $S_t$ 选择动作 $A_t$ 的概率的方向
    \item 除以 $\pi(A_t|S_t, \theta_t)$:防止频繁选择的动作获得不公平优势
\end{itemize}

\section{带基线的REINFORCE}

\subsection{带基线的策略梯度定理}

策略梯度定理可以推广到包含基线 $b(s)$:

\begin{equation}
\nabla_\theta J(\theta) \propto \sum_{s} \mu(s) \sum_{a} [q_\pi(s, a) - b(s)] \nabla_\theta \pi(a|s, \theta)
\label{eq:policy_gradient_baseline}
\end{equation}

\textbf{基线的条件}:
\begin{itemize}
    \item 基线可以是任何函数,甚至是随机变量
    \item 只要基线不依赖于动作 $a$,等式就成立
    \item 因为:$\sum_{a} b(s) \nabla_\theta \pi(a|s, \theta) = b(s) \nabla_\theta \sum_{a} \pi(a|s, \theta) = b(s) \nabla_\theta 1 = 0$
\end{itemize}

\subsection{带基线的REINFORCE更新}

\begin{equation}
\theta_{t+1} = \theta_t + \alpha [G_t - b(S_t)] \frac{\nabla_\theta \pi(A_t|S_t, \theta_t)}{\pi(A_t|S_t, \theta_t)}
\label{eq:reinforce_baseline}
\end{equation}

\textbf{等价形式}:
\begin{equation}
\theta_{t+1} = \theta_t + \alpha [G_t - b(S_t)] \nabla_\theta \ln \pi(A_t|S_t, \theta_t)
\end{equation}

\textbf{基线的效果}:
\begin{itemize}
    \item \textbf{期望值不变}:基线不改变更新的期望值
    \item \textbf{减少方差}:合适的基线可以显著减少方差,加速学习
    \item \textbf{状态相关}:基线应该随状态变化
\end{itemize}

\subsection{使用状态价值函数作为基线}

\textbf{自然选择}:使用状态价值函数的估计 $\hat{v}(S_t, w)$ 作为基线

\begin{equation}
b(S_t) = \hat{v}(S_t, w)
\end{equation}

其中 $w \in \mathbb{R}^d$ 是价值函数的权重向量。

\subsection{带基线的REINFORCE算法}

\begin{algorithm}[H]
\caption{带基线的REINFORCE(回合制)}
\begin{algorithmic}[1]
\REQUIRE 可微策略参数化 $\pi(a|s, \theta)$
\REQUIRE 可微状态价值函数参数化 $\hat{v}(s, w)$
\ENSURE 最优策略 $\pi_*$
\STATE \textbf{算法参数}:步长 $\alpha_\theta > 0$,$\alpha_w > 0$
\STATE \textbf{初始化}:策略参数 $\theta$ 和状态价值权重 $w$(如设为0)
\REPEAT
    \STATE 生成回合:$S_0, A_0, R_1, \ldots, S_{T-1}, A_{T-1}, R_T$,遵循 $\pi(\cdot|\cdot, \theta)$
    \FOR{回合的每一步 $t = 0, 1, \ldots, T-1$}
        \STATE $G \gets \sum_{k=t+1}^{T} \gamma^{k-t-1} R_k$ \COMMENT{计算回报 $G_t$}
        \STATE $\delta \gets G - \hat{v}(S_t, w)$ \COMMENT{TD误差}
        \STATE $w \gets w + \alpha_w \delta \nabla_w \hat{v}(S_t, w)$ \COMMENT{更新价值函数}
        \STATE $\theta \gets \theta + \alpha_\theta \gamma^t \delta \nabla_\theta \ln \pi(A_t|S_t, \theta)$ \COMMENT{更新策略}
    \ENDFOR
\UNTIL{收敛}
\RETURN $\pi_* = \pi(\cdot|\cdot, \theta)$
\end{algorithmic}
\end{algorithm}

\textbf{关键点}:
\begin{itemize}
    \item 同时学习策略和价值函数
    \item 价值函数仅用作基线,不用于自举
    \item 使用蒙特卡洛方法学习价值函数
\end{itemize}

\section{Actor-Critic方法}

\subsection{什么是Actor-Critic?}

\textbf{定义}:
\begin{itemize}
    \item \textbf{Actor}:学习的策略(演员)
    \item \textbf{Critic}:学习的价值函数(评论家)
    \item \textbf{关键区别}:Critic 用于\textbf{自举}(bootstrapping),而不仅仅是基线
\end{itemize}

\textbf{REINFORCE with Baseline vs Actor-Critic}:
\begin{itemize}
    \item \textbf{REINFORCE with Baseline}:价值函数仅用作基线,不使用自举
    \item \textbf{Actor-Critic}:价值函数用于自举,引入偏差但减少方差
\end{itemize}

\subsection{一步Actor-Critic}

\textbf{更新规则}:
\begin{equation}
\theta_{t+1} = \theta_t + \alpha [R_{t+1} + \gamma \hat{v}(S_{t+1}, w) - \hat{v}(S_t, w)] \nabla_\theta \ln \pi(A_t|S_t, \theta_t)
\label{eq:one_step_actor_critic}
\end{equation}

\textbf{TD误差}:
\begin{equation}
\delta_t = R_{t+1} + \gamma \hat{v}(S_{t+1}, w) - \hat{v}(S_t, w)
\end{equation}

\textbf{更新规则(使用TD误差)}:
\begin{equation}
\theta_{t+1} = \theta_t + \alpha \delta_t \nabla_\theta \ln \pi(A_t|S_t, \theta_t)
\end{equation}

\subsection{一步Actor-Critic算法}

\begin{algorithm}[H]
\caption{一步Actor-Critic(回合制)}
\begin{algorithmic}[1]
\REQUIRE 可微策略参数化 $\pi(a|s, \theta)$
\REQUIRE 可微状态价值函数参数化 $\hat{v}(s, w)$
\ENSURE 最优策略 $\pi_*$
\STATE \textbf{算法参数}:步长 $\alpha_\theta > 0$,$\alpha_w > 0$
\STATE \textbf{初始化}:策略参数 $\theta$ 和状态价值权重 $w$(如设为0)
\REPEAT
    \STATE 初始化 $S$(回合的起始状态)
    \STATE $I \gets 1$ \COMMENT{折扣因子修正}
    \WHILE{$S$ 不是终止状态}
        \STATE $A \sim \pi(\cdot|S, \theta)$ \COMMENT{根据策略选择动作}
        \STATE 执行动作 $A$,观察 $S'$,$R$
        \STATE $\delta \gets R + \gamma \hat{v}(S', w) - \hat{v}(S, w)$ \COMMENT{TD误差}
        \IF{$S'$ 是终止状态}
            \STATE $\hat{v}(S', w) \gets 0$
        \ENDIF
        \STATE $w \gets w + \alpha_w \delta \nabla_w \hat{v}(S, w)$ \COMMENT{更新Critic}
        \STATE $\theta \gets \theta + \alpha_\theta I \delta \nabla_\theta \ln \pi(A|S, \theta)$ \COMMENT{更新Actor}
        \STATE $I \gets \gamma I$
        \STATE $S \gets S'$
    \ENDWHILE
\UNTIL{收敛}
\RETURN $\pi_* = \pi(\cdot|\cdot, \theta)$
\end{algorithmic}
\end{algorithm}

\textbf{特点}:
\begin{itemize}
    \item \textbf{完全在线}:每个时间步都可以更新
    \item \textbf{增量式}:不需要等待回合结束
    \item \textbf{使用自举}:减少方差,但引入偏差
\end{itemize}

\subsection{优势函数(Advantage Function)}

\textbf{定义}:
\begin{equation}
A_\pi(s, a) = q_\pi(s, a) - v_\pi(s)
\label{eq:advantage_function}
\end{equation}

\textbf{解释}:
\begin{itemize}
    \item 优势函数表示动作 $a$ 相对于平均动作的"优势"
    \item 如果 $A_\pi(s, a) > 0$,则动作 $a$ 比平均动作好
    \item 如果 $A_\pi(s, a) < 0$,则动作 $a$ 比平均动作差
\end{itemize}

\textbf{使用优势函数的策略梯度}:
\begin{equation}
\nabla_\theta J(\theta) \propto \sum_{s} \mu(s) \sum_{a} A_\pi(s, a) \nabla_\theta \pi(a|s, \theta)
\end{equation}

\textbf{TD误差作为优势估计}:
\begin{equation}
\delta_t = R_{t+1} + \gamma v_\pi(S_{t+1}) - v_\pi(S_t) \approx A_\pi(S_t, A_t)
\end{equation}

\subsection{带资格迹的Actor-Critic}

\textbf{推广到 $n$ 步和 $\lambda$-回报}:
\begin{itemize}
    \item 一步回报:$G_{t:t+1} = R_{t+1} + \gamma \hat{v}(S_{t+1}, w)$
    \item $n$ 步回报:$G_{t:t+n}$
    \item $\lambda$-回报:$G_t^\lambda$
\end{itemize}

\textbf{后向视图}:使用资格迹

\begin{algorithm}[H]
\caption{带资格迹的Actor-Critic(回合制)}
\begin{algorithmic}[1]
\REQUIRE 可微策略参数化 $\pi(a|s, \theta)$
\REQUIRE 可微状态价值函数参数化 $\hat{v}(s, w)$
\ENSURE 最优策略 $\pi_*$
\STATE \textbf{算法参数}:资格迹衰减率 $\lambda_\theta \in [0, 1]$,$\lambda_w \in [0, 1]$;步长 $\alpha_\theta > 0$,$\alpha_w > 0$
\STATE \textbf{初始化}:策略参数 $\theta$ 和状态价值权重 $w$(如设为0)
\REPEAT
    \STATE 初始化 $S$(回合的起始状态)
    \STATE $z_\theta \gets 0$($d'$ 维资格迹向量)
    \STATE $z_w \gets 0$($d$ 维资格迹向量)
    \STATE $I \gets 1$
    \WHILE{$S$ 不是终止状态}
        \STATE $A \sim \pi(\cdot|S, \theta)$
        \STATE 执行动作 $A$,观察 $S'$,$R$
        \STATE $\delta \gets R + \gamma \hat{v}(S', w) - \hat{v}(S, w)$
        \IF{$S'$ 是终止状态}
            \STATE $\hat{v}(S', w) \gets 0$
        \ENDIF
        \STATE $z_w \gets \gamma \lambda_w z_w + \nabla_w \hat{v}(S, w)$ \COMMENT{更新Critic资格迹}
        \STATE $z_\theta \gets \gamma \lambda_\theta z_\theta + I \nabla_\theta \ln \pi(A|S, \theta)$ \COMMENT{更新Actor资格迹}
        \STATE $w \gets w + \alpha_w \delta z_w$ \COMMENT{更新Critic}
        \STATE $\theta \gets \theta + \alpha_\theta \delta z_\theta$ \COMMENT{更新Actor}
        \STATE $I \gets \gamma I$
        \STATE $S \gets S'$
    \ENDWHILE
\UNTIL{收敛}
\RETURN $\pi_* = \pi(\cdot|\cdot, \theta)$
\end{algorithmic}
\end{algorithm}

\section{持续问题的策略梯度}

\subsection{性能指标}

对于持续问题(没有回合边界),性能定义为平均奖励率:

\begin{equation}
J(\theta) = r(\pi) = \lim_{h \to \infty} \frac{1}{h} \sum_{t=1}^{h} \mathbb{E}[R_t | S_0, A_{0:t-1} \sim \pi]
\end{equation}

\subsection{持续问题的策略梯度定理}

\begin{theorem}[持续问题的策略梯度定理]
对于持续问题,性能指标的梯度为:
\begin{equation}
\nabla_\theta J(\theta) = \sum_{s} \mu(s) \sum_{a} [q_\pi(s, a) - v_\pi(s)] \nabla_\theta \pi(a|s, \theta)
\end{equation}

其中 $\mu(s)$ 是在策略 $\pi$ 下的稳态分布。
\end{theorem}

\textbf{关键区别}:
\begin{itemize}
    \item 回合制:$\nabla_\theta J(\theta) \propto \sum_{s} \mu(s) \sum_{a} q_\pi(s, a) \nabla_\theta \pi(a|s, \theta)$
    \item 持续问题:$\nabla_\theta J(\theta) = \sum_{s} \mu(s) \sum_{a} A_\pi(s, a) \nabla_\theta \pi(a|s, \theta)$
    \item 持续问题中使用优势函数,而不是动作价值函数
\end{itemize}

\section{连续动作的策略参数化}

\subsection{问题}

对于连续动作空间,不能使用 soft-max 分布。需要其他参数化方式。

\subsection{高斯策略}

\textbf{高斯策略}:
\begin{equation}
\pi(a|s, \theta) = \frac{1}{\sigma(s, \theta) \sqrt{2\pi}} \exp\left(-\frac{(a - \mu(s, \theta))^2}{2\sigma(s, \theta)^2}\right)
\end{equation}

其中:
\begin{itemize}
    \item $\mu(s, \theta)$:均值(可以是状态的函数)
    \item $\sigma(s, \theta)$:标准差(可以是状态的函数或常数)
\end{itemize}

\textbf{参数化}:
\begin{itemize}
    \item 均值:$\mu(s, \theta) = \theta_\mu^T x_\mu(s)$(线性)或神经网络
    \item 标准差:$\sigma(s, \theta) = \exp(\theta_\sigma^T x_\sigma(s))$(保证为正)
\end{itemize}

\subsection{其他连续动作分布}

\begin{itemize}
    \item \textbf{Beta分布}:适用于有界动作空间
    \item \textbf{混合分布}:多个高斯分布的混合
    \item \textbf{确定性策略}:$\mu(s, \theta)$ 直接作为动作(如 DDPG)
\end{itemize}

\section{核心公式总结}

\subsection{策略梯度定理}

\textbf{回合制}:
\begin{equation}
\nabla_\theta J(\theta) \propto \sum_{s} \mu(s) \sum_{a} q_\pi(s, a) \nabla_\theta \pi(a|s, \theta)
\end{equation}

\textbf{持续问题}:
\begin{equation}
\nabla_\theta J(\theta) = \sum_{s} \mu(s) \sum_{a} A_\pi(s, a) \nabla_\theta \pi(a|s, \theta)
\end{equation}

\subsection{REINFORCE}

\begin{equation}
\theta_{t+1} = \theta_t + \alpha G_t \nabla_\theta \ln \pi(A_t|S_t, \theta_t)
\end{equation}

\subsection{带基线的REINFORCE}

\begin{equation}
\theta_{t+1} = \theta_t + \alpha [G_t - b(S_t)] \nabla_\theta \ln \pi(A_t|S_t, \theta_t)
\end{equation}

\subsection{一步Actor-Critic}

\begin{equation}
\theta_{t+1} = \theta_t + \alpha \delta_t \nabla_\theta \ln \pi(A_t|S_t, \theta_t)
\end{equation}

其中:
\begin{equation}
\delta_t = R_{t+1} + \gamma \hat{v}(S_{t+1}, w) - \hat{v}(S_t, w)
\end{equation}

\subsection{优势函数}

\begin{equation}
A_\pi(s, a) = q_\pi(s, a) - v_\pi(s)
\end{equation}

\section{方法对比}

\begin{table}[H]
\centering
\caption{策略梯度方法对比}
\begin{tabular}{|l|l|l|l|}
\hline
\textbf{方法} & \textbf{更新规则} & \textbf{偏差} & \textbf{方差} \\
\hline
\textbf{REINFORCE} & $G_t \nabla_\theta \ln \pi$ & 无偏 & 高 \\
\hline
\textbf{REINFORCE+Baseline} & $[G_t - b(S_t)] \nabla_\theta \ln \pi$ & 无偏 & 中等 \\
\hline
\textbf{Actor-Critic} & $\delta_t \nabla_\theta \ln \pi$ & 有偏 & 低 \\
\hline
\end{tabular}
\end{table}

\section{总结}

\subsection{核心要点}

\begin{enumerate}
    \item \textbf{策略梯度方法}直接学习参数化策略,无需价值函数即可选择动作
    
    \item \textbf{策略梯度定理}提供了计算性能梯度的解析表达式,不涉及状态分布的导数
    
    \item \textbf{REINFORCE}是蒙特卡洛策略梯度方法,使用完整回报,无偏但方差大
    
    \item \textbf{基线}可以减少方差而不改变期望值,常用状态价值函数作为基线
    
    \item \textbf{Actor-Critic}方法同时学习策略和价值函数,使用自举减少方差但引入偏差
    
    \item \textbf{优势函数} $A_\pi(s, a) = q_\pi(s, a) - v_\pi(s)$ 是策略梯度的自然形式
\end{enumerate}

\subsection{关键优势}

\begin{itemize}
    \item 可以学习随机最优策略
    \item 可以接近确定性策略
    \item 策略可能比价值函数更简单
    \item 可以注入先验知识
    \item 更强的收敛保证
\end{itemize}

\subsection{应用}

策略梯度方法是现代深度强化学习的基础,包括:
\begin{itemize}
    \item PPO(Proximal Policy Optimization)
    \item TRPO(Trust Region Policy Optimization)
    \item A3C(Asynchronous Advantage Actor-Critic)
    \item SAC(Soft Actor-Critic)
    \item DDPG(Deep Deterministic Policy Gradient)
\end{itemize}

\end{document}

