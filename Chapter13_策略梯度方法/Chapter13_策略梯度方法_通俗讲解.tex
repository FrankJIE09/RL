\documentclass[12pt,a4paper]{article}
\usepackage[UTF8]{ctex}
\usepackage{amsmath}
\usepackage{amssymb}
\usepackage{amsthm}
\usepackage{geometry}
\usepackage{hyperref}
\usepackage{enumitem}
\usepackage{booktabs}

\geometry{left=2.5cm,right=2.5cm,top=2.5cm,bottom=2.5cm}

\title{第13章:策略梯度方法通俗讲解}
\subtitle{用生活例子理解策略梯度}
\author{强化学习笔记}
\date{\today}

\begin{document}

\maketitle

\tableofcontents
\newpage

\section{引言}

策略梯度方法是强化学习中的一类重要方法。与之前学习动作价值函数的方法不同,策略梯度方法直接学习"怎么做"(策略),而不是"做这个动作值多少钱"(动作价值)。

\textbf{核心区别}:
\begin{itemize}
    \item \textbf{之前的方法}:先学习"每个动作值多少钱",然后选择最值钱的动作
    \item \textbf{策略梯度方法}:直接学习"应该怎么做",不需要知道每个动作值多少钱
\end{itemize}

\section{什么是策略梯度方法?}

\subsection{通俗理解}

\textbf{策略梯度方法就像}:

\begin{quote}
\textbf{"直接学习怎么做,而不是先学习值多少钱"}
\end{quote}

\textbf{生活例子:学做菜}

\textbf{方法1:先学价值再选择(之前的方法)}
\begin{enumerate}
    \item 你学习"放一勺盐值多少钱"(动作价值)
    \item 你学习"放两勺盐值多少钱"(动作价值)
    \item 你学习"放三勺盐值多少钱"(动作价值)
    \item 你选择"最值钱的动作"(如放两勺盐)
\end{enumerate}

\textbf{方法2:直接学习策略(策略梯度方法)}
\begin{enumerate}
    \item 你直接学习"应该放多少盐"(策略)
    \item 你不需要知道"放一勺盐值多少钱"
    \item 你直接学习"在什么情况下放多少盐"
    \item 你按照学到的策略来做菜
\end{enumerate}

\subsection{核心思想}

\textbf{策略梯度方法的核心}:
\begin{itemize}
    \item 直接学习参数化的策略 $\pi(a|s, \theta)$
    \item 通过梯度上升优化性能指标 $J(\theta)$
    \item 使用策略梯度定理计算梯度
    \item 可以学习随机最优策略
\end{itemize}

\textbf{类比}:
\begin{itemize}
    \item \textbf{策略} $\pi(a|s, \theta)$:你的"做事方式"(参数化)
    \item \textbf{性能指标} $J(\theta)$:你的"做事效果"(好坏程度)
    \item \textbf{梯度上升}:朝着"效果更好"的方向调整"做事方式"
    \item \textbf{策略梯度定理}:告诉你"如何调整做事方式"才能"效果更好"
\end{itemize}

\section{策略参数化:把你的做事方式写成公式}

\subsection{通俗理解}

\textbf{策略参数化就像}:

\begin{quote}
\textbf{"把你的做事方式写成数学公式,可以调整参数来改变做事方式"}
\end{quote}

\textbf{生活例子:学做菜}

\textbf{策略参数化}:
\begin{itemize}
    \item 你的做菜策略:$\pi(\text{放盐}|s, \theta) = \text{在状态 } s \text{ 放盐的概率}$
    \item 参数 $\theta$:控制你放盐的"偏好"
    \item 调整 $\theta$:改变你放盐的方式
\end{itemize}

\textbf{Soft-max动作偏好}:
\begin{equation}
\pi(a|s, \theta) = \frac{e^{h(s, a, \theta)}}{\sum_{b} e^{h(s, b, \theta)}}
\end{equation}

\textbf{通俗解释}:
\begin{itemize}
    \item $h(s, a, \theta)$:你对动作 $a$ 的"偏好分数"
    \item 偏好分数越高,选择这个动作的概率越大
    \item Soft-max:把偏好分数转换成概率(所有动作的概率加起来等于1)
\end{itemize}

\textbf{类比}:
\begin{itemize}
    \item 就像你给每个动作打分,分数高的动作更容易被选择
    \item 但所有动作的概率加起来必须等于1(你总要选择一个动作)
\end{itemize}

\subsection{策略参数化的优势}

\textbf{1. 可以接近确定性策略}

\textbf{通俗理解}:
\begin{itemize}
    \item 你可以学习"几乎总是选择某个动作"的策略
    \item 就像你学会了"几乎总是放两勺盐"
    \item 而 $\varepsilon$-贪婪方法总是有 $\varepsilon$ 概率选择随机动作
\end{itemize}

\textbf{2. 可以学习随机最优策略}

\textbf{通俗理解}:
\begin{itemize}
    \item 在某些情况下,最优策略可能是随机的
    \item 就像在扑克中,有时需要"虚张声势"(随机选择)
    \item 策略梯度方法可以学习这种随机策略
    \item 而动作价值方法无法自然地找到随机最优策略
\end{itemize}

\textbf{3. 策略可能更简单}

\textbf{通俗理解}:
\begin{itemize}
    \item 对于某些问题,策略函数比动作价值函数更简单
    \item 就像"怎么做"比"每个动作值多少钱"更容易描述
    \item 策略方法可能学习更快,得到更好的结果
\end{itemize}

\section{策略梯度定理:如何调整做事方式?}

\subsection{问题:如何调整参数?}

\textbf{问题}:
\begin{itemize}
    \item 你有一个策略 $\pi(a|s, \theta)$(你的做事方式)
    \item 你想调整参数 $\theta$,使性能 $J(\theta)$ 更好(效果更好)
    \item 但如何调整?调整多少?
\end{itemize}

\textbf{策略梯度定理}:
\begin{equation}
\nabla_\theta J(\theta) \propto \sum_{s} \mu(s) \sum_{a} q_\pi(s, a) \nabla_\theta \pi(a|s, \theta)
\end{equation}

\textbf{通俗解释}:
\begin{itemize}
    \item $\nabla_\theta J(\theta)$:性能对参数的梯度(如何调整参数)
    \item $q_\pi(s, a)$:动作价值(这个动作值多少钱)
    \item $\nabla_\theta \pi(a|s, \theta)$:策略对参数的梯度(如何调整策略)
    \item 定理告诉我们:朝着"值钱的动作"的方向调整策略
\end{itemize}

\textbf{类比:学做菜}
\begin{itemize}
    \item 你发现"放两勺盐"效果很好(值钱)
    \item 你朝着"增加放两勺盐的概率"的方向调整参数
    \item 你调整参数,使"放两勺盐"的概率增加
\end{itemize}

\subsection{关键洞察}

\textbf{策略梯度定理的关键}:
\begin{itemize}
    \item 梯度不涉及状态分布的导数
    \item 只需要知道策略参数化和动作价值函数
    \item 这使得我们可以估计性能梯度,即使不知道策略变化对状态分布的影响
\end{itemize}

\textbf{通俗理解}:
\begin{itemize}
    \item 你不需要知道"改变做事方式会影响哪些状态"
    \item 你只需要知道"这个动作值多少钱"和"如何调整策略"
    \item 定理告诉你如何调整,使"值钱的动作"更容易被选择
\end{itemize}

\section{REINFORCE:最简单的策略梯度方法}

\subsection{通俗理解}

\textbf{REINFORCE就像}:

\begin{quote}
\textbf{"做完一件事后,根据结果好坏来调整做事方式"}
\end{quote}

\textbf{生活例子:学做菜}

\textbf{过程}:
\begin{enumerate}
    \item 你按照当前策略做菜(如"放两勺盐")
    \item 你吃完菜,评价结果(如"太咸了"或"正好")
    \item 如果结果好,你增加"放两勺盐"的概率
    \item 如果结果不好,你减少"放两勺盐"的概率
\end{enumerate}

\textbf{REINFORCE更新规则}:
\begin{equation}
\theta_{t+1} = \theta_t + \alpha G_t \nabla_\theta \ln \pi(A_t|S_t, \theta_t)
\end{equation}

\textbf{通俗解释}:
\begin{itemize}
    \item $G_t$:回报(结果好坏)
    \item $\nabla_\theta \ln \pi(A_t|S_t, \theta_t)$:增加选择动作 $A_t$ 的概率的方向
    \item 如果 $G_t$ 大(结果好),参数朝着"增加选择 $A_t$ 的概率"的方向调整
    \item 如果 $G_t$ 小(结果不好),参数朝着"减少选择 $A_t$ 的概率"的方向调整
\end{itemize}

\subsection{REINFORCE的特点}

\textbf{优点}:
\begin{itemize}
    \item \textbf{简单直观}:结果好就增加概率,结果不好就减少概率
    \item \textbf{无偏估计}:使用完整回报,不引入偏差
    \item \textbf{理论收敛}:期望更新方向与性能梯度方向相同
\end{itemize}

\textbf{缺点}:
\begin{itemize}
    \item \textbf{高方差}:结果可能波动很大,导致学习不稳定
    \item \textbf{学习慢}:高方差导致学习速度慢
    \item \textbf{需要等待回合结束}:必须做完整个任务才能更新
\end{itemize}

\textbf{类比}:
\begin{itemize}
    \item 就像你每次做完菜才评价,然后调整
    \item 如果每次做菜的结果波动很大,你很难判断"放两勺盐"是好是坏
    \item 你需要做很多次菜,才能稳定地学习
\end{itemize}

\section{带基线的REINFORCE:减少波动}

\subsection{问题:方差太大}

\textbf{问题}:
\begin{itemize}
    \item REINFORCE的方差很大,导致学习不稳定
    \item 就像你每次做菜的结果波动很大,很难判断"放两勺盐"是好是坏
\end{itemize}

\textbf{解决方案:使用基线}

\textbf{带基线的REINFORCE更新}:
\begin{equation}
\theta_{t+1} = \theta_t + \alpha [G_t - b(S_t)] \nabla_\theta \ln \pi(A_t|S_t, \theta_t)
\end{equation}

\textbf{通俗解释}:
\begin{itemize}
    \item $G_t$:回报(结果好坏)
    \item $b(S_t)$:基线(这个状态的"平均结果")
    \item $G_t - b(S_t)$:相对于平均结果的"好坏"
    \item 如果 $G_t > b(S_t)$(比平均好),增加概率
    \item 如果 $G_t < b(S_t)$(比平均差),减少概率
\end{itemize}

\textbf{类比:学做菜}
\begin{itemize}
    \item 你发现"放两勺盐"的结果是8分($G_t = 8$)
    \item 这个状态的"平均结果"是6分($b(S_t) = 6$)
    \item 你判断"放两勺盐"比平均好($8 - 6 = 2$),增加概率
    \item 这比直接使用8分更稳定(减少了波动)
\end{itemize}

\subsection{基线的效果}

\textbf{基线的特点}:
\begin{itemize}
    \item \textbf{不改变期望值}:基线的期望值不影响更新的期望方向
    \item \textbf{减少方差}:合适的基线可以显著减少方差,加速学习
    \item \textbf{状态相关}:基线应该随状态变化
\end{itemize}

\textbf{使用状态价值函数作为基线}:
\begin{equation}
b(S_t) = \hat{v}(S_t, w)
\end{equation}

\textbf{通俗理解}:
\begin{itemize}
    \item 使用"这个状态的平均价值"作为基线
    \item 就像使用"这个状态的平均结果"作为参考
    \item 判断动作是"比平均好"还是"比平均差"
\end{itemize}

\section{Actor-Critic:演员和评论家}

\subsection{什么是Actor-Critic?}

\textbf{Actor-Critic就像}:

\begin{quote}
\textbf{"演员(Actor)按照策略表演,评论家(Critic)评价好坏,演员根据评价调整表演方式"}
\end{quote}

\textbf{生活例子:学做菜}

\textbf{Actor(演员)}:
\begin{itemize}
    \item 你的做菜策略(如何做菜)
    \item 你按照策略做菜(如"放两勺盐")
\end{itemize}

\textbf{Critic(评论家)}:
\begin{itemize}
    \item 价值函数(评价好坏)
    \item 评价"这个状态值多少钱"(如"这个状态值6分")
\end{itemize}

\textbf{交互过程}:
\begin{enumerate}
    \item Actor按照策略做菜(放两勺盐)
    \item Critic评价结果(这个状态值6分)
    \item Actor根据评价调整策略(如果评价好,增加放两勺盐的概率)
\end{enumerate}

\subsection{Actor-Critic vs REINFORCE with Baseline}

\textbf{关键区别}:

\textbf{REINFORCE with Baseline}:
\begin{itemize}
    \item 价值函数仅用作基线,不使用自举
    \item 就像你使用"平均结果"作为参考,但不预测未来
    \item 必须等待回合结束才能更新
\end{itemize}

\textbf{Actor-Critic}:
\begin{itemize}
    \item 价值函数用于自举,引入偏差但减少方差
    \item 就像你使用"预测的未来结果"来评价当前动作
    \item 可以立即更新,不需要等待回合结束
\end{itemize}

\textbf{一步Actor-Critic更新}:
\begin{equation}
\theta_{t+1} = \theta_t + \alpha \delta_t \nabla_\theta \ln \pi(A_t|S_t, \theta_t)
\end{equation}

其中:
\begin{equation}
\delta_t = R_{t+1} + \gamma \hat{v}(S_{t+1}, w) - \hat{v}(S_t, w)
\end{equation}

\textbf{通俗解释}:
\begin{itemize}
    \item $\delta_t$:TD误差(预测误差)
    \item $R_{t+1} + \gamma \hat{v}(S_{t+1}, w)$:预测的回报(立即奖励 + 未来价值)
    \item $\hat{v}(S_t, w)$:当前状态的价值
    \item 如果预测比当前价值高,说明动作好,增加概率
    \item 如果预测比当前价值低,说明动作差,减少概率
\end{itemize}

\subsection{优势函数:动作相对于平均的优势}

\textbf{优势函数定义}:
\begin{equation}
A_\pi(s, a) = q_\pi(s, a) - v_\pi(s)
\end{equation}

\textbf{通俗理解}:
\begin{itemize}
    \item $q_\pi(s, a)$:动作 $a$ 的价值(这个动作值多少钱)
    \item $v_\pi(s)$:状态 $s$ 的平均价值(这个状态的平均值)
    \item $A_\pi(s, a)$:动作 $a$ 相对于平均的"优势"
    \item 如果 $A_\pi(s, a) > 0$,动作比平均好
    \item 如果 $A_\pi(s, a) < 0$,动作比平均差
\end{itemize}

\textbf{类比:学做菜}
\begin{itemize}
    \item "放两勺盐"的价值是8分($q_\pi = 8$)
    \item 这个状态的平均价值是6分($v_\pi = 6$)
    \item "放两勺盐"的优势是 $8 - 6 = 2$(比平均好2分)
    \item 你增加"放两勺盐"的概率
\end{itemize}

\textbf{TD误差作为优势估计}:
\begin{equation}
\delta_t = R_{t+1} + \gamma v_\pi(S_{t+1}) - v_\pi(S_t) \approx A_\pi(S_t, A_t)
\end{equation}

\textbf{通俗理解}:
\begin{itemize}
    \item TD误差可以近似表示优势
    \item 如果TD误差大,说明动作比预期好
    \item 如果TD误差小,说明动作比预期差
\end{itemize}

\section{完整例子:学做菜的策略梯度}

\subsection{场景设置}

\textbf{场景}:你正在学习如何做一道菜,需要决定放多少盐。

\textbf{状态}:
\begin{itemize}
    \item $s_1$:菜还没放盐
    \item $s_2$:菜已经放了一勺盐
    \item $s_3$:菜已经放了两勺盐
\end{itemize}

\textbf{动作}:
\begin{itemize}
    \item $a_1$:放一勺盐
    \item $a_2$:放两勺盐
    \item $a_3$:放三勺盐
\end{itemize}

\textbf{策略}:$\pi(a|s, \theta)$(在状态 $s$ 选择动作 $a$ 的概率)

\subsection{REINFORCE过程}

\textbf{第1次做菜}:
\begin{enumerate}
    \item 在状态 $s_1$,你按照策略选择动作 $a_2$(放两勺盐)
    \item 你做完菜,评价结果:$G_1 = 8$(8分,结果很好)
    \item 你更新策略:增加"在状态 $s_1$ 放两勺盐"的概率
    \begin{equation}
    \theta \gets \theta + \alpha \times 8 \times \nabla_\theta \ln \pi(a_2|s_1, \theta)
    \end{equation}
    \item 因为结果好(8分),你增加"放两勺盐"的概率
\end{enumerate}

\textbf{第2次做菜}:
\begin{enumerate}
    \item 在状态 $s_1$,你按照更新后的策略选择动作 $a_2$(放两勺盐,概率增加了)
    \item 你做完菜,评价结果:$G_2 = 7$(7分,结果还可以)
    \item 你更新策略:稍微增加"在状态 $s_1$ 放两勺盐"的概率
    \begin{equation}
    \theta \gets \theta + \alpha \times 7 \times \nabla_\theta \ln \pi(a_2|s_1, \theta)
    \end{equation}
\end{enumerate}

\textbf{继续学习}:
\begin{itemize}
    \item 你继续做菜,根据结果调整策略
    \item 如果"放两勺盐"的结果好,你增加概率
    \item 如果"放两勺盐"的结果不好,你减少概率
    \item 最终,你学会了"在状态 $s_1$ 应该放两勺盐"
\end{itemize}

\subsection{带基线的REINFORCE过程}

\textbf{第1次做菜}:
\begin{enumerate}
    \item 在状态 $s_1$,你按照策略选择动作 $a_2$(放两勺盐)
    \item 你做完菜,评价结果:$G_1 = 8$(8分)
    \item 你估计状态 $s_1$ 的平均价值:$\hat{v}(s_1, w) = 6$(平均6分)
    \item 你计算优势:$8 - 6 = 2$(比平均好2分)
    \item 你更新策略:增加"在状态 $s_1$ 放两勺盐"的概率
    \begin{equation}
    \theta \gets \theta + \alpha \times (8 - 6) \times \nabla_\theta \ln \pi(a_2|s_1, \theta)
    \end{equation}
    \item 因为比平均好(+2),你增加"放两勺盐"的概率
\end{enumerate}

\textbf{优势}:
\begin{itemize}
    \item 使用基线减少了波动
    \item 即使结果波动大,你也能稳定地学习
    \item 你判断的是"比平均好多少",而不是"绝对好坏"
\end{itemize}

\subsection{Actor-Critic过程}

\textbf{第1次做菜}:
\begin{enumerate}
    \item 在状态 $s_1$,Actor按照策略选择动作 $a_2$(放两勺盐)
    \item 你执行动作,转移到状态 $s_2$,获得奖励 $R = 2$
    \item Critic评价:$\hat{v}(s_1, w) = 6$,$\hat{v}(s_2, w) = 7$
    \item 计算TD误差:$\delta = 2 + 0.9 \times 7 - 6 = 2.3$(预测比当前价值高)
    \item Actor更新策略:增加"在状态 $s_1$ 放两勺盐"的概率
    \begin{equation}
    \theta \gets \theta + \alpha \times 2.3 \times \nabla_\theta \ln \pi(a_2|s_1, \theta)
    \end{equation}
    \item 因为TD误差大(预测好),你增加"放两勺盐"的概率
\end{enumerate}

\textbf{优势}:
\begin{itemize}
    \item 可以立即更新,不需要等待做完整个菜
    \item 使用预测减少方差,学习更快
    \item Actor和Critic同时学习,相互促进
\end{itemize}

\section{方法对比}

\begin{center}
\begin{tabular}{|l|l|l|l|}
\hline
\textbf{方法} & \textbf{通俗理解} & \textbf{优点} & \textbf{缺点} \\
\hline
\textbf{REINFORCE} & 做完后根据结果调整 & 简单直观,无偏 & 方差大,学习慢 \\
\hline
\textbf{REINFORCE+Baseline} & 做完后根据"比平均好多少"调整 & 减少方差 & 仍需等待完成 \\
\hline
\textbf{Actor-Critic} & 边做边根据预测调整 & 可以立即更新,学习快 & 有偏差 \\
\hline
\end{tabular}
\end{center}

\section{总结}

\subsection{核心概念(通俗版)}

\begin{enumerate}
    \item \textbf{策略梯度方法}:直接学习"怎么做",而不是"值多少钱"
    
    \item \textbf{策略参数化}:把你的做事方式写成公式,可以调整参数
    
    \item \textbf{策略梯度定理}:告诉你如何调整参数,使"值钱的动作"更容易被选择
    
    \item \textbf{REINFORCE}:做完后根据结果好坏来调整做事方式
    
    \item \textbf{基线}:使用"平均结果"作为参考,减少波动
    
    \item \textbf{Actor-Critic}:演员按照策略表演,评论家评价好坏,演员根据评价调整
\end{enumerate}

\subsection{关键公式(简化版)}

\textbf{策略梯度定理}:
\begin{equation}
\text{如何调整参数} \propto \text{动作价值} \times \text{如何调整策略}
\end{equation}

\textbf{REINFORCE}:
\begin{equation}
\text{新参数} = \text{旧参数} + \alpha \times \text{回报} \times \text{增加动作概率的方向}
\end{equation}

\textbf{带基线的REINFORCE}:
\begin{equation}
\text{新参数} = \text{旧参数} + \alpha \times (\text{回报} - \text{基线}) \times \text{增加动作概率的方向}
\end{equation}

\textbf{Actor-Critic}:
\begin{equation}
\text{新参数} = \text{旧参数} + \alpha \times \text{TD误差} \times \text{增加动作概率的方向}
\end{equation}

\subsection{记忆技巧}

\begin{itemize}
    \item \textbf{策略梯度}:直接学"怎么做",不学"值多少钱"
    \item \textbf{REINFORCE}:做完后根据结果调整
    \item \textbf{基线}:用"平均结果"作参考,减少波动
    \item \textbf{Actor-Critic}:演员表演,评论家评价,演员调整
    \item \textbf{优势函数}:动作相对于平均的"优势"
\end{itemize}

\end{document}

