\documentclass[12pt,a4paper]{article}
\usepackage[UTF8]{ctex}
\usepackage{amsmath}
\usepackage{amssymb}
\usepackage{amsthm}
\usepackage{geometry}
\usepackage{hyperref}
\usepackage{booktabs}
\usepackage{array}
\usepackage{enumitem}
\usepackage{float}

\geometry{left=2.5cm,right=2.5cm,top=2.5cm,bottom=2.5cm}

\title{第13章:策略梯度方法重要公式总结}
\author{强化学习笔记}
\date{\today}

\newtheorem{definition}{定义}
\newtheorem{theorem}{定理}

\begin{document}

\maketitle

\tableofcontents
\newpage

\section{策略参数化}

\subsection{参数化策略}

\begin{equation}
\pi(a|s, \theta) = \Pr\{A_t = a | S_t = s, \theta_t = \theta\}
\end{equation}

其中 $\theta \in \mathbb{R}^{d'}$ 是策略的参数向量。

\subsection{Soft-max动作偏好参数化}

\begin{equation}
\pi(a|s, \theta) = \frac{e^{h(s, a, \theta)}}{\sum_{b} e^{h(s, b, \theta)}}
\label{eq:softmax_policy}
\end{equation}

其中 $h(s, a, \theta) \in \mathbb{R}$ 是状态-动作对的数值偏好。

\subsubsection{线性偏好}
\begin{equation}
h(s, a, \theta) = \theta^T x(s, a)
\end{equation}

\subsubsection{神经网络偏好}
$h(s, a, \theta)$ 由深度神经网络计算。

\section{性能指标}

\subsection{回合制任务}

\begin{equation}
J(\theta) = v_{\pi_\theta}(s_0)
\label{eq:episodic_performance}
\end{equation}

其中 $s_0$ 是起始状态,$v_{\pi_\theta}$ 是策略 $\pi_\theta$ 的真实价值函数。

\subsection{持续任务}

\begin{equation}
J(\theta) = r(\pi) = \lim_{h \to \infty} \frac{1}{h} \sum_{t=1}^{h} \mathbb{E}[R_t | S_0, A_{0:t-1} \sim \pi]
\end{equation}

\section{策略梯度方法的基本更新公式}

\subsection{公式13.1:基本更新}

\begin{equation}
\theta_{t+1} = \theta_t + \alpha \widehat{\nabla J}(\theta_t)
\label{eq:13.1}
\end{equation}

其中:
\begin{itemize}
    \item $\theta_t \in \mathbb{R}^{d'}$:时刻 $t$ 的策略参数向量
    \item $\alpha > 0$:步长参数(学习率)
    \item $\widehat{\nabla J}(\theta_t) \in \mathbb{R}^{d'}$:性能指标 $J(\theta)$ 关于参数 $\theta_t$ 的梯度的随机估计
    \item $\mathbb{E}[\widehat{\nabla J}(\theta_t)] \approx \nabla J(\theta_t)$:估计的期望近似等于真实梯度
\end{itemize}

\section{策略梯度定理}

\subsection{策略梯度定理(回合制)}

\begin{theorem}[策略梯度定理]
对于回合制任务,性能指标的梯度为:
\begin{equation}
\nabla_\theta J(\theta) \propto \sum_{s} \mu(s) \sum_{a} q_\pi(s, a) \nabla_\theta \pi(a|s, \theta)
\label{eq:policy_gradient_theorem}
\end{equation}

其中:
\begin{itemize}
    \item $\mu(s)$ 是在策略 $\pi$ 下的状态分布(on-policy distribution)
    \item $q_\pi(s, a)$ 是动作价值函数
    \item $\propto$ 表示"正比于"
\end{itemize}
\end{theorem}

\subsection{策略梯度定理(持续问题)}

\begin{theorem}[持续问题的策略梯度定理]
对于持续问题,性能指标的梯度为:
\begin{equation}
\nabla_\theta J(\theta) = \sum_{s} \mu(s) \sum_{a} [q_\pi(s, a) - v_\pi(s)] \nabla_\theta \pi(a|s, \theta)
\end{equation}

其中 $\mu(s)$ 是在策略 $\pi$ 下的稳态分布。
\end{theorem}

\subsection{策略梯度定理的期望形式}

\begin{equation}
\nabla_\theta J(\theta) \propto \mathbb{E}_\pi \left[\sum_{a} q_\pi(S_t, a) \nabla_\theta \pi(a|S_t, \theta)\right]
\label{eq:policy_gradient_expectation}
\end{equation}

\section{REINFORCE:蒙特卡洛策略梯度}

\subsection{REINFORCE梯度推导}

\begin{align}
\nabla_\theta J(\theta) &= \mathbb{E}_\pi \left[\sum_{a} \pi(a|S_t, \theta) q_\pi(S_t, a) \frac{\nabla_\theta \pi(a|S_t, \theta)}{\pi(a|S_t, \theta)}\right] \\
&= \mathbb{E}_\pi \left[q_\pi(S_t, A_t) \frac{\nabla_\theta \pi(A_t|S_t, \theta)}{\pi(A_t|S_t, \theta)}\right] \\
&= \mathbb{E}_\pi \left[G_t \frac{\nabla_\theta \pi(A_t|S_t, \theta)}{\pi(A_t|S_t, \theta)}\right]
\label{eq:reinforce_gradient}
\end{align}

其中使用了 $\mathbb{E}_\pi[G_t | S_t, A_t] = q_\pi(S_t, A_t)$。

\subsection{REINFORCE更新规则}

\begin{equation}
\theta_{t+1} = \theta_t + \alpha G_t \frac{\nabla_\theta \pi(A_t|S_t, \theta_t)}{\pi(A_t|S_t, \theta_t)}
\label{eq:reinforce_update}
\end{equation}

\subsection{REINFORCE更新规则(对数形式)}

\begin{equation}
\theta_{t+1} = \theta_t + \alpha G_t \nabla_\theta \ln \pi(A_t|S_t, \theta_t)
\label{eq:reinforce_log}
\end{equation}

其中使用了恒等式:$\nabla \ln x = \frac{\nabla x}{x}$。

\subsection{资格向量}

\begin{equation}
\nabla_\theta \ln \pi(A_t|S_t, \theta_t)
\end{equation}

这是策略参数化在算法中唯一出现的地方。

\section{带基线的REINFORCE}

\subsection{带基线的策略梯度定理}

策略梯度定理可以推广到包含基线 $b(s)$:

\begin{equation}
\nabla_\theta J(\theta) \propto \sum_{s} \mu(s) \sum_{a} [q_\pi(s, a) - b(s)] \nabla_\theta \pi(a|s, \theta)
\label{eq:policy_gradient_baseline}
\end{equation}

\subsection{基线的条件}

基线可以是任何函数,甚至是随机变量,只要基线不依赖于动作 $a$,等式就成立:

\begin{equation}
\sum_{a} b(s) \nabla_\theta \pi(a|s, \theta) = b(s) \nabla_\theta \sum_{a} \pi(a|s, \theta) = b(s) \nabla_\theta 1 = 0
\end{equation}

\subsection{带基线的REINFORCE更新}

\begin{equation}
\theta_{t+1} = \theta_t + \alpha [G_t - b(S_t)] \frac{\nabla_\theta \pi(A_t|S_t, \theta_t)}{\pi(A_t|S_t, \theta_t)}
\label{eq:reinforce_baseline}
\end{equation}

\subsection{带基线的REINFORCE更新(对数形式)}

\begin{equation}
\theta_{t+1} = \theta_t + \alpha [G_t - b(S_t)] \nabla_\theta \ln \pi(A_t|S_t, \theta_t)
\end{equation}

\subsection{使用状态价值函数作为基线}

\begin{equation}
b(S_t) = \hat{v}(S_t, w)
\end{equation}

其中 $w \in \mathbb{R}^d$ 是价值函数的权重向量。

\section{Actor-Critic方法}

\subsection{一步Actor-Critic更新规则}

\begin{equation}
\theta_{t+1} = \theta_t + \alpha [R_{t+1} + \gamma \hat{v}(S_{t+1}, w) - \hat{v}(S_t, w)] \nabla_\theta \ln \pi(A_t|S_t, \theta_t)
\label{eq:one_step_actor_critic}
\end{equation}

\subsection{TD误差}

\begin{equation}
\delta_t = R_{t+1} + \gamma \hat{v}(S_{t+1}, w) - \hat{v}(S_t, w)
\end{equation}

\subsection{Actor-Critic更新规则(使用TD误差)}

\begin{equation}
\theta_{t+1} = \theta_t + \alpha \delta_t \nabla_\theta \ln \pi(A_t|S_t, \theta_t)
\end{equation}

\subsection{Critic更新规则}

\begin{equation}
w \gets w + \alpha_w \delta_t \nabla_w \hat{v}(S_t, w)
\end{equation}

其中 $\alpha_w > 0$ 是Critic的学习率。

\section{优势函数}

\subsection{优势函数定义}

\begin{equation}
A_\pi(s, a) = q_\pi(s, a) - v_\pi(s)
\label{eq:advantage_function}
\end{equation}

\subsection{使用优势函数的策略梯度}

\begin{equation}
\nabla_\theta J(\theta) \propto \sum_{s} \mu(s) \sum_{a} A_\pi(s, a) \nabla_\theta \pi(a|s, \theta)
\end{equation}

\subsection{TD误差作为优势估计}

\begin{equation}
\delta_t = R_{t+1} + \gamma v_\pi(S_{t+1}) - v_\pi(S_t) \approx A_\pi(S_t, A_t)
\end{equation}

\section{带资格迹的Actor-Critic}

\subsection{Critic资格迹更新}

\begin{equation}
z_w \gets \gamma \lambda_w z_w + \nabla_w \hat{v}(S, w)
\end{equation}

\subsection{Actor资格迹更新}

\begin{equation}
z_\theta \gets \gamma \lambda_\theta z_\theta + I \nabla_\theta \ln \pi(A|S, \theta)
\end{equation}

其中 $I$ 是折扣因子修正项。

\subsection{Critic更新(使用资格迹)}

\begin{equation}
w \gets w + \alpha_w \delta z_w
\end{equation}

\subsection{Actor更新(使用资格迹)}

\begin{equation}
\theta \gets \theta + \alpha_\theta \delta z_\theta
\end{equation}

\section{连续动作的策略参数化}

\subsection{高斯策略}

\begin{equation}
\pi(a|s, \theta) = \frac{1}{\sigma(s, \theta) \sqrt{2\pi}} \exp\left(-\frac{(a - \mu(s, \theta))^2}{2\sigma(s, \theta)^2}\right)
\end{equation}

其中:
\begin{itemize}
    \item $\mu(s, \theta)$:均值(可以是状态的函数)
    \item $\sigma(s, \theta)$:标准差(可以是状态的函数或常数)
\end{itemize}

\subsection{参数化}

\subsubsection{均值参数化}
\begin{equation}
\mu(s, \theta) = \theta_\mu^T x_\mu(s)
\end{equation}

或使用神经网络。

\subsubsection{标准差参数化}
\begin{equation}
\sigma(s, \theta) = \exp(\theta_\sigma^T x_\sigma(s))
\end{equation}

保证标准差为正。

\section{核心公式总结}

\subsection{策略梯度定理}

\textbf{回合制}:
\begin{equation}
\nabla_\theta J(\theta) \propto \sum_{s} \mu(s) \sum_{a} q_\pi(s, a) \nabla_\theta \pi(a|s, \theta)
\end{equation}

\textbf{持续问题}:
\begin{equation}
\nabla_\theta J(\theta) = \sum_{s} \mu(s) \sum_{a} A_\pi(s, a) \nabla_\theta \pi(a|s, \theta)
\end{equation}

\subsection{REINFORCE}

\begin{equation}
\theta_{t+1} = \theta_t + \alpha G_t \nabla_\theta \ln \pi(A_t|S_t, \theta_t)
\end{equation}

\subsection{带基线的REINFORCE}

\begin{equation}
\theta_{t+1} = \theta_t + \alpha [G_t - b(S_t)] \nabla_\theta \ln \pi(A_t|S_t, \theta_t)
\end{equation}

\subsection{一步Actor-Critic}

\begin{equation}
\theta_{t+1} = \theta_t + \alpha \delta_t \nabla_\theta \ln \pi(A_t|S_t, \theta_t)
\end{equation}

其中:
\begin{equation}
\delta_t = R_{t+1} + \gamma \hat{v}(S_{t+1}, w) - \hat{v}(S_t, w)
\end{equation}

\subsection{优势函数}

\begin{equation}
A_\pi(s, a) = q_\pi(s, a) - v_\pi(s)
\end{equation}

\section{方法对比}

\begin{table}[H]
\centering
\caption{策略梯度方法对比}
\begin{tabular}{|l|l|l|l|}
\hline
\textbf{方法} & \textbf{更新规则} & \textbf{偏差} & \textbf{方差} \\
\hline
\textbf{REINFORCE} & $G_t \nabla_\theta \ln \pi$ & 无偏 & 高 \\
\hline
\textbf{REINFORCE+Baseline} & $[G_t - b(S_t)] \nabla_\theta \ln \pi$ & 无偏 & 中等 \\
\hline
\textbf{Actor-Critic} & $\delta_t \nabla_\theta \ln \pi$ & 有偏 & 低 \\
\hline
\end{tabular}
\end{table}

\section{关键关系}

\subsection{价值函数关系}

\textbf{状态价值到动作价值}:
\begin{equation}
v_\pi(s) = \sum_{a} \pi(a|s) q_\pi(s, a)
\end{equation}

\textbf{动作价值到状态价值}:
\begin{equation}
q_\pi(s, a) = \sum_{s', r} p(s', r | s, a) [r + \gamma v_\pi(s')]
\end{equation}

\subsection{策略梯度与价值函数}

\textbf{策略梯度定理}将性能梯度与动作价值函数联系起来:
\begin{equation}
\nabla_\theta J(\theta) \propto \sum_{s} \mu(s) \sum_{a} q_\pi(s, a) \nabla_\theta \pi(a|s, \theta)
\end{equation}

\textbf{关键洞察}:梯度不涉及状态分布的导数,只需要知道策略参数化和动作价值函数。

\section{符号说明}

\subsection{策略相关}
\begin{itemize}
    \item $\pi(a|s, \theta)$:参数化策略,在状态 $s$ 选择动作 $a$ 的概率
    \item $\theta \in \mathbb{R}^{d'}$:策略参数向量
    \item $h(s, a, \theta)$:状态-动作对的数值偏好
    \item $\nabla_\theta \pi(a|s, \theta)$:策略关于参数的梯度
    \item $\nabla_\theta \ln \pi(a|s, \theta)$:策略对数的梯度(资格向量)
\end{itemize}

\subsection{性能与价值}
\begin{itemize}
    \item $J(\theta)$:性能指标(回合制:$v_{\pi_\theta}(s_0)$,持续问题:$r(\pi)$)
    \item $v_\pi(s)$:策略 $\pi$ 下状态 $s$ 的价值函数
    \item $q_\pi(s, a)$:策略 $\pi$ 下状态-动作对 $(s, a)$ 的价值函数
    \item $A_\pi(s, a)$:优势函数,$A_\pi(s, a) = q_\pi(s, a) - v_\pi(s)$
    \item $\mu(s)$:在策略 $\pi$ 下的状态分布
\end{itemize}

\subsection{更新相关}
\begin{itemize}
    \item $\alpha$:步长参数(学习率)
    \item $\alpha_\theta$:Actor的学习率
    \item $\alpha_w$:Critic的学习率
    \item $G_t$:从时刻 $t$ 开始的回报
    \item $\delta_t$:TD误差
    \item $b(s)$:基线函数
    \item $z_\theta$:Actor的资格迹
    \item $z_w$:Critic的资格迹
    \item $\lambda_\theta$:Actor的资格迹衰减率
    \item $\lambda_w$:Critic的资格迹衰减率
\end{itemize}

\subsection{其他}
\begin{itemize}
    \item $\gamma$:折扣因子,$\gamma \in [0, 1]$
    \item $R_{t+1}$:时刻 $t+1$ 的奖励
    \item $S_t$:时刻 $t$ 的状态
    \item $A_t$:时刻 $t$ 的动作
    \item $w \in \mathbb{R}^d$:价值函数的权重向量
    \item $\hat{v}(s, w)$:参数化的状态价值函数估计
\end{itemize}

\end{document}

