\documentclass[12pt,a4paper]{article}
\usepackage[UTF8]{ctex}
\usepackage{amsmath}
\usepackage{amssymb}
\usepackage{amsthm}
\usepackage{geometry}
\usepackage{hyperref}
\usepackage{enumitem}
\usepackage{booktabs}

\geometry{left=2.5cm,right=2.5cm,top=2.5cm,bottom=2.5cm}

\title{策略梯度定理期望形式推导}
\subtitle{从求和形式到期望形式的详细推导}
\author{强化学习笔记}
\date{\today}

\newtheorem{definition}{定义}
\newtheorem{theorem}{定理}
\newtheorem{proposition}{命题}
\newtheorem{example}{示例}

\begin{document}

\maketitle

\tableofcontents
\newpage

\section{问题}

\textbf{问题}:如何从策略梯度定理的求和形式:
\begin{equation}
\nabla_\theta J(\theta) \propto \sum_{s} \mu(s) \sum_{a} q_\pi(s, a) \nabla_\theta \pi(a|s, \theta)
\label{eq:sum_form}
\end{equation}

推导出期望形式:
\begin{equation}
\nabla_\theta J(\theta) \propto \mathbb{E}_\pi \left[\sum_{a} q_\pi(S_t, a) \nabla_\theta \pi(a|S_t, \theta)\right]
\label{eq:expectation_form}
\end{equation}

\section{关键概念}

\subsection{状态分布 $\mu(s)$}

\textbf{定义}:
\begin{equation}
\mu(s) = \sum_{k=0}^{\infty} \Pr(s_0 \to s, k, \pi)
\end{equation}

\textbf{含义}:
\begin{itemize}
    \item $\mu(s)$ 是在策略 $\pi$ 下,状态 $s$ 的长期访问频率
    \item 归一化后:$\sum_{s} \mu(s) = 1$(或等于平均episode长度)
\end{itemize}

\subsection{期望的定义}

\textbf{离散随机变量的期望}:
\begin{equation}
\mathbb{E}[X] = \sum_{x} x \Pr(X = x)
\end{equation}

\textbf{加权平均}:
\begin{equation}
\sum_{s} \mu(s) f(s) = \mathbb{E}_\mu[f(S)]
\end{equation}

其中 $S$ 是按照分布 $\mu(s)$ 分布的随机变量。

\section{推导过程}

\subsection{步骤1:从求和形式开始}

\textbf{策略梯度定理(求和形式)}:
\begin{equation}
\nabla_\theta J(\theta) \propto \sum_{s} \mu(s) \sum_{a} q_\pi(s, a) \nabla_\theta \pi(a|s, \theta)
\label{eq:step1}
\end{equation}

\subsection{步骤2:识别加权平均}

\textbf{观察}:
\begin{itemize}
    \item $\sum_{s} \mu(s) \cdots$ 是一个加权求和
    \item 权重是状态分布 $\mu(s)$
    \item 这可以写成期望的形式
\end{itemize}

\textbf{定义函数}:
\begin{equation}
f(s) = \sum_{a} q_\pi(s, a) \nabla_\theta \pi(a|s, \theta)
\end{equation}

\textbf{重写}:
\begin{equation}
\nabla_\theta J(\theta) \propto \sum_{s} \mu(s) f(s)
\label{eq:step2}
\end{equation}

\subsection{步骤3:转换为期望}

\textbf{关键洞察}:
\begin{itemize}
    \item $\sum_{s} \mu(s) f(s)$ 是函数 $f(s)$ 关于分布 $\mu(s)$ 的期望
    \item 如果 $S$ 是按照分布 $\mu(s)$ 分布的随机变量,则:
    \begin{equation}
    \sum_{s} \mu(s) f(s) = \mathbb{E}_\mu[f(S)]
    \end{equation}
\end{itemize}

\textbf{应用}:
\begin{equation}
\nabla_\theta J(\theta) \propto \sum_{s} \mu(s) f(s) = \mathbb{E}_\mu[f(S)]
\end{equation}

\textbf{展开}:
\begin{equation}
\nabla_\theta J(\theta) \propto \mathbb{E}_\mu\left[\sum_{a} q_\pi(S, a) \nabla_\theta \pi(a|S, \theta)\right]
\label{eq:step3}
\end{equation}

\subsection{步骤4:理解分布 $\mu(s)$}

\textbf{关键}:
\begin{itemize}
    \item 状态分布 $\mu(s)$ 是在策略 $\pi$ 下的状态分布
    \item 如果我们按照策略 $\pi$ 与环境交互,状态 $S_t$ 的分布就是 $\mu(s)$
    \item 因此,$\mathbb{E}_\mu[\cdots]$ 等价于 $\mathbb{E}_\pi[\cdots]$(在策略 $\pi$ 下的期望)
\end{itemize}

\textbf{最终形式}:
\begin{equation}
\nabla_\theta J(\theta) \propto \mathbb{E}_\pi\left[\sum_{a} q_\pi(S_t, a) \nabla_\theta \pi(a|S_t, \theta)\right]
\label{eq:final}
\end{equation}

其中 $S_t$ 是按照策略 $\pi$ 下的状态分布 $\mu(s)$ 分布的随机变量。

\section{详细数学推导}

\subsection{从求和到期望的严格推导}

\textbf{步骤1:求和形式}
\begin{equation}
\nabla_\theta J(\theta) \propto \sum_{s} \mu(s) \sum_{a} q_\pi(s, a) \nabla_\theta \pi(a|s, \theta)
\end{equation}

\textbf{步骤2:定义随机变量}
\begin{itemize}
    \item 设 $S$ 是随机变量,取值于状态空间 $\mathcal{S}$
    \item $S$ 的分布是 $\mu(s)$:$\Pr(S = s) = \mu(s)$
\end{itemize}

\textbf{步骤3:期望的定义}
\begin{equation}
\mathbb{E}_\mu[f(S)] = \sum_{s} \Pr(S = s) f(s) = \sum_{s} \mu(s) f(s)
\end{equation}

\textbf{步骤4:应用期望定义}
\begin{align}
\nabla_\theta J(\theta) &\propto \sum_{s} \mu(s) \sum_{a} q_\pi(s, a) \nabla_\theta \pi(a|s, \theta) \\
                         &= \sum_{s} \mu(s) f(s) \quad \text{(其中 } f(s) = \sum_{a} q_\pi(s, a) \nabla_\theta \pi(a|s, \theta) \text{)} \\
                         &= \mathbb{E}_\mu[f(S)] \\
                         &= \mathbb{E}_\mu\left[\sum_{a} q_\pi(S, a) \nabla_\theta \pi(a|S, \theta)\right]
\end{align}

\textbf{步骤5:理解分布}
\begin{itemize}
    \item 分布 $\mu(s)$ 是在策略 $\pi$ 下的状态分布
    \item 如果我们按照策略 $\pi$ 与环境交互,状态 $S_t$ 的分布就是 $\mu(s)$
    \item 因此,$\mathbb{E}_\mu[\cdots]$ 等价于 $\mathbb{E}_\pi[\cdots]$
\end{itemize}

\textbf{步骤6:最终形式}
\begin{equation}
\nabla_\theta J(\theta) \propto \mathbb{E}_\pi\left[\sum_{a} q_\pi(S_t, a) \nabla_\theta \pi(a|S_t, \theta)\right]
\end{equation}

\section{具体例子}

\subsection{例子:简单的3状态问题}

\textbf{环境}:3个状态 $s_0, s_1, s_2$

\textbf{状态分布}(假设已知):
\begin{align}
\mu(s_0) &= 0.5 \\
\mu(s_1) &= 0.3 \\
\mu(s_2) &= 0.2
\end{align}

验证:$\sum_{s} \mu(s) = 0.5 + 0.3 + 0.2 = 1$

\textbf{求和形式}:
\begin{align}
\nabla_\theta J(\theta) &\propto \sum_{s} \mu(s) \sum_{a} q_\pi(s, a) \nabla_\theta \pi(a|s, \theta) \\
                         &= \mu(s_0) \sum_{a} q_\pi(s_0, a) \nabla_\theta \pi(a|s_0, \theta) \\
                         &\quad + \mu(s_1) \sum_{a} q_\pi(s_1, a) \nabla_\theta \pi(a|s_1, \theta) \\
                         &\quad + \mu(s_2) \sum_{a} q_\pi(s_2, a) \nabla_\theta \pi(a|s_2, \theta) \\
                         &= 0.5 \times f(s_0) + 0.3 \times f(s_1) + 0.2 \times f(s_2)
\end{align}

其中 $f(s) = \sum_{a} q_\pi(s, a) \nabla_\theta \pi(a|s, \theta)$。

\textbf{期望形式}:
\begin{equation}
\nabla_\theta J(\theta) \propto \mathbb{E}_\pi\left[\sum_{a} q_\pi(S_t, a) \nabla_\theta \pi(a|S_t, \theta)\right]
\end{equation}

\textbf{计算期望}:
\begin{align}
\mathbb{E}_\pi\left[\sum_{a} q_\pi(S_t, a) \nabla_\theta \pi(a|S_t, \theta)\right] &= \sum_{s} \Pr(S_t = s) \sum_{a} q_\pi(s, a) \nabla_\theta \pi(a|s, \theta) \\
                                                                                    &= \sum_{s} \mu(s) \sum_{a} q_\pi(s, a) \nabla_\theta \pi(a|s, \theta) \\
                                                                                    &= 0.5 \times f(s_0) + 0.3 \times f(s_1) + 0.2 \times f(s_2)
\end{align}

\textbf{验证}:
\begin{itemize}
    \item 求和形式和期望形式得到相同的结果
    \item 这证明了推导的正确性
\end{itemize}

\section{为什么使用期望形式?}

\subsection{优势1:便于采样}

\textbf{期望形式的优势}:
\begin{itemize}
    \item 可以从样本中估计期望
    \item 如果我们按照策略 $\pi$ 与环境交互,状态 $S_t$ 的分布就是 $\mu(s)$
    \item 我们可以用样本平均来估计期望
\end{itemize}

\textbf{样本估计}:
\begin{equation}
\nabla_\theta J(\theta) \propto \frac{1}{n} \sum_{i=1}^{n} \sum_{a} q_\pi(S_t^{(i)}, a) \nabla_\theta \pi(a|S_t^{(i)}, \theta)
\end{equation}

其中 $S_t^{(i)}$ 是第 $i$ 个样本状态。

\subsection{优势2:数学上的简洁性}

\textbf{期望形式更简洁}:
\begin{itemize}
    \item 期望形式:$\mathbb{E}_\pi[\cdots]$
    \item 求和形式:$\sum_{s} \mu(s) \cdots$
    \item 期望形式更简洁,更易于理解和操作
\end{itemize}

\subsection{优势3:便于理论分析}

\textbf{期望的性质}:
\begin{itemize}
    \item 期望的线性性质:$\mathbb{E}[aX + bY] = a\mathbb{E}[X] + b\mathbb{E}[Y]$
    \item 期望的独立性:如果 $X$ 和 $Y$ 独立,$\mathbb{E}[XY] = \mathbb{E}[X]\mathbb{E}[Y]$
    \item 这些性质便于理论分析
\end{itemize}

\section{从期望形式到REINFORCE}

\subsection{进一步推导}

\textbf{期望形式}:
\begin{equation}
\nabla_\theta J(\theta) \propto \mathbb{E}_\pi\left[\sum_{a} q_\pi(S_t, a) \nabla_\theta \pi(a|S_t, \theta)\right]
\end{equation}

\textbf{引入动作概率}:
\begin{align}
\nabla_\theta J(\theta) &\propto \mathbb{E}_\pi\left[\sum_{a} \pi(a|S_t, \theta) q_\pi(S_t, a) \frac{\nabla_\theta \pi(a|S_t, \theta)}{\pi(a|S_t, \theta)}\right] \\
                         &= \mathbb{E}_\pi\left[\sum_{a} \pi(a|S_t, \theta) q_\pi(S_t, a) \nabla_\theta \ln \pi(a|S_t, \theta)\right]
\end{align}

\textbf{引入实际选择的动作}:
\begin{itemize}
    \item 设 $A_t$ 是按照策略 $\pi(\cdot|S_t, \theta)$ 选择的动作
    \item 则:
    \begin{equation}
    \mathbb{E}_\pi\left[\sum_{a} \pi(a|S_t, \theta) q_\pi(S_t, a) \nabla_\theta \ln \pi(a|S_t, \theta)\right] = \mathbb{E}_\pi\left[q_\pi(S_t, A_t) \nabla_\theta \ln \pi(A_t|S_t, \theta)\right]
    \end{equation}
\end{itemize}

\textbf{使用回报}:
\begin{itemize}
    \item 因为 $\mathbb{E}_\pi[G_t | S_t, A_t] = q_\pi(S_t, A_t)$
    \item 所以:
    \begin{equation}
    \nabla_\theta J(\theta) \propto \mathbb{E}_\pi\left[G_t \nabla_\theta \ln \pi(A_t|S_t, \theta)\right]
    \end{equation}
\end{itemize}

\textbf{REINFORCE更新}:
\begin{equation}
\theta_{t+1} = \theta_t + \alpha G_t \nabla_\theta \ln \pi(A_t|S_t, \theta_t)
\end{equation}

\section{总结}

\subsection{推导链条}

\begin{enumerate}
    \item \textbf{策略梯度定理(求和形式)}:
    \begin{equation}
    \nabla_\theta J(\theta) \propto \sum_{s} \mu(s) \sum_{a} q_\pi(s, a) \nabla_\theta \pi(a|s, \theta)
    \end{equation}
    
    \item \textbf{识别加权平均}:
    \begin{equation}
    \sum_{s} \mu(s) f(s) = \mathbb{E}_\mu[f(S)]
    \end{equation}
    
    \item \textbf{转换为期望}:
    \begin{equation}
    \nabla_\theta J(\theta) \propto \mathbb{E}_\mu\left[\sum_{a} q_\pi(S, a) \nabla_\theta \pi(a|S, \theta)\right]
    \end{equation}
    
    \item \textbf{理解分布}:
    \begin{itemize}
        \item 分布 $\mu(s)$ 是在策略 $\pi$ 下的状态分布
        \item $\mathbb{E}_\mu[\cdots] = \mathbb{E}_\pi[\cdots]$
    \end{itemize}
    
    \item \textbf{最终形式}:
    \begin{equation}
    \nabla_\theta J(\theta) \propto \mathbb{E}_\pi\left[\sum_{a} q_\pi(S_t, a) \nabla_\theta \pi(a|S_t, \theta)\right]
    \end{equation}
\end{enumerate}

\subsection{关键洞察}

\begin{quote}
\textbf{从求和到期望的转换}:加权求和 $\sum_{s} \mu(s) f(s)$ 可以写成期望 $\mathbb{E}_\mu[f(S)]$,其中 $S$ 是按照分布 $\mu(s)$ 分布的随机变量。由于 $\mu(s)$ 是在策略 $\pi$ 下的状态分布,所以 $\mathbb{E}_\mu[\cdots] = \mathbb{E}_\pi[\cdots]$。
\end{quote}

\subsection{关键公式}

\textbf{期望的定义}:
\begin{equation}
\mathbb{E}_\mu[f(S)] = \sum_{s} \mu(s) f(s)
\end{equation}

\textbf{策略梯度定理(期望形式)}:
\begin{equation}
\nabla_\theta J(\theta) \propto \mathbb{E}_\pi\left[\sum_{a} q_\pi(S_t, a) \nabla_\theta \pi(a|S_t, \theta)\right]
\end{equation}

\textbf{REINFORCE}:
\begin{equation}
\theta_{t+1} = \theta_t + \alpha G_t \nabla_\theta \ln \pi(A_t|S_t, \theta_t)
\end{equation}

\end{document}

