\documentclass[12pt,a4paper]{article}
\usepackage[UTF8]{ctex}
\usepackage{amsmath}
\usepackage{amssymb}
\usepackage{amsthm}
\usepackage{geometry}
\usepackage{hyperref}
\usepackage{enumitem}
\usepackage{booktabs}

\geometry{left=2.5cm,right=2.5cm,top=2.5cm,bottom=2.5cm}

\title{状态分布 $\mu(s)$ 详解}
\subtitle{策略梯度定理中的关键概念}
\author{强化学习笔记}
\date{\today}

\newtheorem{definition}{定义}
\newtheorem{theorem}{定理}
\newtheorem{example}{示例}
\newtheorem{remark}{注记}

\begin{document}

\maketitle

\tableofcontents
\newpage

\section{引言}

在策略梯度定理中,我们经常看到 $\sum_{s} \mu(s)$ 这样的表达式。本文详细解释状态分布 $\mu(s)$ 的含义、计算方法和在策略梯度中的作用。

\section{状态分布 $\mu(s)$ 的定义}

\subsection{基本定义}

\begin{definition}[状态分布 $\mu(s)$]
\textbf{状态分布} $\mu(s)$ 是在策略 $\pi$ 下,状态 $s$ 的\textbf{长期访问频率}或\textbf{稳态分布}。

对于回合制任务:
\begin{equation}
\mu(s) = \sum_{k=0}^{\infty} \Pr(s_0 \to s, k, \pi)
\end{equation}

其中 $\Pr(s_0 \to s, k, \pi)$ 是从起始状态 $s_0$ 开始,在 $k$ 步后到达状态 $s$ 的概率。
\end{definition}

\textbf{通俗理解}:
\begin{itemize}
    \item $\mu(s)$ 表示"在策略 $\pi$ 下,状态 $s$ 被访问的频率"
    \item 如果 $\mu(s_1) = 0.3$,$\mu(s_2) = 0.7$,则状态 $s_2$ 比 $s_1$ 更常被访问
    \item 所有状态的分布加起来等于1(归一化后)
\end{itemize}

\subsection{归一化形式}

\textbf{归一化状态分布}:
\begin{equation}
\mu(s) = \frac{\sum_{k=0}^{\infty} \Pr(s_0 \to s, k, \pi)}{\sum_{s'} \sum_{k=0}^{\infty} \Pr(s_0 \to s', k, \pi)}
\end{equation}

\textbf{性质}:
\begin{equation}
\sum_{s} \mu(s) = 1
\end{equation}

\textbf{含义}:
\begin{itemize}
    \item 所有状态的访问频率加起来等于1
    \item $\mu(s)$ 可以理解为"状态 $s$ 被访问的概率"(在长期平均意义下)
\end{itemize}

\section{$\sum_{s} \mu(s)$ 的含义}

\subsection{基本含义}

\textbf{表达式}:$\sum_{s} \mu(s)$

\textbf{计算}:
\begin{equation}
\sum_{s} \mu(s) = \mu(s_1) + \mu(s_2) + \mu(s_3) + \cdots
\end{equation}

\textbf{结果}:
\begin{itemize}
    \item 如果 $\mu(s)$ 是归一化的,则 $\sum_{s} \mu(s) = 1$
    \item 如果 $\mu(s)$ 是未归一化的(访问次数),则 $\sum_{s} \mu(s)$ 等于总访问次数
\end{itemize}

\subsection{在策略梯度定理中的作用}

\textbf{策略梯度定理}:
\begin{equation}
\nabla_\theta J(\theta) \propto \sum_{s} \mu(s) \sum_{a} q_\pi(s, a) \nabla_\theta \pi(a|s, \theta)
\end{equation}

\textbf{$\sum_{s} \mu(s)$ 的作用}:
\begin{itemize}
    \item 这是一个\textbf{加权求和},权重是状态分布 $\mu(s)$
    \item 经常被访问的状态($\mu(s)$ 大)对梯度的贡献更大
    \item 很少被访问的状态($\mu(s)$ 小)对梯度的贡献较小
    \item 这符合直觉:我们应该更关注经常访问的状态
\end{itemize}

\section{具体计算例子}

\subsection{例子1:简单的3状态问题}

\textbf{环境}:
\begin{itemize}
    \item 状态:$s_0$(起始)→ $s_1$ → $s_2$ → $G$(终止)
    \item 策略:总是向右移动
    \item 折扣因子:$\gamma = 1$(无折扣)
\end{itemize}

\textbf{计算状态分布}:

\textbf{状态 $s_0$}:
\begin{itemize}
    \item $\Pr(s_0 \to s_0, 0, \pi) = 1$(第0步,在起始状态)
    \item $\Pr(s_0 \to s_0, 1, \pi) = 0$(第1步,已经离开)
    \item $\Pr(s_0 \to s_0, 2, \pi) = 0$(第2步,已经离开)
    \item $\mu(s_0) = 1 + 0 + 0 + \cdots = 1$
\end{itemize}

\textbf{状态 $s_1$}:
\begin{itemize}
    \item $\Pr(s_0 \to s_1, 0, \pi) = 0$(第0步,还在 $s_0$)
    \item $\Pr(s_0 \to s_1, 1, \pi) = 1$(第1步,到达 $s_1$)
    \item $\Pr(s_0 \to s_1, 2, \pi) = 0$(第2步,已经离开)
    \item $\mu(s_1) = 0 + 1 + 0 + \cdots = 1$
\end{itemize}

\textbf{状态 $s_2$}:
\begin{itemize}
    \item $\Pr(s_0 \to s_2, 0, \pi) = 0$(第0步,还在 $s_0$)
    \item $\Pr(s_0 \to s_2, 1, \pi) = 0$(第1步,还在 $s_1$)
    \item $\Pr(s_0 \to s_2, 2, \pi) = 1$(第2步,到达 $s_2$)
    \item $\mu(s_2) = 0 + 0 + 1 + \cdots = 1$
\end{itemize}

\textbf{归一化}:
\begin{itemize}
    \item 总访问次数:$\sum_{s} \mu(s) = 1 + 1 + 1 = 3$
    \item 归一化后:$\mu(s_0) = \frac{1}{3}$,$\mu(s_1) = \frac{1}{3}$,$\mu(s_2) = \frac{1}{3}$
    \item $\sum_{s} \mu(s) = \frac{1}{3} + \frac{1}{3} + \frac{1}{3} = 1$
\end{itemize}

\subsection{例子2:有循环的情况}

\textbf{环境}:
\begin{itemize}
    \item 状态:$s_0$(起始)→ $s_1$ → $s_2$ → $s_1$ → $\cdots$(循环)
    \item 策略:总是向右移动(在 $s_2$ 后回到 $s_1$)
    \item 这是一个持续任务(没有终止状态)
\end{itemize}

\textbf{计算状态分布}(稳态分布):

\textbf{状态 $s_0$}:
\begin{itemize}
    \item 只在第0步被访问一次
    \item 之后不再被访问(因为总是向右)
    \item $\mu(s_0) = 1$(未归一化)
\end{itemize}

\textbf{状态 $s_1$}:
\begin{itemize}
    \item 在第1步被访问
    \item 在第4步、第7步、$\cdots$ 被访问(循环)
    \item 访问次数:$1 + 1 + 1 + \cdots = \infty$(在持续任务中)
    \item 稳态分布:$\mu(s_1) = \frac{1}{2}$(归一化后,因为 $s_1$ 和 $s_2$ 各占一半时间)
\end{itemize}

\textbf{状态 $s_2$}:
\begin{itemize}
    \item 在第2步、第5步、第8步、$\cdots$ 被访问
    \item 稳态分布:$\mu(s_2) = \frac{1}{2}$(归一化后)
\end{itemize}

\textbf{归一化}:
\begin{itemize}
    \item 对于持续任务,使用稳态分布
    \item $\mu(s_0) = 0$(在稳态中不被访问)
    \item $\mu(s_1) = \frac{1}{2}$,$\mu(s_2) = \frac{1}{2}$
    \item $\sum_{s} \mu(s) = 0 + \frac{1}{2} + \frac{1}{2} = 1$
\end{itemize}

\section{$\sum_{s} \mu(s)$ 在策略梯度定理中的作用}

\subsection{策略梯度定理}

\textbf{策略梯度定理}:
\begin{equation}
\nabla_\theta J(\theta) \propto \sum_{s} \mu(s) \sum_{a} q_\pi(s, a) \nabla_\theta \pi(a|s, \theta)
\label{eq:policy_gradient}
\end{equation}

\textbf{$\sum_{s} \mu(s)$ 的作用}:
\begin{itemize}
    \item 这是一个\textbf{加权平均},权重是状态分布 $\mu(s)$
    \item 对每个状态 $s$,计算 $\sum_{a} q_\pi(s, a) \nabla_\theta \pi(a|s, \theta)$
    \item 然后用 $\mu(s)$ 加权求和
    \item 经常被访问的状态权重更大
\end{itemize}

\subsection{直观理解}

\textbf{类比:加权平均}

想象你在计算"平均成绩":
\begin{itemize}
    \item 每个状态 $s$ 对应一个"成绩":$\sum_{a} q_\pi(s, a) \nabla_\theta \pi(a|s, \theta)$
    \item 状态分布 $\mu(s)$ 是"权重"(这个状态出现的频率)
    \item $\sum_{s} \mu(s)$ 是"加权平均":$\frac{\sum_{s} \mu(s) \times \text{成绩}_s}{\sum_{s} \mu(s)}$
    \item 如果 $\mu(s)$ 已归一化,$\sum_{s} \mu(s) = 1$,就是加权平均
\end{itemize}

\textbf{为什么使用状态分布作为权重?}
\begin{itemize}
    \item 经常被访问的状态对性能的影响更大
    \item 我们应该更关注这些状态
    \item 使用 $\mu(s)$ 作为权重是合理的
\end{itemize}

\section{具体计算例子}

\subsection{例子:计算策略梯度}

\textbf{环境}:3个状态 $s_0, s_1, s_2$,2个动作 $a_1, a_2$

\textbf{状态分布}(假设已知):
\begin{align}
\mu(s_0) &= 0.5 \quad \text{(经常访问)} \\
\mu(s_1) &= 0.3 \quad \text{(偶尔访问)} \\
\mu(s_2) &= 0.2 \quad \text{(很少访问)}
\end{align}

验证:$\sum_{s} \mu(s) = 0.5 + 0.3 + 0.2 = 1$(归一化)

\textbf{动作价值函数}(假设已知):
\begin{align}
q_\pi(s_0, a_1) &= 5, \quad q_\pi(s_0, a_2) = 3 \\
q_\pi(s_1, a_1) &= 4, \quad q_\pi(s_1, a_2) = 2 \\
q_\pi(s_2, a_1) &= 6, \quad q_\pi(s_2, a_2) = 1
\end{align}

\textbf{策略梯度}(假设 $\nabla_\theta \pi(a_1|s, \theta) = [1, 0]^T$,$\nabla_\theta \pi(a_2|s, \theta) = [0, 1]^T$):

\textbf{对状态 $s_0$ 的贡献}:
\begin{align}
\sum_{a} q_\pi(s_0, a) \nabla_\theta \pi(a|s_0, \theta) &= q_\pi(s_0, a_1) \nabla_\theta \pi(a_1|s_0, \theta) + q_\pi(s_0, a_2) \nabla_\theta \pi(a_2|s_0, \theta) \\
&= 5 \times [1, 0]^T + 3 \times [0, 1]^T \\
&= [5, 3]^T
\end{align}

加权后:$\mu(s_0) \times [5, 3]^T = 0.5 \times [5, 3]^T = [2.5, 1.5]^T$

\textbf{对状态 $s_1$ 的贡献}:
\begin{align}
\sum_{a} q_\pi(s_1, a) \nabla_\theta \pi(a|s_1, \theta) &= 4 \times [1, 0]^T + 2 \times [0, 1]^T \\
&= [4, 2]^T
\end{align}

加权后:$\mu(s_1) \times [4, 2]^T = 0.3 \times [4, 2]^T = [1.2, 0.6]^T$

\textbf{对状态 $s_2$ 的贡献}:
\begin{align}
\sum_{a} q_\pi(s_2, a) \nabla_\theta \pi(a|s_2, \theta) &= 6 \times [1, 0]^T + 1 \times [0, 1]^T \\
&= [6, 1]^T
\end{align}

加权后:$\mu(s_2) \times [6, 1]^T = 0.2 \times [6, 1]^T = [1.2, 0.2]^T$

\textbf{总梯度}:
\begin{align}
\nabla_\theta J(\theta) &\propto \sum_{s} \mu(s) \sum_{a} q_\pi(s, a) \nabla_\theta \pi(a|s, \theta) \\
                         &= [2.5, 1.5]^T + [1.2, 0.6]^T + [1.2, 0.2]^T \\
                         &= [4.9, 2.3]^T
\end{align}

\textbf{观察}:
\begin{itemize}
    \item 状态 $s_0$ 的贡献最大($[2.5, 1.5]^T$),因为 $\mu(s_0) = 0.5$ 最大
    \item 状态 $s_2$ 的贡献较小($[1.2, 0.2]^T$),因为 $\mu(s_2) = 0.2$ 最小
    \item 即使 $s_2$ 的动作价值很高($q_\pi(s_2, a_1) = 6$),但由于访问频率低,贡献较小
\end{itemize}

\section{状态分布的计算方法}

\subsection{回合制任务}

\textbf{方法1:直接计算}
\begin{equation}
\mu(s) = \sum_{k=0}^{\infty} \Pr(s_0 \to s, k, \pi)
\end{equation}

\textbf{方法2:通过状态转移矩阵}

定义状态转移概率矩阵 $\mathbf{P}_\pi$,其中:
\begin{equation}
P_\pi[i, j] = \sum_{a} \pi(a|s_i) \sum_{r} p(s_j, r | s_i, a)
\end{equation}

则状态分布满足:
\begin{equation}
\boldsymbol{\mu} = \boldsymbol{\mu}_0 + \boldsymbol{\mu}_0 \mathbf{P}_\pi + \boldsymbol{\mu}_0 \mathbf{P}_\pi^2 + \cdots
\end{equation}

其中 $\boldsymbol{\mu}_0$ 是初始状态分布(通常是 $[1, 0, 0, \ldots]^T$,从 $s_0$ 开始)。

\subsection{持续任务}

\textbf{稳态分布}:
\begin{equation}
\boldsymbol{\mu} = \boldsymbol{\mu} \mathbf{P}_\pi
\end{equation}

即状态分布是状态转移矩阵的左特征向量(特征值为1)。

\section{$\sum_{s} \mu(s)$ 的性质}

\subsection{归一化性质}

\textbf{如果 $\mu(s)$ 是归一化的}:
\begin{equation}
\sum_{s} \mu(s) = 1
\end{equation}

\textbf{如果 $\mu(s)$ 是未归一化的(访问次数)}:
\begin{equation}
\sum_{s} \mu(s) = \text{总访问次数}
\end{equation}

\subsection{在策略梯度定理中}

\textbf{策略梯度定理}:
\begin{equation}
\nabla_\theta J(\theta) \propto \sum_{s} \mu(s) \sum_{a} q_\pi(s, a) \nabla_\theta \pi(a|s, \theta)
\end{equation}

\textbf{关键}:
\begin{itemize}
    \item 符号 $\propto$ 表示"正比于"
    \item 在回合制任务中,$\sum_{s} \mu(s)$ 等于平均episode长度
    \item 在持续任务中,$\sum_{s} \mu(s) = 1$(归一化)
    \item 常数因子可以被步长 $\alpha$ 吸收
\end{itemize}

\section{具体例子:Gridworld}

\subsection{问题设置}

\textbf{环境}:3×3 Gridworld
\begin{center}
\begin{tabular}{|c|c|c|}
\hline
终止 & $s_2$ & $s_3$ \\
\hline
$s_4$ & $s_5$ & $s_6$ \\
\hline
$s_7$ & $s_8$ & 终止 \\
\hline
\end{tabular}
\end{center}

\textbf{策略}:等概率随机策略(每个动作概率 $0.25$)

\textbf{起始状态}:$s_5$(中心)

\subsection{计算状态分布}

\textbf{方法}:通过模拟或解析计算

\textbf{假设结果}(简化):
\begin{align}
\mu(s_2) &= 0.15 \\
\mu(s_3) &= 0.10 \\
\mu(s_4) &= 0.15 \\
\mu(s_5) &= 0.20 \quad \text{(起始状态,访问频率高)} \\
\mu(s_6) &= 0.10 \\
\mu(s_7) &= 0.15 \\
\mu(s_8) &= 0.15
\end{align}

\textbf{验证}:
\begin{equation}
\sum_{s} \mu(s) = 0.15 + 0.10 + 0.15 + 0.20 + 0.10 + 0.15 + 0.15 = 1.0
\end{equation}

\textbf{观察}:
\begin{itemize}
    \item 起始状态 $s_5$ 的访问频率最高($0.20$)
    \item 边缘状态($s_2, s_3, s_4, s_6, s_7, s_8$)的访问频率较低
    \item 这符合直觉:从中心开始,更容易访问中心附近的状态
\end{itemize}

\subsection{在策略梯度中的应用}

\textbf{策略梯度计算}:
\begin{equation}
\nabla_\theta J(\theta) \propto \sum_{s} \mu(s) \sum_{a} q_\pi(s, a) \nabla_\theta \pi(a|s, \theta)
\end{equation}

\textbf{对每个状态的贡献}:
\begin{itemize}
    \item 状态 $s_5$:贡献 $= 0.20 \times \sum_{a} q_\pi(s_5, a) \nabla_\theta \pi(a|s_5, \theta)$(权重最大)
    \item 状态 $s_2$:贡献 $= 0.15 \times \sum_{a} q_\pi(s_2, a) \nabla_\theta \pi(a|s_2, \theta)$
    \item 状态 $s_3$:贡献 $= 0.10 \times \sum_{a} q_\pi(s_3, a) \nabla_\theta \pi(a|s_3, \theta)$(权重最小)
    \item $\cdots$
\end{itemize}

\textbf{总梯度}:
\begin{equation}
\nabla_\theta J(\theta) \propto 0.20 \times \text{贡献}_{s_5} + 0.15 \times \text{贡献}_{s_2} + 0.15 \times \text{贡献}_{s_4} + \cdots
\end{equation}

\textbf{关键}:
\begin{itemize}
    \item 经常访问的状态(如 $s_5$)对梯度的贡献更大
    \item 很少访问的状态(如 $s_3$)对梯度的贡献较小
    \item 这符合直觉:我们应该更关注经常访问的状态
\end{itemize}

\section{为什么使用状态分布?}

\subsection{理论原因}

\textbf{策略梯度定理的推导}:
\begin{itemize}
    \item 从性能指标 $J(\theta) = v_{\pi_\theta}(s_0)$ 开始
    \item 计算梯度 $\nabla_\theta v_{\pi_\theta}(s_0)$
    \item 通过展开和递归,得到:
    \begin{equation}
    \nabla_\theta v_{\pi_\theta}(s_0) = \sum_{s} \mu(s) \sum_{a} q_\pi(s, a) \nabla_\theta \pi(a|s, \theta)
    \end{equation}
    \item 状态分布 $\mu(s)$ 自然出现在推导中
\end{itemize}

\textbf{关键洞察}:
\begin{itemize}
    \item 梯度不涉及状态分布的导数
    \item 只需要知道状态分布 $\mu(s)$ 的值
    \item 这使我们可以估计梯度,即使不知道策略变化对状态分布的影响
\end{itemize}

\subsection{实际原因}

\textbf{加权平均的合理性}:
\begin{itemize}
    \item 经常访问的状态对性能的影响更大
    \item 我们应该更关注这些状态
    \item 使用 $\mu(s)$ 作为权重是合理的
\end{itemize}

\textbf{样本效率}:
\begin{itemize}
    \item 在策略 $\pi$ 下,状态 $s$ 被访问的频率是 $\mu(s)$
    \item 如果我们按照策略 $\pi$ 采样,状态 $s$ 出现的频率就是 $\mu(s)$
    \item 这使我们可以从样本中估计梯度
\end{itemize}

\section{总结}

\subsection{核心概念}

\begin{enumerate}
    \item \textbf{状态分布 $\mu(s)$}:
    \begin{itemize}
        \item 在策略 $\pi$ 下,状态 $s$ 的长期访问频率
        \item 归一化后:$\sum_{s} \mu(s) = 1$
    \end{itemize}
    
    \item \textbf{$\sum_{s} \mu(s)$}:
    \begin{itemize}
        \item 如果 $\mu(s)$ 归一化:$\sum_{s} \mu(s) = 1$
        \item 如果 $\mu(s)$ 未归一化:$\sum_{s} \mu(s) =$ 总访问次数
    \end{itemize}
    
    \item \textbf{在策略梯度中的作用}:
    \begin{itemize}
        \item 作为加权求和的权重
        \item 经常访问的状态权重更大
        \item 使梯度更关注重要的状态
    \end{itemize}
\end{enumerate}

\subsection{关键公式}

\textbf{状态分布定义}:
\begin{equation}
\mu(s) = \sum_{k=0}^{\infty} \Pr(s_0 \to s, k, \pi)
\end{equation}

\textbf{归一化}:
\begin{equation}
\sum_{s} \mu(s) = 1
\end{equation}

\textbf{策略梯度定理}:
\begin{equation}
\nabla_\theta J(\theta) \propto \sum_{s} \mu(s) \sum_{a} q_\pi(s, a) \nabla_\theta \pi(a|s, \theta)
\end{equation}

\subsection{直观理解}

\begin{quote}
\textbf{$\sum_{s} \mu(s)$} 是一个加权求和,权重是状态分布 $\mu(s)$。经常被访问的状态($\mu(s)$ 大)对梯度的贡献更大,很少被访问的状态($\mu(s)$ 小)对梯度的贡献较小。这符合直觉:我们应该更关注经常访问的状态,因为它们对性能的影响更大。
\end{quote}

\end{document}

