\documentclass[12pt,a4paper]{article}
\usepackage[UTF8]{ctex}
\usepackage{amsmath}
\usepackage{amssymb}
\usepackage{amsthm}
\usepackage{geometry}
\usepackage{hyperref}
\usepackage{enumitem}
\usepackage{booktabs}

\geometry{left=2.5cm,right=2.5cm,top=2.5cm,bottom=2.5cm}

\title{全期望公式推导详解}
\subtitle{Law of Total Expectation的详细推导}
\author{强化学习笔记}
\date{\today}

\newtheorem{definition}{定义}
\newtheorem{theorem}{定理}
\newtheorem{proposition}{命题}
\newtheorem{example}{示例}
\newtheorem{proof}{证明}

\begin{document}

\maketitle

\tableofcontents
\newpage

\section{问题}

\textbf{问题}:全期望公式(Law of Total Expectation):
\begin{equation}
\mathbb{E}_\pi[X] = \mathbb{E}_\pi[\mathbb{E}[X | S_t]]
\label{eq:total_expectation}
\end{equation}

是怎么来的?

\section{全期望公式的一般形式}

\subsection{离散随机变量}

\textbf{全期望公式(一般形式)}:
\begin{equation}
\mathbb{E}[X] = \mathbb{E}[\mathbb{E}[X | Y]]
\label{eq:general_form}
\end{equation}

\textbf{含义}:
\begin{itemize}
    \item $X$ 和 $Y$ 是随机变量
    \item $\mathbb{E}[X | Y]$ 是给定 $Y$ 时 $X$ 的条件期望
    \item 对条件期望再取期望,得到 $X$ 的期望
\end{itemize}

\subsection{展开形式}

\textbf{离散情况}:
\begin{equation}
\mathbb{E}[X] = \sum_{y} \Pr(Y = y) \mathbb{E}[X | Y = y]
\label{eq:discrete_form}
\end{equation}

\textbf{连续情况}:
\begin{equation}
\mathbb{E}[X] = \int_{-\infty}^{\infty} f_Y(y) \mathbb{E}[X | Y = y] \, dy
\label{eq:continuous_form}
\end{equation}

其中 $f_Y(y)$ 是 $Y$ 的概率密度函数。

\section{从条件期望的定义推导}

\subsection{条件期望的定义}

\textbf{离散随机变量}:
\begin{equation}
\mathbb{E}[X | Y = y] = \sum_{x} x \Pr(X = x | Y = y)
\label{eq:conditional_expectation}
\end{equation}

\textbf{含义}:
\begin{itemize}
    \item 给定 $Y = y$,$X$ 的期望值
    \item 这是关于 $X$ 的加权平均,权重是条件概率 $\Pr(X = x | Y = y)$
\end{itemize}

\subsection{推导过程}

\textbf{步骤1:期望的定义}

\textbf{离散随机变量}:
\begin{equation}
\mathbb{E}[X] = \sum_{x} x \Pr(X = x)
\label{eq:expectation_definition}
\end{equation}

\textbf{步骤2:使用全概率公式}

\textbf{全概率公式}:
\begin{equation}
\Pr(X = x) = \sum_{y} \Pr(X = x, Y = y) = \sum_{y} \Pr(Y = y) \Pr(X = x | Y = y)
\label{eq:total_probability}
\end{equation}

\textbf{步骤3:代入期望定义}

\begin{align}
\mathbb{E}[X] &= \sum_{x} x \Pr(X = x) \\
               &= \sum_{x} x \sum_{y} \Pr(Y = y) \Pr(X = x | Y = y) \\
               &= \sum_{x} \sum_{y} x \Pr(Y = y) \Pr(X = x | Y = y) \\
               &= \sum_{y} \Pr(Y = y) \sum_{x} x \Pr(X = x | Y = y) \\
               &= \sum_{y} \Pr(Y = y) \mathbb{E}[X | Y = y] \\
               &= \mathbb{E}[\mathbb{E}[X | Y]]
\end{align}

\textbf{关键步骤}:
\begin{itemize}
    \item 交换求和顺序:$\sum_{x} \sum_{y} = \sum_{y} \sum_{x}$
    \item 识别条件期望:$\sum_{x} x \Pr(X = x | Y = y) = \mathbb{E}[X | Y = y]$
    \item 得到全期望公式:$\sum_{y} \Pr(Y = y) \mathbb{E}[X | Y = y] = \mathbb{E}[\mathbb{E}[X | Y]]$
\end{itemize}

\section{应用到强化学习}

\subsection{公式}

\textbf{全期望公式(强化学习)}:
\begin{equation}
\mathbb{E}_\pi[X] = \mathbb{E}_\pi[\mathbb{E}[X | S_t]]
\label{eq:rl_form}
\end{equation}

\textbf{含义}:
\begin{itemize}
    \item $X$ 是随机变量(例如:回报 $G_t$、动作价值 $q_\pi(S_t, A_t)$ 等)
    \item $S_t$ 是状态(随机变量)
    \item $\mathbb{E}[X | S_t]$ 是给定状态 $S_t$ 时 $X$ 的条件期望
    \item 对条件期望再取期望(关于状态分布),得到 $X$ 的期望
\end{itemize}

\subsection{展开形式}

\textbf{离散状态空间}:
\begin{equation}
\mathbb{E}_\pi[X] = \sum_{s} \mu(s) \mathbb{E}[X | S_t = s]
\label{eq:rl_discrete_form}
\end{equation}

其中 $\mu(s)$ 是在策略 $\pi$ 下的状态分布。

\subsection{推导过程}

\textbf{步骤1:期望的定义}

\begin{equation}
\mathbb{E}_\pi[X] = \sum_{x} x \Pr(X = x)
\end{equation}

\textbf{步骤2:使用全概率公式}

\begin{equation}
\Pr(X = x) = \sum_{s} \Pr(S_t = s) \Pr(X = x | S_t = s) = \sum_{s} \mu(s) \Pr(X = x | S_t = s)
\end{equation}

\textbf{步骤3:代入期望定义}

\begin{align}
\mathbb{E}_\pi[X] &= \sum_{x} x \Pr(X = x) \\
                   &= \sum_{x} x \sum_{s} \mu(s) \Pr(X = x | S_t = s) \\
                   &= \sum_{x} \sum_{s} x \mu(s) \Pr(X = x | S_t = s) \\
                   &= \sum_{s} \mu(s) \sum_{x} x \Pr(X = x | S_t = s) \\
                   &= \sum_{s} \mu(s) \mathbb{E}[X | S_t = s] \\
                   &= \mathbb{E}_\pi[\mathbb{E}[X | S_t]]
\end{align}

\textbf{关键步骤}:
\begin{itemize}
    \item 交换求和顺序:$\sum_{x} \sum_{s} = \sum_{s} \sum_{x}$
    \item 识别条件期望:$\sum_{x} x \Pr(X = x | S_t = s) = \mathbb{E}[X | S_t = s]$
    \item 得到全期望公式:$\sum_{s} \mu(s) \mathbb{E}[X | S_t = s] = \mathbb{E}_\pi[\mathbb{E}[X | S_t]]$
\end{itemize}

\section{具体例子}

\subsection{例子1:简单的3状态问题}

\textbf{场景}:
\begin{itemize}
    \item 3个状态:$s_1, s_2, s_3$
    \item 状态分布:$\mu(s_1) = 0.5, \mu(s_2) = 0.3, \mu(s_3) = 0.2$
    \item 随机变量 $X$ 的分布(给定状态):
    \begin{align}
    \Pr(X = 1 | S_t = s_1) &= 0.6, \quad \Pr(X = 2 | S_t = s_1) = 0.4 \\
    \Pr(X = 1 | S_t = s_2) &= 0.3, \quad \Pr(X = 2 | S_t = s_2) = 0.7 \\
    \Pr(X = 1 | S_t = s_3) &= 0.8, \quad \Pr(X = 2 | S_t = s_3) = 0.2
    \end{align}
\end{itemize}

\textbf{方法1:直接计算期望}

\textbf{步骤1:计算边际分布}

\begin{align}
\Pr(X = 1) &= \sum_{s} \mu(s) \Pr(X = 1 | S_t = s) \\
           &= 0.5 \times 0.6 + 0.3 \times 0.3 + 0.2 \times 0.8 \\
           &= 0.3 + 0.09 + 0.16 \\
           &= 0.55
\end{align}

\begin{align}
\Pr(X = 2) &= \sum_{s} \mu(s) \Pr(X = 2 | S_t = s) \\
           &= 0.5 \times 0.4 + 0.3 \times 0.7 + 0.2 \times 0.2 \\
           &= 0.2 + 0.21 + 0.04 \\
           &= 0.45
\end{align}

\textbf{步骤2:计算期望}

\begin{align}
\mathbb{E}_\pi[X] &= \sum_{x} x \Pr(X = x) \\
                  &= 1 \times 0.55 + 2 \times 0.45 \\
                  &= 0.55 + 0.90 \\
                  &= 1.45
\end{align}

\textbf{方法2:使用全期望公式}

\textbf{步骤1:计算条件期望}

\begin{align}
\mathbb{E}[X | S_t = s_1] &= 1 \times 0.6 + 2 \times 0.4 = 0.6 + 0.8 = 1.4 \\
\mathbb{E}[X | S_t = s_2] &= 1 \times 0.3 + 2 \times 0.7 = 0.3 + 1.4 = 1.7 \\
\mathbb{E}[X | S_t = s_3] &= 1 \times 0.8 + 2 \times 0.2 = 0.8 + 0.4 = 1.2
\end{align}

\textbf{步骤2:对条件期望取期望}

\begin{align}
\mathbb{E}_\pi[\mathbb{E}[X | S_t]] &= \sum_{s} \mu(s) \mathbb{E}[X | S_t = s] \\
                                    &= 0.5 \times 1.4 + 0.3 \times 1.7 + 0.2 \times 1.2 \\
                                    &= 0.7 + 0.51 + 0.24 \\
                                    &= 1.45
\end{align}

\textbf{验证}:两种方法得到相同的结果 $1.45$!

\subsection{例子2:强化学习中的应用}

\textbf{场景}:
\begin{itemize}
    \item 状态 $S_t$,动作 $A_t$,回报 $G_t$
    \item 我们想计算:$\mathbb{E}_\pi[G_t]$
\end{itemize}

\textbf{方法1:直接计算(困难)}

需要知道 $G_t$ 的完整分布,这通常很困难。

\textbf{方法2:使用全期望公式}

\begin{align}
\mathbb{E}_\pi[G_t] &= \mathbb{E}_\pi[\mathbb{E}[G_t | S_t]] \\
                     &= \sum_{s} \mu(s) \mathbb{E}[G_t | S_t = s] \\
                     &= \sum_{s} \mu(s) v_\pi(s)
\end{align}

其中 $v_\pi(s) = \mathbb{E}[G_t | S_t = s]$ 是状态价值函数。

\textbf{优势}:
\begin{itemize}
    \item 不需要知道 $G_t$ 的完整分布
    \item 只需要知道状态价值函数 $v_\pi(s)$ 和状态分布 $\mu(s)$
    \item 这通常更容易计算或估计
\end{itemize}

\section{全期望公式的直观理解}

\subsection{两层平均}

\textbf{全期望公式}:
\begin{equation}
\mathbb{E}_\pi[X] = \mathbb{E}_\pi[\mathbb{E}[X | S_t]]
\end{equation}

\textbf{直观理解}:
\begin{itemize}
    \item \textbf{内层平均}:$\mathbb{E}[X | S_t = s]$ 是给定状态 $s$ 时 $X$ 的平均值
    \item \textbf{外层平均}:$\mathbb{E}_\pi[\cdots]$ 是对所有状态的平均(按照状态分布 $\mu(s)$)
    \item \textbf{全期望}:先对每个状态计算条件期望,再对所有状态求平均
\end{itemize}

\subsection{类比:班级平均分}

\textbf{场景}:
\begin{itemize}
    \item 有3个班级:$s_1, s_2, s_3$
    \item 每个班级的学生人数:$\mu(s_1) = 50, \mu(s_2) = 30, \mu(s_3) = 20$(总人数 $100$)
    \item 每个班级的平均分:$\mathbb{E}[X | S_t = s_1] = 80, \mathbb{E}[X | S_t = s_2] = 90, \mathbb{E}[X | S_t = s_3] = 70$
\end{itemize}

\textbf{方法1:直接计算全校平均分}

需要知道每个学生的分数,然后求平均(困难)。

\textbf{方法2:使用全期望公式}

\begin{align}
\text{全校平均分} &= \mathbb{E}_\pi[X] \\
                  &= \mathbb{E}_\pi[\mathbb{E}[X | S_t]] \\
                  &= \frac{50}{100} \times 80 + \frac{30}{100} \times 90 + \frac{20}{100} \times 70 \\
                  &= 0.5 \times 80 + 0.3 \times 90 + 0.2 \times 70 \\
                  &= 40 + 27 + 14 \\
                  &= 81
\end{align}

\textbf{优势}:
\begin{itemize}
    \item 不需要知道每个学生的分数
    \item 只需要知道每个班级的平均分和人数
    \item 这通常更容易计算
\end{itemize}

\section{在策略梯度中的应用}

\subsection{从求和到期望}

\textbf{策略梯度定理(求和形式)}:
\begin{equation}
\nabla_\theta J(\theta) \propto \sum_{s} \mu(s) \sum_{a} q_\pi(s, a) \nabla_\theta \pi(a|s, \theta)
\end{equation}

\textbf{使用全期望公式}:
\begin{align}
\nabla_\theta J(\theta) &\propto \sum_{s} \mu(s) \sum_{a} q_\pi(s, a) \nabla_\theta \pi(a|s, \theta) \\
                         &= \sum_{s} \mu(s) \mathbb{E}\left[\sum_{a} q_\pi(S_t, a) \nabla_\theta \pi(a|S_t, \theta) \middle| S_t = s\right] \\
                         &= \mathbb{E}_\pi\left[\sum_{a} q_\pi(S_t, a) \nabla_\theta \pi(a|S_t, \theta)\right]
\end{align}

\textbf{关键}:
\begin{itemize}
    \item 内层:$\sum_{a} q_\pi(s, a) \nabla_\theta \pi(a|s, \theta)$ 是给定状态 $s$ 时的值
    \item 外层:$\sum_{s} \mu(s) \cdots$ 是对所有状态的平均
    \item 这就是全期望公式的应用
\end{itemize}

\subsection{从动作求和到动作期望}

\textbf{策略梯度定理(期望形式)}:
\begin{equation}
\nabla_\theta J(\theta) \propto \mathbb{E}_\pi\left[\sum_{a} q_\pi(S_t, a) \nabla_\theta \pi(a|S_t, \theta)\right]
\end{equation}

\textbf{引入动作概率}:
\begin{align}
\nabla_\theta J(\theta) &= \mathbb{E}_\pi\left[\sum_{a} \pi(a|S_t, \theta) q_\pi(S_t, a) \frac{\nabla_\theta \pi(a|S_t, \theta)}{\pi(a|S_t, \theta)}\right] \\
                         &= \mathbb{E}_\pi\left[\sum_{a} \pi(a|S_t, \theta) q_\pi(S_t, a) \nabla_\theta \ln \pi(a|S_t, \theta)\right]
\end{align}

\textbf{使用全期望公式(关于动作)}:
\begin{align}
\nabla_\theta J(\theta) &= \mathbb{E}_\pi\left[\sum_{a} \pi(a|S_t, \theta) q_\pi(S_t, a) \nabla_\theta \ln \pi(a|S_t, \theta)\right] \\
                         &= \mathbb{E}_\pi\left[\mathbb{E}\left[q_\pi(S_t, A_t) \nabla_\theta \ln \pi(A_t|S_t, \theta) \middle| S_t\right]\right] \\
                         &= \mathbb{E}_\pi\left[q_\pi(S_t, A_t) \nabla_\theta \ln \pi(A_t|S_t, \theta)\right]
\end{align}

\textbf{关键}:
\begin{itemize}
    \item 内层:$\sum_{a} \pi(a|S_t, \theta) \cdots$ 是给定状态 $S_t$ 时关于动作的期望
    \item 外层:$\mathbb{E}_\pi[\cdots]$ 是对所有状态的平均
    \item 全期望公式使我们能够简化表达式
\end{itemize}

\section{全期望公式的证明}

\subsection{离散情况}

\textbf{定理}(全期望公式):
\begin{equation}
\mathbb{E}[X] = \mathbb{E}[\mathbb{E}[X | Y]]
\end{equation}

\textbf{证明}:

\textbf{步骤1:条件期望的定义}

\begin{equation}
\mathbb{E}[X | Y = y] = \sum_{x} x \Pr(X = x | Y = y)
\end{equation}

\textbf{步骤2:期望的定义}

\begin{equation}
\mathbb{E}[X] = \sum_{x} x \Pr(X = x)
\end{equation}

\textbf{步骤3:使用全概率公式}

\begin{equation}
\Pr(X = x) = \sum_{y} \Pr(X = x, Y = y) = \sum_{y} \Pr(Y = y) \Pr(X = x | Y = y)
\end{equation}

\textbf{步骤4:代入期望定义}

\begin{align}
\mathbb{E}[X] &= \sum_{x} x \Pr(X = x) \\
               &= \sum_{x} x \sum_{y} \Pr(Y = y) \Pr(X = x | Y = y) \\
               &= \sum_{x} \sum_{y} x \Pr(Y = y) \Pr(X = x | Y = y) \\
               &= \sum_{y} \Pr(Y = y) \sum_{x} x \Pr(X = x | Y = y) \\
               &= \sum_{y} \Pr(Y = y) \mathbb{E}[X | Y = y] \\
               &= \mathbb{E}[\mathbb{E}[X | Y]]
\end{align}

\textbf{证毕}!

\subsection{连续情况}

\textbf{定理}(全期望公式,连续情况):
\begin{equation}
\mathbb{E}[X] = \mathbb{E}[\mathbb{E}[X | Y]]
\end{equation}

\textbf{证明}:

\textbf{步骤1:条件期望的定义}

\begin{equation}
\mathbb{E}[X | Y = y] = \int_{-\infty}^{\infty} x f_{X|Y}(x | y) \, dx
\end{equation}

其中 $f_{X|Y}(x | y)$ 是条件概率密度函数。

\textbf{步骤2:期望的定义}

\begin{equation}
\mathbb{E}[X] = \int_{-\infty}^{\infty} x f_X(x) \, dx
\end{equation}

\textbf{步骤3:使用全概率公式}

\begin{equation}
f_X(x) = \int_{-\infty}^{\infty} f_{X,Y}(x, y) \, dy = \int_{-\infty}^{\infty} f_Y(y) f_{X|Y}(x | y) \, dy
\end{equation}

\textbf{步骤4:代入期望定义}

\begin{align}
\mathbb{E}[X] &= \int_{-\infty}^{\infty} x f_X(x) \, dx \\
               &= \int_{-\infty}^{\infty} x \int_{-\infty}^{\infty} f_Y(y) f_{X|Y}(x | y) \, dy \, dx \\
               &= \int_{-\infty}^{\infty} \int_{-\infty}^{\infty} x f_Y(y) f_{X|Y}(x | y) \, dy \, dx \\
               &= \int_{-\infty}^{\infty} f_Y(y) \int_{-\infty}^{\infty} x f_{X|Y}(x | y) \, dx \, dy \\
               &= \int_{-\infty}^{\infty} f_Y(y) \mathbb{E}[X | Y = y] \, dy \\
               &= \mathbb{E}[\mathbb{E}[X | Y]]
\end{align}

\textbf{证毕}!

\section{总结}

\subsection{全期望公式}

\textbf{一般形式}:
\begin{equation}
\mathbb{E}[X] = \mathbb{E}[\mathbb{E}[X | Y]]
\end{equation}

\textbf{展开形式(离散)}:
\begin{equation}
\mathbb{E}[X] = \sum_{y} \Pr(Y = y) \mathbb{E}[X | Y = y]
\end{equation}

\textbf{展开形式(连续)}:
\begin{equation}
\mathbb{E}[X] = \int_{-\infty}^{\infty} f_Y(y) \mathbb{E}[X | Y = y] \, dy
\end{equation}

\subsection{推导过程}

\begin{enumerate}
    \item \textbf{期望的定义}:$\mathbb{E}[X] = \sum_{x} x \Pr(X = x)$
    \item \textbf{全概率公式}:$\Pr(X = x) = \sum_{y} \Pr(Y = y) \Pr(X = x | Y = y)$
    \item \textbf{代入并交换求和顺序}:$\sum_{x} \sum_{y} = \sum_{y} \sum_{x}$
    \item \textbf{识别条件期望}:$\sum_{x} x \Pr(X = x | Y = y) = \mathbb{E}[X | Y = y]$
    \item \textbf{得到全期望公式}:$\sum_{y} \Pr(Y = y) \mathbb{E}[X | Y = y] = \mathbb{E}[\mathbb{E}[X | Y]]$
\end{enumerate}

\subsection{直观理解}

\begin{quote}
\textbf{全期望公式}:先对每个条件($Y = y$)计算条件期望,再对所有条件求平均(按照 $Y$ 的分布),得到 $X$ 的期望。这是"两层平均"的思想。
\end{quote}

\subsection{在强化学习中的应用}

\begin{itemize}
    \item \textbf{从求和到期望}:$\sum_{s} \mu(s) \cdots = \mathbb{E}_\pi[\cdots]$
    \item \textbf{从动作求和到动作期望}:$\sum_{a} \pi(a|S_t, \theta) \cdots = \mathbb{E}[\cdots | S_t]$
    \item \textbf{简化表达式}:使策略梯度定理从求和形式转换为期望形式
\end{itemize}

\end{document}

