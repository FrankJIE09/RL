\documentclass[12pt,a4paper]{article}
\usepackage[UTF8]{ctex}
\usepackage{amsmath}
\usepackage{amssymb}
\usepackage{amsthm}
\usepackage{geometry}
\usepackage{hyperref}
\usepackage{enumitem}
\usepackage{booktabs}

\geometry{left=2.5cm,right=2.5cm,top=2.5cm,bottom=2.5cm}

\title{公式13.1详细解释}
\subtitle{策略梯度方法的基本更新公式}
\author{强化学习笔记}
\date{\today}

\newtheorem{definition}{定义}
\newtheorem{theorem}{定理}
\newtheorem{example}{示例}
\newtheorem{remark}{注记}

\begin{document}

\maketitle

\tableofcontents
\newpage

\section{公式13.1}

\subsection{公式}

\begin{equation}
\theta_{t+1} = \theta_t + \alpha \widehat{\nabla J}(\theta_t)
\label{eq:13.1}
\end{equation}

其中:
\begin{itemize}
    \item $\theta_t \in \mathbb{R}^{d'}$:时刻 $t$ 的策略参数向量
    \item $\theta_{t+1} \in \mathbb{R}^{d'}$:时刻 $t+1$ 的策略参数向量
    \item $\alpha > 0$:步长参数(学习率)
    \item $\widehat{\nabla J}(\theta_t) \in \mathbb{R}^{d'}$:性能指标 $J(\theta)$ 关于参数 $\theta_t$ 的梯度的随机估计
    \item $\mathbb{E}[\widehat{\nabla J}(\theta_t)] \approx \nabla J(\theta_t)$:估计的期望近似等于真实梯度
\end{itemize}

\section{公式的组成部分}

\subsection{参数向量 $\theta_t$}

\textbf{定义}:
\begin{itemize}
    \item $\theta_t$ 是策略的参数向量
    \item 维度为 $d'$(策略参数的个数)
    \item 决定了策略 $\pi(a|s, \theta_t)$ 的行为
\end{itemize}

\textbf{通俗理解}:
\begin{itemize}
    \item $\theta_t$ 就像你的"做事方式的参数"
    \item 调整 $\theta_t$ 可以改变你的做事方式
    \item 例如:在soft-max策略中,$\theta_t$ 决定了动作偏好 $h(s, a, \theta_t)$
\end{itemize}

\textbf{例子}:
\begin{itemize}
    \item 如果策略是线性soft-max:$\pi(a|s, \theta) = \frac{e^{\theta^T x(s, a)}}{\sum_b e^{\theta^T x(s, b)}}$
    \item 那么 $\theta$ 就是线性权重向量
    \item 调整 $\theta$ 可以改变每个动作被选择的概率
\end{itemize}

\subsection{步长参数 $\alpha$}

\textbf{定义}:
\begin{itemize}
    \item $\alpha > 0$ 是步长参数(学习率)
    \item 控制参数更新的幅度
    \item 通常是一个小的正数(如 0.01, 0.1)
\end{itemize}

\textbf{作用}:
\begin{itemize}
    \item 如果 $\alpha$ 太大:更新幅度大,可能不稳定,甚至发散
    \item 如果 $\alpha$ 太小:更新幅度小,学习速度慢
    \item 需要选择合适的 $\alpha$ 来平衡学习速度和稳定性
\end{itemize}

\textbf{通俗理解}:
\begin{itemize}
    \item $\alpha$ 就像"调整做事方式的幅度"
    \item 如果调整幅度太大,可能改变太多,导致不稳定
    \item 如果调整幅度太小,改变太少,学习速度慢
\end{itemize}

\subsection{梯度估计 $\widehat{\nabla J}(\theta_t)$}

\textbf{定义}:
\begin{itemize}
    \item $\widehat{\nabla J}(\theta_t)$ 是性能指标 $J(\theta)$ 关于参数 $\theta_t$ 的梯度的随机估计
    \item 是一个 $d'$ 维向量
    \item 期望值近似等于真实梯度:$\mathbb{E}[\widehat{\nabla J}(\theta_t)] \approx \nabla J(\theta_t)$
\end{itemize}

\textbf{关键特性}:
\begin{itemize}
    \item \textbf{随机估计}:不是精确的梯度,而是基于样本的估计
    \item \textbf{无偏或近似无偏}:期望值应该等于或近似等于真实梯度
    \item \textbf{可以有不同的实现}:不同的策略梯度方法使用不同的估计方式
\end{itemize}

\textbf{通俗理解}:
\begin{itemize}
    \item $\widehat{\nabla J}(\theta_t)$ 就像"告诉你如何调整做事方式才能更好"
    \item 这是一个估计值,基于你观察到的结果
    \item 不同的方法(如REINFORCE、Actor-Critic)使用不同的估计方式
\end{itemize}

\section{公式的含义}

\subsection{梯度上升}

\textbf{目标}:最大化性能指标 $J(\theta)$

\textbf{方法}:梯度上升
\begin{itemize}
    \item 梯度 $\nabla J(\theta_t)$ 指向性能增加最快的方向
    \item 沿着梯度方向更新参数,使性能增加
    \item 这是优化问题的标准方法
\end{itemize}

\textbf{更新规则}:
\begin{equation}
\theta_{t+1} = \theta_t + \alpha \nabla J(\theta_t)
\end{equation}

\textbf{公式13.1的区别}:
\begin{itemize}
    \item 使用随机估计 $\widehat{\nabla J}(\theta_t)$ 而不是精确梯度 $\nabla J(\theta_t)$
    \item 这是随机梯度上升(Stochastic Gradient Ascent)
    \item 适用于无法精确计算梯度的情况
\end{itemize}

\subsection{随机梯度上升}

\textbf{为什么使用随机估计?}
\begin{itemize}
    \item 在强化学习中,通常无法精确计算梯度
    \item 需要从样本中估计梯度
    \item 使用随机估计是可行的替代方案
\end{itemize}

\textbf{随机梯度上升的特点}:
\begin{itemize}
    \item 使用随机估计而不是精确梯度
    \item 期望更新方向与梯度方向相同
    \item 在适当条件下可以收敛到局部最优
\end{itemize}

\textbf{收敛条件}:
\begin{itemize}
    \item 估计应该是无偏或近似无偏的
    \item 步长需要满足某些条件(如 Robbins-Monro 条件)
    \item 在适当条件下,算法会收敛
\end{itemize}

\section{公式的直观理解}

\subsection{通俗类比}

\textbf{类比1:爬山}

想象你在爬一座山(最大化性能):
\begin{itemize}
    \item $\theta_t$:你当前的位置(当前的做事方式)
    \item $J(\theta_t)$:你当前位置的高度(当前的性能)
    \item $\nabla J(\theta_t)$:最陡的上坡方向(性能增加最快的方向)
    \item $\widehat{\nabla J}(\theta_t)$:对最陡方向的估计(基于你观察到的信息)
    \item $\alpha$:你每步走的距离(调整幅度)
    \item $\theta_{t+1}$:你下一步的位置(更新后的做事方式)
\end{itemize}

\textbf{更新过程}:
\begin{enumerate}
    \item 你观察当前方向(估计梯度)
    \item 你朝着估计的最陡方向走一步(更新参数)
    \item 你到达新位置(新参数)
    \item 重复这个过程,逐渐接近山顶(最优策略)
\end{enumerate}

\textbf{类比2:学做菜}

想象你在学习如何做菜(优化策略):
\begin{itemize}
    \item $\theta_t$:你当前的做菜方式(如"放两勺盐")
    \item $J(\theta_t)$:你当前做菜的效果(如"8分")
    \item $\widehat{\nabla J}(\theta_t)$:告诉你如何调整做菜方式才能更好(如"增加放盐的量")
    \item $\alpha$:你调整的幅度(如"稍微增加")
    \item $\theta_{t+1}$:你调整后的做菜方式(如"放两勺半盐")
\end{itemize}

\textbf{更新过程}:
\begin{enumerate}
    \item 你按照当前方式做菜(使用策略 $\pi(\cdot|\cdot, \theta_t)$)
    \item 你观察结果,估计如何调整(计算 $\widehat{\nabla J}(\theta_t)$)
    \item 你稍微调整做菜方式(更新 $\theta_t$ 到 $\theta_{t+1}$)
    \item 你继续学习,逐渐找到最优做菜方式
\end{enumerate}

\section{不同策略梯度方法的实现}

\subsection{REINFORCE}

\textbf{梯度估计}:
\begin{equation}
\widehat{\nabla J}(\theta_t) = G_t \nabla_\theta \ln \pi(A_t|S_t, \theta_t)
\end{equation}

\textbf{更新公式}:
\begin{equation}
\theta_{t+1} = \theta_t + \alpha G_t \nabla_\theta \ln \pi(A_t|S_t, \theta_t)
\end{equation}

\textbf{特点}:
\begin{itemize}
    \item 使用完整回报 $G_t$
    \item 无偏估计
    \item 高方差
\end{itemize}

\subsection{带基线的REINFORCE}

\textbf{梯度估计}:
\begin{equation}
\widehat{\nabla J}(\theta_t) = [G_t - b(S_t)] \nabla_\theta \ln \pi(A_t|S_t, \theta_t)
\end{equation}

\textbf{更新公式}:
\begin{equation}
\theta_{t+1} = \theta_t + \alpha [G_t - b(S_t)] \nabla_\theta \ln \pi(A_t|S_t, \theta_t)
\end{equation}

\textbf{特点}:
\begin{itemize}
    \item 使用基线 $b(S_t)$ 减少方差
    \item 仍然是无偏估计
    \item 方差比REINFORCE小
\end{itemize}

\subsection{Actor-Critic}

\textbf{梯度估计}:
\begin{equation}
\widehat{\nabla J}(\theta_t) = \delta_t \nabla_\theta \ln \pi(A_t|S_t, \theta_t)
\end{equation}

其中 $\delta_t = R_{t+1} + \gamma \hat{v}(S_{t+1}, w) - \hat{v}(S_t, w)$ 是TD误差。

\textbf{更新公式}:
\begin{equation}
\theta_{t+1} = \theta_t + \alpha \delta_t \nabla_\theta \ln \pi(A_t|S_t, \theta_t)
\end{equation}

\textbf{特点}:
\begin{itemize}
    \item 使用TD误差而不是完整回报
    \item 有偏估计(因为使用了价值函数估计)
    \item 低方差
    \item 可以立即更新
\end{itemize}

\section{公式的数学性质}

\subsection{期望更新方向}

\textbf{关键性质}:
\begin{equation}
\mathbb{E}[\theta_{t+1} - \theta_t] = \mathbb{E}[\alpha \widehat{\nabla J}(\theta_t)] = \alpha \mathbb{E}[\widehat{\nabla J}(\theta_t)] \approx \alpha \nabla J(\theta_t)
\end{equation}

\textbf{含义}:
\begin{itemize}
    \item 参数更新的期望方向与梯度方向相同
    \item 这是随机梯度上升的核心性质
    \item 保证了算法朝着性能增加的方向更新
\end{itemize}

\subsection{收敛性}

\textbf{收敛条件}(Robbins-Monro条件):
\begin{enumerate}
    \item $\sum_{t=1}^{\infty} \alpha_t = \infty$(步长之和发散)
    \item $\sum_{t=1}^{\infty} \alpha_t^2 < \infty$(步长平方和收敛)
\end{enumerate}

\textbf{含义}:
\begin{itemize}
    \item 条件1:确保有足够的更新,能够到达最优解
    \item 条件2:确保步长逐渐减小,最终收敛
    \item 如果使用常数步长,条件2不满足,但可能仍然收敛(在适当条件下)
\end{itemize}

\textbf{收敛结果}:
\begin{itemize}
    \item 在适当条件下,算法会收敛到局部最优
    \item 对于策略梯度方法,通常收敛到局部最优策略
    \item 全局最优的保证需要额外的条件
\end{itemize}

\section{具体例子}

\subsection{例子1:简单的2动作问题}

\textbf{场景}:有两个动作 $a_1$ 和 $a_2$,策略参数 $\theta = [\theta_1, \theta_2]^T$。

\textbf{策略}:soft-max策略
\begin{equation}
\pi(a_1|s, \theta) = \frac{e^{\theta_1}}{e^{\theta_1} + e^{\theta_2}}, \quad \pi(a_2|s, \theta) = \frac{e^{\theta_2}}{e^{\theta_1} + e^{\theta_2}}
\end{equation}

\textbf{初始参数}:$\theta_0 = [0, 0]^T$(两个动作等概率)

\textbf{第1次更新}(使用REINFORCE):
\begin{enumerate}
    \item 在状态 $s$,选择动作 $a_1$(根据策略 $\pi(\cdot|s, \theta_0)$)
    \item 获得回报 $G_0 = 5$
    \item 计算梯度估计:
    \begin{align}
    \widehat{\nabla J}(\theta_0) &= G_0 \nabla_\theta \ln \pi(a_1|s, \theta_0) \\
                                  &= 5 \times [1 - \pi(a_1|s, \theta_0), -\pi(a_1|s, \theta_0)]^T \\
                                  &= 5 \times [0.5, -0.5]^T \\
                                  &= [2.5, -2.5]^T
    \end{align}
    \item 更新参数(假设 $\alpha = 0.1$):
    \begin{align}
    \theta_1 &= \theta_0 + \alpha \widehat{\nabla J}(\theta_0) \\
             &= [0, 0]^T + 0.1 \times [2.5, -2.5]^T \\
             &= [0.25, -0.25]^T
    \end{align}
    \item 新策略:$\pi(a_1|s, \theta_1) \approx 0.62$,$\pi(a_2|s, \theta_1) \approx 0.38$
    \item 动作 $a_1$ 的概率增加了(因为回报好)
\end{enumerate}

\subsection{例子2:参数更新的几何解释}

\textbf{参数空间}:
\begin{itemize}
    \item 参数空间是 $d'$ 维空间
    \item 每个点 $\theta$ 对应一个策略
    \item 性能指标 $J(\theta)$ 定义了一个"性能曲面"
\end{itemize}

\textbf{更新过程}:
\begin{enumerate}
    \item 当前位置:$\theta_t$(对应性能 $J(\theta_t)$)
    \item 梯度方向:$\nabla J(\theta_t)$(指向性能增加最快的方向)
    \item 估计方向:$\widehat{\nabla J}(\theta_t)$(对梯度方向的估计)
    \item 更新:$\theta_{t+1} = \theta_t + \alpha \widehat{\nabla J}(\theta_t)$
    \item 新位置:$\theta_{t+1}$(对应性能 $J(\theta_{t+1})$,通常 $J(\theta_{t+1}) > J(\theta_t)$)
\end{enumerate}

\textbf{可视化}:
\begin{itemize}
    \item 在2维参数空间中,可以画出等高线图
    \item 梯度方向垂直于等高线,指向更高性能的方向
    \item 更新沿着估计的梯度方向移动
\end{itemize}

\section{公式的变体}

\subsection{不同的步长策略}

\textbf{常数步长}:
\begin{equation}
\theta_{t+1} = \theta_t + \alpha \widehat{\nabla J}(\theta_t)
\end{equation}
\begin{itemize}
    \item $\alpha$ 是常数
    \item 简单,但可能不收敛
    \item 适合非平稳问题
\end{itemize}

\textbf{递减步长}:
\begin{equation}
\theta_{t+1} = \theta_t + \alpha_t \widehat{\nabla J}(\theta_t)
\end{equation}
\begin{itemize}
    \item $\alpha_t$ 随时间递减(如 $\alpha_t = \frac{1}{t}$)
    \item 满足 Robbins-Monro 条件
    \item 保证收敛
\end{itemize}

\textbf{自适应步长}:
\begin{equation}
\theta_{t+1} = \theta_t + \alpha_t \widehat{\nabla J}(\theta_t)
\end{equation}
\begin{itemize}
    \item $\alpha_t$ 根据梯度大小自适应调整
    \item 如 Adam 优化器
    \item 通常收敛更快
\end{itemize}

\subsection{不同的梯度估计}

\textbf{REINFORCE}:
\begin{equation}
\widehat{\nabla J}(\theta_t) = G_t \nabla_\theta \ln \pi(A_t|S_t, \theta_t)
\end{equation}

\textbf{带基线的REINFORCE}:
\begin{equation}
\widehat{\nabla J}(\theta_t) = [G_t - b(S_t)] \nabla_\theta \ln \pi(A_t|S_t, \theta_t)
\end{equation}

\textbf{Actor-Critic}:
\begin{equation}
\widehat{\nabla J}(\theta_t) = \delta_t \nabla_\theta \ln \pi(A_t|S_t, \theta_t)
\end{equation}

\textbf{自然策略梯度}:
\begin{equation}
\widehat{\nabla J}(\theta_t) = F^{-1}(\theta_t) \widehat{\nabla J}(\theta_t)
\end{equation}
其中 $F(\theta_t)$ 是 Fisher 信息矩阵。

\section{总结}

\subsection{核心要点}

\begin{enumerate}
    \item \textbf{公式13.1}是策略梯度方法的基本更新公式
    
    \item \textbf{梯度上升}:沿着性能增加最快的方向更新参数
    
    \item \textbf{随机估计}:使用随机估计而不是精确梯度
    
    \item \textbf{期望方向}:期望更新方向与梯度方向相同
    
    \item \textbf{不同实现}:不同的策略梯度方法使用不同的梯度估计
\end{enumerate}

\subsection{关键公式}

\textbf{基本更新公式}:
\begin{equation}
\theta_{t+1} = \theta_t + \alpha \widehat{\nabla J}(\theta_t)
\end{equation}

\textbf{期望更新方向}:
\begin{equation}
\mathbb{E}[\theta_{t+1} - \theta_t] = \alpha \mathbb{E}[\widehat{\nabla J}(\theta_t)] \approx \alpha \nabla J(\theta_t)
\end{equation}

\textbf{收敛条件}:
\begin{align}
\sum_{t=1}^{\infty} \alpha_t &= \infty \\
\sum_{t=1}^{\infty} \alpha_t^2 &< \infty
\end{align}

\subsection{直观理解}

\begin{quote}
\textbf{公式13.1}就像"根据观察到的结果,稍微调整做事方式,使效果更好"。

你当前的做法是 $\theta_t$,你观察结果并估计如何调整($\widehat{\nabla J}(\theta_t)$),你按照估计的方向稍微调整($\alpha \widehat{\nabla J}(\theta_t)$),你得到新的做法 $\theta_{t+1}$,你继续这个过程,逐渐找到最优做法。
\end{quote}

\end{document}

