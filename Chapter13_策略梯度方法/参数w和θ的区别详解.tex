\documentclass[12pt,a4paper]{article}
\usepackage[UTF8]{ctex}
\usepackage{amsmath}
\usepackage{amssymb}
\usepackage{amsthm}
\usepackage{geometry}
\usepackage{hyperref}
\usepackage{enumitem}
\usepackage{booktabs}

\geometry{left=2.5cm,right=2.5cm,top=2.5cm,bottom=2.5cm}

\title{参数$w$和$\theta$的区别详解}
\subtitle{Actor-Critic方法中两个参数的含义、区别和联系}
\author{强化学习笔记}
\date{\today}

\newtheorem{definition}{定义}
\newtheorem{theorem}{定理}
\newtheorem{example}{示例}
\newtheorem{remark}{注记}

\begin{document}

\maketitle

\tableofcontents
\newpage

\section{问题}

\textbf{问题}:在Actor-Critic方法中,$w$是什么?$\theta$是什么?它们有什么区别?

\section{参数定义}

\subsection{$\theta$:Actor的参数}

\textbf{定义}:
\begin{itemize}
    \item $\theta$ 是策略(policy)的参数
    \item 策略函数:$\pi(a|s, \theta)$
    \item 表示在状态 $s$ 下,选择动作 $a$ 的概率(或概率密度)
    \item $\theta$ 决定了策略的行为
\end{itemize}

\textbf{作用}:
\begin{itemize}
    \item 控制Actor如何选择动作
    \item 通过更新 $\theta$,可以改进策略
    \item 目标是找到最优策略参数 $\theta^*$,使得期望回报最大
\end{itemize}

\textbf{更新规则}:
\begin{equation}
\theta_{t+1} = \theta_t + \alpha \delta_t \nabla_\theta \ln \pi(A_t|S_t, \theta_t)
\label{eq:theta_update}
\end{equation}

其中:
\begin{itemize}
    \item $\alpha$ 是学习率(Actor的学习率)
    \item $\delta_t$ 是TD误差,由Critic提供
    \item $\nabla_\theta \ln \pi(A_t|S_t, \theta_t)$ 是策略梯度的对数形式
\end{itemize}

\subsection{$w$:Critic的参数}

\textbf{定义}:
\begin{itemize}
    \item $w$ 是价值函数(value function)的参数
    \item 价值函数:$\hat{v}(s, w)$
    \item 表示状态 $s$ 的价值估计
    \item $w$ 决定了价值函数的形状
\end{itemize}

\textbf{作用}:
\begin{itemize}
    \item 控制Critic如何评估状态价值
    \item 通过更新 $w$,可以改进价值估计
    \item 目标是找到最优价值函数参数 $w^*$,使得价值估计准确
\end{itemize}

\textbf{更新规则}:
\begin{equation}
w_{t+1} = w_t + \beta \delta_t \nabla_w \hat{v}(S_t, w_t)
\label{eq:w_update}
\end{equation}

其中:
\begin{itemize}
    \item $\beta$ 是学习率(Critic的学习率)
    \item $\delta_t$ 是TD误差
    \item $\nabla_w \hat{v}(S_t, w_t)$ 是价值函数关于参数 $w$ 的梯度
\end{itemize}

\section{主要区别}

\subsection{区别1:所属组件不同}

\textbf{$\theta$(Actor的参数)}:
\begin{itemize}
    \item 属于Actor组件
    \item 用于参数化策略 $\pi(a|s, \theta)$
    \item 控制动作选择
\end{itemize}

\textbf{$w$(Critic的参数)}:
\begin{itemize}
    \item 属于Critic组件
    \item 用于参数化价值函数 $\hat{v}(s, w)$
    \item 控制价值估计
\end{itemize}

\subsection{区别2:函数不同}

\textbf{$\theta$ 参数化的函数}:
\begin{equation}
\pi(a|s, \theta) : \mathcal{S} \times \Theta \to [0, 1]
\end{equation}

\begin{itemize}
    \item 输入:状态 $s$ 和参数 $\theta$
    \item 输出:动作概率分布(或动作概率密度)
    \item 类型:策略函数
\end{itemize}

\textbf{$w$ 参数化的函数}:
\begin{equation}
\hat{v}(s, w) : \mathcal{S} \times \mathcal{W} \to \mathbb{R}
\end{equation}

\begin{itemize}
    \item 输入:状态 $s$ 和参数 $w$
    \item 输出:状态价值(实数)
    \item 类型:价值函数
\end{itemize}

\subsection{区别3:输出不同}

\textbf{$\theta$ 的输出}:
\begin{itemize}
    \item 输出动作概率分布:$\pi(\cdot|s, \theta)$
    \item 例如:$\pi(\text{左}|s, \theta) = 0.3, \pi(\text{右}|s, \theta) = 0.7$
    \item 用于选择动作
\end{itemize}

\textbf{$w$ 的输出}:
\begin{itemize}
    \item 输出状态价值:$\hat{v}(s, w)$
    \item 例如:$\hat{v}(s, w) = 5.2$
    \item 用于评估状态
\end{itemize}

\subsection{区别4:更新目标不同}

\textbf{$\theta$ 的更新目标}:
\begin{itemize}
    \item 最大化期望回报:$J(\theta) = \mathbb{E}_\pi[G_t | S_0 = s_0]$
    \item 通过策略梯度方法更新
    \item 目标是找到最优策略
\end{itemize}

\textbf{$w$ 的更新目标}:
\begin{itemize}
    \item 最小化价值估计误差
    \item 通过TD误差更新
    \item 目标是准确估计状态价值
\end{itemize}

\subsection{区别5:更新方式不同}

\textbf{$\theta$ 的更新}:
\begin{equation}
\theta_{t+1} = \theta_t + \alpha \delta_t \nabla_\theta \ln \pi(A_t|S_t, \theta_t)
\end{equation}

\begin{itemize}
    \item 使用策略梯度:$\nabla_\theta \ln \pi(A_t|S_t, \theta_t)$
    \item 更新方向:沿着策略梯度方向(最大化回报)
    \item 学习率:$\alpha$(通常较小)
\end{itemize}

\textbf{$w$ 的更新}:
\begin{equation}
w_{t+1} = w_t + \beta \delta_t \nabla_w \hat{v}(S_t, w_t)
\end{equation}

\begin{itemize}
    \item 使用价值函数梯度:$\nabla_w \hat{v}(S_t, w_t)$
    \item 更新方向:沿着价值函数梯度方向(最小化误差)
    \item 学习率:$\beta$(通常较大)
\end{itemize}

\subsection{区别6:学习率不同}

\textbf{$\theta$ 的学习率 $\alpha$}:
\begin{itemize}
    \item 通常较小(例如:$\alpha = 0.001$)
    \item 因为策略更新需要谨慎,避免策略变化太快
    \item 策略变化太快会导致价值函数估计不准确
\end{itemize}

\textbf{$w$ 的学习率 $\beta$}:
\begin{itemize}
    \item 通常较大(例如:$\beta = 0.01$)
    \item 因为价值函数更新可以更快,需要快速适应策略变化
    \item 价值函数估计需要及时反映策略的变化
\end{itemize}

\textbf{关系}:
\begin{equation}
\beta > \alpha
\end{equation}

通常Critic的学习率比Actor的学习率大。

\section{具体例子}

\subsection{例子:神经网络参数化}

\textbf{场景}:使用神经网络来参数化策略和价值函数

\textbf{Actor网络(参数 $\theta$)}:
\begin{itemize}
    \item 输入:状态 $s$(例如:图像、特征向量)
    \item 输出:动作概率分布 $\pi(\cdot|s, \theta)$
    \item 参数:$\theta = \{W_1, b_1, W_2, b_2, \ldots\}$(权重和偏置)
    \item 例如:$\theta$ 包含1000个参数
\end{itemize}

\textbf{Critic网络(参数 $w$)}:
\begin{itemize}
    \item 输入:状态 $s$(例如:图像、特征向量)
    \item 输出:状态价值 $\hat{v}(s, w)$(标量)
    \item 参数:$w = \{W_1', b_1', W_2', b_2', \ldots\}$(权重和偏置)
    \item 例如:$w$ 包含800个参数
\end{itemize}

\textbf{区别}:
\begin{itemize}
    \item $\theta$ 和 $w$ 是完全不同的参数集合
    \item 它们属于不同的神经网络
    \item 它们有不同的结构和参数数量
\end{itemize}

\subsection{例子:线性函数近似}

\textbf{场景}:使用线性函数来近似策略和价值函数

\textbf{Actor(参数 $\theta$)}:
\begin{equation}
\pi(a|s, \theta) = \frac{\exp(\theta^T \phi(s, a))}{\sum_{a'} \exp(\theta^T \phi(s, a'))}
\end{equation}

\begin{itemize}
    \item $\theta$ 是权重向量(例如:$\theta \in \mathbb{R}^{100}$)
    \item $\phi(s, a)$ 是状态-动作特征向量
    \item 输出:动作概率分布
\end{itemize}

\textbf{Critic(参数 $w$)}:
\begin{equation}
\hat{v}(s, w) = w^T \phi(s)
\end{equation}

\begin{itemize}
    \item $w$ 是权重向量(例如:$w \in \mathbb{R}^{50}$)
    \item $\phi(s)$ 是状态特征向量
    \item 输出:状态价值(标量)
\end{itemize}

\textbf{区别}:
\begin{itemize}
    \item $\theta$ 和 $w$ 是不同的权重向量
    \item 它们有不同的维度($\theta$ 是100维,$w$ 是50维)
    \item 它们用于不同的函数(策略函数 vs 价值函数)
\end{itemize}

\subsection{例子:Gridworld中的参数}

\textbf{场景}:简单的Gridworld问题

\textbf{Actor(参数 $\theta$)}:
\begin{itemize}
    \item 状态:$s = (x, y)$(位置坐标)
    \item 动作:$a \in \{\text{上}, \text{下}, \text{左}, \text{右}\}$
    \item 策略:$\pi(a|s, \theta) = \text{softmax}(\theta^T \phi(s, a))$
    \item 参数:$\theta$ 是一个 $4 \times 2$ 的矩阵(4个动作,2个状态特征)
    \item 例如:$\theta = \begin{pmatrix} 0.5 & -0.3 \\ 0.2 & 0.1 \\ -0.1 & 0.4 \\ 0.3 & -0.2 \end{pmatrix}$
\end{itemize}

\textbf{Critic(参数 $w$)}:
\begin{itemize}
    \item 状态:$s = (x, y)$(位置坐标)
    \item 价值函数:$\hat{v}(s, w) = w^T \phi(s)$
    \item 参数:$w$ 是一个 $2 \times 1$ 的向量(2个状态特征)
    \item 例如:$w = \begin{pmatrix} 1.2 \\ -0.5 \end{pmatrix}$
\end{itemize}

\textbf{区别}:
\begin{itemize}
    \item $\theta$ 是 $4 \times 2$ 矩阵,用于策略函数
    \item $w$ 是 $2 \times 1$ 向量,用于价值函数
    \item 它们是完全不同的参数
\end{itemize}

\section{联系和协作}

\subsection{通过TD误差联系}

\textbf{TD误差}:
\begin{equation}
\delta_t = R_{t+1} + \gamma \hat{v}(S_{t+1}, w) - \hat{v}(S_t, w)
\label{eq:td_error}
\end{equation}

\textbf{联系}:
\begin{itemize}
    \item TD误差 $\delta_t$ 同时用于更新 $\theta$ 和 $w$
    \item $\theta$ 的更新依赖于Critic提供的 $\delta_t$
    \item $w$ 的更新也依赖于 $\delta_t$
    \item $\delta_t$ 是两者之间的桥梁
\end{itemize}

\subsection{相互依赖}

\textbf{$\theta$ 依赖 $w$}:
\begin{itemize}
    \item Actor需要Critic提供的TD误差 $\delta_t$ 来更新策略
    \item $\delta_t$ 的计算需要价值函数 $\hat{v}(s, w)$
    \item 如果 $w$ 不准确,$\delta_t$ 也不准确,导致 $\theta$ 的更新不准确
\end{itemize}

\textbf{$w$ 依赖 $\theta$}:
\begin{itemize}
    \item Critic需要Actor选择的动作来产生数据(状态转移、奖励)
    \item 如果 $\theta$ 不好,产生的数据质量差,导致 $w$ 的更新不准确
    \item 策略 $\pi(a|s, \theta)$ 决定了状态分布,影响价值函数的估计
\end{itemize}

\subsection{协同更新}

\textbf{同时更新}:
\begin{enumerate}
    \item Actor选择动作 $A_t$,与环境交互
    \item 获得状态 $S_{t+1}$ 和奖励 $R_{t+1}$
    \item Critic计算TD误差:$\delta_t = R_{t+1} + \gamma \hat{v}(S_{t+1}, w) - \hat{v}(S_t, w)$
    \item 同时更新 $\theta$ 和 $w$:
    \begin{align}
    \theta_{t+1} &= \theta_t + \alpha \delta_t \nabla_\theta \ln \pi(A_t|S_t, \theta_t) \\
    w_{t+1} &= w_t + \beta \delta_t \nabla_w \hat{v}(S_t, w_t)
    \end{align}
\end{enumerate}

\textbf{关键}:
\begin{itemize}
    \item $\theta$ 和 $w$ 同时更新,相互影响
    \item 它们需要协调工作,才能达到最优性能
    \item 如果一方更新太快或太慢,都会影响另一方的学习
\end{itemize}

\section{参数维度对比}

\subsection{维度可能不同}

\textbf{$\theta$ 的维度}:
\begin{itemize}
    \item 取决于策略函数的复杂度
    \item 例如:如果使用神经网络,$\theta$ 可能包含数千个参数
    \item 如果使用线性函数,$\theta$ 可能只有几十个参数
\end{itemize}

\textbf{$w$ 的维度}:
\begin{itemize}
    \item 取决于价值函数的复杂度
    \item 例如:如果使用神经网络,$w$ 可能包含数百个参数
    \item 如果使用线性函数,$w$ 可能只有几个参数
\end{itemize}

\textbf{关系}:
\begin{itemize}
    \item $\theta$ 和 $w$ 的维度通常不同
    \item 它们是完全独立的参数空间
    \item 没有直接的关系
\end{itemize}

\subsection{共享特征的情况}

\textbf{共享特征提取器}:
\begin{itemize}
    \item 有时Actor和Critic共享底层的特征提取器
    \item 例如:共享卷积层来提取图像特征
    \item 但 $\theta$ 和 $w$ 仍然不同(上层参数不同)
\end{itemize}

\textbf{例子}:
\begin{itemize}
    \item 共享部分:特征提取网络(例如:CNN的前几层)
    \item Actor专用:$\theta_{\text{actor}}$(策略头)
    \item Critic专用:$w_{\text{critic}}$(价值头)
    \item 总参数:$\{\theta_{\text{shared}}, \theta_{\text{actor}}, w_{\text{critic}}\}$
\end{itemize}

\section{总结}

\subsection{参数定义}

\textbf{$\theta$(Actor的参数)}:
\begin{itemize}
    \item 策略函数的参数:$\pi(a|s, \theta)$
    \item 控制动作选择
    \item 更新目标:最大化期望回报
    \item 更新方式:策略梯度
\end{itemize}

\textbf{$w$(Critic的参数)}:
\begin{itemize}
    \item 价值函数的参数:$\hat{v}(s, w)$
    \item 控制价值估计
    \item 更新目标:最小化价值估计误差
    \item 更新方式:TD误差
\end{itemize}

\subsection{主要区别}

\begin{table}[h]
\centering
\begin{tabular}{lcc}
\toprule
\textbf{特征} & \textbf{$\theta$(Actor)} & \textbf{$w$(Critic)} \\
\midrule
所属组件 & Actor & Critic \\
参数化函数 & 策略 $\pi(a|s, \theta)$ & 价值函数 $\hat{v}(s, w)$ \\
输出类型 & 动作概率分布 & 状态价值(标量) \\
更新目标 & 最大化期望回报 & 最小化价值估计误差 \\
更新方式 & 策略梯度 & TD误差 \\
学习率 & $\alpha$(通常较小) & $\beta$(通常较大) \\
\bottomrule
\end{tabular}
\caption{$\theta$ 和 $w$ 的主要区别}
\label{tab:comparison}
\end{table}

\subsection{联系和协作}

\begin{quote}
\textbf{$\theta$ 和 $w$ 通过TD误差 $\delta_t$ 联系}:
\begin{itemize}
    \item TD误差同时用于更新 $\theta$ 和 $w$
    \item $\theta$ 依赖 $w$ 提供的TD误差来更新策略
    \item $w$ 依赖 $\theta$ 产生的数据来更新价值函数
    \item 它们相互依赖、协同工作
\end{itemize}
\end{quote}

\subsection{关键公式}

\textbf{TD误差}:
\begin{equation}
\delta_t = R_{t+1} + \gamma \hat{v}(S_{t+1}, w) - \hat{v}(S_t, w)
\end{equation}

\textbf{Actor更新}:
\begin{equation}
\theta_{t+1} = \theta_t + \alpha \delta_t \nabla_\theta \ln \pi(A_t|S_t, \theta_t)
\end{equation}

\textbf{Critic更新}:
\begin{equation}
w_{t+1} = w_t + \beta \delta_t \nabla_w \hat{v}(S_t, w_t)
\end{equation}

\subsection{直观理解}

\begin{itemize}
    \item \textbf{$\theta$}:就像"演员"的"演技参数",决定了如何表演(选择动作)
    \item \textbf{$w$}:就像"评论家"的"评分标准",决定了如何评分(评估状态)
    \item \textbf{TD误差 $\delta_t$}:就像"评论家"给出的"评分",告诉"演员"表演的好坏
    \item \textbf{协同工作}:"演员"根据"评分"改进"演技","评论家"根据"表演"改进"评分标准"
\end{itemize}

\end{document}

