\documentclass[12pt,a4paper]{article}
\usepackage[UTF8]{ctex}
\usepackage{amsmath}
\usepackage{amssymb}
\usepackage{amsthm}
\usepackage{geometry}
\usepackage{hyperref}
\usepackage{enumitem}
\usepackage{booktabs}

\geometry{left=2.5cm,right=2.5cm,top=2.5cm,bottom=2.5cm}

\title{梯度上升直观理解:具体计算例子}
\subtitle{$\theta_t$、$J(\theta_t)$、$\nabla J(\theta_t)$ 的详细解释}
\author{强化学习笔记}
\date{\today}

\newtheorem{definition}{定义}
\newtheorem{example}{示例}

\begin{document}

\maketitle

\tableofcontents
\newpage

\section{引言}

本文通过具体的数值计算例子来解释梯度上升中的核心概念:
\begin{itemize}
    \item $\theta_t$:当前的位置(当前的做事方式)
    \item $J(\theta_t)$:当前位置的高度(当前的性能)
    \item $\nabla J(\theta_t)$:最陡的上坡方向(性能增加最快的方向)
\end{itemize}

\section{简单例子:2维参数空间}

\subsection{问题设置}

\textbf{场景}:学习做菜,需要决定放多少盐。

\textbf{策略参数化}:
\begin{itemize}
    \item 参数 $\theta = [\theta_1, \theta_2]^T$(2维)
    \item $\theta_1$:控制"放一勺盐"的偏好
    \item $\theta_2$:控制"放两勺盐"的偏好
    \item 策略:$\pi(\text{一勺}|\theta) = \frac{e^{\theta_1}}{e^{\theta_1} + e^{\theta_2}}$,$\pi(\text{两勺}|\theta) = \frac{e^{\theta_2}}{e^{\theta_1} + e^{\theta_2}}$
\end{itemize}

\textbf{性能函数}(假设已知):
\begin{equation}
J(\theta) = -(\theta_1 - 1)^2 - (\theta_2 - 2)^2 + 5
\end{equation}

这是一个简单的二次函数,在 $\theta = [1, 2]^T$ 处达到最大值 $J = 5$。

\subsection{当前位置 $\theta_t$}

\textbf{初始位置}:$\theta_0 = [0, 0]^T$

\textbf{含义}:
\begin{itemize}
    \item $\theta_1 = 0$:对"放一勺盐"的偏好为0
    \item $\theta_2 = 0$:对"放两勺盐"的偏好为0
    \item 策略:$\pi(\text{一勺}|\theta_0) = \frac{e^0}{e^0 + e^0} = 0.5$,$\pi(\text{两勺}|\theta_0) = 0.5$(等概率)
\end{itemize}

\textbf{可视化}:
\begin{center}
参数空间中的点:$\theta_0 = (0, 0)$
\end{center}

\subsection{当前位置的高度 $J(\theta_t)$}

\textbf{计算}:
\begin{align}
J(\theta_0) &= -(\theta_1 - 1)^2 - (\theta_2 - 2)^2 + 5 \\
            &= -(0 - 1)^2 - (0 - 2)^2 + 5 \\
            &= -1 - 4 + 5 \\
            &= 0
\end{align}

\textbf{含义}:
\begin{itemize}
    \item 当前位置的性能是 $0$
    \item 这是"做菜效果"的量化
    \item 性能越高,做菜效果越好
\end{itemize}

\textbf{可视化}:
\begin{center}
在"性能山"上,当前位置的高度是 $0$
\end{center}

\subsection{梯度 $\nabla J(\theta_t)$}

\textbf{梯度定义}:
\begin{equation}
\nabla J(\theta) = \left[\frac{\partial J}{\partial \theta_1}, \frac{\partial J}{\partial \theta_2}\right]^T
\end{equation}

\textbf{计算偏导数}:
\begin{align}
\frac{\partial J}{\partial \theta_1} &= \frac{\partial}{\partial \theta_1}[-(\theta_1 - 1)^2 - (\theta_2 - 2)^2 + 5] \\
                                     &= -2(\theta_1 - 1) \\
                                     &= -2(0 - 1) = 2
\end{align}

\begin{align}
\frac{\partial J}{\partial \theta_2} &= \frac{\partial}{\partial \theta_2}[-(\theta_1 - 1)^2 - (\theta_2 - 2)^2 + 5] \\
                                     &= -2(\theta_2 - 2) \\
                                     &= -2(0 - 2) = 4
\end{align}

\textbf{梯度}:
\begin{equation}
\nabla J(\theta_0) = [2, 4]^T
\end{equation}

\textbf{含义}:
\begin{itemize}
    \item 梯度是一个向量,指向性能增加最快的方向
    \item $\frac{\partial J}{\partial \theta_1} = 2$:增加 $\theta_1$ 会使性能增加(每增加1单位,性能增加约2)
    \item $\frac{\partial J}{\partial \theta_2} = 4$:增加 $\theta_2$ 会使性能增加更多(每增加1单位,性能增加约4)
    \item 梯度方向 $[2, 4]^T$ 指向性能增加最快的方向
\end{itemize}

\textbf{可视化}:
\begin{center}
在参数空间中,从点 $(0, 0)$ 出发,梯度 $[2, 4]^T$ 指向右上方
\end{center}

\subsection{更新:$\theta_{t+1} = \theta_t + \alpha \nabla J(\theta_t)$}

\textbf{假设}:$\alpha = 0.1$(步长)

\textbf{计算}:
\begin{align}
\theta_1 &= \theta_0 + \alpha \nabla J(\theta_0) \\
         &= [0, 0]^T + 0.1 \times [2, 4]^T \\
         &= [0, 0]^T + [0.2, 0.4]^T \\
         &= [0.2, 0.4]^T
\end{align}

\textbf{新位置的性能}:
\begin{align}
J(\theta_1) &= -(0.2 - 1)^2 - (0.4 - 2)^2 + 5 \\
            &= -(-0.8)^2 - (-1.6)^2 + 5 \\
            &= -0.64 - 2.56 + 5 \\
            &= 1.8
\end{align}

\textbf{性能提升}:
\begin{itemize}
    \item 旧性能:$J(\theta_0) = 0$
    \item 新性能:$J(\theta_1) = 1.8$
    \item 性能提升了 $1.8$!
\end{itemize}

\subsection{继续迭代}

\textbf{第2次迭代}:

\textbf{计算梯度}:
\begin{align}
\nabla J(\theta_1) &= [-2(\theta_1 - 1), -2(\theta_2 - 2)]^T \\
                   &= [-2(0.2 - 1), -2(0.4 - 2)]^T \\
                   &= [-2(-0.8), -2(-1.6)]^T \\
                   &= [1.6, 3.2]^T
\end{align}

\textbf{更新参数}:
\begin{align}
\theta_2 &= \theta_1 + \alpha \nabla J(\theta_1) \\
         &= [0.2, 0.4]^T + 0.1 \times [1.6, 3.2]^T \\
         &= [0.2, 0.4]^T + [0.16, 0.32]^T \\
         &= [0.36, 0.72]^T
\end{align}

\textbf{新位置的性能}:
\begin{align}
J(\theta_2) &= -(0.36 - 1)^2 - (0.72 - 2)^2 + 5 \\
            &= -(-0.64)^2 - (-1.28)^2 + 5 \\
            &= -0.4096 - 1.6384 + 5 \\
            &= 2.952
\end{align}

\textbf{性能提升}:
\begin{itemize}
    \item 旧性能:$J(\theta_1) = 1.8$
    \item 新性能:$J(\theta_2) = 2.952$
    \item 性能提升了 $1.152$
\end{itemize}

\textbf{迭代过程总结}:

\begin{center}
\begin{tabular}{|c|c|c|c|}
\hline
\textbf{迭代} & \textbf{$\theta_t$} & \textbf{$J(\theta_t)$} & \textbf{$\nabla J(\theta_t)$} \\
\hline
0 & $[0, 0]^T$ & $0$ & $[2, 4]^T$ \\
\hline
1 & $[0.2, 0.4]^T$ & $1.8$ & $[1.6, 3.2]^T$ \\
\hline
2 & $[0.36, 0.72]^T$ & $2.952$ & $[1.28, 2.56]^T$ \\
\hline
3 & $[0.488, 0.976]^T$ & $3.805$ & $[1.024, 2.048]^T$ \\
\hline
$\cdots$ & $\cdots$ & $\cdots$ & $\cdots$ \\
\hline
$\infty$ & $[1, 2]^T$ & $5$ & $[0, 0]^T$ \\
\hline
\end{tabular}
\end{center}

\textbf{观察}:
\begin{itemize}
    \item 参数逐渐接近最优值 $[1, 2]^T$
    \item 性能逐渐增加,接近最大值 $5$
    \item 梯度逐渐减小,接近 $[0, 0]^T$(在最优解处梯度为0)
\end{itemize}

\section{为什么梯度指向最陡的上坡方向?}

\subsection{方向导数的概念}

\textbf{问题}:从当前位置 $\theta_t$ 出发,在哪个方向性能增加最快?

\textbf{方向导数}:
\begin{equation}
\frac{\partial J}{\partial \mathbf{u}} = \nabla J(\theta_t) \cdot \mathbf{u}
\end{equation}

其中 $\mathbf{u}$ 是单位方向向量。

\textbf{关键结果}:
\begin{itemize}
    \item 方向导数在 $\mathbf{u} = \frac{\nabla J(\theta_t)}{\|\nabla J(\theta_t)\|}$ 时最大
    \item 即:梯度方向是性能增加最快的方向
\end{itemize}

\subsection{具体例子}

\textbf{当前位置}:$\theta_0 = [0, 0]^T$,梯度:$\nabla J(\theta_0) = [2, 4]^T$

\textbf{考虑不同方向}:

\textbf{方向1:沿着梯度方向} $\mathbf{u}_1 = \frac{[2, 4]^T}{\sqrt{2^2 + 4^2}} = \frac{[2, 4]^T}{\sqrt{20}} = [0.447, 0.894]^T$

方向导数:
\begin{align}
\frac{\partial J}{\partial \mathbf{u}_1} &= \nabla J(\theta_0) \cdot \mathbf{u}_1 \\
                                         &= [2, 4]^T \cdot [0.447, 0.894]^T \\
                                         &= 2 \times 0.447 + 4 \times 0.894 \\
                                         &= 0.894 + 3.576 = 4.47
\end{align}

\textbf{方向2:垂直方向} $\mathbf{u}_2 = [1, 0]^T$

方向导数:
\begin{align}
\frac{\partial J}{\partial \mathbf{u}_2} &= \nabla J(\theta_0) \cdot \mathbf{u}_2 \\
                                         &= [2, 4]^T \cdot [1, 0]^T \\
                                         &= 2 \times 1 + 4 \times 0 = 2
\end{align}

\textbf{方向3:另一个方向} $\mathbf{u}_3 = [0, 1]^T$

方向导数:
\begin{align}
\frac{\partial J}{\partial \mathbf{u}_3} &= \nabla J(\theta_0) \cdot \mathbf{u}_3 \\
                                         &= [2, 4]^T \cdot [0, 1]^T \\
                                         &= 2 \times 0 + 4 \times 1 = 4
\end{align}

\textbf{比较}:
\begin{itemize}
    \item 沿着梯度方向:方向导数 $= 4.47$(最大!)
    \item 沿着 $\theta_1$ 轴:方向导数 $= 2$
    \item 沿着 $\theta_2$ 轴:方向导数 $= 4$
    \item 梯度方向确实是性能增加最快的方向
\end{itemize}

\section{更直观的例子:1维情况}

\subsection{问题设置}

\textbf{场景}:只有一个参数 $\theta$(1维)

\textbf{性能函数}:
\begin{equation}
J(\theta) = -(\theta - 3)^2 + 10
\end{equation}

这是一个开口向下的抛物线,在 $\theta = 3$ 处达到最大值 $J = 10$。

\subsection{当前位置 $\theta_t$}

\textbf{初始位置}:$\theta_0 = 0$

\textbf{含义}:
\begin{itemize}
    \item 当前的做事方式对应参数值 $0$
    \item 例如:当前放盐的偏好是 $0$
\end{itemize}

\subsection{当前位置的高度 $J(\theta_t)$}

\textbf{计算}:
\begin{align}
J(\theta_0) &= -(0 - 3)^2 + 10 \\
            &= -9 + 10 \\
            &= 1
\end{align}

\textbf{含义}:
\begin{itemize}
    \item 当前位置的性能是 $1$
    \item 在"性能曲线"上,当前位置的高度是 $1$
\end{itemize}

\textbf{可视化}:
\begin{center}
性能曲线:$J(\theta) = -(\theta - 3)^2 + 10$ \\
在 $\theta = 0$ 处,$J = 1$(高度为1)
\end{center}

\subsection{梯度 $\nabla J(\theta_t)$}

\textbf{计算}:
\begin{align}
\nabla J(\theta_0) &= \frac{d}{d\theta}[-(\theta - 3)^2 + 10] \Big|_{\theta = 0} \\
                   &= -2(\theta - 3) \Big|_{\theta = 0} \\
                   &= -2(0 - 3) \\
                   &= 6
\end{align}

\textbf{含义}:
\begin{itemize}
    \item 梯度是 $6$(1维情况下是标量)
    \item 正数表示:增加 $\theta$ 会使性能增加
    \item 数值 $6$ 表示:每增加 $\theta$ 1单位,性能大约增加 $6$
    \item 这是"最陡的上坡方向"(在1维中,就是向右)
\end{itemize}

\textbf{可视化}:
\begin{center}
在 $\theta = 0$ 处,曲线的斜率是 $6$(向上倾斜)
\end{center}

\subsection{更新}

\textbf{假设}:$\alpha = 0.1$

\textbf{计算}:
\begin{align}
\theta_1 &= \theta_0 + \alpha \nabla J(\theta_0) \\
         &= 0 + 0.1 \times 6 \\
         &= 0.6
\end{align}

\textbf{新位置的性能}:
\begin{align}
J(\theta_1) &= -(0.6 - 3)^2 + 10 \\
            &= -(-2.4)^2 + 10 \\
            &= -5.76 + 10 \\
            &= 4.24
\end{align}

\textbf{性能提升}:
\begin{itemize}
    \item 旧性能:$J(\theta_0) = 1$
    \item 新性能:$J(\theta_1) = 4.24$
    \item 性能提升了 $3.24$!
\end{itemize}

\subsection{继续迭代}

\textbf{第2次迭代}:

梯度:
\begin{align}
\nabla J(\theta_1) &= -2(\theta_1 - 3) \\
                   &= -2(0.6 - 3) \\
                   &= -2(-2.4) = 4.8
\end{align}

更新:
\begin{align}
\theta_2 &= \theta_1 + \alpha \nabla J(\theta_1) \\
         &= 0.6 + 0.1 \times 4.8 \\
         &= 0.6 + 0.48 = 1.08
\end{align}

性能:
\begin{align}
J(\theta_2) &= -(1.08 - 3)^2 + 10 \\
            &= -(-1.92)^2 + 10 \\
            &= -3.6864 + 10 = 6.3136
\end{align}

\textbf{迭代过程}:

\begin{center}
\begin{tabular}{|c|c|c|c|}
\hline
\textbf{迭代} & \textbf{$\theta_t$} & \textbf{$J(\theta_t)$} & \textbf{$\nabla J(\theta_t)$} \\
\hline
0 & $0$ & $1$ & $6$ \\
\hline
1 & $0.6$ & $4.24$ & $4.8$ \\
\hline
2 & $1.08$ & $6.31$ & $3.84$ \\
\hline
3 & $1.464$ & $7.64$ & $3.072$ \\
\hline
$\cdots$ & $\cdots$ & $\cdots$ & $\cdots$ \\
\hline
$\infty$ & $3$ & $10$ & $0$ \\
\hline
\end{tabular}
\end{center}

\textbf{观察}:
\begin{itemize}
    \item 参数逐渐接近最优值 $\theta = 3$
    \item 性能逐渐增加,接近最大值 $J = 10$
    \item 梯度逐渐减小,接近 $0$(在最优解处梯度为0)
\end{itemize}

\section{为什么梯度是最陡方向?数学证明}

\subsection{方向导数}

\textbf{定义}:函数 $J(\theta)$ 在点 $\theta_t$ 沿方向 $\mathbf{u}$(单位向量)的方向导数为:
\begin{equation}
\frac{\partial J}{\partial \mathbf{u}} = \lim_{h \to 0} \frac{J(\theta_t + h\mathbf{u}) - J(\theta_t)}{h} = \nabla J(\theta_t) \cdot \mathbf{u}
\end{equation}

\textbf{关键}:方向导数等于梯度与方向向量的点积。

\subsection{最大化方向导数}

\textbf{问题}:在哪个方向 $\mathbf{u}$ 上,方向导数最大?

\textbf{使用柯西-施瓦茨不等式}:
\begin{equation}
|\nabla J(\theta_t) \cdot \mathbf{u}| \leq \|\nabla J(\theta_t)\| \|\mathbf{u}\| = \|\nabla J(\theta_t)\|
\end{equation}

等号成立当且仅当 $\mathbf{u} = \frac{\nabla J(\theta_t)}{\|\nabla J(\theta_t)\|}$。

\textbf{结论}:
\begin{itemize}
    \item 方向导数在 $\mathbf{u} = \frac{\nabla J(\theta_t)}{\|\nabla J(\theta_t)\|}$ 时最大
    \item 最大值为 $\|\nabla J(\theta_t)\|$
    \item 梯度方向确实是性能增加最快的方向
\end{itemize}

\section{具体例子:策略梯度中的计算}

\subsection{问题设置}

\textbf{场景}:2动作问题,策略参数 $\theta = [\theta_1, \theta_2]^T$

\textbf{策略}:
\begin{equation}
\pi(a_1|s, \theta) = \frac{e^{\theta_1}}{e^{\theta_1} + e^{\theta_2}}, \quad \pi(a_2|s, \theta) = \frac{e^{\theta_2}}{e^{\theta_1} + e^{\theta_2}}
\end{equation}

\textbf{假设性能函数}(简化,实际中需要通过采样估计):
\begin{equation}
J(\theta) = 5\pi(a_1|s, \theta) + 3\pi(a_2|s, \theta)
\end{equation}

这表示:动作 $a_1$ 的期望回报是 $5$,动作 $a_2$ 的期望回报是 $3$。

\subsection{当前位置}

\textbf{初始参数}:$\theta_0 = [0, 0]^T$

\textbf{策略}:
\begin{align}
\pi(a_1|s, \theta_0) &= \frac{e^0}{e^0 + e^0} = 0.5 \\
\pi(a_2|s, \theta_0) &= \frac{e^0}{e^0 + e^0} = 0.5
\end{align}

\textbf{性能}:
\begin{align}
J(\theta_0) &= 5 \times 0.5 + 3 \times 0.5 \\
            &= 2.5 + 1.5 = 4
\end{align}

\subsection{计算梯度}

\textbf{计算 $\frac{\partial J}{\partial \theta_1}$}:

首先计算 $\frac{\partial \pi(a_1|s, \theta)}{\partial \theta_1}$:
\begin{align}
\frac{\partial \pi(a_1|s, \theta)}{\partial \theta_1} &= \frac{\partial}{\partial \theta_1}\left[\frac{e^{\theta_1}}{e^{\theta_1} + e^{\theta_2}}\right] \\
&= \frac{e^{\theta_1}(e^{\theta_1} + e^{\theta_2}) - e^{\theta_1} \cdot e^{\theta_1}}{(e^{\theta_1} + e^{\theta_2})^2} \\
&= \frac{e^{\theta_1} e^{\theta_2}}{(e^{\theta_1} + e^{\theta_2})^2} \\
&= \pi(a_1|s, \theta) \pi(a_2|s, \theta)
\end{align}

在 $\theta_0 = [0, 0]^T$ 处:
\begin{align}
\frac{\partial \pi(a_1|s, \theta_0)}{\partial \theta_1} &= 0.5 \times 0.5 = 0.25
\end{align}

类似地:
\begin{align}
\frac{\partial \pi(a_2|s, \theta_0)}{\partial \theta_1} &= -\pi(a_1|s, \theta_0) \pi(a_2|s, \theta_0) = -0.25
\end{align}

因此:
\begin{align}
\frac{\partial J}{\partial \theta_1} &= 5 \times 0.25 + 3 \times (-0.25) \\
                                     &= 1.25 - 0.75 = 0.5
\end{align}

\textbf{计算 $\frac{\partial J}{\partial \theta_2}$}:

类似地:
\begin{align}
\frac{\partial \pi(a_1|s, \theta_0)}{\partial \theta_2} &= -0.25 \\
\frac{\partial \pi(a_2|s, \theta_0)}{\partial \theta_2} &= 0.25
\end{align}

因此:
\begin{align}
\frac{\partial J}{\partial \theta_2} &= 5 \times (-0.25) + 3 \times 0.25 \\
                                     &= -1.25 + 0.75 = -0.5
\end{align}

\textbf{梯度}:
\begin{equation}
\nabla J(\theta_0) = [0.5, -0.5]^T
\end{equation}

\textbf{含义}:
\begin{itemize}
    \item $\frac{\partial J}{\partial \theta_1} = 0.5 > 0$:增加 $\theta_1$ 会使性能增加
    \item $\frac{\partial J}{\partial \theta_2} = -0.5 < 0$:增加 $\theta_2$ 会使性能减少
    \item 应该增加 $\theta_1$,减少 $\theta_2$
    \item 这符合直觉:动作 $a_1$ 的回报更高(5 vs 3),应该增加选择 $a_1$ 的概率
\end{itemize}

\subsection{更新}

\textbf{假设}:$\alpha = 0.5$

\textbf{计算}:
\begin{align}
\theta_1 &= \theta_0 + \alpha \nabla J(\theta_0) \\
         &= [0, 0]^T + 0.5 \times [0.5, -0.5]^T \\
         &= [0, 0]^T + [0.25, -0.25]^T \\
         &= [0.25, -0.25]^T
\end{align}

\textbf{新策略}:
\begin{align}
\pi(a_1|s, \theta_1) &= \frac{e^{0.25}}{e^{0.25} + e^{-0.25}} \approx 0.562 \\
\pi(a_2|s, \theta_1) &= \frac{e^{-0.25}}{e^{0.25} + e^{-0.25}} \approx 0.438
\end{align}

\textbf{新性能}:
\begin{align}
J(\theta_1) &= 5 \times 0.562 + 3 \times 0.438 \\
            &= 2.81 + 1.314 = 4.124
\end{align}

\textbf{性能提升}:
\begin{itemize}
    \item 旧性能:$J(\theta_0) = 4$
    \item 新性能:$J(\theta_1) = 4.124$
    \item 性能提升了 $0.124$
    \item 动作 $a_1$ 的概率从 $0.5$ 增加到 $0.562$(符合预期)
\end{itemize}

\section{总结}

\subsection{核心概念}

\begin{enumerate}
    \item \textbf{$\theta_t$:当前位置}
    \begin{itemize}
        \item 当前的做事方式(策略参数)
        \item 在参数空间中是一个点
    \end{itemize}
    
    \item \textbf{$J(\theta_t)$:当前位置的高度}
    \begin{itemize}
        \item 当前的性能(做事的 effect)
        \item 在"性能山"上的高度
    \end{itemize}
    
    \item \textbf{$\nabla J(\theta_t)$:最陡的上坡方向}
    \begin{itemize}
        \item 性能增加最快的方向
        \item 是一个向量,指向更高的性能
        \item 数学上:方向导数最大的方向
    \end{itemize}
\end{enumerate}

\subsection{更新公式}

\begin{equation}
\theta_{t+1} = \theta_t + \alpha \nabla J(\theta_t)
\end{equation}

\textbf{含义}:
\begin{itemize}
    \item 从当前位置 $\theta_t$ 出发
    \item 沿着梯度方向(最陡上坡方向)走一步
    \item 步长为 $\alpha$
    \item 到达新位置 $\theta_{t+1}$,性能更高
\end{itemize}

\subsection{关键洞察}

\begin{quote}
\textbf{梯度方向是性能增加最快的方向},这是数学上的严格结果。沿着梯度方向更新参数,可以最快地提高性能。
\end{quote}

\end{document}

