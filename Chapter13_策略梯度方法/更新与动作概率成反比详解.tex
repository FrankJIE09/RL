\documentclass[12pt,a4paper]{article}
\usepackage[UTF8]{ctex}
\usepackage{amsmath}
\usepackage{amssymb}
\usepackage{amsthm}
\usepackage{geometry}
\usepackage{hyperref}
\usepackage{enumitem}
\usepackage{booktabs}
\usepackage{graphicx}

\geometry{left=2.5cm,right=2.5cm,top=2.5cm,bottom=2.5cm}

\title{更新与动作概率成反比详解}
\subtitle{REINFORCE算法中"与动作概率成反比"的直观理解}
\author{强化学习笔记}
\date{\today}

\newtheorem{definition}{定义}
\newtheorem{theorem}{定理}
\newtheorem{example}{示例}
\newtheorem{remark}{注记}

\begin{document}

\maketitle

\tableofcontents
\newpage

\section{问题}

\textbf{问题}:REINFORCE算法中,为什么说"更新与回报成正比,与动作概率成反比"?特别是"与动作概率成反比"怎么理解?

\textbf{REINFORCE更新规则}:
\begin{equation}
\theta_{t+1} = \theta_t + \alpha G_t \nabla_\theta \ln \pi(A_t|S_t, \theta_t)
\label{eq:reinforce}
\end{equation}

\textbf{关键项}:
\begin{equation}
\nabla_\theta \ln \pi(A_t|S_t, \theta_t) = \frac{\nabla_\theta \pi(A_t|S_t, \theta_t)}{\pi(A_t|S_t, \theta_t)}
\label{eq:log_gradient}
\end{equation}

\textbf{观察}:
\begin{itemize}
    \item 更新量 $\propto G_t$(与回报成正比)
    \item 更新量 $\propto \frac{1}{\pi(A_t|S_t, \theta_t)}$(与动作概率成反比)
\end{itemize}

\section{为什么与动作概率成反比?}

\subsection{核心原因:补偿采样偏差}

\textbf{问题}:
\begin{itemize}
    \item 我们使用样本(实际选择的动作)来估计梯度
    \item 如果某个动作被选择的概率大,它会被频繁采样
    \item 如果某个动作被选择的概率小,它会被很少采样
    \item 这会导致采样偏差:频繁选择的动作会被多次更新,而很少选择的动作很少被更新
\end{itemize}

\textbf{解决方案}:
\begin{itemize}
    \item 对于概率大的动作,每次更新的幅度应该小一些(除以大的概率)
    \item 对于概率小的动作,每次更新的幅度应该大一些(除以小的概率)
    \item 这样可以补偿采样频率的差异,确保所有动作的更新是公平的
\end{itemize}

\subsection{数学解释}

\textbf{期望更新}:
\begin{equation}
\mathbb{E}_\pi\left[\frac{G_t}{\pi(A_t|S_t, \theta)} \nabla_\theta \pi(A_t|S_t, \theta)\right] = \sum_{a} \pi(a|S_t, \theta) \frac{G_t}{\pi(a|S_t, \theta)} \nabla_\theta \pi(a|S_t, \theta)
\end{equation}

\textbf{关键}:
\begin{itemize}
    \item 如果 $A_t = a$,更新项是 $\frac{G_t}{\pi(a|S_t, \theta)} \nabla_\theta \pi(a|S_t, \theta)$
    \item 概率 $\pi(a|S_t, \theta)$ 在期望中作为权重出现
    \item 除以 $\pi(a|S_t, \theta)$ 后,权重被抵消,得到无偏估计
\end{itemize}

\textbf{验证}:
\begin{align}
\mathbb{E}_\pi\left[\frac{G_t}{\pi(A_t|S_t, \theta)} \nabla_\theta \pi(A_t|S_t, \theta)\right] &= \sum_{a} \pi(a|S_t, \theta) \frac{G_t}{\pi(a|S_t, \theta)} \nabla_\theta \pi(a|S_t, \theta) \\
                                                                                                &= \sum_{a} G_t \nabla_\theta \pi(a|S_t, \theta)
\end{align}

\textbf{结果}:
\begin{itemize}
    \item 每个动作的贡献都是 $G_t \nabla_\theta \pi(a|S_t, \theta)$
    \item 不依赖于动作被选择的概率
    \item 这是无偏的梯度估计
\end{itemize}

\section{直观理解}

\subsection{类比1:抽奖系统}

\textbf{场景}:
\begin{itemize}
    \item 有3个动作:$a_1, a_2, a_3$
    \item 选择概率:$\pi(a_1) = 0.8, \pi(a_2) = 0.15, \pi(a_3) = 0.05$
    \item 回报:$G_t = 10$(假设相同)
\end{itemize}

\textbf{问题}:
\begin{itemize}
    \item 如果直接使用 $G_t \nabla_\theta \pi(A_t|S_t, \theta)$ 更新
    \item 动作 $a_1$ 会被更新 $80\%$ 的时间
    \item 动作 $a_2$ 会被更新 $15\%$ 的时间
    \item 动作 $a_3$ 会被更新 $5\%$ 的时间
    \item 这会导致不公平:$a_1$ 被过度更新,$a_2$ 和 $a_3$ 被欠更新
\end{itemize}

\textbf{解决方案}:
\begin{itemize}
    \item 使用 $\frac{G_t}{\pi(A_t|S_t, \theta)} \nabla_\theta \pi(A_t|S_t, \theta)$ 更新
    \item 动作 $a_1$:更新量 $\propto \frac{1}{0.8} = 1.25$(较小)
    \item 动作 $a_2$:更新量 $\propto \frac{1}{0.15} = 6.67$(较大)
    \item 动作 $a_3$:更新量 $\propto \frac{1}{0.05} = 20$(很大)
\end{itemize}

\textbf{效果}:
\begin{itemize}
    \item 虽然 $a_1$ 被更新得更频繁,但每次更新的幅度较小
    \item 虽然 $a_2$ 和 $a_3$ 被更新得较少,但每次更新的幅度较大
    \item 长期来看,所有动作的更新是公平的
\end{itemize}

\textbf{数学验证}:
\begin{align}
\text{期望更新} &= 0.8 \times \frac{10}{0.8} \times \nabla_\theta \pi(a_1) + 0.15 \times \frac{10}{0.15} \times \nabla_\theta \pi(a_2) + 0.05 \times \frac{10}{0.05} \times \nabla_\theta \pi(a_3) \\
                &= 10 \times \nabla_\theta \pi(a_1) + 10 \times \nabla_\theta \pi(a_2) + 10 \times \nabla_\theta \pi(a_3) \\
                &= 10 \times \sum_{a} \nabla_\theta \pi(a)
\end{align}

每个动作的贡献都是 $10 \times \nabla_\theta \pi(a)$,公平!

\subsection{类比2:学习做菜}

\textbf{场景}:
\begin{itemize}
    \item 你在学习做3道菜:宫保鸡丁、麻婆豆腐、糖醋里脊
    \item 你经常做宫保鸡丁(概率 $0.7$),偶尔做麻婆豆腐(概率 $0.25$),很少做糖醋里脊(概率 $0.05$)
    \item 每次做菜后,根据味道好坏(回报)调整做法(更新参数)
\end{itemize}

\textbf{问题}:
\begin{itemize}
    \item 如果直接根据回报更新
    \item 宫保鸡丁会被更新 $70\%$ 的时间
    \item 麻婆豆腐会被更新 $25\%$ 的时间
    \item 糖醋里脊只会被更新 $5\%$ 的时间
    \item 结果:宫保鸡丁学得很好,但糖醋里脊学得很差
\end{itemize}

\textbf{解决方案}:
\begin{itemize}
    \item 使用"重要性权重":$\frac{1}{\text{选择概率}}$
    \item 宫保鸡丁:更新量 $\propto \frac{1}{0.7} = 1.43$(较小)
    \item 麻婆豆腐:更新量 $\propto \frac{1}{0.25} = 4$(较大)
    \item 糖醋里脊:更新量 $\propto \frac{1}{0.05} = 20$(很大)
\end{itemize}

\textbf{效果}:
\begin{itemize}
    \item 虽然你经常做宫保鸡丁,但每次调整的幅度较小
    \item 虽然你很少做糖醋里脊,但每次调整的幅度很大
    \item 长期来看,三道菜的学习进度是平衡的
\end{itemize}

\subsection{类比3:投票系统}

\textbf{场景}:
\begin{itemize}
    \item 有3个候选人:A、B、C
    \item 投票概率:$\Pr(A) = 0.6, \Pr(B) = 0.3, \Pr(C) = 0.1$
    \item 每次投票后,根据结果更新支持度
\end{itemize}

\textbf{问题}:
\begin{itemize}
    \item 如果直接根据投票结果更新
    \item 候选人A会被更新 $60\%$ 的时间
    \item 候选人B会被更新 $30\%$ 的时间
    \item 候选人C只会被更新 $10\%$ 的时间
    \item 结果:A的支持度被过度更新,C的支持度被欠更新
\end{itemize}

\textbf{解决方案}:
\begin{itemize}
    \item 使用"权重调整":$\frac{1}{\text{投票概率}}$
    \item 候选人A:更新量 $\propto \frac{1}{0.6} = 1.67$(较小)
    \item 候选人B:更新量 $\propto \frac{1}{0.3} = 3.33$(较大)
    \item 候选人C:更新量 $\propto \frac{1}{0.1} = 10$(很大)
\end{itemize}

\textbf{效果}:
\begin{itemize}
    \item 虽然A被投票的次数多,但每次更新的幅度较小
    \item 虽然C被投票的次数少,但每次更新的幅度很大
    \item 长期来看,所有候选人的更新是公平的
\end{itemize}

\section{数学推导}

\subsection{从期望形式到重要性采样}

\textbf{策略梯度定理(期望形式)}:
\begin{equation}
\nabla_\theta J(\theta) = \mathbb{E}_\pi\left[\sum_{a} q_\pi(S_t, a) \nabla_\theta \pi(a|S_t, \theta)\right]
\end{equation}

\textbf{引入动作概率}:
\begin{align}
\nabla_\theta J(\theta) &= \mathbb{E}_\pi\left[\sum_{a} \pi(a|S_t, \theta) q_\pi(S_t, a) \frac{\nabla_\theta \pi(a|S_t, \theta)}{\pi(a|S_t, \theta)}\right] \\
                         &= \mathbb{E}_\pi\left[\sum_{a} \pi(a|S_t, \theta) q_\pi(S_t, a) \nabla_\theta \ln \pi(a|S_t, \theta)\right]
\end{align}

\textbf{引入实际选择的动作}:
\begin{align}
\nabla_\theta J(\theta) &= \mathbb{E}_\pi\left[\sum_{a} \pi(a|S_t, \theta) q_\pi(S_t, a) \nabla_\theta \ln \pi(a|S_t, \theta)\right] \\
                         &= \mathbb{E}_\pi\left[q_\pi(S_t, A_t) \nabla_\theta \ln \pi(A_t|S_t, \theta)\right]
\end{align}

\textbf{使用回报}:
\begin{align}
\nabla_\theta J(\theta) &= \mathbb{E}_\pi\left[q_\pi(S_t, A_t) \nabla_\theta \ln \pi(A_t|S_t, \theta)\right] \\
                         &= \mathbb{E}_\pi\left[G_t \nabla_\theta \ln \pi(A_t|S_t, \theta)\right]
\end{align}

\textbf{展开对数梯度}:
\begin{equation}
\nabla_\theta J(\theta) = \mathbb{E}_\pi\left[G_t \frac{\nabla_\theta \pi(A_t|S_t, \theta)}{\pi(A_t|S_t, \theta)}\right]
\end{equation}

\textbf{关键观察}:
\begin{itemize}
    \item 更新项是 $\frac{G_t}{\pi(A_t|S_t, \theta)} \nabla_\theta \pi(A_t|S_t, \theta)$
    \item 分母是 $\pi(A_t|S_t, \theta)$,所以与动作概率成反比
    \item 这是重要性采样(Importance Sampling)的形式
\end{itemize}

\subsection{重要性采样解释}

\textbf{重要性采样}:
\begin{equation}
\mathbb{E}_{p(x)}[f(x)] = \mathbb{E}_{q(x)}\left[\frac{p(x)}{q(x)} f(x)\right]
\end{equation}

\textbf{应用到策略梯度}:
\begin{itemize}
    \item 我们想计算:$\sum_{a} q_\pi(S_t, a) \nabla_\theta \pi(a|S_t, \theta)$
    \item 这是关于均匀分布的期望
    \item 但我们只能从分布 $\pi(\cdot|S_t, \theta)$ 采样
    \item 所以需要使用重要性采样:
    \begin{align}
    \sum_{a} q_\pi(S_t, a) \nabla_\theta \pi(a|S_t, \theta) &= \sum_{a} \pi(a|S_t, \theta) \frac{1}{\pi(a|S_t, \theta)} q_\pi(S_t, a) \nabla_\theta \pi(a|S_t, \theta) \\
                                                              &= \mathbb{E}_\pi\left[\frac{1}{\pi(A_t|S_t, \theta)} q_\pi(S_t, A_t) \nabla_\theta \pi(A_t|S_t, \theta)\right]
    \end{align}
\end{itemize}

\textbf{结果}:
\begin{itemize}
    \item 重要性权重是 $\frac{1}{\pi(A_t|S_t, \theta)}$
    \item 这就是"与动作概率成反比"的来源
\end{itemize}

\section{具体例子}

\subsection{例子:3动作问题}

\textbf{场景}:
\begin{itemize}
    \item 状态 $S_t$,3个动作:$a_1, a_2, a_3$
    \item 策略:
    \begin{align}
    \pi(a_1|S_t, \theta) &= 0.7 \\
    \pi(a_2|S_t, \theta) &= 0.25 \\
    \pi(a_3|S_t, \theta) &= 0.05
    \end{align}
    \item 回报:$G_t = 10$(假设相同)
    \item 策略梯度(假设):$\nabla_\theta \pi(a_1|S_t, \theta) = [0.1, -0.1]^T$
\end{itemize}

\textbf{情况1:不使用重要性权重(错误)}

\textbf{更新规则}:
\begin{equation}
\theta_{t+1} = \theta_t + \alpha G_t \nabla_\theta \pi(A_t|S_t, \theta_t)
\end{equation}

\textbf{期望更新}:
\begin{align}
\mathbb{E}_\pi[G_t \nabla_\theta \pi(A_t|S_t, \theta)] &= 0.7 \times 10 \times [0.1, -0.1]^T + 0.25 \times 10 \times \nabla_\theta \pi(a_2) + 0.05 \times 10 \times \nabla_\theta \pi(a_3) \\
                                                         &= 7 \times [0.1, -0.1]^T + 2.5 \times \nabla_\theta \pi(a_2) + 0.5 \times \nabla_\theta \pi(a_3)
\end{align}

\textbf{问题}:
\begin{itemize}
    \item 动作 $a_1$ 的贡献是 $7 \times [0.1, -0.1]^T$(很大)
    \item 动作 $a_3$ 的贡献是 $0.5 \times \nabla_\theta \pi(a_3)$(很小)
    \item 这会导致不公平:$a_1$ 被过度更新,$a_3$ 被欠更新
\end{itemize}

\textbf{情况2:使用重要性权重(正确)}

\textbf{更新规则}:
\begin{equation}
\theta_{t+1} = \theta_t + \alpha G_t \frac{\nabla_\theta \pi(A_t|S_t, \theta_t)}{\pi(A_t|S_t, \theta_t)}
\end{equation}

\textbf{期望更新}:
\begin{align}
\mathbb{E}_\pi\left[G_t \frac{\nabla_\theta \pi(A_t|S_t, \theta)}{\pi(A_t|S_t, \theta)}\right] &= 0.7 \times 10 \times \frac{[0.1, -0.1]^T}{0.7} + 0.25 \times 10 \times \frac{\nabla_\theta \pi(a_2)}{0.25} + 0.05 \times 10 \times \frac{\nabla_\theta \pi(a_3)}{0.05} \\
                                                                                                  &= 10 \times [0.1, -0.1]^T + 10 \times \nabla_\theta \pi(a_2) + 10 \times \nabla_\theta \pi(a_3) \\
                                                                                                  &= 10 \times \sum_{a} \nabla_\theta \pi(a|S_t, \theta)
\end{align}

\textbf{效果}:
\begin{itemize}
    \item 每个动作的贡献都是 $10 \times \nabla_\theta \pi(a)$
    \item 不依赖于动作被选择的概率
    \item 这是公平的、无偏的梯度估计
\end{itemize}

\textbf{单次更新示例}:

假设实际选择的动作是 $a_3$(概率 $0.05$):

\textbf{不使用重要性权重}:
\begin{align}
\text{更新量} &= 10 \times \nabla_\theta \pi(a_3|S_t, \theta) \\
              &= 10 \times [0.05, -0.05]^T \\
              &= [0.5, -0.5]^T
\end{align}

\textbf{使用重要性权重}:
\begin{align}
\text{更新量} &= 10 \times \frac{\nabla_\theta \pi(a_3|S_t, \theta)}{0.05} \\
              &= 10 \times \frac{[0.05, -0.05]^T}{0.05} \\
              &= 10 \times [1, -1]^T \\
              &= [10, -10]^T
\end{align}

\textbf{对比}:
\begin{itemize}
    \item 不使用重要性权重:更新量是 $[0.5, -0.5]^T$(很小)
    \item 使用重要性权重:更新量是 $[10, -10]^T$(很大)
    \item 因为 $a_3$ 很少被选择,所以需要更大的更新量来补偿
\end{itemize}

\section{为什么需要成反比?}

\subsection{原因1:无偏性}

\textbf{目标}:
\begin{itemize}
    \item 我们想估计:$\sum_{a} q_\pi(S_t, a) \nabla_\theta \pi(a|S_t, \theta)$
    \item 这是对所有动作的求和
    \item 但我们只能从分布 $\pi(\cdot|S_t, \theta)$ 采样
\end{itemize}

\textbf{无偏估计}:
\begin{itemize}
    \item 使用重要性采样:$\frac{1}{\pi(A_t|S_t, \theta)} q_\pi(S_t, A_t) \nabla_\theta \pi(A_t|S_t, \theta)$
    \item 期望值:
    \begin{align}
    \mathbb{E}_\pi\left[\frac{1}{\pi(A_t|S_t, \theta)} q_\pi(S_t, A_t) \nabla_\theta \pi(A_t|S_t, \theta)\right] &= \sum_{a} \pi(a|S_t, \theta) \frac{1}{\pi(a|S_t, \theta)} q_\pi(S_t, a) \nabla_\theta \pi(a|S_t, \theta) \\
                                                                                                                    &= \sum_{a} q_\pi(S_t, a) \nabla_\theta \pi(a|S_t, \theta)
    \end{align}
    \item 这是无偏的!
\end{itemize}

\subsection{原因2:公平性}

\textbf{问题}:
\begin{itemize}
    \item 如果某个动作被选择的概率大,它会被频繁更新
    \item 如果某个动作被选择的概率小,它会被很少更新
    \item 这会导致不公平:频繁选择的动作被过度更新,很少选择的动作被欠更新
\end{itemize}

\textbf{解决方案}:
\begin{itemize}
    \item 对于概率大的动作,每次更新的幅度应该小一些(除以大的概率)
    \item 对于概率小的动作,每次更新的幅度应该大一些(除以小的概率)
    \item 这样可以补偿采样频率的差异,确保所有动作的更新是公平的
\end{itemize}

\subsection{原因3:方差控制}

\textbf{问题}:
\begin{itemize}
    \item 重要性权重 $\frac{1}{\pi(A_t|S_t, \theta)}$ 可能会很大(如果 $\pi(A_t|S_t, \theta)$ 很小)
    \item 这会导致高方差,使学习不稳定
\end{itemize}

\textbf{解决方案}:
\begin{itemize}
    \item 使用基线(baseline)来减少方差
    \item 例如:$G_t - b(S_t)$ 而不是 $G_t$
    \item 这不会改变期望值,但可以减少方差
\end{itemize}

\section{总结}

\subsection{关键理解}

\begin{quote}
\textbf{"与动作概率成反比"}的意思是:更新量 $\propto \frac{1}{\pi(A_t|S_t, \theta)}$。这是为了补偿采样偏差,确保所有动作的更新是公平的、无偏的。
\end{quote}

\subsection{核心原因}

\begin{enumerate}
    \item \textbf{无偏性}:使用重要性采样,确保梯度估计是无偏的
    \item \textbf{公平性}:补偿采样频率的差异,确保所有动作的更新是公平的
    \item \textbf{正确性}:这是从策略梯度定理严格推导出来的,不是随意设计的
\end{enumerate}

\subsection{直观理解}

\begin{itemize}
    \item \textbf{概率大的动作}:被频繁选择,但每次更新的幅度较小(除以大的概率)
    \item \textbf{概率小的动作}:被很少选择,但每次更新的幅度较大(除以小的概率)
    \item \textbf{长期效果}:所有动作的更新是平衡的、公平的
\end{itemize}

\subsection{数学形式}

\textbf{REINFORCE更新}:
\begin{equation}
\theta_{t+1} = \theta_t + \alpha G_t \frac{\nabla_\theta \pi(A_t|S_t, \theta_t)}{\pi(A_t|S_t, \theta_t)}
\end{equation}

\textbf{关键项}:
\begin{equation}
\frac{\nabla_\theta \pi(A_t|S_t, \theta_t)}{\pi(A_t|S_t, \theta_t)} = \nabla_\theta \ln \pi(A_t|S_t, \theta_t)
\end{equation}

\textbf{重要性权重}:
\begin{equation}
\frac{1}{\pi(A_t|S_t, \theta_t)}
\end{equation}

这就是"与动作概率成反比"的数学表达!

\end{document}

