\documentclass[12pt,a4paper]{article}
\usepackage[UTF8]{ctex}
\usepackage{amsmath}
\usepackage{amssymb}
\usepackage{amsthm}
\usepackage{geometry}
\usepackage{hyperref}
\usepackage{enumitem}
\usepackage{booktabs}
\usepackage{tikz}

\geometry{left=2.5cm,right=2.5cm,top=2.5cm,bottom=2.5cm}

\title{例题13.1:短走廊动作反转详解}
\subtitle{策略参数化 vs 动作价值参数化}
\author{强化学习笔记}
\date{\today}

\newtheorem{definition}{定义}
\newtheorem{theorem}{定理}
\newtheorem{example}{示例}
\newtheorem{remark}{注记}

\begin{document}

\maketitle

\tableofcontents
\newpage

\section{问题描述}

\subsection{环境设置}

\textbf{短走廊Gridworld}:

\begin{center}
\begin{tikzpicture}[scale=1.5]
    \draw[thick] (0,0) rectangle (4,1);
    \node at (0.5,0.5) {S};
    \node at (1.5,0.5) {$s_1$};
    \node at (2.5,0.5) {$s_2$};
    \node at (3.5,0.5) {G};
    \draw[->] (0.7,0.5) -- (1.3,0.5);
    \draw[->] (1.7,0.5) -- (2.3,0.5);
    \draw[->] (2.7,0.5) -- (3.3,0.5);
    \node[below] at (1,0) {左:不动};
    \node[below] at (2,0) {左$\leftrightarrow$右};
    \node[below] at (3,0) {正常};
\end{tikzpicture}
\end{center}

\textbf{状态空间}:
\begin{itemize}
    \item $S$:起始状态
    \item $s_1$:第1个非终止状态
    \item $s_2$:第2个非终止状态
    \item $G$:终止状态(目标)
\end{itemize}

\textbf{动作空间}:
\begin{itemize}
    \item 每个状态只有两个动作:\textbf{左}(left)和\textbf{右}(right)
\end{itemize}

\textbf{奖励}:
\begin{itemize}
    \item 每步奖励:$R = -1$
    \item 到达终止状态时episode结束
\end{itemize}

\subsection{关键特性:动作反转}

\textbf{状态 $S$(起始状态)}:
\begin{itemize}
    \item \textbf{右}:正常移动到 $s_1$
    \item \textbf{左}:撞墙,状态不变(仍在 $S$)
\end{itemize}

\textbf{状态 $s_1$(第1个非终止状态)}:
\begin{itemize}
    \item \textbf{关键}:动作被\textbf{反转}了!
    \item \textbf{右}:实际移动到\textbf{左}(移动到 $S$)
    \item \textbf{左}:实际移动到\textbf{右}(移动到 $s_2$)
\end{itemize}

\textbf{状态 $s_2$(第2个非终止状态)}:
\begin{itemize}
    \item \textbf{右}:正常移动到 $G$(终止状态)
    \item \textbf{左}:正常移动到 $s_1$
\end{itemize}

\section{函数逼近的挑战}

\subsection{特征定义}

\textbf{动作特征}:
\begin{align}
x(s, \text{right}) &= [1, 0]^T \quad \text{对所有 } s \\
x(s, \text{left}) &= [0, 1]^T \quad \text{对所有 } s
\end{align}

\textbf{关键问题}:
\begin{itemize}
    \item 所有状态的特征表示\textbf{完全相同}
    \item 无法区分不同状态
    \item 函数逼近器认为所有状态是相同的
\end{itemize}

\subsection{为什么这很困难?}

\textbf{问题}:
\begin{itemize}
    \item 在状态 $s_1$,动作"右"和"左"的效果被反转了
    \item 但在其他状态($S$ 和 $s_2$),动作是正常的
    \item 由于所有状态的特征相同,函数逼近器无法区分它们
    \item 必须对所有状态使用相同的策略
\end{itemize}

\textbf{挑战}:
\begin{itemize}
    \item 动作价值方法(如Q-learning)需要为每个状态-动作对学习不同的价值
    \item 但由于特征相同,所有状态的动作价值必须相同
    \item 这限制了策略的选择
\end{itemize}

\section{动作价值方法的限制}

\subsection{$\varepsilon$-贪婪方法的限制}

\textbf{动作价值参数化}:
\begin{equation}
Q(s, a) = \theta^T x(s, a)
\end{equation}

由于 $x(s, \text{right}) = [1, 0]^T$ 和 $x(s, \text{left}) = [0, 1]^T$ 对所有 $s$ 相同:
\begin{align}
Q(s, \text{right}) &= \theta_1 \quad \text{对所有 } s \\
Q(s, \text{left}) &= \theta_2 \quad \text{对所有 } s
\end{align}

\textbf{$\varepsilon$-贪婪策略}:
\begin{itemize}
    \item 如果 $\theta_1 > \theta_2$:在所有状态以概率 $1 - \varepsilon/2$ 选择"右"
    \item 如果 $\theta_2 > \theta_1$:在所有状态以概率 $1 - \varepsilon/2$ 选择"左"
    \item 如果 $\theta_1 = \theta_2$:在所有状态等概率选择
\end{itemize}

\textbf{限制}:
\begin{itemize}
    \item 只能选择两种策略之一:
    \begin{enumerate}
        \item 在所有状态以高概率选择"右"(概率 $1 - \varepsilon/2$)
        \item 在所有状态以高概率选择"左"(概率 $1 - \varepsilon/2$)
    \end{enumerate}
    \item 无法学习状态特定的概率
    \item 如果 $\varepsilon = 0.1$,则选择"右"的概率是 $0.95$,选择"左"的概率是 $0.05$
\end{itemize}

\subsection{性能分析}

\textbf{策略1:在所有状态以高概率选择"右"}($\varepsilon = 0.1$,概率 $0.95$)

\textbf{路径分析}:
\begin{enumerate}
    \item 从 $S$ 开始
    \item 在 $S$:以概率 $0.95$ 选择"右" $\to$ 移动到 $s_1$(奖励 $-1$)
    \item 在 $s_1$:以概率 $0.95$ 选择"右" $\to$ 但动作反转,实际移动到 $S$(奖励 $-1$)
    \item 可能陷入循环:$S \to s_1 \to S \to s_1 \to \cdots$
    \item 偶尔(概率 $0.05$)选择"左",可能到达 $s_2$,然后到达 $G$
    \item 期望步数很大,价值 $v(S) < -44$
\end{enumerate}

\textbf{策略2:在所有状态以高概率选择"左"}($\varepsilon = 0.1$,概率 $0.95$)

\textbf{路径分析}:
\begin{enumerate}
    \item 从 $S$ 开始
    \item 在 $S$:以概率 $0.95$ 选择"左" $\to$ 撞墙,状态不变(奖励 $-1$)
    \item 可能长时间停留在 $S$,无法前进
    \item 偶尔(概率 $0.05$)选择"右",可能到达 $s_1$,然后可能到达 $s_2$,最后到达 $G$
    \item 期望步数更大,价值 $v(S) < -82$
\end{enumerate}

\section{策略参数化的优势}

\subsection{策略参数化}

\textbf{策略参数化}:
\begin{equation}
\pi(a|s, \theta) = \frac{e^{h(s, a, \theta)}}{\sum_{b} e^{h(s, b, \theta)}}
\end{equation}

\textbf{关键}:
\begin{itemize}
    \item 即使特征相同,策略参数化可以学习状态特定的概率
    \item 可以通过调整参数 $\theta$ 来改变每个状态选择动作的概率
    \item 不需要为每个状态学习不同的动作价值
\end{itemize}

\subsection{最优随机策略}

\textbf{问题}:在所有状态,应该以什么概率选择"右"?

\textbf{分析}:
\begin{itemize}
    \item 由于所有状态特征相同,策略必须在所有状态使用相同的概率
    \item 但可以通过调整这个概率来优化性能
    \item 最优概率不是 $0.95$($\varepsilon$-贪婪)或 $0.05$,而是某个中间值
\end{itemize}

\textbf{最优概率}:
\begin{itemize}
    \item 通过计算,最优概率约为 $0.59$
    \item 即:在所有状态以概率 $0.59$ 选择"右",以概率 $0.41$ 选择"左"
    \item 这个策略的价值:$v(S) \approx -11.6$
\end{itemize}

\textbf{为什么这个概率最优?}
\begin{itemize}
    \item 概率 $0.59$ 平衡了不同路径的期望回报
    \item 既不会像"总是右"那样陷入循环
    \item 也不会像"总是左"那样无法前进
    \item 找到了一个平衡点,使得期望步数最小
\end{itemize}

\section{详细计算}

\subsection{策略定义}

\textbf{策略}:在所有状态以概率 $p$ 选择"右",以概率 $1-p$ 选择"左"

\begin{equation}
\pi(\text{right}|s, p) = p, \quad \pi(\text{left}|s, p) = 1-p \quad \text{对所有 } s
\end{equation}

\subsection{状态价值函数}

\textbf{状态 $G$(终止状态)}:
\begin{equation}
v(G) = 0
\end{equation}

\textbf{状态 $s_2$}:
\begin{align}
v(s_2) &= \pi(\text{right}|s_2) \times [R + \gamma v(G)] + \pi(\text{left}|s_2) \times [R + \gamma v(s_1)] \\
       &= p \times [-1 + 0] + (1-p) \times [-1 + \gamma v(s_1)] \\
       &= -1 + (1-p) \gamma v(s_1)
\end{align}

\textbf{状态 $s_1$}(动作反转):
\begin{align}
v(s_1) &= \pi(\text{right}|s_1) \times [R + \gamma v(S)] + \pi(\text{left}|s_1) \times [R + \gamma v(s_2)] \\
       &= p \times [-1 + \gamma v(S)] + (1-p) \times [-1 + \gamma v(s_2)] \\
       &= -1 + p \gamma v(S) + (1-p) \gamma v(s_2)
\end{align}

注意:在 $s_1$,动作"右"实际移动到 $S$,动作"左"实际移动到 $s_2$。

\textbf{状态 $S$(起始状态)}:
\begin{align}
v(S) &= \pi(\text{right}|S) \times [R + \gamma v(s_1)] + \pi(\text{left}|S) \times [R + \gamma v(S)] \\
     &= p \times [-1 + \gamma v(s_1)] + (1-p) \times [-1 + \gamma v(S)] \\
     &= -1 + p \gamma v(s_1) + (1-p) \gamma v(S)
\end{align}

注意:在 $S$,动作"左"撞墙,状态不变。

\subsection{求解最优概率}

\textbf{目标}:找到 $p$ 使得 $v(S)$ 最大(或使得期望步数最小)

\textbf{方法}:
\begin{itemize}
    \item 对于每个 $p$,求解线性方程组得到 $v(S)$
    \item 找到使 $v(S)$ 最大的 $p$
\end{itemize}

\textbf{结果}:
\begin{itemize}
    \item 最优概率:$p^* \approx 0.59$
    \item 最优价值:$v^*(S) \approx -11.6$
\end{itemize}

\subsection{性能对比}

\begin{center}
\begin{tabular}{|l|l|l|}
\hline
\textbf{方法} & \textbf{策略} & \textbf{价值 $v(S)$} \\
\hline
$\varepsilon$-贪婪(右) & $p = 0.95$ & $< -44$ \\
\hline
$\varepsilon$-贪婪(左) & $p = 0.05$ & $< -82$ \\
\hline
\textbf{最优随机策略} & $p = 0.59$ & $\approx -11.6$ \\
\hline
\end{tabular}
\end{center}

\textbf{关键观察}:
\begin{itemize}
    \item 最优随机策略($p = 0.59$)比 $\varepsilon$-贪婪方法好得多
    \item 策略参数化可以学习这个最优概率
    \item 动作价值方法($\varepsilon$-贪婪)无法学习这个最优概率
\end{itemize}

\section{为什么策略参数化更好?}

\subsection{灵活性}

\textbf{策略参数化}:
\begin{itemize}
    \item 可以直接学习任意概率分布
    \item 可以学习 $p = 0.59$ 这样的最优概率
    \item 不受 $\varepsilon$-贪婪的限制
\end{itemize}

\textbf{动作价值参数化}:
\begin{itemize}
    \item 受限于 $\varepsilon$-贪婪策略
    \item 只能选择 $p = 1 - \varepsilon/2$ 或 $p = \varepsilon/2$
    \item 无法学习中间的最优概率
\end{itemize}

\subsection{问题本质}

\textbf{这个问题的特点}:
\begin{itemize}
    \item 所有状态在函数逼近下看起来相同
    \item 但不同状态需要不同的动作(由于动作反转)
    \item 最优策略是随机的,且概率是特定的($p = 0.59$)
\end{itemize}

\textbf{策略参数化的优势}:
\begin{itemize}
    \item 可以学习随机最优策略
    \item 可以学习特定的概率值
    \item 不受动作价值估计的限制
\end{itemize}

\section{策略梯度方法的应用}

\subsection{REINFORCE在这个问题中}

\textbf{策略参数化}:
\begin{equation}
\pi(\text{right}|s, \theta) = \frac{e^{\theta}}{e^{\theta} + e^{-\theta}} = \frac{1}{1 + e^{-2\theta}}
\end{equation}

这是一个简化的参数化,其中:
\begin{itemize}
    \item $\theta$ 控制选择"右"的概率
    \item 如果 $\theta = 0$,则 $p = 0.5$(等概率)
    \item 如果 $\theta > 0$,则 $p > 0.5$(偏向"右")
    \item 如果 $\theta < 0$,则 $p < 0.5$(偏向"左")
\end{itemize}

\textbf{REINFORCE更新}:
\begin{equation}
\theta_{t+1} = \theta_t + \alpha G_t \nabla_\theta \ln \pi(A_t|S_t, \theta_t)
\end{equation}

\textbf{学习过程}:
\begin{itemize}
    \item 从随机初始值 $\theta_0$ 开始
    \item 根据回报 $G_t$ 调整 $\theta$
    \item 逐渐收敛到最优值 $\theta^*$,使得 $p^* \approx 0.59$
\end{itemize}

\subsection{实验结果}

\textbf{REINFORCE的表现}:
\begin{itemize}
    \item 可以学习到接近最优的策略($p \approx 0.59$)
    \item 价值 $v(S)$ 接近 $-11.6$
    \item 比 $\varepsilon$-贪婪方法好得多
\end{itemize}

\textbf{带基线的REINFORCE}:
\begin{itemize}
    \item 使用基线可以减少方差
    \item 学习更快,更稳定
    \item 仍然可以学习到最优策略
\end{itemize}

\section{关键洞察}

\subsection{策略参数化的优势}

\textbf{1. 可以学习随机最优策略}
\begin{itemize}
    \item 在这个问题中,最优策略是随机的($p = 0.59$)
    \item 策略参数化可以学习这个随机策略
    \item 动作价值方法无法自然地找到随机最优策略
\end{itemize}

\textbf{2. 不受动作价值估计的限制}
\begin{itemize}
    \item 策略参数化不依赖于动作价值的准确估计
    \item 可以直接优化策略参数
    \item 即使动作价值估计不准确,也可以学习好的策略
\end{itemize}

\textbf{3. 可以学习状态特定的概率}
\begin{itemize}
    \item 虽然在这个问题中所有状态使用相同概率
    \item 但策略参数化可以学习状态特定的概率
    \item 这在更复杂的问题中很有用
\end{itemize}

\subsection{动作价值方法的限制}

\textbf{1. 受限于$\varepsilon$-贪婪}
\begin{itemize}
    \item 只能选择 $p = 1 - \varepsilon/2$ 或 $p = \varepsilon/2$
    \item 无法学习中间的最优概率
    \item 在这个问题中表现很差
\end{itemize}

\textbf{2. 需要准确的动作价值估计}
\begin{itemize}
    \item 由于特征相同,所有状态的动作价值必须相同
    \item 这限制了策略的选择
    \item 无法学习状态特定的策略
\end{itemize}

\section{总结}

\subsection{问题要点}

\begin{enumerate}
    \item \textbf{环境}:短走廊,在中间状态动作被反转
    
    \item \textbf{挑战}:所有状态特征相同,无法区分
    
    \item \textbf{动作价值方法}:受限于 $\varepsilon$-贪婪,只能选择 $p = 0.95$ 或 $p = 0.05$,性能差($v(S) < -44$ 或 $< -82$)
    
    \item \textbf{策略参数化方法}:可以学习最优概率 $p = 0.59$,性能好($v(S) \approx -11.6$)
\end{enumerate}

\subsection{关键结论}

\begin{quote}
\textbf{策略参数化的优势}:在某些问题中,策略函数比动作价值函数更简单,策略参数化方法可以学习随机最优策略,而动作价值方法受限于 $\varepsilon$-贪婪策略,无法学习最优的随机策略。
\end{quote}

\subsection{实际意义}

\begin{itemize}
    \item 这个例子说明了策略梯度方法的一个重要优势
    \item 在某些问题中,策略参数化比动作价值参数化更合适
    \item 策略梯度方法可以学习随机最优策略
    \item 这在许多实际问题中很重要(如扑克中的虚张声势)
\end{itemize}

\end{document}

