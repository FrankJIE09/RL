\documentclass[12pt,a4paper]{article}
\usepackage[UTF8]{ctex}
\usepackage{amsmath}
\usepackage{amssymb}
\usepackage{amsthm}
\usepackage{geometry}
\usepackage{hyperref}
\usepackage{enumitem}
\usepackage{booktabs}

\geometry{left=2.5cm,right=2.5cm,top=2.5cm,bottom=2.5cm}

\title{Critic更新公式详解}
\subtitle{$w_{t+1} = w_t + \beta \delta_t \nabla_w \hat{v}(S_t, w_t)$ 的详细解释}
\author{强化学习笔记}
\date{\today}

\newtheorem{definition}{定义}
\newtheorem{theorem}{定理}
\newtheorem{example}{示例}
\newtheorem{remark}{注记}

\begin{document}

\maketitle

\tableofcontents
\newpage

\section{问题}

\textbf{问题}:如何理解Critic的更新公式?
\begin{equation}
w_{t+1} = w_t + \beta \delta_t \nabla_w \hat{v}(S_t, w_t)
\label{eq:critic_update}
\end{equation}

\section{公式组成}

\subsection{公式分解}

\begin{equation}
w_{t+1} = w_t + \beta \delta_t \nabla_w \hat{v}(S_t, w_t)
\end{equation}

\textbf{组成部分}:
\begin{itemize}
    \item $w_t$:当前时刻的价值函数参数
    \item $w_{t+1}$:下一时刻的价值函数参数(更新后)
    \item $\beta$:学习率(Critic的学习率)
    \item $\delta_t$:TD误差(Temporal Difference Error)
    \item $\nabla_w \hat{v}(S_t, w_t)$:价值函数关于参数 $w$ 的梯度
\end{itemize}

\subsection{各部分的含义}

\textbf{$w_t$ 和 $w_{t+1}$}:
\begin{itemize}
    \item $w_t$:当前时刻的价值函数参数
    \item $w_{t+1}$:更新后的价值函数参数
    \item 通过更新,$w$ 会逐渐接近最优值 $w^*$
\end{itemize}

\textbf{$\beta$(学习率)}:
\begin{itemize}
    \item Critic的学习率,控制更新步长
    \item 通常 $\beta > \alpha$(Actor的学习率)
    \item 例如:$\beta = 0.01, 0.1, 0.5$ 等
    \item 如果 $\beta$ 太大,更新不稳定;如果 $\beta$ 太小,学习太慢
\end{itemize}

\textbf{$\delta_t$(TD误差)}:
\begin{equation}
\delta_t = R_{t+1} + \gamma \hat{v}(S_{t+1}, w_t) - \hat{v}(S_t, w_t)
\label{eq:td_error}
\end{equation}

\begin{itemize}
    \item 表示实际回报与预期回报的差异
    \item $\delta_t > 0$:实际回报比预期好,价值估计偏低
    \item $\delta_t < 0$:实际回报比预期差,价值估计偏高
    \item $\delta_t = 0$:价值估计准确
\end{itemize}

\textbf{$\nabla_w \hat{v}(S_t, w_t)$(梯度)}:
\begin{itemize}
    \item 价值函数关于参数 $w$ 的梯度
    \item 表示在参数空间 $w$ 中,哪个方向可以最快增加 $\hat{v}(S_t, w_t)$
    \item 是一个向量,维度与 $w$ 相同
\end{itemize}

\section{为什么这样更新?}

\subsection{目标:最小化TD误差}

\textbf{目标函数}:
\begin{equation}
L(w) = \frac{1}{2} \delta_t^2 = \frac{1}{2} [R_{t+1} + \gamma \hat{v}(S_{t+1}, w) - \hat{v}(S_t, w)]^2
\label{eq:loss_function}
\end{equation}

\textbf{含义}:
\begin{itemize}
    \item 我们希望最小化TD误差的平方
    \item 如果 $\delta_t = 0$,说明价值估计准确
    \item 通过最小化 $L(w)$,我们可以改进价值函数
\end{itemize}

\subsection{梯度下降}

\textbf{梯度下降更新}:
\begin{equation}
w_{t+1} = w_t - \alpha \nabla_w L(w_t)
\label{eq:gradient_descent}
\end{equation}

\textbf{计算梯度}:
\begin{align}
\nabla_w L(w) &= \nabla_w \left[\frac{1}{2} \delta_t^2\right] \\
              &= \delta_t \nabla_w \delta_t \\
              &= \delta_t \nabla_w [R_{t+1} + \gamma \hat{v}(S_{t+1}, w) - \hat{v}(S_t, w)] \\
              &= \delta_t [\gamma \nabla_w \hat{v}(S_{t+1}, w) - \nabla_w \hat{v}(S_t, w)]
\end{align}

\textbf{简化(半梯度方法)}:
\begin{itemize}
    \item 在实际应用中,通常只考虑当前状态 $S_t$ 的梯度
    \item 忽略下一状态 $S_{t+1}$ 的梯度(因为 $S_{t+1}$ 可能还没有更新)
    \item 这称为"半梯度"(semi-gradient)方法
\end{itemize}

\textbf{半梯度更新}:
\begin{align}
\nabla_w L(w) &\approx -\delta_t \nabla_w \hat{v}(S_t, w) \\
w_{t+1} &= w_t - \beta (-\delta_t \nabla_w \hat{v}(S_t, w)) \\
        &= w_t + \beta \delta_t \nabla_w \hat{v}(S_t, w)
\end{align}

这就是Critic的更新公式!

\subsection{直观理解}

\textbf{更新方向}:
\begin{itemize}
    \item 如果 $\delta_t > 0$:实际回报比预期好,价值估计偏低
    \begin{itemize}
        \item 更新方向:$+\delta_t \nabla_w \hat{v}(S_t, w_t)$(增加 $\hat{v}(S_t, w_t)$)
        \item 结果:提高当前状态的价值估计
    \end{itemize}
    \item 如果 $\delta_t < 0$:实际回报比预期差,价值估计偏高
    \begin{itemize}
        \item 更新方向:$+\delta_t \nabla_w \hat{v}(S_t, w_t)$(减少 $\hat{v}(S_t, w_t)$)
        \item 结果:降低当前状态的价值估计
    \end{itemize}
    \item 如果 $\delta_t = 0$:价值估计准确
    \begin{itemize}
        \item 更新量:$0$(不更新)
        \item 结果:保持当前参数不变
    \end{itemize}
\end{itemize}

\section{详细推导}

\subsection{从TD学习到Critic更新}

\textbf{TD(0)更新}:
\begin{equation}
v(S_t) \leftarrow v(S_t) + \alpha [R_{t+1} + \gamma v(S_{t+1}) - v(S_t)]
\label{eq:td0_update}
\end{equation}

\textbf{函数近似}:
\begin{itemize}
    \item 在函数近似中,$v(S_t)$ 被替换为 $\hat{v}(S_t, w)$
    \item 我们需要更新参数 $w$,而不是直接更新价值函数
\end{itemize}

\textbf{参数更新}:
\begin{equation}
w_{t+1} = w_t + \alpha [R_{t+1} + \gamma \hat{v}(S_{t+1}, w_t) - \hat{v}(S_t, w_t)] \nabla_w \hat{v}(S_t, w_t)
\end{equation}

\textbf{使用TD误差}:
\begin{equation}
w_{t+1} = w_t + \beta \delta_t \nabla_w \hat{v}(S_t, w_t)
\end{equation}

其中 $\delta_t = R_{t+1} + \gamma \hat{v}(S_{t+1}, w_t) - \hat{v}(S_t, w_t)$。

\subsection{严格的数学推导}

\textbf{目标}:最小化TD误差的平方
\begin{equation}
L(w) = \frac{1}{2} [R_{t+1} + \gamma \hat{v}(S_{t+1}, w) - \hat{v}(S_t, w)]^2
\end{equation}

\textbf{梯度}:
\begin{align}
\nabla_w L(w) &= \frac{\partial L(w)}{\partial w} \\
              &= \frac{\partial}{\partial w} \left[\frac{1}{2} [R_{t+1} + \gamma \hat{v}(S_{t+1}, w) - \hat{v}(S_t, w)]^2\right] \\
              &= [R_{t+1} + \gamma \hat{v}(S_{t+1}, w) - \hat{v}(S_t, w)] \times \frac{\partial}{\partial w} [R_{t+1} + \gamma \hat{v}(S_{t+1}, w) - \hat{v}(S_t, w)] \\
              &= \delta_t \times [\gamma \nabla_w \hat{v}(S_{t+1}, w) - \nabla_w \hat{v}(S_t, w)]
\end{align}

\textbf{半梯度方法}:
\begin{itemize}
    \item 在实际应用中,通常忽略 $\nabla_w \hat{v}(S_{t+1}, w)$
    \item 只考虑 $\nabla_w \hat{v}(S_t, w)$
    \item 这称为"半梯度"(semi-gradient)方法
\end{itemize}

\textbf{半梯度更新}:
\begin{align}
\nabla_w L(w) &\approx -\delta_t \nabla_w \hat{v}(S_t, w) \\
w_{t+1} &= w_t - \beta \nabla_w L(w_t) \\
        &= w_t - \beta (-\delta_t \nabla_w \hat{v}(S_t, w_t)) \\
        &= w_t + \beta \delta_t \nabla_w \hat{v}(S_t, w_t)
\end{align}

\section{具体例子}

\subsection{例子1:线性函数近似}

\textbf{场景}:
\begin{itemize}
    \item 价值函数:$\hat{v}(s, w) = w^T \phi(s)$
    \item 状态特征:$\phi(s) = [1, s_1, s_2]^T$(3维特征向量)
    \item 参数:$w = [w_0, w_1, w_2]^T$(3维参数向量)
    \item 当前状态:$S_t = (2, 3)$,特征:$\phi(S_t) = [1, 2, 3]^T$
\end{itemize}

\textbf{计算梯度}:
\begin{align}
\nabla_w \hat{v}(S_t, w) &= \nabla_w [w^T \phi(S_t)] \\
                          &= \phi(S_t) \\
                          &= [1, 2, 3]^T
\end{align}

\textbf{假设}:
\begin{itemize}
    \item 当前参数:$w_t = [0.5, 0.3, -0.2]^T$
    \item 当前状态价值:$\hat{v}(S_t, w_t) = 0.5 \times 1 + 0.3 \times 2 + (-0.2) \times 3 = 0.5$
    \item 下一状态价值:$\hat{v}(S_{t+1}, w_t) = 1.0$
    \item 奖励:$R_{t+1} = -1$
    \item 折扣因子:$\gamma = 0.9$
    \item 学习率:$\beta = 0.01$
\end{itemize}

\textbf{计算TD误差}:
\begin{align}
\delta_t &= R_{t+1} + \gamma \hat{v}(S_{t+1}, w_t) - \hat{v}(S_t, w_t) \\
         &= -1 + 0.9 \times 1.0 - 0.5 \\
         &= -1 + 0.9 - 0.5 \\
         &= -0.6
\end{align}

\textbf{更新参数}:
\begin{align}
w_{t+1} &= w_t + \beta \delta_t \nabla_w \hat{v}(S_t, w_t) \\
        &= [0.5, 0.3, -0.2]^T + 0.01 \times (-0.6) \times [1, 2, 3]^T \\
        &= [0.5, 0.3, -0.2]^T + (-0.006) \times [1, 2, 3]^T \\
        &= [0.5, 0.3, -0.2]^T + [-0.006, -0.012, -0.018]^T \\
        &= [0.494, 0.288, -0.218]^T
\end{align}

\textbf{解释}:
\begin{itemize}
    \item TD误差 $\delta_t = -0.6 < 0$,说明价值估计偏高
    \item 更新是负的,会降低价值估计
    \item 参数 $w$ 被更新,使得 $\hat{v}(S_t, w)$ 降低
\end{itemize}

\textbf{验证}:
\begin{align}
\hat{v}(S_t, w_{t+1}) &= w_{t+1}^T \phi(S_t) \\
                      &= 0.494 \times 1 + 0.288 \times 2 + (-0.218) \times 3 \\
                      &= 0.494 + 0.576 - 0.654 \\
                      &= 0.416
\end{align}

确实降低了(从 $0.5$ 降到 $0.416$)!

\subsection{例子2:神经网络}

\textbf{场景}:
\begin{itemize}
    \item 价值函数:$\hat{v}(s, w)$ 是一个神经网络
    \item 输入:状态 $s$(例如:图像、特征向量)
    \item 输出:状态价值(标量)
    \item 参数:$w = \{W_1, b_1, W_2, b_2, \ldots\}$(权重和偏置)
\end{itemize}

\textbf{计算梯度}:
\begin{itemize}
    \item 使用反向传播(backpropagation)计算 $\nabla_w \hat{v}(S_t, w_t)$
    \item 这是一个向量,维度与 $w$ 相同
    \item 例如:如果 $w$ 有1000个参数,$\nabla_w \hat{v}(S_t, w_t)$ 也是1000维向量
\end{itemize}

\textbf{更新过程}:
\begin{enumerate}
    \item 前向传播:计算 $\hat{v}(S_t, w_t)$ 和 $\hat{v}(S_{t+1}, w_t)$
    \item 计算TD误差:$\delta_t = R_{t+1} + \gamma \hat{v}(S_{t+1}, w_t) - \hat{v}(S_t, w_t)$
    \item 反向传播:计算 $\nabla_w \hat{v}(S_t, w_t)$
    \item 更新参数:$w_{t+1} = w_t + \beta \delta_t \nabla_w \hat{v}(S_t, w_t)$
\end{enumerate}

\textbf{具体数值}:
\begin{itemize}
    \item 假设 $\delta_t = 0.5$(TD误差为正)
    \item 假设 $\beta = 0.01$(学习率)
    \item 假设 $\nabla_w \hat{v}(S_t, w_t)$ 的某个元素是 $0.3$
    \item 该参数的更新量:$\beta \delta_t \times 0.3 = 0.01 \times 0.5 \times 0.3 = 0.0015$
\end{itemize}

\subsection{例子3:Gridworld}

\textbf{场景}:
\begin{itemize}
    \item 状态:$S_t = (2, 3)$(Gridworld中的位置)
    \item 动作:向右移动
    \item 奖励:$R_{t+1} = -1$(每步的代价)
    \item 下一状态:$S_{t+1} = (2, 4)$
    \item 折扣因子:$\gamma = 0.9$
\end{itemize}

\textbf{价值函数}:
\begin{itemize}
    \item 使用表格形式:$\hat{v}(s, w) = w[s]$(每个状态一个参数)
    \item 参数:$w$ 是一个字典,$w[(2, 3)] = 5.0, w[(2, 4)] = 6.0$
\end{itemize}

\textbf{计算梯度}:
\begin{itemize}
    \item 对于表格形式,$\nabla_w \hat{v}(S_t, w_t)$ 是一个稀疏向量
    \item 只有对应状态 $S_t$ 的位置是 $1$,其他位置是 $0$
    \item 例如:如果 $S_t = (2, 3)$,则 $\nabla_w \hat{v}(S_t, w_t)[(2, 3)] = 1$,其他为 $0$
\end{itemize}

\textbf{计算TD误差}:
\begin{align}
\delta_t &= R_{t+1} + \gamma \hat{v}(S_{t+1}, w_t) - \hat{v}(S_t, w_t) \\
         &= -1 + 0.9 \times 6.0 - 5.0 \\
         &= -1 + 5.4 - 5.0 \\
         &= -0.6
\end{align}

\textbf{更新参数}:
\begin{align}
w_{t+1}[(2, 3)] &= w_t[(2, 3)] + \beta \delta_t \times 1 \\
                 &= 5.0 + 0.01 \times (-0.6) \times 1 \\
                 &= 5.0 - 0.006 \\
                 &= 4.994
\end{align}

\textbf{解释}:
\begin{itemize}
    \item 只有状态 $(2, 3)$ 的参数被更新
    \item 其他状态的参数保持不变
    \item 状态 $(2, 3)$ 的价值从 $5.0$ 降到 $4.994$
\end{itemize}

\section{更新方向分析}

\subsection{TD误差为正的情况}

\textbf{场景}:$\delta_t > 0$

\textbf{含义}:
\begin{itemize}
    \item 实际回报比预期好
    \item 价值估计偏低
    \item 需要提高价值估计
\end{itemize}

\textbf{更新}:
\begin{equation}
w_{t+1} = w_t + \beta \times (\text{正数}) \times \nabla_w \hat{v}(S_t, w_t)
\end{equation}

\textbf{结果}:
\begin{itemize}
    \item 更新方向与梯度方向相同
    \item 会增加 $\hat{v}(S_t, w_t)$
    \item 提高当前状态的价值估计
\end{itemize}

\subsection{TD误差为负的情况}

\textbf{场景}:$\delta_t < 0$

\textbf{含义}:
\begin{itemize}
    \item 实际回报比预期差
    \item 价值估计偏高
    \item 需要降低价值估计
\end{itemize}

\textbf{更新}:
\begin{equation}
w_{t+1} = w_t + \beta \times (\text{负数}) \times \nabla_w \hat{v}(S_t, w_t)
\end{equation}

\textbf{结果}:
\begin{itemize}
    \item 更新方向与梯度方向相反
    \item 会减少 $\hat{v}(S_t, w_t)$
    \item 降低当前状态的价值估计
\end{itemize}

\subsection{TD误差为零的情况}

\textbf{场景}:$\delta_t = 0$

\textbf{含义}:
\begin{itemize}
    \item 实际回报与预期一致
    \item 价值估计准确
    \item 不需要更新
\end{itemize}

\textbf{更新}:
\begin{equation}
w_{t+1} = w_t + \beta \times 0 \times \nabla_w \hat{v}(S_t, w_t) = w_t
\end{equation}

\textbf{结果}:
\begin{itemize}
    \item 参数保持不变
    \item 价值估计已经准确
\end{itemize}

\section{与Actor更新的对比}

\subsection{Actor更新}

\textbf{公式}:
\begin{equation}
\theta_{t+1} = \theta_t + \alpha \delta_t \nabla_\theta \ln \pi(A_t|S_t, \theta_t)
\label{eq:actor_update}
\end{equation}

\textbf{特点}:
\begin{itemize}
    \item 使用策略梯度:$\nabla_\theta \ln \pi(A_t|S_t, \theta_t)$
    \item 学习率:$\alpha$(通常较小)
    \item 目标:最大化期望回报
\end{itemize}

\subsection{Critic更新}

\textbf{公式}:
\begin{equation}
w_{t+1} = w_t + \beta \delta_t \nabla_w \hat{v}(S_t, w_t)
\end{equation}

\textbf{特点}:
\begin{itemize}
    \item 使用价值函数梯度:$\nabla_w \hat{v}(S_t, w_t)$
    \item 学习率:$\beta$(通常较大)
    \item 目标:最小化价值估计误差
\end{itemize}

\subsection{对比}

\begin{table}[h]
\centering
\begin{tabular}{lcc}
\toprule
\textbf{特征} & \textbf{Actor更新} & \textbf{Critic更新} \\
\midrule
参数 & $\theta$(策略参数) & $w$(价值函数参数) \\
梯度 & $\nabla_\theta \ln \pi(A_t|S_t, \theta_t)$ & $\nabla_w \hat{v}(S_t, w_t)$ \\
学习率 & $\alpha$(通常较小) & $\beta$(通常较大) \\
目标 & 最大化期望回报 & 最小化价值估计误差 \\
\bottomrule
\end{tabular}
\caption{Actor和Critic更新的对比}
\label{tab:comparison}
\end{table}

\section{总结}

\subsection{公式}

\begin{equation}
w_{t+1} = w_t + \beta \delta_t \nabla_w \hat{v}(S_t, w_t)
\end{equation}

\subsection{组成部分}

\begin{itemize}
    \item $w_t$:当前时刻的价值函数参数
    \item $w_{t+1}$:更新后的价值函数参数
    \item $\beta$:学习率(Critic的学习率)
    \item $\delta_t$:TD误差($R_{t+1} + \gamma \hat{v}(S_{t+1}, w_t) - \hat{v}(S_t, w_t)$)
    \item $\nabla_w \hat{v}(S_t, w_t)$:价值函数关于参数 $w$ 的梯度
\end{itemize}

\subsection{更新原理}

\begin{quote}
\textbf{Critic的更新原理}:
\begin{itemize}
    \item 目标:最小化TD误差的平方
    \item 方法:梯度下降(半梯度方法)
    \item 更新方向:沿着价值函数梯度的方向
    \item 更新幅度:由TD误差和学习率决定
\end{itemize}
\end{quote}

\subsection{直观理解}

\begin{itemize}
    \item \textbf{$\delta_t > 0$}:实际回报比预期好,提高价值估计
    \item \textbf{$\delta_t < 0$}:实际回报比预期差,降低价值估计
    \item \textbf{$\delta_t = 0$}:价值估计准确,不更新
\end{itemize}

\subsection{关键公式}

\textbf{TD误差}:
\begin{equation}
\delta_t = R_{t+1} + \gamma \hat{v}(S_{t+1}, w_t) - \hat{v}(S_t, w_t)
\end{equation}

\textbf{Critic更新}:
\begin{equation}
w_{t+1} = w_t + \beta \delta_t \nabla_w \hat{v}(S_t, w_t)
\end{equation}

\textbf{Actor更新}:
\begin{equation}
\theta_{t+1} = \theta_t + \alpha \delta_t \nabla_\theta \ln \pi(A_t|S_t, \theta_t)
\end{equation}

\end{document}

