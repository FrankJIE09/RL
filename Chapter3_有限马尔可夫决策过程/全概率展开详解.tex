\documentclass[12pt,a4paper]{article}
\usepackage[UTF8]{ctex}
\usepackage{amsmath}
\usepackage{amssymb}
\usepackage{amsthm}
\usepackage{geometry}
\usepackage{hyperref}
\usepackage{enumitem}
\usepackage{xcolor}
\usepackage{bm}
\usepackage{tikz}

\geometry{margin=2.5cm}

\title{全概率展开详解}
\subtitle{从条件期望到全概率展开的完整推导}
\author{}
\date{}

\newtheorem{definition}{定义}
\newtheorem{theorem}{定理}
\newtheorem{proposition}{命题}
\newtheorem{example}{示例}
\newtheorem{remark}{注记}

\begin{document}

\maketitle

\tableofcontents
\newpage

\section{问题提出}

在推导贝尔曼最优性方程时,我们需要将:

\begin{equation}
v_*(s) = \max_\pi \mathbb{E}_\pi[R_{t+1} + \gamma G_{t+1} | S_t = s]
\label{eq:initial}
\end{equation}

展开为:

\begin{equation}
v_*(s) = \max_\pi \sum_{a \in \mathcal{A}(s)} \pi(a | s) \sum_{s', r} p(s', r | s, a) \mathbb{E}_\pi[R_{t+1} + \gamma G_{t+1} | S_t = s, A_t = a, S_{t+1} = s', R_{t+1} = r]
\label{eq:expanded}
\end{equation}

这个展开过程使用了\textbf{全期望公式}(Law of Total Expectation),需要分两步进行:
\begin{enumerate}
    \item 对动作 $A_t$ 进行展开
    \item 对下一状态 $S_{t+1}$ 和奖励 $R_{t+1}$ 进行展开
\end{enumerate}

\section{全期望公式}

\subsection{全期望公式(Law of Total Expectation)}

\begin{theorem}[全期望公式]
对于随机变量 $X$ 和 $Y$,有:
\begin{equation}
\mathbb{E}[X] = \mathbb{E}[\mathbb{E}[X | Y]]
\label{eq:total_expectation}
\end{equation}
\end{theorem}

\textbf{离散情况下的展开形式}:
\begin{equation}
\mathbb{E}[X] = \sum_y \Pr\{Y = y\} \cdot \mathbb{E}[X | Y = y]
\label{eq:total_expectation_discrete}
\end{equation}

\textbf{解释}:
\begin{itemize}
    \item 外层期望:对所有可能的 $Y$ 值求期望
    \item 内层期望:给定 $Y = y$ 的条件下,$X$ 的期望
    \item 权重:每个 $y$ 的权重是其概率 $\Pr\{Y = y\}$
\end{itemize}

\subsection{多变量情况}

对于多个随机变量,全期望公式可以连续应用:

\begin{theorem}[多变量全期望公式]
对于随机变量 $X$、$Y$ 和 $Z$,有:
\begin{equation}
\mathbb{E}[X | Z] = \mathbb{E}[\mathbb{E}[X | Y, Z] | Z] = \sum_y \Pr\{Y = y | Z\} \cdot \mathbb{E}[X | Y = y, Z]
\label{eq:multivariate_total_expectation}
\end{equation}
\end{theorem}

\section{第一步:对动作进行展开}

\subsection{问题设置}

我们想要展开:
\begin{equation}
\mathbb{E}_\pi[R_{t+1} + \gamma G_{t+1} | S_t = s]
\end{equation}

\textbf{关键观察}:在状态 $s$,我们不知道会采取什么动作。动作 $A_t$ 是一个随机变量,其分布由策略 $\pi$ 决定。

\subsection{应用全期望公式}

根据全期望公式,我们可以对动作 $A_t$ 进行展开:

\begin{align}
\mathbb{E}_\pi[R_{t+1} + \gamma G_{t+1} | S_t = s] 
&= \mathbb{E}_\pi[\mathbb{E}_\pi[R_{t+1} + \gamma G_{t+1} | S_t = s, A_t] | S_t = s] \\
&= \sum_{a \in \mathcal{A}(s)} \Pr\{A_t = a | S_t = s\} \cdot \mathbb{E}_\pi[R_{t+1} + \gamma G_{t+1} | S_t = s, A_t = a] \\
&= \sum_{a \in \mathcal{A}(s)} \pi(a | s) \cdot \mathbb{E}_\pi[R_{t+1} + \gamma G_{t+1} | S_t = s, A_t = a]
\label{eq:action_expansion}
\end{align}

\textbf{解释}:
\begin{itemize}
    \item 第一步:使用全期望公式,将期望分解为"对动作求期望"
    \item 第二步:在离散情况下,期望变成求和
    \item 第三步:$\Pr\{A_t = a | S_t = s\} = \pi(a | s)$ 是策略概率
\end{itemize}

\subsection{直观理解}

这个展开的含义是:
\begin{quote}
\textbf{在状态 $s$ 的期望回报 = 对所有可能动作的期望回报进行加权平均,权重是策略概率。}
\end{quote}

数学上:
\begin{itemize}
    \item 外层求和:遍历所有可能的动作 $a$
    \item 权重:$\pi(a | s)$ 表示在状态 $s$ 选择动作 $a$ 的概率
    \item 内层期望:$\mathbb{E}_\pi[R_{t+1} + \gamma G_{t+1} | S_t = s, A_t = a]$ 表示给定状态 $s$ 和动作 $a$ 的期望回报
\end{itemize}

\section{第二步:对下一状态和奖励进行展开}

\subsection{问题设置}

现在我们需要展开:
\begin{equation}
\mathbb{E}_\pi[R_{t+1} + \gamma G_{t+1} | S_t = s, A_t = a]
\end{equation}

\textbf{关键观察}:即使知道了状态 $s$ 和动作 $a$,下一状态 $S_{t+1}$ 和奖励 $R_{t+1}$ 仍然是随机的(由环境动态 $p(s', r | s, a)$ 决定)。

\subsection{应用全期望公式}

根据全期望公式,我们可以对 $(S_{t+1}, R_{t+1})$ 进行展开:

\begin{align}
\mathbb{E}_\pi[R_{t+1} + \gamma G_{t+1} | S_t = s, A_t = a]
&= \mathbb{E}_\pi[\mathbb{E}_\pi[R_{t+1} + \gamma G_{t+1} | S_t = s, A_t = a, S_{t+1}, R_{t+1}] | S_t = s, A_t = a] \\
&= \sum_{s', r} \Pr\{S_{t+1} = s', R_{t+1} = r | S_t = s, A_t = a\} \\
&\quad \times \mathbb{E}_\pi[R_{t+1} + \gamma G_{t+1} | S_t = s, A_t = a, S_{t+1} = s', R_{t+1} = r] \\
&= \sum_{s', r} p(s', r | s, a) \cdot \mathbb{E}_\pi[R_{t+1} + \gamma G_{t+1} | S_t = s, A_t = a, S_{t+1} = s', R_{t+1} = r]
\label{eq:state_reward_expansion}
\end{align}

\textbf{解释}:
\begin{itemize}
    \item 第一步:使用全期望公式,将期望分解为"对下一状态和奖励求期望"
    \item 第二步:在离散情况下,期望变成对 $(s', r)$ 的求和
    \item 第三步:$\Pr\{S_{t+1} = s', R_{t+1} = r | S_t = s, A_t = a\} = p(s', r | s, a)$ 是环境动态
\end{itemize}

\subsection{直观理解}

这个展开的含义是:
\begin{quote}
\textbf{给定状态 $s$ 和动作 $a$ 的期望回报 = 对所有可能的下一状态和奖励的期望回报进行加权平均,权重是环境动态概率。}
\end{quote}

数学上:
\begin{itemize}
    \item 外层求和:遍历所有可能的 $(s', r)$ 组合
    \item 权重:$p(s', r | s, a)$ 表示在状态 $s$ 采取动作 $a$ 后,转移到状态 $s'$ 并获得奖励 $r$ 的概率
    \item 内层期望:$\mathbb{E}_\pi[R_{t+1} + \gamma G_{t+1} | S_t = s, A_t = a, S_{t+1} = s', R_{t+1} = r]$ 表示给定完整信息的期望回报
\end{itemize}

\section{完整展开过程}

\subsection{逐步展开}

现在我们将两步展开结合起来:

\textbf{步骤1}:从初始表达式开始
\begin{equation}
v_*(s) = \max_\pi \mathbb{E}_\pi[R_{t+1} + \gamma G_{t+1} | S_t = s]
\end{equation}

\textbf{步骤2}:对动作进行展开(使用公式 \eqref{eq:action_expansion})
\begin{align}
v_*(s) &= \max_\pi \sum_{a \in \mathcal{A}(s)} \pi(a | s) \cdot \mathbb{E}_\pi[R_{t+1} + \gamma G_{t+1} | S_t = s, A_t = a]
\end{align}

\textbf{步骤3}:对下一状态和奖励进行展开(使用公式 \eqref{eq:state_reward_expansion})
\begin{align}
v_*(s) &= \max_\pi \sum_{a \in \mathcal{A}(s)} \pi(a | s) \sum_{s', r} p(s', r | s, a) \\
&\quad \times \mathbb{E}_\pi[R_{t+1} + \gamma G_{t+1} | S_t = s, A_t = a, S_{t+1} = s', R_{t+1} = r]
\label{eq:final_expanded}
\end{align}

这就是我们想要的展开形式!

\subsection{展开的层次结构}

展开过程可以可视化为:

\begin{center}
\begin{tikzpicture}[node distance=1.5cm]
    \node (root) {$\mathbb{E}_\pi[R_{t+1} + \gamma G_{t+1} | S_t = s]$};
    \node (action) [below left=of root] {对动作 $A_t$ 展开};
    \node (state) [below right=of root] {对 $(S_{t+1}, R_{t+1})$ 展开};
    \node (result) [below=of action, xshift=2cm] {$\sum_a \pi(a|s) \sum_{s',r} p(s',r|s,a) \mathbb{E}_\pi[\cdots | \cdots, A_t=a, S_{t+1}=s', R_{t+1}=r]$};
    
    \draw[->] (root) -- (action);
    \draw[->] (root) -- (state);
    \draw[->] (action) -- (result);
    \draw[->] (state) -- (result);
\end{tikzpicture}
\end{center}

\section{为什么需要这样展开?}

\subsection{展开的目的}

展开的目的是将条件期望中的"未知量"逐步具体化:

\begin{enumerate}
    \item \textbf{初始状态}:只知道 $S_t = s$,不知道会采取什么动作
    \item \textbf{第一步展开}:考虑所有可能的动作,用策略概率加权
    \item \textbf{第二步展开}:对于每个动作,考虑所有可能的下一状态和奖励,用环境动态概率加权
    \item \textbf{最终结果}:得到完全确定条件下的期望(给定 $S_t, A_t, S_{t+1}, R_{t+1}$)
\end{enumerate}

\subsection{展开的必要性}

为什么不能直接计算 $\mathbb{E}_\pi[R_{t+1} + \gamma G_{t+1} | S_t = s]$?

\textbf{原因}:
\begin{itemize}
    \item 我们不知道在状态 $s$ 会采取什么动作(这由策略 $\pi$ 决定,是随机的)
    \item 即使知道了动作,我们也不知道会转移到什么状态、获得什么奖励(这由环境动态 $p$ 决定,也是随机的)
    \item 因此,必须对所有可能的情况进行加权平均
\end{itemize}

\section{简化:利用已知信息}

\subsection{关键概念:$r$ 与 $R_{t+1}$ 的区别}

在理解简化过程之前,需要明确一个重要概念:

\begin{remark}[随机变量 vs 具体值]
\begin{itemize}
    \item \textbf{$R_{t+1}$}:这是一个\textbf{随机变量},表示 $t+1$ 时刻的奖励,其值是不确定的
    \item \textbf{$r$}:这是一个\textbf{具体的数值}(实数),是 $R_{t+1}$ 的一个可能取值
    \item \textbf{关系}:$r \in \mathcal{R}$,其中 $\mathcal{R}$ 是所有可能的奖励值集合
\end{itemize}
\end{remark}

\textbf{重要区别}:
\begin{itemize}
    \item 当我们写 $\mathbb{E}_\pi[R_{t+1} | S_t = s, A_t = a]$ 时,$R_{t+1}$ 是随机变量,需要求期望
    \item 当我们写 $\mathbb{E}_\pi[R_{t+1} | \ldots, R_{t+1} = r]$ 时,条件中已经指定了 $R_{t+1} = r$,此时 $R_{t+1}$ 不再是随机变量,而是确定的值 $r$
\end{itemize}

\subsection{简化条件期望}

在展开的最后一步,我们有:
\begin{equation}
\mathbb{E}_\pi[R_{t+1} + \gamma G_{t+1} | S_t = s, A_t = a, S_{t+1} = s', R_{t+1} = r]
\end{equation}

\textbf{关键观察}:在这个条件下,$R_{t+1}$ 和 $S_{t+1}$ 都是已知的(等于 $r$ 和 $s'$),因此:

\begin{align}
\mathbb{E}_\pi[R_{t+1} + \gamma G_{t+1} | S_t = s, A_t = a, S_{t+1} = s', R_{t+1} = r]
&= \mathbb{E}_\pi[r + \gamma G_{t+1} | S_t = s, A_t = a, S_{t+1} = s', R_{t+1} = r] \\
&= r + \gamma \mathbb{E}_\pi[G_{t+1} | S_t = s, A_t = a, S_{t+1} = s', R_{t+1} = r]
\end{align}

\textbf{为什么 $R_{t+1}$ 变成了 $r$?}

因为条件中已经指定了 $R_{t+1} = r$,所以:
\begin{itemize}
    \item $R_{t+1}$ 不再是随机变量,而是确定的值 $r$
    \item 对于确定的常数 $r$,有 $\mathbb{E}_\pi[r | \ldots] = r$
    \item 因此,$\mathbb{E}_\pi[R_{t+1} | \ldots, R_{t+1} = r] = r$
\end{itemize}

\textbf{数学表达}:
\begin{align}
\mathbb{E}_\pi[R_{t+1} | S_t = s, A_t = a, S_{t+1} = s', R_{t+1} = r] 
&= \sum_{r' \in \mathcal{R}} r' \cdot \Pr\{R_{t+1} = r' | S_t = s, A_t = a, S_{t+1} = s', R_{t+1} = r\} \\
&= r \cdot 1 + \sum_{r' \neq r} r' \cdot 0 \\
&= r
\end{align}

因为当条件包含 $R_{t+1} = r$ 时,$\Pr\{R_{t+1} = r' | \ldots, R_{t+1} = r\} = 1$ 当且仅当 $r' = r$,否则为 $0$。

\subsection{利用马尔可夫性质}

由于马尔可夫性质,给定 $S_{t+1} = s'$,$G_{t+1}$ 的分布只依赖于 $s'$ 和后续的策略选择,不依赖于 $S_t$、$A_t$ 或 $R_{t+1}$:

\begin{align}
\mathbb{E}_\pi[G_{t+1} | S_t = s, A_t = a, S_{t+1} = s', R_{t+1} = r]
&= \mathbb{E}_\pi[G_{t+1} | S_{t+1} = s'] \\
&= v_\pi(s')
\end{align}

因此:
\begin{align}
\mathbb{E}_\pi[R_{t+1} + \gamma G_{t+1} | S_t = s, A_t = a, S_{t+1} = s', R_{t+1} = r]
&= r + \gamma v_\pi(s')
\end{align}

\subsection{最终形式}

将简化结果代入展开式 \eqref{eq:final_expanded}:

\begin{align}
v_*(s) &= \max_\pi \sum_{a \in \mathcal{A}(s)} \pi(a | s) \sum_{s', r} p(s', r | s, a) [r + \gamma v_\pi(s')]
\end{align}

这就是贝尔曼最优性方程的中间形式!

\section{数值示例}

\subsection{简单例子}

考虑一个简单的状态,有两个动作 $a_1$ 和 $a_2$:

\textbf{给定信息}:
\begin{itemize}
    \item 策略:$\pi(a_1 | s) = 0.6$,$\pi(a_2 | s) = 0.4$
    \item 动作 $a_1$:以概率 $0.8$ 转移到 $s_1'$ 并获得奖励 $r_1 = 1$,以概率 $0.2$ 转移到 $s_2'$ 并获得奖励 $r_2 = 0$
    \item 动作 $a_2$:以概率 $1.0$ 转移到 $s_3'$ 并获得奖励 $r_3 = 2$
    \item 假设 $v_\pi(s_1') = 5$,$v_\pi(s_2') = 3$,$v_\pi(s_3') = 4$,$\gamma = 0.9$
\end{itemize}

\textbf{计算过程}:

\textbf{步骤1}:计算每个动作的期望回报

对于动作 $a_1$:
\begin{align}
q_\pi(s, a_1) &= \sum_{s', r} p(s', r | s, a_1) [r + \gamma v_\pi(s')] \\
              &= 0.8 \times [1 + 0.9 \times 5] + 0.2 \times [0 + 0.9 \times 3] \\
              &= 0.8 \times 5.5 + 0.2 \times 2.7 \\
              &= 4.4 + 0.54 = 4.94
\end{align}

对于动作 $a_2$:
\begin{align}
q_\pi(s, a_2) &= \sum_{s', r} p(s', r | s, a_2) [r + \gamma v_\pi(s')] \\
              &= 1.0 \times [2 + 0.9 \times 4] \\
              &= 1.0 \times 5.6 = 5.6
\end{align}

\textbf{步骤2}:对动作进行加权平均

\begin{align}
v_\pi(s) &= \sum_{a} \pi(a | s) q_\pi(s, a) \\
         &= 0.6 \times 4.94 + 0.4 \times 5.6 \\
         &= 2.964 + 2.24 = 5.204
\end{align}

\textbf{验证}:这等价于直接展开:

\begin{align}
v_\pi(s) &= \sum_{a} \pi(a | s) \sum_{s', r} p(s', r | s, a) [r + \gamma v_\pi(s')] \\
         &= 0.6 \times [0.8 \times (1 + 4.5) + 0.2 \times (0 + 2.7)] \\
         &\quad + 0.4 \times [1.0 \times (2 + 3.6)] \\
         &= 0.6 \times 4.94 + 0.4 \times 5.6 = 5.204 \quad \checkmark
\end{align}

\section{总结}

\subsection{展开的关键步骤}

\begin{enumerate}
    \item \textbf{对动作展开}:使用策略概率 $\pi(a | s)$ 作为权重
    \item \textbf{对状态和奖励展开}:使用环境动态 $p(s', r | s, a)$ 作为权重
    \item \textbf{简化条件期望}:利用已知信息和马尔可夫性质
\end{enumerate}

\subsection{数学工具}

展开过程主要使用了:
\begin{itemize}
    \item \textbf{全期望公式}:$\mathbb{E}[X] = \mathbb{E}[\mathbb{E}[X | Y]]$
    \item \textbf{条件概率}:$\Pr\{A | B\} = \frac{\Pr\{A, B\}}{\Pr\{B\}}$
    \item \textbf{马尔可夫性质}:未来只依赖于当前状态
\end{itemize}

\subsection{直观理解}

展开的本质是:
\begin{quote}
\textbf{将不确定的期望分解为所有可能情况的加权平均,权重是相应的概率。}
\end{quote}

这就像计算一个班级的平均分:
\begin{itemize}
    \item 不是直接计算"班级平均分"
    \item 而是先计算"每个小组的平均分"
    \item 然后按小组人数加权平均
    \item 对于每个小组,又需要计算"每个学生的分数"的加权平均
\end{itemize}

\vspace{1cm}

\textbf{参考文献}:
\begin{itemize}
    \item Sutton, R. S., \& Barto, A. G. (2018). \textit{Reinforcement Learning: An Introduction} (2nd Edition). MIT Press, Chapter 3.
\end{itemize}

\end{document}

