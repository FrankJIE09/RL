\documentclass[12pt,a4paper]{article}
\usepackage[UTF8]{ctex}
\usepackage{amsmath}
\usepackage{amssymb}
\usepackage{amsthm}
\usepackage{geometry}
\usepackage{hyperref}
\usepackage{enumitem}
\usepackage{xcolor}
\usepackage{bm}

\geometry{margin=2.5cm}

\title{期望符号 $\mathbb{E}$ 详解}
\subtitle{期望与概率的关系}
\author{}
\date{}

\newtheorem{definition}{定义}
\newtheorem{theorem}{定理}
\newtheorem{proposition}{命题}
\newtheorem{example}{示例}

\begin{document}

\maketitle

\tableofcontents
\newpage

\section{引言}

在强化学习和概率论中,$\mathbb{E}[\cdot]$ 表示\textbf{期望(Expectation)},$\Pr\{\cdot\}$ 表示\textbf{概率(Probability)}。本文档详细解释期望符号的含义,以及它与概率的关系和区别。

\section{期望符号 $\mathbb{E}$}

\subsection{定义}

\begin{definition}[期望]
对于随机变量 $X$,其期望(也称为期望值或均值)定义为:
\begin{equation}
\mathbb{E}[X] = \sum_{x} x \cdot \Pr\{X = x\}
\end{equation}
对于离散随机变量,或
\begin{equation}
\mathbb{E}[X] = \int_{-\infty}^{\infty} x \cdot f_X(x) \, dx
\end{equation}
对于连续随机变量,其中 $f_X(x)$ 是概率密度函数。
\end{definition}

\subsection{符号说明}

\begin{itemize}
    \item \textbf{字母}:$\mathbb{E}$ 是花体(blackboard bold)的 E
    \item \textbf{读法}:读作"期望"或"E of X"
    \item \textbf{含义}:表示随机变量的\textbf{平均值}或\textbf{期望值}
    \item \textbf{其他记号}:有时也写作 $E[X]$ 或 $\text{E}[X]$
\end{itemize}

\subsection{条件期望}

\begin{definition}[条件期望]
给定事件 $A$,随机变量 $X$ 的条件期望定义为:
\begin{equation}
\mathbb{E}[X | A] = \sum_{x} x \cdot \Pr\{X = x | A\}
\end{equation}
\end{definition}

\textbf{在强化学习中的应用}:
\begin{equation}
\mathbb{E}[R_{t+1} | S_t = s, A_t = a] = \sum_{r} r \cdot \Pr\{R_{t+1} = r | S_t = s, A_t = a\}
\end{equation}

\section{概率符号 $\Pr$}

\subsection{定义}

\begin{definition}[概率]
对于事件 $A$,其概率定义为:
\begin{equation}
\Pr\{A\} = \text{事件 } A \text{ 发生的概率}
\end{equation}
\end{definition}

\subsection{符号说明}

\begin{itemize}
    \item \textbf{字母}:$\Pr$ 是 Probability 的缩写
    \item \textbf{读法}:读作"概率"或"P of A"
    \item \textbf{含义}:表示事件发生的\textbf{可能性}
    \item \textbf{值域}:$\Pr\{A\} \in [0, 1]$
    \item \textbf{其他记号}:有时也写作 $P(A)$ 或 $\mathbb{P}(A)$
\end{itemize}

\section{期望与概率的关系}

\subsection{基本关系}

\begin{proposition}[期望与概率的关系]
期望是概率的加权平均:
\begin{equation}
\mathbb{E}[X] = \sum_{x} x \cdot \Pr\{X = x\}
\end{equation}
其中权重是概率值。
\end{proposition}

\textbf{关键点}:
\begin{itemize}
    \item \textbf{概率} $\Pr\{X = x\}$:事件发生的可能性(0 到 1 之间的数)
    \item \textbf{期望} $\mathbb{E}[X]$:所有可能值的加权平均(可以是任意实数)
    \item \textbf{关系}:期望 = 值 × 概率 的求和
\end{itemize}

\subsection{简单例子}

\textbf{例子:投掷骰子}

设 $X$ 表示骰子的点数,$X \in \{1, 2, 3, 4, 5, 6\}$。

\textbf{概率}:
\begin{align}
\Pr\{X = 1\} &= \frac{1}{6} \\
\Pr\{X = 2\} &= \frac{1}{6} \\
&\vdots \\
\Pr\{X = 6\} &= \frac{1}{6}
\end{align}

\textbf{期望}:
\begin{align}
\mathbb{E}[X] &= \sum_{x=1}^6 x \cdot \Pr\{X = x\} \\
              &= 1 \cdot \frac{1}{6} + 2 \cdot \frac{1}{6} + 3 \cdot \frac{1}{6} + 4 \cdot \frac{1}{6} + 5 \cdot \frac{1}{6} + 6 \cdot \frac{1}{6} \\
              &= \frac{1 + 2 + 3 + 4 + 5 + 6}{6} \\
              &= \frac{21}{6} = 3.5
\end{align}

\textbf{解释}:
\begin{itemize}
    \item 概率:每个点数出现的概率是 $1/6$
    \item 期望:平均点数是 $3.5$(虽然永远不可能出现 $3.5$ 点)
    \item 关系:期望是"如果投掷很多次,平均会得到多少点"
\end{itemize}

\section{期望与概率的区别}

\subsection{本质区别}

\begin{center}
\begin{tabular}{|l|l|l|}
\hline
\textbf{特性} & \textbf{概率 $\Pr$} & \textbf{期望 $\mathbb{E}$} \\
\hline
\textbf{输入} & 事件 & 随机变量 \\
\hline
\textbf{输出} & $[0, 1]$ 之间的数 & 任意实数 \\
\hline
\textbf{含义} & 事件发生的可能性 & 随机变量的平均值 \\
\hline
\textbf{单位} & 无单位(比例) & 与随机变量同单位 \\
\hline
\textbf{求和} & 所有事件的概率和为 1 & 不一定有约束 \\
\hline
\end{tabular}
\end{center}

\subsection{具体对比}

\textbf{概率 $\Pr\{X = x\}$}:
\begin{itemize}
    \item 回答:"$X$ 等于 $x$ 的概率是多少?"
    \item 答案:0 到 1 之间的数
    \item 例如:$\Pr\{X = 5\} = 0.3$ 表示"$X$ 等于 5 的概率是 30\%"
\end{itemize}

\textbf{期望 $\mathbb{E}[X]$}:
\begin{itemize}
    \item 回答:"$X$ 的平均值是多少?"
    \item 答案:任意实数
    \item 例如:$\mathbb{E}[X] = 4.2$ 表示"$X$ 的平均值是 4.2"
\end{itemize}

\section{在强化学习中的应用}

\subsection{期望奖励}

在 MDP 中,我们经常计算\textbf{期望奖励}:

\begin{equation}
r(s, a) = \mathbb{E}[R_{t+1} | S_t = s, A_t = a]
\end{equation}

\textbf{展开形式}:
\begin{align}
r(s, a) &= \sum_{r \in \mathcal{R}} r \cdot \Pr\{R_{t+1} = r | S_t = s, A_t = a\} \\
        &= \sum_{r} r \sum_{s'} p(s', r | s, a)
\end{align}

\textbf{解释}:
\begin{itemize}
    \item \textbf{概率}:$\Pr\{R_{t+1} = r | S_t = s, A_t = a\}$ 表示在给定条件下,奖励为 $r$ 的概率
    \item \textbf{期望}:$\mathbb{E}[R_{t+1} | S_t = s, A_t = a]$ 表示在给定条件下,奖励的期望值(平均值)
    \item \textbf{关系}:期望 = 所有可能的奖励值 × 对应概率 的求和
\end{itemize}

\subsection{状态价值函数}

状态价值函数定义为期望回报:

\begin{equation}
v_\pi(s) = \mathbb{E}_\pi[G_t | S_t = s]
\end{equation}

\textbf{解释}:
\begin{itemize}
    \item $\mathbb{E}_\pi$ 表示在策略 $\pi$ 下的期望
    \item $G_t$ 是回报(随机变量)
    \item $v_\pi(s)$ 是回报的期望值(平均值)
\end{itemize}

\subsection{动作价值函数}

动作价值函数也定义为期望回报:

\begin{equation}
q_\pi(s, a) = \mathbb{E}_\pi[G_t | S_t = s, A_t = a]
\end{equation}

\section{期望的性质}

\subsection{线性性质}

\begin{theorem}[期望的线性性]
对于随机变量 $X$ 和 $Y$,以及常数 $a$ 和 $b$:
\begin{equation}
\mathbb{E}[aX + bY] = a\mathbb{E}[X] + b\mathbb{E}[Y]
\end{equation}
\end{theorem}

\textbf{应用}:
\begin{align}
\mathbb{E}[R_{t+1} + \gamma G_{t+1}] &= \mathbb{E}[R_{t+1}] + \gamma \mathbb{E}[G_{t+1}] \\
                                     &= r(s, a) + \gamma v_\pi(s')
\end{align}

\subsection{全期望公式}

\begin{theorem}[全期望公式(Law of Total Expectation)]
对于随机变量 $X$ 和 $Y$:
\begin{equation}
\mathbb{E}[X] = \mathbb{E}[\mathbb{E}[X | Y]]
\end{equation}
\end{theorem}

\textbf{在强化学习中的应用}:
\begin{align}
r(s, a) &= \mathbb{E}[R_{t+1} | S_t = s, A_t = a] \\
        &= \sum_{s'} p(s' | s, a) \cdot \mathbb{E}[R_{t+1} | S_t = s, A_t = a, S_{t+1} = s'] \\
        &= \sum_{s'} p(s' | s, a) \cdot r(s, a, s')
\end{align}

\section{期望 vs 概率:详细对比}

\subsection{数学形式对比}

\textbf{概率}:
\begin{equation}
\Pr\{X = x\} = \text{某个数,值域 } [0, 1]
\end{equation}

\textbf{期望}:
\begin{equation}
\mathbb{E}[X] = \sum_{x} x \cdot \Pr\{X = x\} = \text{某个数,值域 } \mathbb{R}
\end{equation}

\subsection{计算过程对比}

\textbf{计算概率}:
\begin{enumerate}
    \item 列出所有可能的事件
    \item 计算每个事件的概率
    \item 验证:所有概率和为 1
\end{enumerate}

\textbf{计算期望}:
\begin{enumerate}
    \item 列出所有可能的值和对应的概率
    \item 对每个值,计算"值 × 概率"
    \item 对所有"值 × 概率"求和
\end{enumerate}

\subsection{实际例子对比}

\textbf{例子:奖励分布}

假设在状态 $s$ 采取动作 $a$,奖励的可能值和概率:

\begin{center}
\begin{tabular}{|c|c|}
\hline
奖励 $r$ & 概率 $\Pr\{R_{t+1} = r | S_t = s, A_t = a\}$ \\
\hline
$+10$ & $0.2$ \\
$+5$ & $0.5$ \\
$0$ & $0.2$ \\
$-5$ & $0.1$ \\
\hline
\end{tabular}
\end{center}

\textbf{概率计算}:
\begin{itemize}
    \item $\Pr\{R_{t+1} = +10 | S_t = s, A_t = a\} = 0.2$
    \item $\Pr\{R_{t+1} = +5 | S_t = s, A_t = a\} = 0.5$
    \item $\Pr\{R_{t+1} = 0 | S_t = s, A_t = a\} = 0.2$
    \item $\Pr\{R_{t+1} = -5 | S_t = s, A_t = a\} = 0.1$
    \item 验证:$0.2 + 0.5 + 0.2 + 0.1 = 1.0$ ✓
\end{itemize}

\textbf{期望计算}:
\begin{align}
\mathbb{E}[R_{t+1} | S_t = s, A_t = a] &= \sum_{r} r \cdot \Pr\{R_{t+1} = r | S_t = s, A_t = a\} \\
                                       &= (+10) \cdot 0.2 + (+5) \cdot 0.5 + (0) \cdot 0.2 + (-5) \cdot 0.1 \\
                                       &= 2.0 + 2.5 + 0.0 - 0.5 \\
                                       &= 4.0
\end{align}

\textbf{对比}:
\begin{itemize}
    \item \textbf{概率}:告诉我们"每个奖励值出现的可能性"
    \item \textbf{期望}:告诉我们"平均会收到多少奖励"
    \item \textbf{关系}:期望是概率的加权平均
\end{itemize}

\section{常见符号和记号}

\subsection{期望的多种表示}

\begin{itemize}
    \item $\mathbb{E}[X]$:标准表示(花体 E)
    \item $E[X]$:简化表示
    \item $\text{E}[X]$:文本形式
    \item $\langle X \rangle$:物理学中常用(但强化学习中较少用)
\end{itemize}

\subsection{概率的多种表示}

\begin{itemize}
    \item $\Pr\{A\}$:标准表示
    \item $P(A)$:简化表示
    \item $\mathbb{P}(A)$:花体 P
    \item $p(A)$:小写 p(通常用于概率质量/密度函数)
\end{itemize}

\subsection{条件期望和条件概率}

\textbf{条件概率}:
\begin{equation}
\Pr\{X = x | Y = y\} = \frac{\Pr\{X = x, Y = y\}}{\Pr\{Y = y\}}
\end{equation}

\textbf{条件期望}:
\begin{equation}
\mathbb{E}[X | Y = y] = \sum_{x} x \cdot \Pr\{X = x | Y = y\}
\end{equation}

\section{在 MDP 中的具体应用}

\subsection{双参数期望奖励}

\begin{equation}
r(s, a) = \mathbb{E}[R_{t+1} | S_t = s, A_t = a]
\end{equation}

\textbf{展开}:
\begin{align}
r(s, a) &= \sum_{r} r \cdot \Pr\{R_{t+1} = r | S_t = s, A_t = a\} \\
        &= \sum_{r} r \sum_{s'} p(s', r | s, a)
\end{align}

\textbf{解释}:
\begin{itemize}
    \item 使用\textbf{概率} $p(s', r | s, a)$ 来计算
    \item 得到\textbf{期望} $r(s, a)$
    \item 期望是概率的加权平均
\end{itemize}

\subsection{三参数期望奖励}

\begin{equation}
r(s, a, s') = \mathbb{E}[R_{t+1} | S_t = s, A_t = a, S_{t+1} = s']
\end{equation}

\textbf{展开}:
\begin{align}
r(s, a, s') &= \sum_{r} r \cdot \Pr\{R_{t+1} = r | S_t = s, A_t = a, S_{t+1} = s'\} \\
            &= \frac{\sum_{r} r \cdot p(s', r | s, a)}{p(s' | s, a)}
\end{align}

\textbf{解释}:
\begin{itemize}
    \item 使用\textbf{条件概率}来计算
    \item 得到\textbf{条件期望}
    \item 期望依赖于额外的条件(下一状态 $s'$)
\end{itemize}

\section{常见误解}

\subsection{误解 1:期望就是概率}

\textbf{误解}:
\begin{quote}
"$\mathbb{E}[X]$ 和 $\Pr\{X = x\}$ 是一样的,都表示概率。"
\end{quote}

\textbf{纠正}:
\begin{itemize}
    \item \textbf{概率}:$\Pr\{X = x\}$ 是 0 到 1 之间的数,表示事件发生的可能性
    \item \textbf{期望}:$\mathbb{E}[X]$ 是任意实数,表示随机变量的平均值
    \item \textbf{关系}:期望是用概率计算出来的,但本身不是概率
\end{itemize}

\subsection{误解 2:期望总是等于某个可能值}

\textbf{误解}:
\begin{quote}
"期望应该等于某个可能的 $x$ 值。"
\end{quote}

\textbf{纠正}:
\begin{itemize}
    \item 期望是\textbf{平均值},不一定是可能的值
    \item 例如:投掷骰子的期望是 $3.5$,但永远不可能出现 $3.5$ 点
    \item 期望是"如果重复很多次,平均会得到多少"
\end{itemize}

\subsection{误解 3:期望和概率可以互换}

\textbf{误解}:
\begin{quote}
"$\mathbb{E}[X]$ 和 $\Pr\{X = x\}$ 可以互换使用。"
\end{quote}

\textbf{纠正}:
\begin{itemize}
    \item 它们有不同的含义和用途
    \item 概率描述"可能性"
    \item 期望描述"平均值"
    \item 不能互换,但有关系:期望 = 值 × 概率 的求和
\end{itemize}

\section{总结}

\subsection{核心概念}

\begin{enumerate}
    \item \textbf{期望 $\mathbb{E}[X]$}:
    \begin{itemize}
        \item 表示随机变量的\textbf{平均值}或\textbf{期望值}
        \item 值域:任意实数 $\mathbb{R}$
        \item 计算:$\mathbb{E}[X] = \sum_{x} x \cdot \Pr\{X = x\}$
    \end{itemize}
    
    \item \textbf{概率 $\Pr\{A\}$}:
    \begin{itemize}
        \item 表示事件发生的\textbf{可能性}
        \item 值域:$[0, 1]$
        \item 所有事件的概率和为 1
    \end{itemize}
    
    \item \textbf{关系}:
    \begin{itemize}
        \item 期望是概率的\textbf{加权平均}
        \item 期望 = 值 × 概率 的求和
        \item 概率用于计算期望
    \end{itemize}
\end{enumerate}

\subsection{关键区别}

\begin{center}
\begin{tabular}{|l|l|l|}
\hline
\textbf{特性} & \textbf{概率 $\Pr$} & \textbf{期望 $\mathbb{E}$} \\
\hline
\textbf{输入} & 事件 & 随机变量 \\
\hline
\textbf{输出} & $[0, 1]$ & $\mathbb{R}$ \\
\hline
\textbf{含义} & 可能性 & 平均值 \\
\hline
\textbf{单位} & 无单位 & 与变量同单位 \\
\hline
\textbf{求和} & 所有概率和为 1 & 无特殊约束 \\
\hline
\end{tabular}
\end{center}

\subsection{记忆技巧}

\begin{itemize}
    \item \textbf{概率} = "可能性" = 0 到 1 之间的数
    \item \textbf{期望} = "平均值" = 任意实数
    \item \textbf{关系} = 期望 = 值 × 概率 的求和
    \item \textbf{类比} = 概率是"权重",期望是"加权平均"
\end{itemize}

\vspace{1cm}

\textbf{参考文献}:
\begin{itemize}
    \item Sutton, R. S., \& Barto, A. G. (2018). Reinforcement Learning: An Introduction (2nd Edition). MIT Press.
    \item 任何概率论和数理统计教材
\end{itemize}

\end{document}

