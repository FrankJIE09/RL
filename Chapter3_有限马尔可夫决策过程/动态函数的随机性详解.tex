\documentclass[12pt,a4paper]{article}
\usepackage[UTF8]{ctex}
\usepackage{amsmath}
\usepackage{amssymb}
\usepackage{amsthm}
\usepackage{geometry}
\usepackage{hyperref}
\usepackage{enumitem}
\usepackage{xcolor}
\usepackage{bm}
\usepackage{tikz}

\geometry{margin=2.5cm}

\title{动态函数的随机性详解}
\subtitle{为什么需要求和?确定性 vs 随机性}
\author{}
\date{}

\newtheorem{definition}{定义}
\newtheorem{theorem}{定理}
\newtheorem{proposition}{命题}
\newtheorem{example}{示例}

\begin{document}

\maketitle

\tableofcontents
\newpage

\section{问题提出}

一个常见的困惑是:
\begin{quote}
\textit{"在状态 $s$ 采取动作 $a$ 后,应该只有一个下一状态和一个奖励,为什么需要求和?为什么 $\sum_{s' \in \mathcal{S}} \sum_{r \in \mathcal{R}} p(s', r | s, a) = 1$?"}
\end{quote}

本文档详细解释这个重要概念。

\section{核心答案}

\subsection{关键理解}

\textbf{重要事实}:在状态 $s$ 采取动作 $a$ 后,结果可能是\textbf{随机的},而不是确定性的!

\begin{itemize}
    \item \textbf{不是}:必然有一个确定的结果
    \item \textbf{而是}:可能有\textbf{多个可能的结果},每个结果有一定的概率
    \item \textbf{求和的意义}:所有可能结果的概率和为 1
\end{itemize}

\subsection{确定性是随机性的特例}

\begin{proposition}[确定性情况]
如果环境是\textbf{确定性的},则对每个 $(s, a)$,只有一个 $(s', r)$ 的概率为 1,其他所有 $(s', r)$ 的概率为 0。在这种情况下,求和仍然成立,但只有一项非零。
\end{proposition}

\textbf{示例}:
\begin{itemize}
    \item 如果 $(s, a)$ 总是导致 $(s', r)$,则:
    \begin{equation}
    p(s', r | s, a) = 1, \quad p(\tilde{s}, \tilde{r} | s, a) = 0 \text{ 对所有 } (\tilde{s}, \tilde{r}) \neq (s', r)
    \end{equation}
    \item 求和:$1 + 0 + 0 + \cdots = 1$ ✓
\end{itemize}

\section{为什么结果可能是随机的?}

\subsection{环境的不确定性}

\textbf{原因 1:环境的随机性}

环境本身可能具有\textbf{内在的随机性}:

\begin{example}[随机环境]
考虑一个简单的游戏:
\begin{itemize}
    \item 状态:当前位置
    \item 动作:投掷骰子
    \item 结果:骰子的结果是随机的(1-6),每个结果有概率 $1/6$
    \item 下一状态:取决于骰子结果
    \item 奖励:取决于骰子结果
\end{itemize}

在这种情况下,即使采取相同的动作,结果也可能不同!
\end{example}

\subsection{不完全可观测性}

\textbf{原因 2:隐藏信息}

智能体可能无法完全观测到环境的所有信息:

\begin{example}[部分可观测]
考虑一个机器人:
\begin{itemize}
    \item 状态:机器人观测到的信息(可能不完整)
    \item 动作:移动指令
    \item 结果:由于环境中有隐藏因素(如其他移动的物体、不确定的地形),结果可能是随机的
\end{itemize}
\end{example}

\subsection{建模不确定性}

\textbf{原因 3:建模不确定性}

即使环境本身可能是确定性的,但由于我们\textbf{不知道}完整的物理规律,我们将其建模为随机的:

\begin{example}[建模不确定性]
考虑一个复杂的物理系统:
\begin{itemize}
    \item 理论上,如果知道所有物理参数,结果是确定的
    \item 但实际上,我们无法精确知道所有参数
    \item 因此,我们使用概率分布来建模这种不确定性
\end{itemize}
\end{example}

\section{具体例子}

\subsection{例子 1:网格世界中的随机性}

考虑一个网格世界,智能体在状态 $s_1$ 采取动作"向右移动":

\textbf{可能的结果}:
\begin{center}
\begin{tabular}{|c|c|c|}
\hline
下一状态 $s'$ & 奖励 $r$ & 概率 $p(s', r | s_1, \text{右})$ \\
\hline
$s_2$(成功移动) & $+1$ & $0.7$ \\
$s_2$(成功移动) & $0$ & $0.1$ \\
$s_1$(撞墙) & $-1$ & $0.2$ \\
\hline
\end{tabular}
\end{center}

\textbf{解释}:
\begin{itemize}
    \item 70\% 的概率:成功移动到 $s_2$,收到奖励 $+1$
    \item 10\% 的概率:成功移动到 $s_2$,但收到奖励 $0$(可能因为其他原因)
    \item 20\% 的概率:撞墙,留在 $s_1$,收到惩罚 $-1$
\end{itemize}

\textbf{验证归一化}:
\begin{equation}
0.7 + 0.1 + 0.2 = 1.0 \quad \checkmark
\end{equation}

\textbf{关键点}:
\begin{itemize}
    \item 即使采取相同的动作,结果也可能不同
    \item 有\textbf{多个可能的结果},每个有不同概率
    \item 所有可能结果的概率和为 1
\end{itemize}

\subsection{例子 2:游戏中的随机性}

考虑一个简单的纸牌游戏:

\textbf{状态}:当前手牌
\textbf{动作}:抽一张牌
\textbf{结果}:
\begin{itemize}
    \item 可能抽到不同的牌(每张牌有不同概率)
    \item 不同的牌导致不同的下一状态
    \item 不同的牌导致不同的奖励
\end{itemize}

\textbf{动态函数}:
\begin{equation}
p(\text{下一手牌}, \text{奖励} | \text{当前手牌}, \text{抽牌动作})
\end{equation}

由于牌是随机抽取的,结果必然是随机的!

\subsection{例子 3:机器人导航}

考虑一个机器人在不确定环境中导航:

\textbf{状态}:机器人位置和传感器读数
\textbf{动作}:向前移动
\textbf{结果}:
\begin{itemize}
    \item 可能成功移动(如果地面平坦)
    \item 可能滑倒(如果地面湿滑)
    \item 可能被障碍物阻挡(如果障碍物移动)
\end{itemize}

每种结果有不同的概率,导致不同的下一状态和奖励。

\section{求和的意义}

\subsection{归一化条件}

\begin{equation}
\sum_{s' \in \mathcal{S}} \sum_{r \in \mathcal{R}} p(s', r | s, a) = 1
\label{eq:normalization}
\end{equation}

\textbf{含义}:
\begin{quote}
\textit{"在状态 $s$ 采取动作 $a$ 后,\textbf{必然}会有某个下一状态和某个奖励。所有可能的结果的概率和必须为 1。"}
\end{quote}

\textbf{关键理解}:
\begin{itemize}
    \item \textbf{必然性}:确实必然会有某个结果
    \item \textbf{随机性}:但\textbf{哪个}结果是随机的
    \item \textbf{概率和}:所有可能结果的概率和为 1,保证了"必然会有某个结果"
\end{itemize}

\subsection{概率分布的性质}

这是\textbf{概率分布}的基本性质:

\begin{enumerate}
    \item \textbf{非负性}:$p(s', r | s, a) \geq 0$ 对所有 $(s', r)$
    \item \textbf{归一性}:$\sum_{s', r} p(s', r | s, a) = 1$
    \item \textbf{完备性}:所有可能的结果都被包含在求和中
\end{enumerate}

\subsection{可视化理解}

\begin{center}
\begin{tikzpicture}[scale=1.2]
    % 当前状态和动作
    \node[rectangle, draw, fill=blue!20] (sa) at (0, 2) {$(s, a)$};
    
    % 可能的结果
    \node[circle, draw, fill=green!20] (r1) at (-2, 0) {$(s'_1, r_1)$\\$p = 0.3$};
    \node[circle, draw, fill=green!20] (r2) at (0, 0) {$(s'_2, r_2)$\\$p = 0.5$};
    \node[circle, draw, fill=green!20] (r3) at (2, 0) {$(s'_3, r_3)$\\$p = 0.2$};
    
    % 箭头
    \draw[->, thick] (sa) -- (r1);
    \draw[->, thick] (sa) -- (r2);
    \draw[->, thick] (sa) -- (r3);
    
    % 标签
    \node[below] at (0, -1) {所有概率和:$0.3 + 0.5 + 0.2 = 1.0$};
    \node[above] at (0, 3) {必然会有某个结果,但哪个结果是随机的};
\end{tikzpicture}
\end{center}

\section{确定性情况}

\subsection{确定性环境的特征}

如果环境是\textbf{确定性的},则:

\begin{proposition}[确定性环境]
对于每个 $(s, a)$,存在唯一的 $(s', r)$ 使得:
\begin{equation}
p(s', r | s, a) = 1
\end{equation}
而所有其他 $(s', r)$ 的概率为 0。
\end{proposition}

\subsection{确定性情况下的求和}

\textbf{示例}:

假设在状态 $s_1$ 采取动作 $a_1$,总是(100\%)导致状态 $s_2$ 和奖励 $r = +1$。

则:
\begin{align}
p(s_2, +1 | s_1, a_1) &= 1.0 \\
p(s_1, +1 | s_1, a_1) &= 0.0 \\
p(s_2, -1 | s_1, a_1) &= 0.0 \\
p(s_3, +1 | s_1, a_1) &= 0.0 \\
&\vdots
\end{align}

\textbf{求和}:
\begin{align}
\sum_{s' \in \mathcal{S}} \sum_{r \in \mathcal{R}} p(s', r | s_1, a_1) &= p(s_2, +1 | s_1, a_1) + \text{所有其他项} \\
                                                                      &= 1.0 + 0.0 + 0.0 + \cdots \\
                                                                      &= 1.0 \quad \checkmark
\end{align}

\textbf{关键点}:
\begin{itemize}
    \item 求和仍然需要,因为理论上所有 $(s', r)$ 都是可能的
    \item 但只有一项非零(概率为 1)
    \item 其他所有项都是 0
    \item 这是随机性的\textbf{特例}
\end{itemize}

\section{为什么需要求和?}

\subsection{数学上的必要性}

\textbf{概率论的要求}:

概率分布必须满足归一化条件:
\begin{equation}
\sum_{\text{所有可能结果}} \Pr\{\text{结果}\} = 1
\end{equation}

这保证了:
\begin{itemize}
    \item 所有可能结果的概率和为 1
    \item "必然会有某个结果"这个事实
    \item 概率分布的有效性
\end{itemize}

\subsection{理论上的完备性}

\textbf{考虑所有可能性}:

即使某些 $(s', r)$ 的概率为 0,我们仍然需要在求和中包含它们,因为:
\begin{itemize}
    \item 理论上,它们都是可能的结果
    \item 我们不知道哪些是 0,哪些是 1(除非已经计算)
    \item 求和保证了理论的完备性
\end{itemize}

\subsection{实际计算}

\textbf{在实际计算中}:

我们通常只考虑概率非零的项:
\begin{equation}
\sum_{s' \in \mathcal{S}} \sum_{r \in \mathcal{R}} p(s', r | s, a) = \sum_{\substack{s', r \\ p(s', r | s, a) > 0}} p(s', r | s, a)
\end{equation}

但理论上,求和应该包含所有可能的 $(s', r)$。

\section{常见误解}

\subsection{误解 1:结果应该是确定的}

\textbf{误解}:
\begin{quote}
"在状态 $s$ 采取动作 $a$ 后,应该只有一个确定的结果,为什么需要求和?"
\end{quote}

\textbf{纠正}:
\begin{itemize}
    \item 在\textbf{随机环境}中,结果不是确定的
    \item 可能有多个可能的结果,每个有不同概率
    \item 求和表示"所有可能结果的概率和"
    \item 即使环境是确定性的,求和仍然需要(只是大部分项为 0)
\end{itemize}

\subsection{误解 2:求和意味着多个结果同时发生}

\textbf{误解}:
\begin{quote}
"求和意味着会有多个结果同时发生?"
\end{quote}

\textbf{纠正}:
\begin{itemize}
    \item \textbf{不是}同时发生多个结果
    \item \textbf{而是}每次只发生一个结果,但\textbf{哪个}结果是随机的
    \item 求和表示"所有可能结果的概率和",保证必然会有某个结果
\end{itemize}

\subsection{误解 3:确定性环境不需要求和}

\textbf{误解}:
\begin{quote}
"如果环境是确定性的,就不需要求和了。"
\end{quote}

\textbf{纠正}:
\begin{itemize}
    \item 即使环境是确定性的,求和仍然需要
    \item 只是大部分项为 0,只有一项为 1
    \item 求和保证了概率分布的有效性
    \item 统一的数学框架适用于确定性和随机性情况
\end{itemize}

\section{实际应用}

\subsection{在价值函数计算中}

在计算价值函数时,我们需要考虑所有可能的结果:

\begin{equation}
v_\pi(s) = \sum_a \pi(a | s) \sum_{s', r} p(s', r | s, a) [r + \gamma v_\pi(s')]
\end{equation}

\textbf{求和的意义}:
\begin{itemize}
    \item 对所有可能的下一状态 $s'$ 和奖励 $r$ 求和
    \item 每个 $(s', r)$ 的贡献是 $p(s', r | s, a) \cdot [r + \gamma v_\pi(s')]$
    \item 这是\textbf{期望值}的计算
\end{itemize}

\subsection{在策略评估中}

在评估策略时,我们需要考虑所有可能的结果:

\begin{equation}
q_\pi(s, a) = \sum_{s', r} p(s', r | s, a) [r + \gamma v_\pi(s')]
\end{equation}

\textbf{求和的意义}:
\begin{itemize}
    \item 计算\textbf{期望回报}
    \item 考虑所有可能的 $(s', r)$ 组合
    \item 每个组合按其概率加权
\end{itemize}

\section{数值示例}

\subsection{示例:随机环境}

考虑在状态 $s$ 采取动作 $a$,可能的结果:

\begin{center}
\begin{tabular}{|c|c|c|c|}
\hline
$s'$ & $r$ & $p(s', r | s, a)$ & 说明 \\
\hline
$s_1$ & $+10$ & $0.6$ & 60\% 概率:到达目标,大奖励 \\
$s_2$ & $+1$ & $0.3$ & 30\% 概率:正常移动,小奖励 \\
$s_3$ & $-5$ & $0.1$ & 10\% 概率:遇到危险,惩罚 \\
\hline
\textbf{总和} & - & \textbf{1.0} & \textbf{必然会有某个结果} \\
\hline
\end{tabular}
\end{center}

\textbf{解释}:
\begin{itemize}
    \item 每次采取动作,只会发生\textbf{一个}结果
    \item 但\textbf{哪个}结果是随机的
    \item 60\% 的概率是 $(s_1, +10)$
    \item 30\% 的概率是 $(s_2, +1)$
    \item 10\% 的概率是 $(s_3, -5)$
    \item 所有概率和为 1,保证必然会有某个结果
\end{itemize}

\subsection{示例:确定性环境}

考虑在状态 $s$ 采取动作 $a$,结果是确定的:

\begin{center}
\begin{tabular}{|c|c|c|c|}
\hline
$s'$ & $r$ & $p(s', r | s, a)$ & 说明 \\
\hline
$s_2$ & $+1$ & $1.0$ & 100\% 概率:确定的结果 \\
$s_1$ & $+1$ & $0.0$ & 不可能 \\
$s_2$ & $-1$ & $0.0$ & 不可能 \\
$s_3$ & $+1$ & $0.0$ & 不可能 \\
$\vdots$ & $\vdots$ & $\vdots$ & $\vdots$ \\
\hline
\textbf{总和} & - & \textbf{1.0} & \textbf{仍然需要求和} \\
\hline
\end{tabular}
\end{center}

\textbf{解释}:
\begin{itemize}
    \item 只有一项概率为 1,其他都是 0
    \item 但求和仍然需要,包含所有可能的 $(s', r)$
    \item 这是随机性的特例
\end{itemize}

\section{总结}

\subsection{核心答案}

\begin{enumerate}
    \item \textbf{结果可能是随机的}:
    \begin{itemize}
        \item 在状态 $s$ 采取动作 $a$ 后,结果不一定是确定的
        \item 可能有多个可能的结果,每个有不同概率
        \item 这是环境的随机性或不确定性的体现
    \end{itemize}
    
    \item \textbf{求和的意义}:
    \begin{itemize}
        \item $\sum_{s', r} p(s', r | s, a) = 1$ 表示所有可能结果的概率和为 1
        \item 这保证了"必然会有某个结果"
        \item 这是概率分布的基本性质
    \end{itemize}
    
    \item \textbf{确定性是特例}:
    \begin{itemize}
        \item 如果环境是确定性的,只有一项概率为 1,其他为 0
        \item 但求和仍然需要
        \item 统一的数学框架适用于所有情况
    \end{itemize}
\end{enumerate}

\subsection{关键理解}

\begin{itemize}
    \item \textbf{必然性}:确实必然会有某个结果
    \item \textbf{随机性}:但\textbf{哪个}结果是随机的(除非环境是确定性的)
    \item \textbf{概率和}:所有可能结果的概率和为 1
    \item \textbf{期望计算}:求和用于计算期望值
\end{itemize}

\subsection{记忆技巧}

\begin{itemize}
    \item \textbf{必然有结果} + \textbf{结果可能随机} = \textbf{需要求和}
    \item \textbf{概率分布} = \textbf{所有可能结果的概率和为 1}
    \item \textbf{确定性} = \textbf{随机性的特例}(只有一项非零)
\end{itemize}

\vspace{1cm}

\textbf{参考文献}:
\begin{itemize}
    \item Sutton, R. S., \& Barto, A. G. (2018). Reinforcement Learning: An Introduction (2nd Edition). MIT Press, Chapter 3, Section 3.1.
    \item 书中说明:"The probabilities given by $p$ completely characterize the environment's dynamics."
\end{itemize}

\end{document}


