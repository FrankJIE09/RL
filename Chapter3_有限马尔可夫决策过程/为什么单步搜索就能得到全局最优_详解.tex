\documentclass[12pt,a4paper]{article}
\usepackage[UTF8]{ctex}
\usepackage{amsmath}
\usepackage{amssymb}
\usepackage{amsthm}
\usepackage{geometry}
\usepackage{hyperref}
\usepackage{enumitem}
\usepackage{xcolor}
\usepackage{bm}
\usepackage{tikz}
\usepackage{booktabs}

\geometry{margin=2.5cm}

\title{为什么单步搜索就能得到全局最优?}
\subtitle{贝尔曼最优性方程的核心洞察}
\author{}
\date{}

\newtheorem{definition}{定义}
\newtheorem{theorem}{定理}
\newtheorem{proposition}{命题}
\newtheorem{example}{示例}
\newtheorem{remark}{注记}

\begin{document}

\maketitle

\tableofcontents
\newpage

\section{引言}

在强化学习中,有一个看似矛盾但极其重要的结论:

\begin{quote}
\textbf{最优的长期(全局)回报期望值可以转化为每个状态对应的一个当前局部量的计算。因此,一次单步搜索就可以产生长期(或说全局)最优的动作序列。}
\end{quote}

这个结论是贝尔曼最优性方程(Bellman Optimality Equation)的核心洞察。本文档将详细解释为什么这个看似不可能的事情是成立的。

\section{问题的本质}

\subsection{全局优化问题的挑战}

在马尔可夫决策过程(MDP)中,我们的目标是找到一个策略 $\pi$,使得从任意初始状态 $s_0$ 开始,期望累积回报最大:

\begin{equation}
\max_\pi \mathbb{E}_\pi\left[\sum_{k=0}^\infty \gamma^k R_{t+k+1} \Big| S_t = s_0\right]
\label{eq:global_objective}
\end{equation}

这是一个\textbf{全局优化问题},因为:
\begin{itemize}
    \item 需要同时考虑所有时间步的决策
    \item 当前决策会影响未来所有时间步的回报
    \item 理论上需要搜索所有可能的动作序列
\end{itemize}

如果状态空间和动作空间都很大,直接求解这个全局优化问题在计算上是不可行的。

\subsection{贝尔曼最优性方程的奇迹}

然而,贝尔曼最优性方程告诉我们,这个全局优化问题可以转化为:

\begin{enumerate}
    \item \textbf{局部计算}:为每个状态 $s$ 计算一个局部量 $v_*(s)$
    \item \textbf{单步决策}:在每个状态,只需比较一步内的动作选择
    \item \textbf{自动全局最优}:这种局部贪婪选择自动产生全局最优策略
\end{enumerate}

这看起来像是一个"免费午餐"——为什么局部最优选择能保证全局最优?答案在于\textbf{最优子结构}(Optimal Substructure)和\textbf{马尔可夫性质}。

\section{最优状态价值函数}

\subsection{定义}

\begin{definition}[最优状态价值函数]
最优状态价值函数 $v_*(s)$ 定义为所有可能策略中,状态 $s$ 的最大价值:
\begin{equation}
v_*(s) \doteq \max_\pi v_\pi(s) = \max_\pi \mathbb{E}_\pi\left[\sum_{k=0}^\infty \gamma^k R_{t+k+1} \Big| S_t = s\right]
\label{eq:optimal_v_definition}
\end{equation}
\end{definition}

\textbf{关键洞察}:$v_*(s)$ 已经\textbf{编码了全局信息}。它表示从状态 $s$ 开始,如果遵循最优策略,能够获得的\textbf{最大期望累积回报}。

\subsection{为什么 $v_*(s)$ 是"局部量"?}

虽然 $v_*(s)$ 表示全局最优回报,但它是一个\textbf{局部量},因为:
\begin{itemize}
    \item 它只依赖于状态 $s$,不依赖于到达 $s$ 的路径
    \item 它可以通过递归关系(贝尔曼方程)局部计算
    \item 一旦计算出所有状态的 $v_*(s)$,就可以存储在表格中,每个状态一个值
\end{itemize}

\section{贝尔曼最优性方程}

\subsection{状态价值函数的贝尔曼最优性方程}

\begin{theorem}[状态价值函数的贝尔曼最优性方程]
最优状态价值函数 $v_*(s)$ 满足以下递归关系:
\begin{equation}
v_*(s) = \max_{a \in \mathcal{A}(s)} \sum_{s', r} p(s', r | s, a) [r + \gamma v_*(s')]
\label{eq:bellman_optimal_v}
\end{equation}
对所有 $s \in \mathcal{S}$ 成立。
\end{theorem}

\subsection{方程的含义}

这个方程告诉我们:

\begin{enumerate}
    \item \textbf{全局最优 = 局部最优}:状态 $s$ 的全局最优价值 $v_*(s)$ 等于在当前状态选择最优动作,然后加上从下一状态开始的最优价值。
    
    \item \textbf{递归结构}:当前状态的最优价值依赖于下一状态的最优价值,形成递归关系。
    
    \item \textbf{单步决策}:要计算 $v_*(s)$,只需要考虑\textbf{一步}内的动作选择,不需要考虑整个动作序列。
\end{enumerate}

\subsection{推导过程}

从最优价值函数的定义出发:

\begin{align}
v_*(s) &= \max_\pi v_\pi(s) \\
       &= \max_\pi \mathbb{E}_\pi\left[\sum_{k=0}^\infty \gamma^k R_{t+k+1} \Big| S_t = s\right]
\end{align}

将回报分解为即时奖励和未来回报:

\begin{align}
v_*(s) &= \max_\pi \mathbb{E}_\pi[R_{t+1} + \gamma G_{t+1} | S_t = s]
\end{align}

对动作和下一状态进行全概率展开:

\begin{align}
v_*(s) &= \max_\pi \sum_{a \in \mathcal{A}(s)} \pi(a | s) \sum_{s', r} p(s', r | s, a) \mathbb{E}_\pi[R_{t+1} + \gamma G_{t+1} | S_t = s, A_t = a, S_{t+1} = s', R_{t+1} = r]
\end{align}

由于马尔可夫性质,给定 $S_{t+1} = s'$,$G_{t+1}$ 的分布只依赖于 $s'$ 和后续策略:

\begin{align}
v_*(s) &= \max_\pi \sum_{a \in \mathcal{A}(s)} \pi(a | s) \sum_{s', r} p(s', r | s, a) [r + \gamma \mathbb{E}_\pi[G_{t+1} | S_{t+1} = s']] \\
       &= \max_\pi \sum_{a \in \mathcal{A}(s)} \pi(a | s) \sum_{s', r} p(s', r | s, a) [r + \gamma v_\pi(s')]
\end{align}

\textbf{关键步骤}:对于最优策略,在状态 $s$ 下应该选择使期望回报最大的动作。因此,最优策略在状态 $s$ 是\textbf{确定性的},只选择最优动作:

\begin{align}
v_*(s) &= \max_{a \in \mathcal{A}(s)} \sum_{s', r} p(s', r | s, a) [r + \gamma v_*(s')]
\end{align}

注意:这里 $v_\pi(s')$ 变成了 $v_*(s')$,因为最优策略在状态 $s'$ 也应该是最优的。

\section{为什么单步搜索就能得到全局最优?}

\subsection{最优子结构}

贝尔曼最优性方程揭示了MDP问题的\textbf{最优子结构}(Optimal Substructure):

\begin{quote}
\textbf{如果从状态 $s$ 开始的最优策略是选择动作 $a$,然后从状态 $s'$ 开始遵循最优策略,那么从状态 $s'$ 开始的最优策略就是从 $s'$ 开始的最优策略。}
\end{quote}

这意味着:
\begin{itemize}
    \item 全局最优解由局部最优解组成
    \item 如果每一步都选择局部最优动作,整个序列就是全局最优的
    \item 不需要考虑"未来可能改变策略"的情况
\end{itemize}

\subsection{贪婪策略的最优性}

\begin{proposition}[贪婪策略的最优性]
定义策略 $\pi_*$ 为:
\begin{equation}
\pi_*(a | s) = \begin{cases}
1, & \text{如果 } a = \arg\max_{a'} \sum_{s', r} p(s', r | s, a') [r + \gamma v_*(s')] \\
0, & \text{否则}
\end{cases}
\label{eq:greedy_policy}
\end{equation}
则 $\pi_*$ 是最优策略。
\end{proposition}

\textbf{证明思路}:

\begin{enumerate}
    \item 贪婪策略 $\pi_*$ 在每一步都选择使 $r + \gamma v_*(s')$ 最大的动作。
    
    \item 由于 $v_*(s')$ 已经是从 $s'$ 开始的全局最优价值,选择使 $r + \gamma v_*(s')$ 最大的动作,就是选择使从 $s$ 开始的全局期望回报最大的动作。
    
    \item 因此,$\pi_*$ 的价值函数 $v_{\pi_*}(s)$ 满足:
    \begin{equation}
    v_{\pi_*}(s) = \max_{a \in \mathcal{A}(s)} \sum_{s', r} p(s', r | s, a) [r + \gamma v_{\pi_*}(s')]
    \end{equation}
    
    \item 这与 $v_*(s)$ 的贝尔曼最优性方程相同,因此 $v_{\pi_*}(s) = v_*(s)$ 对所有 $s$ 成立。
    
    \item 所以 $\pi_*$ 是最优策略。
\end{enumerate}

\subsection{为什么不需要搜索整个动作序列?}

关键在于 $v_*(s')$ 已经\textbf{预计算}了从 $s'$ 开始的全局最优回报。因此:

\begin{itemize}
    \item 在状态 $s$,我们不需要知道"如果选择动作 $a$,未来会采取什么动作序列"
    \item 我们只需要知道"如果选择动作 $a$,会到达状态 $s'$,而从 $s'$ 开始的最优价值是 $v_*(s')$"
    \item 比较所有动作的 $r + \gamma v_*(s')$,选择最大值即可
\end{itemize}

这就是为什么\textbf{一次单步搜索}就能产生全局最优动作序列的原因。

\section{直观理解}

\subsection{类比:最短路径问题}

贝尔曼最优性方程类似于图论中的最短路径问题:

\begin{itemize}
    \item \textbf{全局问题}:找到从起点到终点的最短路径
    \item \textbf{局部计算}:为每个节点计算"从该节点到终点的最短距离"
    \item \textbf{单步决策}:在每个节点,选择使"当前边权重 + 下一节点最短距离"最小的边
    \item \textbf{全局最优}:这种局部贪婪选择自动产生全局最短路径
\end{itemize}

在MDP中:
\begin{itemize}
    \item 节点 = 状态
    \item 边 = 动作
    \item 边权重 = 即时奖励
    \item 节点距离 = 状态价值函数
    \item 折扣因子 = 考虑未来距离的权重
\end{itemize}

\subsection{为什么这可行?}

关键在于\textbf{马尔可夫性质}:

\begin{quote}
\textbf{给定当前状态,未来与过去独立。}
\end{quote}

这意味着:
\begin{itemize}
    \item 从状态 $s$ 开始的最优策略,不依赖于如何到达 $s$
    \item 因此,$v_*(s)$ 是一个\textbf{状态函数},只依赖于 $s$,不依赖于历史
    \item 一旦计算出 $v_*(s)$,它就可以被"重用",无论从哪个路径到达 $s$
\end{itemize}

\section{计算过程}

\subsection{价值迭代算法}

价值迭代(Value Iteration)算法展示了如何通过局部计算得到全局最优:

\begin{algorithm}[H]
\caption{价值迭代算法}
\begin{algorithmic}[1]
\REQUIRE 环境动态 $p(s', r | s, a)$,折扣因子 $\gamma$
\ENSURE 最优价值函数 $v_*$
\STATE 初始化 $v_0(s) = 0$ 对所有 $s$
\REPEAT
    \FOR{每个状态 $s$}
        \STATE $v_{k+1}(s) = \max_{a \in \mathcal{A}(s)} \sum_{s', r} p(s', r | s, a) [r + \gamma v_k(s')]$
    \ENDFOR
\UNTIL{收敛}
\RETURN $v_* = v_k$
\end{algorithmic}
\end{algorithmic}

\textbf{关键观察}:
\begin{itemize}
    \item 每次迭代只更新\textbf{一个状态}的价值(局部计算)
    \item 不需要考虑整个动作序列
    \item 通过迭代,所有状态的价值收敛到全局最优值
\end{itemize}

\subsection{从最优价值函数得到最优策略}

一旦计算出 $v_*(s)$,最优策略可以通过\textbf{单步搜索}得到:

\begin{algorithm}[H]
\caption{从最优价值函数得到最优策略}
\begin{algorithmic}[1]
\REQUIRE 最优价值函数 $v_*$,环境动态 $p(s', r | s, a)$
\ENSURE 最优策略 $\pi_*$
\FOR{每个状态 $s$}
    \STATE $\pi_*(s) = \arg\max_{a \in \mathcal{A}(s)} \sum_{s', r} p(s', r | s, a) [r + \gamma v_*(s')]$
\ENDFOR
\RETURN $\pi_*$
\end{algorithmic}
\end{algorithmic}

\textbf{关键观察}:
\begin{itemize}
    \item 对每个状态,只需要\textbf{比较一步内的动作}
    \item 不需要搜索整个动作序列
    \item 这种局部贪婪选择自动产生全局最优策略
\end{itemize}

\section{数学证明}

\subsection{最优策略的存在性}

\begin{theorem}[最优策略的存在性]
在有限MDP中,如果 $\gamma < 1$ 或所有策略都是适当的(proper),则存在唯一的最优价值函数 $v_*$,以及至少一个最优策略 $\pi_*$。
\end{theorem}

\textbf{证明思路}:
\begin{enumerate}
    \item 贝尔曼最优性方程定义了一个算子 $T_*$:
    \begin{equation}
    (T_* v)(s) = \max_{a \in \mathcal{A}(s)} \sum_{s', r} p(s', r | s, a) [r + \gamma v(s')]
    \end{equation}
    
    \item $T_*$ 是一个\textbf{压缩映射}(contraction mapping),即存在 $\gamma < 1$ 使得:
    \begin{equation}
    \|T_* v_1 - T_* v_2\|_\infty \leq \gamma \|v_1 - v_2\|_\infty
    \end{equation}
    
    \item 根据压缩映射定理,$T_*$ 有唯一不动点 $v_*$,满足 $T_* v_* = v_*$。
    
    \item 这个不动点就是最优价值函数。
    
    \item 从 $v_*$ 构造的贪婪策略 $\pi_*$ 是最优策略。
\end{enumerate}

\subsection{贪婪策略的最优性证明}

\begin{theorem}[贪婪策略的最优性]
如果策略 $\pi$ 相对于价值函数 $v$ 是贪婪的,即:
\begin{equation}
\pi(a | s) = \begin{cases}
1, & \text{如果 } a = \arg\max_{a'} \sum_{s', r} p(s', r | s, a') [r + \gamma v(s')] \\
0, & \text{否则}
\end{cases}
\end{equation}
且 $v$ 满足 $v = T_* v$(即 $v$ 是最优价值函数),则 $\pi$ 是最优策略。
\end{theorem}

\textbf{证明}:

由于 $\pi$ 相对于 $v_*$ 是贪婪的,我们有:
\begin{equation}
\sum_{a} \pi(a | s) \sum_{s', r} p(s', r | s, a) [r + \gamma v_*(s')] = \max_{a} \sum_{s', r} p(s', r | s, a) [r + \gamma v_*(s')] = v_*(s)
\end{equation}

因此,$\pi$ 的价值函数 $v_\pi$ 满足:
\begin{equation}
v_\pi(s) = \sum_{a} \pi(a | s) \sum_{s', r} p(s', r | s, a) [r + \gamma v_\pi(s')]
\end{equation}

由于 $v_*$ 是唯一满足 $v_* = T_* v_*$ 的函数,且 $v_\pi$ 也满足这个方程(因为 $\pi$ 是贪婪的),我们有 $v_\pi = v_*$。

因此,$\pi$ 是最优策略。$\square$

\section{数值示例}

\subsection{简单网格世界}

考虑一个 $3 \times 3$ 网格世界:
\begin{itemize}
    \item 状态:9个格子,编号为 $s_1$ 到 $s_9$
    \item 动作:上、下、左、右
    \item 目标:$s_9$ 是终止状态,到达获得奖励 $+10$
    \item 每步移动奖励:$-1$
    \item 折扣因子:$\gamma = 0.9$
\end{itemize}

\subsection{计算最优价值函数}

使用价值迭代算法,经过多次迭代后,得到最优价值函数 $v_*$(近似值):

\begin{center}
\begin{tabular}{c|c|c}
$v_*(s_1) \approx 5.3$ & $v_*(s_2) \approx 6.2$ & $v_*(s_3) \approx 7.1$ \\
\hline
$v_*(s_4) \approx 6.2$ & $v_*(s_5) \approx 7.1$ & $v_*(s_6) \approx 8.0$ \\
\hline
$v_*(s_7) \approx 7.1$ & $v_*(s_8) \approx 8.0$ & $v_*(s_9) = 10.0$ \\
\end{tabular}
\end{center}

\subsection{单步搜索得到最优策略}

在状态 $s_5$(中心),计算每个动作的价值:

\begin{align}
q_*(s_5, \text{上}) &= -1 + 0.9 \times v_*(s_2) = -1 + 0.9 \times 6.2 = 4.58 \\
q_*(s_5, \text{下}) &= -1 + 0.9 \times v_*(s_8) = -1 + 0.9 \times 8.0 = 6.20 \\
q_*(s_5, \text{左}) &= -1 + 0.9 \times v_*(s_4) = -1 + 0.9 \times 6.2 = 4.58 \\
q_*(s_5, \text{右}) &= -1 + 0.9 \times v_*(s_6) = -1 + 0.9 \times 8.0 = 6.20
\end{align}

因此,在状态 $s_5$ 的最优动作是"下"或"右"(两者价值相同)。

\textbf{关键观察}:
\begin{itemize}
    \item 我们只需要\textbf{比较一步内的动作}(单步搜索)
    \item 不需要考虑"如果选择'上',未来会采取什么动作序列"
    \item $v_*(s_2)$ 已经编码了从 $s_2$ 开始的全局最优回报
    \item 因此,选择使 $r + \gamma v_*(s')$ 最大的动作,就是全局最优选择
\end{itemize}

\section{为什么这很重要?}

\subsection{计算效率}

如果直接搜索所有可能的动作序列:
\begin{itemize}
    \item 状态数:$|\mathcal{S}|$
    \item 每个状态的动作数:$|\mathcal{A}|$
    \item 时间步数:$T$(可能是无穷)
    \item 可能的动作序列数:$|\mathcal{A}|^T$(指数级增长)
\end{itemize}

使用贝尔曼最优性方程:
\begin{itemize}
    \item 需要计算的状态价值数:$|\mathcal{S}|$(线性)
    \item 每个状态的计算:比较 $|\mathcal{A}|$ 个动作(线性)
    \item 总计算复杂度:$O(|\mathcal{S}| \times |\mathcal{A}|)$(多项式)
\end{itemize}

\textbf{从指数级降低到多项式级!}

\subsection{理论基础}

贝尔曼最优性方程是许多强化学习算法的理论基础:
\begin{itemize}
    \item \textbf{价值迭代}:直接使用贝尔曼最优性方程迭代更新价值函数
    \item \textbf{策略迭代}:交替进行策略评估和策略改进
    \item \textbf{Q学习}:使用最优动作价值函数的贝尔曼方程
    \item \textbf{深度Q网络(DQN)}:使用神经网络近似最优Q函数
\end{itemize}

\section{常见误解}

\subsection{误解一:局部最优 = 全局最优总是成立}

\textbf{误解}:在任何优化问题中,局部最优选择都能保证全局最优。

\textbf{澄清}:这\textbf{不是}普遍成立的。只有在具有\textbf{最优子结构}的问题中才成立,而MDP恰好具有这个性质。

\textbf{反例}:在一般的非凸优化问题中,局部最优可能不是全局最优。

\subsection{误解二:不需要考虑未来}

\textbf{误解}:单步搜索意味着不需要考虑未来。

\textbf{澄清}:我们\textbf{确实}考虑了未来,但通过 $v_*(s')$ 间接考虑。$v_*(s')$ 已经编码了从 $s'$ 开始的所有未来最优决策。

\subsection{误解三:贪婪策略总是最优}

\textbf{误解}:在任何情况下,贪婪策略都是最优的。

\textbf{澄清}:只有在\textbf{已知最优价值函数}的情况下,贪婪策略才是最优的。如果价值函数不准确,贪婪策略可能不是最优的。

\section{总结}

\subsection{核心要点}

\begin{enumerate}
    \item \textbf{全局到局部的转化}:
    \begin{itemize}
        \item 最优的长期回报期望值 $v_*(s)$ 可以存储在局部量中
        \item 每个状态只需要一个数值
        \item 通过贝尔曼最优性方程递归计算
    \end{itemize}
    
    \item \textbf{单步搜索的可行性}:
    \begin{itemize}
        \item 由于最优子结构,局部贪婪选择保证全局最优
        \item 不需要搜索整个动作序列
        \item 只需要比较一步内的动作选择
    \end{itemize}
    
    \item \textbf{马尔可夫性质的关键作用}:
    \begin{itemize}
        \item 使得 $v_*(s)$ 只依赖于 $s$,不依赖于历史
        \item 使得最优策略可以"重用"已计算的价值函数
    \end{itemize}
    
    \item \textbf{计算效率的巨大提升}:
    \begin{itemize}
        \item 从指数级复杂度降低到多项式级
        \item 使得大规模MDP问题的求解成为可能
    \end{itemize}
\end{enumerate}

\subsection{数学表达}

贝尔曼最优性方程的核心是:

\begin{equation}
v_*(s) = \max_{a \in \mathcal{A}(s)} \sum_{s', r} p(s', r | s, a) [r + \gamma v_*(s')]
\end{equation}

这个方程告诉我们:
\begin{itemize}
    \item \textbf{左边}:全局最优价值(从 $s$ 开始的长期最优回报)
    \item \textbf{右边}:局部计算(一步内的最优选择 + 未来最优价值)
    \item \textbf{等号}:全局最优 = 局部最优
\end{itemize}

\subsection{哲学意义}

贝尔曼最优性方程揭示了:
\begin{itemize}
    \item \textbf{结构的重要性}:问题的结构(最优子结构)使得复杂问题可以简化
    \item \textbf{递归的力量}:通过递归关系,局部信息可以编码全局信息
    \item \textbf{数学的美妙}:看似不可能的事情,通过正确的数学框架,变得可能
\end{itemize}

\vspace{1cm}

\textbf{参考文献}:
\begin{itemize}
    \item Sutton, R. S., \& Barto, A. G. (2018). \textit{Reinforcement Learning: An Introduction} (2nd Edition). MIT Press, Chapter 3.
    \item Bellman, R. (1957). \textit{Dynamic Programming}. Princeton University Press.
    \item Puterman, M. L. (2014). \textit{Markov Decision Processes: Discrete Stochastic Dynamic Programming}. John Wiley \& Sons.
\end{itemize}

\end{document}

