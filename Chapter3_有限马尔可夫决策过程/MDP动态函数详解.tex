\documentclass[12pt,a4paper]{article}
\usepackage[UTF8]{ctex}
\usepackage{amsmath}
\usepackage{amssymb}
\usepackage{amsthm}
\usepackage{geometry}
\usepackage{hyperref}
\usepackage{enumitem}
\usepackage{xcolor}
\usepackage{bm}
\usepackage{tikz}

\geometry{margin=2.5cm}

\title{MDP 动态函数详解}
\subtitle{四参数动态函数及其衍生形式}
\author{}
\date{}

\newtheorem{definition}{定义}
\newtheorem{theorem}{定理}
\newtheorem{proposition}{命题}
\newtheorem{example}{示例}

\begin{document}

\maketitle

\tableofcontents
\newpage

\section{引言}

在马尔可夫决策过程(MDP)中,环境的动态可以用多种方式表示。本文档详细解释四参数动态函数及其所有衍生形式,包括它们的定义、关系、推导和应用。

\section{四参数动态函数:基础}

\subsection{定义}

\begin{definition}[四参数动态函数]
四参数动态函数 $p(s', r | s, a)$ 定义为:
\begin{equation}
p(s', r | s, a) \doteq \Pr\{S_{t+1} = s', R_{t+1} = r | S_t = s, A_t = a\}
\label{eq:four_param}
\end{equation}
其中:
\begin{itemize}
    \item $s \in \mathcal{S}$:当前状态
    \item $a \in \mathcal{A}(s)$:在当前状态 $s$ 下采取的动作
    \item $s' \in \mathcal{S}$:下一状态
    \item $r \in \mathcal{R}$:奖励值
\end{itemize}
\end{definition}

\subsection{符号说明}

\begin{itemize}
    \item \textbf{函数名}:$p$ 表示概率(probability)
    \item \textbf{参数顺序}:$(s', r | s, a)$
    \begin{itemize}
        \item 竖线 $|$ 前:结果(下一状态和奖励)
        \item 竖线 $|$ 后:条件(当前状态和动作)
    \end{itemize}
    \item \textbf{点等号}:$\doteq$ 表示"定义为",强调这是定义而非推导结果
    \item \textbf{时间索引}:注意 $S_{t+1}$ 和 $R_{t+1}$ 使用 $t+1$,强调它们是联合确定的
\end{itemize}

\subsection{含义}

$p(s', r | s, a)$ 表示:
\begin{quote}
在状态 $s$ 采取动作 $a$ 的条件下,下一状态为 $s'$ \textbf{且}奖励为 $r$ 的\textbf{联合概率}。
\end{quote}

\textbf{关键点}:
\begin{itemize}
    \item 这是\textbf{联合概率}:同时指定了下一状态和奖励
    \item 强调了 $S_{t+1}$ 和 $R_{t+1}$ 的\textbf{联合性}
    \item 完全描述了环境的动态
\end{itemize}

\subsection{归一化条件}

对于任意给定的 $(s, a)$,所有可能的 $(s', r)$ 组合的概率和必须为 1:
\begin{equation}
\sum_{s' \in \mathcal{S}} \sum_{r \in \mathcal{R}} p(s', r | s, a) = 1, \quad \text{对所有 } s \in \mathcal{S}, a \in \mathcal{A}(s)
\label{eq:normalization}
\end{equation}

\textbf{解释}:
\begin{itemize}
    \item 在状态 $s$ 采取动作 $a$ 后,必然会有某个下一状态和某个奖励
    \item 所有可能的结果的概率和为 1
    \item 这保证了 $p$ 是一个有效的概率分布
\end{itemize}

\subsection{函数类型}

\begin{itemize}
    \item \textbf{定义域}:$\mathcal{S} \times \mathcal{R} \times \mathcal{S} \times \mathcal{A} \to [0, 1]$
    \item \textbf{值域}:$[0, 1]$(概率值)
    \item \textbf{确定性}:$p$ 本身是确定性的函数,但它的值表示概率
\end{itemize}

\section{三参数状态转移概率:$p(s' | s, a)$}

\subsection{定义}

\begin{definition}[状态转移概率]
状态转移概率 $p(s' | s, a)$ 定义为:
\begin{equation}
p(s' | s, a) \doteq \Pr\{S_{t+1} = s' | S_t = s, A_t = a\} = \sum_{r \in \mathcal{R}} p(s', r | s, a)
\label{eq:three_param_transition}
\end{equation}
\end{definition}

\subsection{推导过程}

\textbf{从联合概率到边缘概率}:

给定四参数动态函数 $p(s', r | s, a)$,我们可以通过对奖励 $r$ 求和来得到状态转移概率:
\begin{align}
p(s' | s, a) &= \Pr\{S_{t+1} = s' | S_t = s, A_t = a\} \\
             &= \sum_{r \in \mathcal{R}} \Pr\{S_{t+1} = s', R_{t+1} = r | S_t = s, A_t = a\} \\
             &= \sum_{r \in \mathcal{R}} p(s', r | s, a)
\end{align}

\textbf{解释}:
\begin{itemize}
    \item 这是\textbf{边缘化(marginalization)}操作
    \item 对所有可能的奖励值求和,得到只关于状态的概率
    \item 从联合分布 $p(s', r | s, a)$ 得到边缘分布 $p(s' | s, a)$
\end{itemize}

\subsection{含义}

$p(s' | s, a)$ 表示:
\begin{quote}
在状态 $s$ 采取动作 $a$ 的条件下,下一状态为 $s'$ 的概率(不考虑具体奖励值)。
\end{quote}

\textbf{关键点}:
\begin{itemize}
    \item 这是\textbf{条件概率}:只关注状态转移
    \item 忽略了奖励的具体值
    \item 仍然依赖于当前状态和动作
\end{itemize}

\subsection{归一化条件}

对于任意给定的 $(s, a)$:
\begin{equation}
\sum_{s' \in \mathcal{S}} p(s' | s, a) = 1, \quad \text{对所有 } s \in \mathcal{S}, a \in \mathcal{A}(s)
\end{equation}

\textbf{验证}:
\begin{align}
\sum_{s'} p(s' | s, a) &= \sum_{s'} \sum_{r} p(s', r | s, a) \\
                        &= \sum_{s', r} p(s', r | s, a) \\
                        &= 1 \quad \text{(由式 \eqref{eq:normalization})}
\end{align}

\subsection{符号重载}

\begin{itemize}
    \item \textbf{注意}:这里使用了\textbf{符号重载(overloading)}
    \item 函数名都是 $p$,但参数个数不同
    \item $p(s', r | s, a)$:四参数函数
    \item $p(s' | s, a)$:三参数函数
    \item 通过参数个数可以区分它们
\end{itemize}

\section{双参数期望奖励:$r(s, a)$}

\subsection{定义}

\begin{definition}[状态-动作期望奖励]
状态-动作期望奖励 $r(s, a)$ 定义为:
\begin{equation}
r(s, a) \doteq \mathbb{E}[R_{t+1} | S_t = s, A_t = a] = \sum_{r \in \mathcal{R}} r \sum_{s' \in \mathcal{S}} p(s', r | s, a)
\label{eq:two_param_reward}
\end{equation}
\end{definition}

\subsection{推导过程}

\textbf{期望的定义}:

奖励 $R_{t+1}$ 的期望值等于所有可能奖励值的加权平均:
\begin{align}
r(s, a) &= \mathbb{E}[R_{t+1} | S_t = s, A_t = a] \\
        &= \sum_{r \in \mathcal{R}} r \cdot \Pr\{R_{t+1} = r | S_t = s, A_t = a\} \\
        &= \sum_{r \in \mathcal{R}} r \sum_{s' \in \mathcal{S}} \Pr\{S_{t+1} = s', R_{t+1} = r | S_t = s, A_t = a\} \\
        &= \sum_{r \in \mathcal{R}} r \sum_{s' \in \mathcal{S}} p(s', r | s, a)
\end{align}

\textbf{解释}:
\begin{itemize}
    \item 对所有可能的 $(s', r)$ 组合求和
    \item 每个组合的权重是 $r \cdot p(s', r | s, a)$
    \item 结果是对奖励值的期望
\end{itemize}

\subsection{含义}

$r(s, a)$ 表示:
\begin{quote}
在状态 $s$ 采取动作 $a$ 的条件下,期望收到的奖励值(平均奖励)。
\end{quote}

\textbf{关键点}:
\begin{itemize}
    \item 这是\textbf{期望值},不是概率
    \item 值域是实数:$r(s, a) \in \mathbb{R}$
    \item 考虑了所有可能的下一状态和奖励组合
    \item 只依赖于当前状态和动作
\end{itemize}

\subsection{简化形式}

如果奖励只依赖于 $(s, a)$ 而不依赖于 $s'$,则:
\begin{equation}
r(s, a) = \sum_{r \in \mathcal{R}} r \cdot \Pr\{R_{t+1} = r | S_t = s, A_t = a\}
\end{equation}

\section{三参数期望奖励:$r(s, a, s')$}

\subsection{定义}

\begin{definition}[状态-动作-下一状态期望奖励]
状态-动作-下一状态期望奖励 $r(s, a, s')$ 定义为:
\begin{equation}
r(s, a, s') \doteq \mathbb{E}[R_{t+1} | S_t = s, A_t = a, S_{t+1} = s'] = \frac{\sum_{r \in \mathcal{R}} r \cdot p(s', r | s, a)}{p(s' | s, a)}
\label{eq:three_param_reward}
\end{equation}
\end{definition}

\subsection{推导过程}

\textbf{条件期望}:

给定下一状态 $s'$,奖励的期望是:
\begin{align}
r(s, a, s') &= \mathbb{E}[R_{t+1} | S_t = s, A_t = a, S_{t+1} = s'] \\
             &= \sum_{r \in \mathcal{R}} r \cdot \Pr\{R_{t+1} = r | S_t = s, A_t = a, S_{t+1} = s'\}
\end{align}

\textbf{使用条件概率公式}:

根据条件概率的定义:
\begin{equation}
\Pr\{R_{t+1} = r | S_t = s, A_t = a, S_{t+1} = s'\} = \frac{\Pr\{S_{t+1} = s', R_{t+1} = r | S_t = s, A_t = a\}}{\Pr\{S_{t+1} = s' | S_t = s, A_t = a\}}
\end{equation}

因此:
\begin{align}
r(s, a, s') &= \sum_{r \in \mathcal{R}} r \cdot \frac{p(s', r | s, a)}{p(s' | s, a)} \\
             &= \frac{\sum_{r \in \mathcal{R}} r \cdot p(s', r | s, a)}{p(s' | s, a)}
\end{align}

\textbf{解释}:
\begin{itemize}
    \item 这是\textbf{条件期望}:给定下一状态 $s'$ 的期望奖励
    \item 分子:所有可能的奖励值乘以对应的联合概率
    \item 分母:状态转移概率(归一化因子)
\end{itemize}

\subsection{含义}

$r(s, a, s')$ 表示:
\begin{quote}
在状态 $s$ 采取动作 $a$,且已知下一状态为 $s'$ 的条件下,期望收到的奖励值。
\end{quote}

\textbf{关键点}:
\begin{itemize}
    \item 这是\textbf{条件期望},依赖于下一状态
    \item 比 $r(s, a)$ 更具体:不仅知道当前状态和动作,还知道下一状态
    \item 在某些问题中,奖励可能强烈依赖于下一状态
\end{itemize}

\subsection{与 $r(s, a)$ 的关系}

通过全概率公式,$r(s, a)$ 可以表示为:
\begin{align}
r(s, a) &= \sum_{s' \in \mathcal{S}} p(s' | s, a) \cdot r(s, a, s') \\
        &= \sum_{s' \in \mathcal{S}} p(s' | s, a) \cdot \mathbb{E}[R_{t+1} | S_t = s, A_t = a, S_{t+1} = s']
\end{align}

\textbf{解释}:
\begin{itemize}
    \item $r(s, a)$ 是 $r(s, a, s')$ 关于 $s'$ 的加权平均
    \item 权重是状态转移概率 $p(s' | s, a)$
    \item 这是\textbf{全期望公式(law of total expectation)}的应用
\end{itemize}

\section{函数关系总结}

\subsection{层次结构}

\begin{center}
\begin{tikzpicture}[node distance=2cm]
    \node (four) [rectangle, draw, text width=3cm, text centered] {四参数\\$p(s', r | s, a)$\\基础函数};
    \node (three_trans) [rectangle, draw, text width=3cm, text centered, below left=of four] {三参数转移\\$p(s' | s, a)$\\边缘化};
    \node (two_reward) [rectangle, draw, text width=3cm, text centered, below right=of four] {双参数奖励\\$r(s, a)$\\期望};
    \node (three_reward) [rectangle, draw, text width=3cm, text centered, below=of three_trans] {三参数奖励\\$r(s, a, s')$\\条件期望};
    
    \draw[->] (four) -- node[left] {对 $r$ 求和} (three_trans);
    \draw[->] (four) -- node[right] {加权求和} (two_reward);
    \draw[->] (four) -- node[below left] {条件概率} (three_reward);
    \draw[->] (three_trans) -- node[right] {加权平均} (two_reward);
    \draw[->] (three_reward) -- node[below] {全期望} (two_reward);
\end{tikzpicture}
\end{center}

\subsection{推导关系}

\begin{enumerate}
    \item \textbf{从四参数到三参数转移}:
    \begin{equation}
    p(s' | s, a) = \sum_{r \in \mathcal{R}} p(s', r | s, a)
    \end{equation}
    
    \item \textbf{从四参数到双参数奖励}:
    \begin{equation}
    r(s, a) = \sum_{r \in \mathcal{R}} r \sum_{s' \in \mathcal{S}} p(s', r | s, a)
    \end{equation}
    
    \item \textbf{从四参数到三参数奖励}:
    \begin{equation}
    r(s, a, s') = \frac{\sum_{r \in \mathcal{R}} r \cdot p(s', r | s, a)}{p(s' | s, a)}
    \end{equation}
    
    \item \textbf{从三参数奖励到双参数奖励}:
    \begin{equation}
    r(s, a) = \sum_{s' \in \mathcal{S}} p(s' | s, a) \cdot r(s, a, s')
    \end{equation}
\end{enumerate}

\section{数值示例}

\subsection{示例:简单的网格世界}

考虑一个简单的 2×2 网格世界:
\begin{itemize}
    \item 状态:$\mathcal{S} = \{s_1, s_2, s_3, s_4\}$(4个格子)
    \item 动作:$\mathcal{A} = \{\text{上}, \text{下}, \text{左}, \text{右}\}$
    \item 奖励:$\mathcal{R} = \{-1, 0, +10\}$
\end{itemize}

\subsection{四参数动态函数示例}

假设在状态 $s_1$ 采取动作"右":

\begin{center}
\begin{tabular}{|c|c|c|c|}
\hline
$s'$ & $r$ & $p(s', r | s_1, \text{右})$ & 说明 \\
\hline
$s_2$ & $-1$ & $0.8$ & 80\% 概率移动到 $s_2$,奖励 $-1$ \\
$s_2$ & $0$ & $0.1$ & 10\% 概率移动到 $s_2$,奖励 $0$ \\
$s_1$ & $-1$ & $0.1$ & 10\% 概率撞墙,留在 $s_1$,奖励 $-1$ \\
\hline
\end{tabular}
\end{center}

\textbf{验证归一化}:
\begin{equation}
0.8 + 0.1 + 0.1 = 1.0 \quad \checkmark
\end{equation}

\subsection{计算三参数状态转移概率}

\begin{align}
p(s_2 | s_1, \text{右}) &= \sum_{r} p(s_2, r | s_1, \text{右}) \\
                        &= p(s_2, -1 | s_1, \text{右}) + p(s_2, 0 | s_1, \text{右}) \\
                        &= 0.8 + 0.1 = 0.9 \\
p(s_1 | s_1, \text{右}) &= p(s_1, -1 | s_1, \text{右}) = 0.1
\end{align}

\textbf{验证}:
\begin{equation}
p(s_2 | s_1, \text{右}) + p(s_1 | s_1, \text{右}) = 0.9 + 0.1 = 1.0 \quad \checkmark
\end{equation}

\subsection{计算双参数期望奖励}

\begin{align}
r(s_1, \text{右}) &= \sum_{r} r \sum_{s'} p(s', r | s_1, \text{右}) \\
                  &= (-1) \cdot [p(s_2, -1 | s_1, \text{右}) + p(s_1, -1 | s_1, \text{右})] \\
                  &\quad + 0 \cdot p(s_2, 0 | s_1, \text{右}) \\
                  &= (-1) \cdot (0.8 + 0.1) + 0 \cdot 0.1 \\
                  &= -0.9
\end{align}

\subsection{计算三参数期望奖励}

\begin{align}
r(s_1, \text{右}, s_2) &= \frac{\sum_{r} r \cdot p(s_2, r | s_1, \text{右})}{p(s_2 | s_1, \text{右})} \\
                        &= \frac{(-1) \cdot 0.8 + 0 \cdot 0.1}{0.9} \\
                        &= \frac{-0.8}{0.9} \approx -0.889 \\
r(s_1, \text{右}, s_1) &= \frac{(-1) \cdot 0.1}{0.1} = -1.0
\end{align}

\textbf{验证关系}:
\begin{align}
r(s_1, \text{右}) &= p(s_2 | s_1, \text{右}) \cdot r(s_1, \text{右}, s_2) + p(s_1 | s_1, \text{右}) \cdot r(s_1, \text{右}, s_1) \\
                  &= 0.9 \times (-0.889) + 0.1 \times (-1.0) \\
                  &\approx -0.800 - 0.100 = -0.900 \quad \checkmark
\end{align}

\section{应用场景}

\subsection{何时使用四参数函数?}

\textbf{适用情况}:
\begin{itemize}
    \item \textbf{完整建模}:需要完整描述环境动态
    \item \textbf{理论分析}:推导和证明
    \item \textbf{一般性讨论}:最通用的表示
\end{itemize}

\textbf{优点}:
\begin{itemize}
    \item 包含所有信息
    \item 可以推导出其他所有函数
    \item 数学表达最完整
\end{itemize}

\textbf{缺点}:
\begin{itemize}
    \item 参数较多,可能难以估计
    \item 存储空间较大
\end{itemize}

\subsection{何时使用三参数转移概率?}

\textbf{适用情况}:
\begin{itemize}
    \item \textbf{只关心状态转移}:不关心具体奖励值
    \item \textbf{规划问题}:某些规划算法只需要转移概率
    \item \textbf{简化模型}:奖励是确定性的或简单的
\end{itemize}

\textbf{优点}:
\begin{itemize}
    \item 参数较少
    \item 更易估计和存储
\end{itemize}

\textbf{缺点}:
\begin{itemize}
    \item 丢失了奖励信息
    \item 无法完整描述环境
\end{itemize}

\subsection{何时使用双参数奖励?}

\textbf{适用情况}:
\begin{itemize}
    \item \textbf{奖励只依赖于 $(s, a)$}:不依赖于下一状态
    \item \textbf{简化计算}:某些算法只需要期望奖励
    \item \textbf{价值函数计算}:贝尔曼方程中常用
\end{itemize}

\textbf{优点}:
\begin{itemize}
    \item 形式简单
    \item 计算方便
\end{itemize}

\textbf{缺点}:
\begin{itemize}
    \item 忽略了奖励的随机性
    \item 如果奖励依赖于 $s'$,信息丢失
\end{itemize}

\subsection{何时使用三参数奖励?}

\textbf{适用情况}:
\begin{itemize}
    \item \textbf{奖励依赖于下一状态}:不同转移有不同的奖励
    \item \textbf{精确建模}:需要区分不同转移的奖励
    \item \textbf{某些算法}:需要更详细的奖励信息
\end{itemize}

\textbf{优点}:
\begin{itemize}
    \item 比双参数更精确
    \item 保留了状态转移的信息
\end{itemize}

\textbf{缺点}:
\begin{itemize}
    \item 参数较多
    \item 计算稍复杂
\end{itemize}

\section{在贝尔曼方程中的应用}

\subsection{使用四参数函数}

状态价值函数的贝尔曼方程:
\begin{equation}
v_\pi(s) = \sum_a \pi(a | s) \sum_{s', r} p(s', r | s, a) [r + \gamma v_\pi(s')]
\end{equation}

\textbf{优点}:直接使用基础函数,表达清晰

\subsection{使用三参数转移和双参数奖励}

可以改写为:
\begin{equation}
v_\pi(s) = \sum_a \pi(a | s) \sum_{s'} p(s' | s, a) [r(s, a) + \gamma v_\pi(s')]
\end{equation}

\textbf{注意}:这要求奖励不依赖于 $s'$,否则需要使用 $r(s, a, s')$。

\subsection{使用三参数转移和三参数奖励}

更一般的形式:
\begin{equation}
v_\pi(s) = \sum_a \pi(a | s) \sum_{s'} p(s' | s, a) [r(s, a, s') + \gamma v_\pi(s')]
\end{equation}

\textbf{优点}:适用于奖励依赖于下一状态的情况

\section{总结}

\subsection{函数对比表}

\begin{center}
\begin{tabular}{|l|c|c|c|}
\hline
\textbf{函数} & \textbf{参数个数} & \textbf{类型} & \textbf{值域} \\
\hline
$p(s', r | s, a)$ & 4 & 概率 & $[0, 1]$ \\
$p(s' | s, a)$ & 3 & 概率 & $[0, 1]$ \\
$r(s, a)$ & 2 & 期望 & $\mathbb{R}$ \\
$r(s, a, s')$ & 3 & 条件期望 & $\mathbb{R}$ \\
\hline
\end{tabular}
\end{center}

\subsection{关键关系}

\begin{enumerate}
    \item \textbf{基础}:四参数函数 $p(s', r | s, a)$ 是基础,可以推导出其他所有函数
    
    \item \textbf{边缘化}:
    \begin{equation}
    p(s' | s, a) = \sum_{r} p(s', r | s, a)
    \end{equation}
    
    \item \textbf{期望}:
    \begin{equation}
    r(s, a) = \sum_{r} r \sum_{s'} p(s', r | s, a)
    \end{equation}
    
    \item \textbf{条件期望}:
    \begin{equation}
    r(s, a, s') = \frac{\sum_{r} r \cdot p(s', r | s, a)}{p(s' | s, a)}
    \end{equation}
    
    \item \textbf{全期望}:
    \begin{equation}
    r(s, a) = \sum_{s'} p(s' | s, a) \cdot r(s, a, s')
    \end{equation}
\end{enumerate}

\subsection{选择建议}

\begin{itemize}
    \item \textbf{理论工作}:使用四参数函数 $p(s', r | s, a)$
    \item \textbf{简单问题}:如果奖励不依赖于 $s'$,使用 $p(s' | s, a)$ 和 $r(s, a)$
    \item \textbf{复杂问题}:如果奖励依赖于 $s'$,使用 $p(s' | s, a)$ 和 $r(s, a, s')$
    \item \textbf{实际应用}:根据问题的具体特点选择最合适的表示
\end{itemize}

\vspace{1cm}

\textbf{参考文献}:
\begin{itemize}
    \item Sutton, R. S., \& Barto, A. G. (2018). Reinforcement Learning: An Introduction (2nd Edition). MIT Press, Chapter 3, Section 3.1.
\end{itemize}

\end{document}


