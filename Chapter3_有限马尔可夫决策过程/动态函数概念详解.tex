\documentclass[12pt,a4paper]{article}
\usepackage[UTF8]{ctex}
\usepackage{amsmath}
\usepackage{amssymb}
\usepackage{amsthm}
\usepackage{geometry}
\usepackage{hyperref}
\usepackage{enumitem}
\usepackage{xcolor}
\usepackage{bm}
\usepackage{tikz}

\geometry{margin=2.5cm}

\title{动态函数(Dynamics Function)概念详解}
\subtitle{什么是动态函数?为什么叫"动态"?}
\author{}
\date{}

\newtheorem{definition}{定义}
\newtheorem{theorem}{定理}
\newtheorem{proposition}{命题}
\newtheorem{example}{示例}

\begin{document}

\maketitle

\tableofcontents
\newpage

\section{引言}

"动态函数"(Dynamics Function)是马尔可夫决策过程(MDP)中的核心概念。本文档详细解释什么是动态函数,为什么它被称为"动态",以及它在强化学习中的作用。

\section{动态函数的基本定义}

\subsection{定义}

\begin{definition}[动态函数]
动态函数 $p$ 是一个函数,它完全描述了环境的\textbf{动态行为},即:给定当前状态 $s$ 和动作 $a$,它给出了下一状态 $s'$ 和奖励 $r$ 的联合概率分布。

数学上,四参数动态函数定义为:
\begin{equation}
p(s', r | s, a) \doteq \Pr\{S_{t+1} = s', R_{t+1} = r | S_t = s, A_t = a\}
\label{eq:dynamics_def}
\end{equation}
\end{definition}

\subsection{名称来源}

\textbf{"动态"(Dynamics)的含义}:
\begin{itemize}
    \item \textbf{词源}:来自希腊语"dynamis",意为"力量"或"运动"
    \item \textbf{物理学}:在物理学中,"dynamics"研究物体如何随时间变化
    \item \textbf{控制理论}:在控制理论中,"dynamics"描述系统如何响应输入
    \item \textbf{强化学习}:在强化学习中,"dynamics"描述环境如何响应智能体的动作
\end{itemize}

\textbf{为什么叫"动态"函数?}
\begin{itemize}
    \item 它描述了环境的\textbf{变化规律}
    \item 它说明了系统如何\textbf{演化}(从当前状态到下一状态)
    \item 它捕捉了环境的\textbf{行为模式}
    \item 它决定了系统的\textbf{动态特性}
\end{itemize}

\section{动态函数的本质}

\subsection{环境的"规则书"}

动态函数可以看作是环境的\textbf{规则书}或\textbf{说明书}:

\begin{quote}
\textit{"如果你在状态 $s$ 采取动作 $a$,那么环境会按照概率分布 $p(\cdot, \cdot | s, a)$ 返回下一状态和奖励。"}
\end{quote}

\textbf{类比}:
\begin{itemize}
    \item \textbf{游戏规则}:就像游戏规则书,告诉你"在某种情况下采取某个行动,会发生什么"
    \item \textbf{物理定律}:就像物理定律,描述"在某种条件下,系统如何演化"
    \item \textbf{用户手册}:就像设备的使用手册,说明"按某个按钮,设备会如何响应"
\end{itemize}

\subsection{完全描述环境}

\begin{proposition}[完全性]
动态函数 $p(s', r | s, a)$ 完全描述了环境的动态行为。给定 $p$,我们可以回答关于环境行为的任何问题。
\end{proposition}

\textbf{可以回答的问题}:
\begin{enumerate}
    \item 在状态 $s$ 采取动作 $a$,下一状态可能是哪些?概率是多少?
    \item 在状态 $s$ 采取动作 $a$,可能收到哪些奖励?概率是多少?
    \item 在状态 $s$ 采取动作 $a$,期望奖励是多少?
    \item 从状态 $s$ 开始,采取动作序列 $a_1, a_2, \ldots$,可能的状态序列是什么?
\end{enumerate}

\section{动态 vs 静态}

\subsection{静态 vs 动态}

\textbf{静态(Static)}:
\begin{itemize}
    \item 不随时间变化
    \item 固定的、不变的
    \item 例如:状态空间 $\mathcal{S}$、动作空间 $\mathcal{A}$
\end{itemize}

\textbf{动态(Dynamic)}:
\begin{itemize}
    \item 随时间变化
    \item 描述变化过程
    \item 例如:状态转移、奖励产生
\end{itemize}

\subsection{为什么是"动态"的?}

动态函数描述了\textbf{变化过程}:

\begin{center}
\begin{tikzpicture}[node distance=2cm]
    \node (s) [rectangle, draw] {$S_t = s$};
    \node (a) [rectangle, draw, below of=s] {$A_t = a$};
    \node (p) [ellipse, draw, right of=s, xshift=2cm] {$p(s', r | s, a)$};
    \node (snext) [rectangle, draw, right of=p, xshift=2cm] {$S_{t+1} = s'$};
    \node (r) [rectangle, draw, below of=snext] {$R_{t+1} = r$};
    
    \draw[->] (s) -- (p);
    \draw[->] (a) -- (p);
    \draw[->] (p) -- node[above] {变化} (snext);
    \draw[->] (p) -- (r);
    
    \node[below of=p, yshift=-1cm] {动态过程};
\end{tikzpicture}
\end{center}

\textbf{关键点}:
\begin{itemize}
    \item 从 $S_t$ 到 $S_{t+1}$:\textbf{状态变化}
    \item 产生 $R_{t+1}$:\textbf{奖励产生}
    \item 这些都是\textbf{动态过程},不是静态属性
    \item 动态函数描述了这些变化如何发生
\end{itemize}

\section{动态函数的作用}

\subsection{预测功能}

动态函数允许我们\textbf{预测}环境的响应:

\textbf{问题}:在状态 $s$ 采取动作 $a$,下一状态会是什么?

\textbf{答案}:使用动态函数计算状态转移概率:
\begin{equation}
p(s' | s, a) = \sum_{r \in \mathcal{R}} p(s', r | s, a)
\end{equation}

\textbf{问题}:期望收到多少奖励?

\textbf{答案}:使用动态函数计算期望奖励:
\begin{equation}
r(s, a) = \sum_{r \in \mathcal{R}} r \sum_{s' \in \mathcal{S}} p(s', r | s, a)
\end{equation}

\subsection{规划功能}

有了动态函数,我们可以进行\textbf{规划}:

\begin{itemize}
    \item \textbf{前向搜索}:从当前状态开始,模拟未来的可能轨迹
    \item \textbf{价值计算}:计算状态和动作的价值函数
    \item \textbf{策略优化}:找到最优策略
\end{itemize}

\subsection{学习功能}

在强化学习中,我们通常\textbf{不知道}动态函数,需要从经验中学习:

\begin{itemize}
    \item \textbf{模型学习}:从交互数据中估计 $p(s', r | s, a)$
    \item \textbf{基于模型的方法}:先学习模型,再使用模型进行规划
    \item \textbf{无模型方法}:不学习完整模型,直接从经验中学习价值或策略
\end{itemize}

\section{动态函数的特性}

\subsection{概率性}

动态函数给出的是\textbf{概率分布},不是确定性结果:

\begin{itemize}
    \item 在状态 $s$ 采取动作 $a$,结果可能是随机的
    \item $p(s', r | s, a)$ 表示"以某个概率"得到 $(s', r)$
    \item 这反映了环境的\textbf{随机性}或\textbf{不确定性}
\end{itemize}

\textbf{确定性情况}:
\begin{itemize}
    \item 如果环境是确定性的,则对每个 $(s, a)$,只有一个 $(s', r)$ 的概率为 1
    \item 其他所有 $(s', r)$ 的概率为 0
    \item 但动态函数的形式仍然相同
\end{itemize}

\subsection{马尔可夫性}

动态函数体现了\textbf{马尔可夫性质}:

\begin{equation}
p(s', r | s, a) = \Pr\{S_{t+1} = s', R_{t+1} = r | S_t = s, A_t = a\}
\end{equation}

\textbf{关键点}:
\begin{itemize}
    \item 只依赖于\textbf{当前}状态 $S_t = s$ 和动作 $A_t = a$
    \item 不依赖于\textbf{历史}:$S_{t-1}, A_{t-1}, S_{t-2}, \ldots$
    \item 这简化了问题,使得问题可处理
\end{itemize}

\subsection{完整性}

动态函数是\textbf{完整的},包含了所有必要信息:

\begin{itemize}
    \item 状态转移信息:$p(s' | s, a)$
    \item 奖励信息:$r(s, a)$ 或 $r(s, a, s')$
    \item 联合信息:$p(s', r | s, a)$
\end{itemize}

\section{动态函数的表示}

\subsection{表格形式}

对于有限 MDP,动态函数可以用\textbf{表格}表示:

\begin{center}
\begin{tabular}{|c|c|c|c|c|}
\hline
$s$ & $a$ & $s'$ & $r$ & $p(s', r | s, a)$ \\
\hline
$s_1$ & $a_1$ & $s_2$ & $+1$ & $0.8$ \\
$s_1$ & $a_1$ & $s_1$ & $-1$ & $0.2$ \\
$s_1$ & $a_2$ & $s_3$ & $0$ & $1.0$ \\
$\vdots$ & $\vdots$ & $\vdots$ & $\vdots$ & $\vdots$ \\
\hline
\end{tabular}
\end{center}

\subsection{函数形式}

对于连续状态空间,动态函数是\textbf{函数}:
\begin{equation}
p: \mathcal{S} \times \mathcal{R} \times \mathcal{S} \times \mathcal{A} \to [0, 1]
\end{equation}

\subsection{参数化形式}

在实际应用中,动态函数可能被\textbf{参数化}:
\begin{equation}
p_\theta(s', r | s, a)
\end{equation}
其中 $\theta$ 是参数,需要从数据中学习。

\section{动态函数与其他概念的关系}

\subsection{与状态转移矩阵的关系}

在离散状态空间中,动态函数可以表示为\textbf{状态转移矩阵}:

\begin{itemize}
    \item 对每个动作 $a$,有一个转移矩阵 $P^a$
    \item $P^a_{ij} = p(s_j | s_i, a)$
    \item 这是动态函数的简化形式(忽略了奖励)
\end{itemize}

\subsection{与奖励函数的关系}

动态函数包含了奖励信息:

\begin{itemize}
    \item 可以从动态函数中提取奖励函数:$r(s, a)$ 或 $r(s, a, s')$
    \item 奖励函数是动态函数的一部分
    \item 但动态函数还包含状态转移信息
\end{itemize}

\subsection{与模型的关系}

在强化学习中,\textbf{模型(Model)}通常指动态函数:

\begin{itemize}
    \item \textbf{有模型(Model-based)}:知道或学习了动态函数
    \item \textbf{无模型(Model-free)}:不知道动态函数,直接从经验学习
    \item 动态函数就是环境的"模型"
\end{itemize}

\section{实际例子}

\subsection{例子 1:网格世界}

在网格世界中,动态函数描述:
\begin{itemize}
    \item 从某个格子(状态)向某个方向移动(动作)
    \item 可能到达的下一格子(下一状态)
    \item 可能收到的奖励(如:到达目标 $+10$,撞墙 $-1$)
\end{itemize}

\textbf{动态性体现在}:
\begin{itemize}
    \item 状态在\textbf{变化}(从一个格子到另一个格子)
    \item 奖励在\textbf{产生}(根据结果给出奖励)
    \item 这些是\textbf{动态过程},不是静态属性
\end{itemize}

\subsection{例子 2:游戏}

在游戏中(如 Atari 游戏),动态函数描述:
\begin{itemize}
    \item 当前游戏画面(状态)
    \item 玩家的操作(动作,如按左键)
    \item 下一帧游戏画面(下一状态)
    \item 得分变化(奖励)
\end{itemize}

\textbf{动态性体现在}:
\begin{itemize}
    \item 游戏画面在\textbf{变化}(动画、移动)
    \item 游戏状态在\textbf{演化}(敌人移动、物体碰撞)
    \item 这些都是\textbf{动态过程}
\end{itemize}

\subsection{例子 3:机器人控制}

在机器人控制中,动态函数描述:
\begin{itemize}
    \item 机器人的当前姿态和位置(状态)
    \item 发送给电机的控制信号(动作)
    \item 机器人的下一姿态和位置(下一状态)
    \item 任务相关的反馈(奖励,如距离目标的减少)
\end{itemize}

\textbf{动态性体现在}:
\begin{itemize}
    \item 机器人的位置和姿态在\textbf{变化}
    \item 物理系统在\textbf{演化}
    \item 这些是\textbf{动态过程},遵循物理定律
\end{itemize}

\section{为什么需要动态函数?}

\subsection{预测未来}

有了动态函数,我们可以\textbf{预测}:
\begin{itemize}
    \item 如果采取某个动作序列,会发生什么?
    \item 可能到达哪些状态?
    \item 可能收到多少奖励?
\end{itemize}

\subsection{规划最优行为}

有了动态函数,我们可以\textbf{规划}:
\begin{itemize}
    \item 计算最优策略
    \item 评估不同策略的价值
    \item 进行前向搜索
\end{itemize}

\subsection{理解环境}

动态函数帮助我们\textbf{理解}环境:
\begin{itemize}
    \item 环境如何响应动作?
    \item 哪些状态-动作对是好的?
    \item 环境的随机性如何?
\end{itemize}

\section{动态函数的局限性}

\subsection{完全知识的假设}

动态函数假设我们\textbf{完全知道}环境:
\begin{itemize}
    \item 在现实中,我们通常不知道完整的动态函数
    \item 需要从经验中学习或估计
    \item 学习可能不准确或不完整
\end{itemize}

\subsection{计算复杂度}

对于大状态空间:
\begin{itemize}
    \item 存储完整的动态函数需要大量内存
    \item 计算可能非常耗时
    \item 需要近似方法
\end{itemize}

\subsection{非平稳性}

如果环境是\textbf{非平稳}的:
\begin{itemize}
    \item 动态函数会随时间变化
    \item 需要持续更新模型
    \item 增加了学习难度
\end{itemize}

\section{总结}

\subsection{核心概念}

\begin{enumerate}
    \item \textbf{定义}:动态函数 $p(s', r | s, a)$ 描述了环境的动态行为
    
    \item \textbf{为什么叫"动态"}:
    \begin{itemize}
        \item 描述了系统的\textbf{变化过程}
        \item 说明了状态如何\textbf{演化}
        \item 捕捉了环境的\textbf{行为模式}
    \end{itemize}
    
    \item \textbf{作用}:
    \begin{itemize}
        \item 预测环境的响应
        \item 规划最优行为
        \item 理解环境特性
    \end{itemize}
    
    \item \textbf{特性}:
    \begin{itemize}
        \item 概率性(可能随机)
        \item 马尔可夫性(只依赖于当前状态和动作)
        \item 完整性(包含所有必要信息)
    \end{itemize}
\end{enumerate}

\subsection{关键要点}

\begin{itemize}
    \item 动态函数是环境的\textbf{规则书},描述了"如果...那么..."的关系
    
    \item "动态"强调\textbf{变化}和\textbf{演化},而不是静态属性
    
    \item 动态函数完全描述了环境,可以从中推导出所有其他信息
    
    \item 在强化学习中,我们通常需要学习或估计动态函数
    
    \item 动态函数是连接"当前"和"未来"的桥梁
\end{itemize}

\subsection{记忆技巧}

\begin{itemize}
    \item \textbf{动态} = \textbf{变化} = 描述系统如何随时间演化
    
    \item \textbf{函数} = 输入 $(s, a)$,输出 $(s', r)$ 的概率分布
    
    \item \textbf{完全描述} = 知道动态函数,就知道环境的一切
    
    \item \textbf{预测工具} = 可以用动态函数预测未来
\end{itemize}

\vspace{1cm}

\textbf{参考文献}:
\begin{itemize}
    \item Sutton, R. S., \& Barto, A. G. (2018). Reinforcement Learning: An Introduction (2nd Edition). MIT Press, Chapter 3, Section 3.1.
    \item 书中说明:"The function $p$ defines the dynamics of the MDP."
\end{itemize}

\end{document}


