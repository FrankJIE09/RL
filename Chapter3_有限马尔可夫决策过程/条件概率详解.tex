\documentclass[12pt,a4paper]{article}
\usepackage[UTF8]{ctex}
\usepackage{amsmath}
\usepackage{amssymb}
\usepackage{amsthm}
\usepackage{geometry}
\usepackage{hyperref}
\usepackage{enumitem}
\usepackage{xcolor}
\usepackage{bm}
\usepackage{tikz}

\geometry{margin=2.5cm}

\title{条件概率(Conditional Probability)详解}
\subtitle{定义、直观理解与应用}
\author{}
\date{}

\newtheorem{definition}{定义}
\newtheorem{theorem}{定理}
\newtheorem{proposition}{命题}
\newtheorem{example}{示例}

\begin{document}

\maketitle

\tableofcontents
\newpage

\section{引言}

条件概率是概率论中的核心概念,也是理解强化学习中动态函数、价值函数等概念的基础。本文档详细解释条件概率的定义、直观含义、数学性质和实际应用。

\section{条件概率的定义}

\subsection{基本定义}

\begin{definition}[条件概率]
对于事件 $A$ 和 $B$,在事件 $B$ 发生的条件下,事件 $A$ 发生的条件概率定义为:
\begin{equation}
\Pr\{A | B\} = \frac{\Pr\{A \cap B\}}{\Pr\{B\}}, \quad \text{如果 } \Pr\{B\} > 0
\label{eq:conditional_prob}
\end{equation}
其中:
\begin{itemize}
    \item $A \cap B$ 表示事件 $A$ 和 $B$ 同时发生(交集)
    \item $\Pr\{A \cap B\}$ 是 $A$ 和 $B$ 的联合概率
    \item $\Pr\{B\}$ 是事件 $B$ 的概率(条件)
\end{itemize}
\end{definition}

\subsection{符号说明}

\begin{itemize}
    \item \textbf{竖线 $|$}:表示"在...条件下"或"给定..."
    \item \textbf{读法}:$\Pr\{A | B\}$ 读作"在 $B$ 发生的条件下,$A$ 发生的概率"或"给定 $B$,$A$ 的概率"
    \item \textbf{条件}:$B$ 是条件(已知信息)
    \item \textbf{目标}:$A$ 是我们关心的事件
\end{itemize}

\subsection{条件}

\textbf{重要条件}:$\Pr\{B\} > 0$

\begin{itemize}
    \item 如果 $\Pr\{B\} = 0$,则条件概率未定义
    \item 因为分母不能为 0
    \item 这保证了定义的合理性
\end{itemize}

\section{直观理解}

\subsection{基本思想}

条件概率回答的问题是:
\begin{quote}
\textit{"在已知事件 $B$ 发生的情况下,事件 $A$ 发生的概率是多少?"}
\end{quote}

\textbf{关键点}:
\begin{itemize}
    \item 我们有了\textbf{额外的信息}($B$ 发生了)
    \item 这个信息\textbf{改变了}我们对 $A$ 发生概率的估计
    \item 条件概率是"更新后的"概率
\end{itemize}

\subsection{样本空间缩小}

\textbf{直观理解}:

条件概率可以理解为"缩小样本空间":

\begin{enumerate}
    \item \textbf{原始样本空间}:所有可能的结果
    \item \textbf{条件 $B$ 发生}:样本空间缩小到 $B$ 中的结果
    \item \textbf{在缩小后的空间中}:计算 $A$ 发生的概率
\end{enumerate}

\textbf{公式解释}:
\begin{equation}
\Pr\{A | B\} = \frac{\Pr\{A \cap B\}}{\Pr\{B\}} = \frac{\text{既在 } A \text{ 中又在 } B \text{ 中的概率}}{\text{在 } B \text{ 中的概率}}
\end{equation}

\subsection{可视化理解}

\begin{center}
\begin{tikzpicture}[scale=1.5]
    % 样本空间
    \draw[fill=gray!20] (0,0) rectangle (4,3);
    \node at (2,3.3) {样本空间 $\Omega$};
    
    % 事件 B
    \draw[fill=blue!30] (1,0.5) ellipse (1.2 and 1);
    \node at (1,1.5) {$B$};
    
    % 事件 A
    \draw[fill=red!30] (2.5,1) ellipse (1 and 1.2);
    \node at (3,1.5) {$A$};
    
    % 交集 A ∩ B
    \begin{scope}
        \clip (1,0.5) ellipse (1.2 and 1);
        \fill[purple!50] (2.5,1) ellipse (1 and 1.2);
    \end{scope}
    \node[white] at (2,1.2) {$A \cap B$};
    
    % 标注
    \node[below] at (2,0) {$\Pr\{A | B\} = \frac{\text{紫色区域}}{\text{蓝色区域}}$};
\end{tikzpicture}
\end{center}

\textbf{解释}:
\begin{itemize}
    \item 蓝色区域:事件 $B$(条件)
    \item 红色区域:事件 $A$(目标)
    \item 紫色区域:$A \cap B$(同时发生)
    \item 条件概率 = 紫色区域 / 蓝色区域
\end{itemize}

\section{数学推导}

\subsection{从定义出发}

条件概率的定义可以从以下思路理解:

\textbf{思路}:
\begin{enumerate}
    \item 我们知道 $B$ 发生了,所以只考虑 $B$ 中的结果
    \item 在 $B$ 中,$A$ 发生的部分就是 $A \cap B$
    \item 因此,条件概率应该是 $\frac{\Pr\{A \cap B\}}{\Pr\{B\}}$
\end{enumerate}

\subsection{乘法法则}

从条件概率定义可以推导出\textbf{乘法法则}:

\begin{theorem}[乘法法则]
\begin{equation}
\Pr\{A \cap B\} = \Pr\{A | B\} \cdot \Pr\{B\} = \Pr\{B | A\} \cdot \Pr\{A\}
\label{eq:multiplication}
\end{equation}
\end{theorem}

\textbf{证明}:
从定义 $\Pr\{A | B\} = \frac{\Pr\{A \cap B\}}{\Pr\{B\}}$,两边同时乘以 $\Pr\{B\}$:
\begin{equation}
\Pr\{A | B\} \cdot \Pr\{B\} = \Pr\{A \cap B\}
\end{equation}

\textbf{应用}:计算两个事件同时发生的概率。

\subsection{全概率公式}

\begin{theorem}[全概率公式(Law of Total Probability)]
如果事件 $B_1, B_2, \ldots, B_n$ 构成样本空间的一个划分(互斥且完备),则:
\begin{equation}
\Pr\{A\} = \sum_{i=1}^n \Pr\{A | B_i\} \cdot \Pr\{B_i\}
\label{eq:total_prob}
\end{equation}
\end{theorem}

\textbf{解释}:
\begin{itemize}
    \item 将事件 $A$ 的概率分解为多个条件概率的加权和
    \item 权重是各个条件 $B_i$ 的概率
    \item 这是计算复杂事件概率的重要工具
\end{itemize}

\section{具体例子}

\subsection{例子 1:投掷骰子}

\textbf{问题}:投掷一个公平的六面骰子。

\textbf{事件定义}:
\begin{itemize}
    \item $A$:点数为偶数 $\{2, 4, 6\}$
    \item $B$:点数大于 3 $\{4, 5, 6\}$
\end{itemize}

\textbf{无条件概率}:
\begin{align}
\Pr\{A\} &= \frac{3}{6} = \frac{1}{2} \\
\Pr\{B\} &= \frac{3}{6} = \frac{1}{2}
\end{align}

\textbf{联合概率}:
\begin{align}
A \cap B &= \{4, 6\} \quad \text{(既是偶数又大于 3)} \\
\Pr\{A \cap B\} &= \frac{2}{6} = \frac{1}{3}
\end{align}

\textbf{条件概率}:
\begin{align}
\Pr\{A | B\} &= \frac{\Pr\{A \cap B\}}{\Pr\{B\}} = \frac{1/3}{1/2} = \frac{2}{3}
\end{align}

\textbf{解释}:
\begin{itemize}
    \item 无条件:点数为偶数的概率是 $1/2$
    \item 条件:已知点数大于 3,点数为偶数的概率是 $2/3$
    \item 信息 $B$ 改变了我们对 $A$ 的估计
\end{itemize}

\subsection{例子 2:从动态函数理解}

在 MDP 中,动态函数 $p(s', r | s, a)$ 就是条件概率:

\begin{equation}
p(s', r | s, a) = \Pr\{S_{t+1} = s', R_{t+1} = r | S_t = s, A_t = a\}
\end{equation}

\textbf{解释}:
\begin{itemize}
    \item \textbf{条件}:$S_t = s, A_t = a$(当前状态和动作)
    \item \textbf{目标}:$S_{t+1} = s', R_{t+1} = r$(下一状态和奖励)
    \item \textbf{含义}:在给定当前状态和动作的条件下,下一状态和奖励的联合概率
\end{itemize}

\subsection{例子 3:期望奖励}

期望奖励 $r(s, a)$ 使用了条件期望:

\begin{equation}
r(s, a) = \mathbb{E}[R_{t+1} | S_t = s, A_t = a]
\end{equation}

\textbf{展开为条件概率}:
\begin{align}
r(s, a) &= \sum_{r} r \cdot \Pr\{R_{t+1} = r | S_t = s, A_t = a\} \\
        &= \sum_{r} r \sum_{s'} p(s', r | s, a)
\end{align}

\textbf{解释}:
\begin{itemize}
    \item 条件概率:$\Pr\{R_{t+1} = r | S_t = s, A_t = a\}$
    \item 表示:在给定状态 $s$ 和动作 $a$ 的条件下,奖励为 $r$ 的概率
    \item 期望:所有可能的奖励值乘以对应条件概率的求和
\end{itemize}

\section{条件概率的性质}

\subsection{基本性质}

\begin{enumerate}
    \item \textbf{非负性}:$\Pr\{A | B\} \geq 0$
    
    \item \textbf{归一性}:如果 $A_1, A_2, \ldots, A_n$ 构成样本空间的划分,则:
    \begin{equation}
    \sum_{i=1}^n \Pr\{A_i | B\} = 1
    \end{equation}
    
    \item \textbf{概率值域}:$0 \leq \Pr\{A | B\} \leq 1$
\end{enumerate}

\subsection{与无条件概率的关系}

\textbf{一般情况}:
\begin{equation}
\Pr\{A | B\} \neq \Pr\{A\}
\end{equation}
条件概率通常不同于无条件概率。

\textbf{特殊情况:独立性}

如果 $A$ 和 $B$ 独立,则:
\begin{equation}
\Pr\{A | B\} = \Pr\{A\}
\end{equation}
此时,知道 $B$ 发生不改变 $A$ 的概率。

\section{条件概率的链式法则}

\subsection{多个条件的条件概率}

\begin{definition}[多条件条件概率]
对于事件 $A, B, C$,条件概率可以链式应用:
\begin{equation}
\Pr\{A | B, C\} = \frac{\Pr\{A \cap B | C\}}{\Pr\{B | C\}} = \frac{\Pr\{A \cap B \cap C\}}{\Pr\{B \cap C\}}
\end{equation}
\end{definition}

\textbf{在 MDP 中的应用}:

三参数期望奖励使用了条件概率:
\begin{equation}
r(s, a, s') = \mathbb{E}[R_{t+1} | S_t = s, A_t = a, S_{t+1} = s']
\end{equation}

\textbf{对应的条件概率}:
\begin{equation}
\Pr\{R_{t+1} = r | S_t = s, A_t = a, S_{t+1} = s'\} = \frac{p(s', r | s, a)}{p(s' | s, a)}
\end{equation}

\section{贝叶斯定理}

\subsection{贝叶斯公式}

\begin{theorem}[贝叶斯定理(Bayes' Theorem)]
\begin{equation}
\Pr\{A | B\} = \frac{\Pr\{B | A\} \cdot \Pr\{A\}}{\Pr\{B\}}
\label{eq:bayes}
\end{equation}
\end{theorem}

\textbf{证明}:
从乘法法则:
\begin{align}
\Pr\{A \cap B\} &= \Pr\{A | B\} \cdot \Pr\{B\} \\
                &= \Pr\{B | A\} \cdot \Pr\{A\}
\end{align}
因此:
\begin{equation}
\Pr\{A | B\} \cdot \Pr\{B\} = \Pr\{B | A\} \cdot \Pr\{A\}
\end{equation}
两边同时除以 $\Pr\{B\}$ 得到贝叶斯公式。

\textbf{应用}:
\begin{itemize}
    \item 从 $\Pr\{B | A\}$ 计算 $\Pr\{A | B\}$
    \item 在机器学习中用于参数估计
    \item 在强化学习中用于模型学习
\end{itemize}

\section{在强化学习中的应用}

\subsection{动态函数}

动态函数是条件概率:
\begin{equation}
p(s', r | s, a) = \Pr\{S_{t+1} = s', R_{t+1} = r | S_t = s, A_t = a\}
\end{equation}

\textbf{条件}:$S_t = s, A_t = a$(当前状态和动作)
\textbf{目标}:$S_{t+1} = s', R_{t+1} = r$(下一状态和奖励)

\subsection{状态转移概率}

状态转移概率也是条件概率:
\begin{equation}
p(s' | s, a) = \Pr\{S_{t+1} = s' | S_t = s, A_t = a\}
\end{equation}

\textbf{从联合概率得到}:
\begin{equation}
p(s' | s, a) = \frac{\sum_{r} p(s', r | s, a)}{\sum_{s', r} p(s', r | s, a)} = \sum_{r} p(s', r | s, a)
\end{equation}

\subsection{策略}

策略也是条件概率:
\begin{equation}
\pi(a | s) = \Pr\{A_t = a | S_t = s\}
\end{equation}

\textbf{含义}:在状态 $s$ 的条件下,选择动作 $a$ 的概率。

\subsection{价值函数}

价值函数使用条件期望:
\begin{equation}
v_\pi(s) = \mathbb{E}_\pi[G_t | S_t = s] = \mathbb{E}_\pi\left[\sum_{k=0}^{\infty} \gamma^k R_{t+k+1} \middle| S_t = s\right]
\end{equation}

\textbf{条件}:$S_t = s$(当前状态)
\textbf{目标}:回报 $G_t$ 的期望值

\section{常见误解}

\subsection{误解 1:条件概率就是联合概率}

\textbf{误解}:
\begin{quote}
"$\Pr\{A | B\}$ 和 $\Pr\{A \cap B\}$ 是一样的。"
\end{quote}

\textbf{纠正}:
\begin{itemize}
    \item \textbf{联合概率}:$\Pr\{A \cap B\}$ 是 $A$ 和 $B$ 同时发生的概率
    \item \textbf{条件概率}:$\Pr\{A | B\}$ 是在已知 $B$ 发生的条件下,$A$ 发生的概率
    \item \textbf{关系}:$\Pr\{A | B\} = \frac{\Pr\{A \cap B\}}{\Pr\{B\}}$
    \item \textbf{区别}:条件概率的分母是 $\Pr\{B\}$,联合概率没有这个分母
\end{itemize}

\subsection{误解 2:$\Pr\{A | B\} = \Pr\{B | A\}$}

\textbf{误解}:
\begin{quote}
"条件概率是对称的,$\Pr\{A | B\} = \Pr\{B | A\}$。"
\end{quote}

\textbf{纠正}:
\begin{itemize}
    \item 一般情况下,$\Pr\{A | B\} \neq \Pr\{B | A\}$
    \item 它们的关系由贝叶斯定理给出:
    \begin{equation}
    \Pr\{A | B\} = \frac{\Pr\{B | A\} \cdot \Pr\{A\}}{\Pr\{B\}}
    \end{equation}
    \item 只有当 $\Pr\{A\} = \Pr\{B\}$ 时,它们才相等
\end{itemize}

\textbf{例子}:
\begin{itemize}
    \item $\Pr\{\text{下雨} | \text{有云}\}$:有云时下雨的概率(可能较高)
    \item $\Pr\{\text{有云} | \text{下雨}\}$:下雨时有云的概率(几乎为 1)
    \item 这两个概率明显不同!
\end{itemize}

\subsection{误解 3:条件概率总是小于无条件概率}

\textbf{误解}:
\begin{quote}
"有了条件,概率应该变小。"
\end{quote}

\textbf{纠正}:
\begin{itemize}
    \item 条件概率可能大于、等于或小于无条件概率
    \item 取决于条件 $B$ 与事件 $A$ 的关系
    \item 如果 $B$ 有利于 $A$,则 $\Pr\{A | B\} > \Pr\{A\}$
    \item 如果 $B$ 不利于 $A$,则 $\Pr\{A | B\} < \Pr\{A\}$
\end{itemize}

\textbf{例子}:
\begin{itemize}
    \item $\Pr\{\text{偶数} | \text{大于 3}\} = 2/3 > 1/2 = \Pr\{\text{偶数}\}$(条件增加了概率)
    \item $\Pr\{\text{小于 2} | \text{大于 3}\} = 0 < 1/3 = \Pr\{\text{小于 2}\}$(条件减少了概率)
\end{itemize}

\section{条件概率的计算步骤}

\subsection{一般步骤}

计算条件概率 $\Pr\{A | B\}$ 的步骤:

\begin{enumerate}
    \item \textbf{识别条件}:确定事件 $B$(条件)
    \item \textbf{识别目标}:确定事件 $A$(目标)
    \item \textbf{计算联合概率}:$\Pr\{A \cap B\}$
    \item \textbf{计算条件概率}:$\Pr\{B\}$
    \item \textbf{应用公式}:$\Pr\{A | B\} = \frac{\Pr\{A \cap B\}}{\Pr\{B\}}$
\end{enumerate}

\subsection{在 MDP 中的计算}

计算 $p(s', r | s, a)$ 的步骤:

\begin{enumerate}
    \item \textbf{条件}:$S_t = s, A_t = a$
    \item \textbf{目标}:$S_{t+1} = s', R_{t+1} = r$
    \item \textbf{方法}:从经验数据中估计
    \begin{equation}
    p(s', r | s, a) \approx \frac{\text{在 } (s, a) \text{ 后观察到 } (s', r) \text{ 的次数}}{\text{在 } (s, a) \text{ 的总次数}}
    \end{equation}
\end{enumerate}

\section{总结}

\subsection{核心定义}

\begin{equation}
\Pr\{A | B\} = \frac{\Pr\{A \cap B\}}{\Pr\{B\}}, \quad \text{如果 } \Pr\{B\} > 0
\end{equation}

\subsection{关键理解}

\begin{enumerate}
    \item \textbf{条件概率}:在已知 $B$ 发生的条件下,$A$ 发生的概率
    
    \item \textbf{直观理解}:
    \begin{itemize}
        \item 样本空间缩小到 $B$
        \item 在缩小后的空间中计算 $A$ 的概率
        \item 公式:$\frac{A \cap B \text{ 的概率}}{B \text{ 的概率}}$
    \end{itemize}
    
    \item \textbf{与无条件概率的关系}:
    \begin{itemize}
        \item 一般不同:$\Pr\{A | B\} \neq \Pr\{A\}$
        \item 独立性:如果 $A$ 和 $B$ 独立,则 $\Pr\{A | B\} = \Pr\{A\}$
    \end{itemize}
    
    \item \textbf{在强化学习中的应用}:
    \begin{itemize}
        \item 动态函数:$p(s', r | s, a)$
        \item 策略:$\pi(a | s)$
        \item 价值函数:条件期望
    \end{itemize}
\end{enumerate}

\subsection{记忆技巧}

\begin{itemize}
    \item \textbf{条件概率} = "在...条件下"的概率
    \item \textbf{公式} = 联合概率 / 条件概率
    \item \textbf{直观} = 缩小样本空间
    \item \textbf{应用} = 动态函数、策略、价值函数
\end{itemize}

\vspace{1cm}

\textbf{参考文献}:
\begin{itemize}
    \item Sutton, R. S., \& Barto, A. G. (2018). Reinforcement Learning: An Introduction (2nd Edition). MIT Press, Chapter 3.
    \item 任何概率论和数理统计教材
\end{itemize}

\end{document}


