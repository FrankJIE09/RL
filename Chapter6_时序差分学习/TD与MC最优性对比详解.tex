\documentclass[12pt,a4paper]{article}
\usepackage[UTF8]{ctex}
\usepackage{amsmath}
\usepackage{amssymb}
\usepackage{amsthm}
\usepackage{geometry}
\usepackage{hyperref}
\usepackage{enumitem}
\usepackage{xcolor}
\usepackage{bm}
\usepackage{booktabs}
\usepackage{array}

\geometry{margin=2.5cm}

\title{TD与蒙特卡洛方法的最优性对比详解}
\subtitle{为什么说"蒙特卡洛方法只在有限意义下最优,而TD在预测回报方面更相关地最优"?}
\author{}
\date{}

\newtheorem{definition}{定义}
\newtheorem{theorem}{定理}
\newtheorem{proposition}{命题}
\newtheorem{example}{示例}
\newtheorem{remark}{注记}

\begin{document}

\maketitle

\tableofcontents
\newpage

\section{核心问题}

\textbf{关键陈述}:
\begin{quote}
"The Monte Carlo method is optimal only in a limited way, and that TD is optimal in a way that is more relevant to predicting returns."
\end{quote}

\textbf{中文翻译}:
\begin{quote}
"蒙特卡洛方法只在有限意义下最优,而TD在预测回报方面更相关地最优。"
\end{quote}

\textbf{问题}:这句话是什么意思?为什么说MC是"有限的最优",而TD是"更相关地最优"?

\section{背景:批量更新(Batch Updating)}

\subsection{什么是批量更新?}

\textbf{批量更新}:给定有限的经验数据(如10个回合或100个时间步),重复处理这些数据直到算法收敛。

\textbf{过程}:
\begin{enumerate}
    \item 给定价值函数 $V$,对每个时间步计算更新增量
    \item 但价值函数只更新一次(所有增量的和)
    \item 用新的价值函数再次处理所有经验数据
    \item 重复直到收敛
\end{enumerate}

\textbf{关键特性}:
\begin{itemize}
    \item 在批量更新下,TD(0)和常数-$\alpha$ MC都确定性地收敛
    \item 但它们收敛到\textbf{不同的答案}
    \item 理解这两个答案的差异是关键
\end{itemize}

\section{蒙特卡洛方法的"有限最优性"}

\subsection{蒙特卡洛方法的最优性}

\textbf{定理}:批量蒙特卡洛方法总是找到在\textbf{训练集}上最小化均方误差的估计。

\textbf{数学表达}:
\begin{equation}
V_{\text{MC}} = \arg\min_V \sum_{t} [G_t - V(S_t)]^2
\end{equation}

其中求和是对训练集中所有状态访问的。

\textbf{解释}:
\begin{itemize}
    \item 蒙特卡洛方法找到的是在\textbf{已观察到的回报}上误差最小的估计
    \item 它最小化的是\textbf{训练数据上的误差}
    \item 这是"有限的最优性":只对训练数据最优
\end{itemize}

\subsection{为什么是"有限的"?}

\textbf{关键限制}:
\begin{enumerate}
    \item \textbf{只考虑训练数据}:
    \begin{itemize}
        \item MC方法只优化在已观察到的回报上的误差
        \item 它不考虑马尔可夫性质
        \item 它不考虑状态转移的结构
    \end{itemize}
    
    \item \textbf{不利用马尔可夫性质}:
    \begin{itemize}
        \item MC方法直接使用观察到的回报
        \item 它不利用状态之间的转移关系
        \item 它不利用"从A到B的转移"这一信息
    \end{itemize}
    
    \item \textbf{对未来数据的泛化能力有限}:
    \begin{itemize}
        \item 在训练数据上误差最小,不一定在测试数据上误差最小
        \item 如果过程是马尔可夫的,利用转移结构可能更好
    \end{itemize}
\end{enumerate}

\section{TD方法的"更相关的最优性"}

\subsection{TD方法的最优性}

\textbf{定理}:批量TD(0)总是找到对\textbf{最大似然马尔可夫过程模型}完全正确的估计。

\textbf{数学表达}:
\begin{itemize}
    \item 首先,从观察到的数据构建最大似然模型:
    \begin{align}
    \hat{p}(s' | s) &= \frac{\text{从 } s \text{ 到 } s' \text{ 的转移次数}}{\text{从 } s \text{ 出发的总转移次数}} \\
    \hat{r}(s, s') &= \text{从 } s \text{ 到 } s' \text{ 的观察到的奖励的平均值}
    \end{align}
    
    \item 然后,计算在这个模型下完全正确的价值函数估计
    \item 这个估计称为\textbf{确定性等价估计}(Certainty-Equivalence Estimate)
\end{itemize}

\textbf{解释}:
\begin{itemize}
    \item TD方法找到的是对\textbf{马尔可夫过程模型}的最优估计
    \item 它利用状态转移的结构
    \item 它假设模型是准确的,然后计算相应的价值函数
\end{itemize}

\subsection{为什么"更相关"?}

\textbf{关键优势}:
\begin{enumerate}
    \item \textbf{利用马尔可夫性质}:
    \begin{itemize}
        \item TD方法利用状态之间的转移关系
        \item 它知道"从A总是转移到B"
        \item 它利用这个结构信息
    \end{itemize}
    
    \item \textbf{对未来数据的泛化能力更强}:
    \begin{itemize}
        \item 如果过程确实是马尔可夫的,TD的估计在测试数据上误差更小
        \item 它能够预测未观察到的状态转移
        \item 它能够利用转移结构进行推理
    \end{itemize}
    
    \item \textbf{更符合预测回报的目标}:
    \begin{itemize}
        \item 预测回报的目标是估计从状态开始的期望回报
        \item 这需要理解状态转移和奖励结构
        \item TD方法正是基于这个结构进行估计
    \end{itemize}
\end{enumerate}

\section{具体例子:Example 6.4}

\subsection{问题设置}

\textbf{观察到的8个回合}:
\begin{align}
&\text{回合1: } A, 0, B, 0 \\
&\text{回合2-7: } B, 1 \quad \text{(6次)} \\
&\text{回合8: } B, 0
\end{align}

\textbf{问题}:给定这些数据,最优的 $V(A)$ 和 $V(B)$ 是什么?

\subsection{两个合理的答案}

\textbf{答案1:蒙特卡洛方法}

\textbf{对于 $V(B)$}:
\begin{itemize}
    \item 在状态 $B$ 观察到的回报:6次是1,2次是0
    \item 平均回报:$\frac{6 \times 1 + 2 \times 0}{8} = \frac{6}{8} = \frac{3}{4}$
    \item 因此:$V(B) = \frac{3}{4}$
\end{itemize}

\textbf{对于 $V(A)$}:
\begin{itemize}
    \item 从状态 $A$ 只观察到一个回报:0
    \item 因此:$V(A) = 0$
    \item 这个答案在训练数据上误差为0(完美拟合)
\end{itemize}

\textbf{答案2:TD(0)方法}

\textbf{对于 $V(B)$}:
\begin{itemize}
    \item 与MC相同:$V(B) = \frac{3}{4}$
\end{itemize}

\textbf{对于 $V(A)$}:
\begin{itemize}
    \item 观察:100\%的时间从 $A$ 转移到 $B$(奖励为0)
    \item 已知:$V(B) = \frac{3}{4}$
    \item 因此:$V(A) = 0 + \gamma \times \frac{3}{4} = \frac{3}{4}$(假设 $\gamma = 1$)
    \item 这个答案基于马尔可夫过程模型
\end{itemize}

\subsection{哪个答案更好?}

\textbf{在训练数据上}:
\begin{itemize}
    \item MC答案:$V(A) = 0$,误差为0(完美)
    \item TD答案:$V(A) = \frac{3}{4}$,误差不为0
    \item MC在训练数据上更优
\end{itemize}

\textbf{在未来数据上}:
\begin{itemize}
    \item 如果过程是马尔可夫的,从 $A$ 总是转移到 $B$
    \item 而 $B$ 的期望回报是 $\frac{3}{4}$
    \item 因此,$A$ 的期望回报也应该是 $\frac{3}{4}$
    \item TD答案在未来数据上误差更小
\end{itemize}

\textbf{关键洞察}:
\begin{quote}
MC方法在训练数据上最优,但TD方法在预测未来回报方面更优,因为它利用了马尔可夫性质。
\end{quote}

\section{数学形式化}

\subsection{蒙特卡洛方法的最优性}

\textbf{目标函数}:
\begin{equation}
V_{\text{MC}} = \arg\min_V \sum_{(s, G) \in \mathcal{D}} [G - V(s)]^2
\end{equation}

其中 $\mathcal{D}$ 是训练数据集,包含状态-回报对 $(s, G)$。

\textbf{解}:
\begin{equation}
V_{\text{MC}}(s) = \frac{1}{N(s)} \sum_{i=1}^{N(s)} G_i(s)
\end{equation}

其中 $N(s)$ 是状态 $s$ 在训练数据中出现的次数,$G_i(s)$ 是第 $i$ 次访问 $s$ 时的回报。

\textbf{特性}:
\begin{itemize}
    \item 这是训练数据上均方误差的最小值
    \item 在训练数据上误差最小
    \item 但不一定对未来数据最优
\end{itemize}

\subsection{TD方法的最优性}

\textbf{步骤1:构建最大似然模型}

从观察到的数据估计转移概率和奖励:
\begin{align}
\hat{p}(s' | s) &= \frac{N(s, s')}{N(s)} \\
\hat{r}(s, s') &= \frac{1}{N(s, s')} \sum_{i=1}^{N(s, s')} r_i(s, s')
\end{align}

其中 $N(s, s')$ 是从 $s$ 到 $s'$ 的转移次数。

\textbf{步骤2:计算确定性等价估计}

在假设模型准确的情况下,价值函数满足贝尔曼方程:
\begin{equation}
V_{\text{TD}}(s) = \sum_{s'} \hat{p}(s' | s) [\hat{r}(s, s') + \gamma V_{\text{TD}}(s')]
\end{equation}

\textbf{特性}:
\begin{itemize}
    \item 这是对最大似然模型完全正确的估计
    \item 利用状态转移的结构
    \item 对未来数据(如果过程是马尔可夫的)泛化能力更强
\end{itemize}

\section{为什么TD"更相关"?}

\subsection{预测回报的目标}

\textbf{目标}:估计从状态 $s$ 开始的期望回报 $v_\pi(s) = \mathbb{E}_\pi[G_t | S_t = s]$。

\textbf{关键点}:
\begin{itemize}
    \item 这个目标需要理解状态转移和奖励结构
    \item 它需要利用马尔可夫性质
    \item 它需要能够预测未观察到的状态序列
\end{itemize}

\subsection{TD方法的优势}

\textbf{1. 利用马尔可夫性质}:
\begin{itemize}
    \item TD方法基于状态转移结构进行估计
    \item 它知道"从A总是转移到B"
    \item 它利用这个信息来估计 $V(A)$
\end{itemize}

\textbf{2. 泛化能力}:
\begin{itemize}
    \item TD方法能够预测未观察到的状态序列
    \item 它知道即使没有从 $A$ 开始的完整轨迹,也可以利用转移结构
    \item 它能够从部分观察中推断完整信息
\end{itemize}

\textbf{3. 对未来数据的预测}:
\begin{itemize}
    \item 如果过程是马尔可夫的,TD的估计在未来数据上误差更小
    \item 它能够利用转移结构进行推理
    \item 它更符合预测回报的目标
\end{itemize}

\subsection{蒙特卡洛方法的限制}

\textbf{1. 不利用结构}:
\begin{itemize}
    \item MC方法直接使用观察到的回报
    \item 它不考虑状态转移结构
    \item 它不知道"从A总是转移到B"这一信息
\end{itemize}

\textbf{2. 泛化能力有限}:
\begin{itemize}
    \item MC方法需要观察完整的轨迹
    \item 如果某个状态只出现一次,它只能使用那一次的回报
    \item 它不能利用转移结构进行推理
\end{itemize}

\textbf{3. 只优化训练数据}:
\begin{itemize}
    \item MC方法只优化在训练数据上的误差
    \item 它不考虑对未来数据的预测
    \item 这是"有限的最优性"
\end{itemize}

\section{总结}

\subsection{核心区别}

\begin{center}
\begin{tabular}{|l|c|c|}
\hline
\textbf{特性} & \textbf{蒙特卡洛} & \textbf{TD(0)} \\
\hline
\textbf{最优性} & 训练数据上MSE最小 & 最大似然模型的确定性等价估计 \\
\hline
\textbf{利用结构} & 否 & 是(马尔可夫性质) \\
\hline
\textbf{泛化能力} & 有限 & 更强 \\
\hline
\textbf{预测未来} & 较差 & 更好 \\
\hline
\textbf{为什么"有限"?} & 只优化训练数据 & 利用转移结构 \\
\hline
\textbf{为什么"更相关"?} & 不相关 & 更符合预测回报的目标 \\
\hline
\end{tabular}
\end{center}

\subsection{关键洞察}

\begin{quote}
\textbf{蒙特卡洛方法只在有限意义下最优}:它只优化在训练数据上的误差,不利用马尔可夫性质,对未来数据的泛化能力有限。

\textbf{TD方法在预测回报方面更相关地最优}:它利用马尔可夫性质,基于状态转移结构进行估计,对未来数据的预测能力更强,更符合预测回报的目标。
\end{quote}

\subsection{实际意义}

\begin{enumerate}
    \item \textbf{为什么TD收敛更快?}
    \begin{itemize}
        \item TD方法利用状态转移结构
        \item 它能够从部分信息推断完整信息
        \item 它更有效地利用数据
    \end{itemize}
    
    \item \textbf{什么时候TD更好?}
    \begin{itemize}
        \item 当过程是马尔可夫的
        \item 当数据有限时
        \item 当需要预测未来回报时
    \end{itemize}
    
    \item \textbf{什么时候MC更好?}
    \begin{itemize}
        \item 当过程不是马尔可夫的
        \item 当只需要拟合训练数据时
        \item 当有大量完整轨迹数据时
    \end{itemize}
\end{enumerate}

\section{确定性等价估计(Certainty-Equivalence Estimate)}

\subsection{定义}

\textbf{确定性等价估计}:假设从数据估计的模型是准确的,然后计算在该模型下完全正确的价值函数估计。

\textbf{步骤}:
\begin{enumerate}
    \item 从观察数据构建最大似然模型
    \item 假设这个模型是准确的(确定性)
    \item 计算在该模型下的价值函数
\end{enumerate}

\subsection{为什么叫"确定性等价"?}

\textbf{等价性}:
\begin{itemize}
    \item 这个估计等价于假设模型是已知的(确定性)
    \item 而不是近似的(不确定性)
    \item 因此称为"确定性等价"
\end{itemize}

\subsection{TD(0)与确定性等价估计}

\textbf{定理}:批量TD(0)收敛到确定性等价估计。

\textbf{意义}:
\begin{itemize}
    \item TD(0)找到的是对最大似然模型完全正确的估计
    \item 它利用状态转移结构
    \item 它更符合预测回报的目标
\end{itemize}

\section{计算复杂度}

\subsection{直接计算确定性等价估计}

\textbf{复杂度}:
\begin{itemize}
    \item 构建最大似然模型:$O(n^2)$ 内存($n$ 是状态数)
    \item 计算价值函数:$O(n^3)$ 计算步骤
    \item 对于大状态空间,这不可行
\end{itemize}

\subsection{TD方法的优势}

\textbf{复杂度}:
\begin{itemize}
    \item TD方法只需要 $O(n)$ 内存
    \item 通过重复处理训练数据来近似确定性等价估计
    \item 对于大状态空间,TD可能是唯一可行的方法
\end{itemize}

\textbf{关键洞察}:
\begin{quote}
TD方法能够以 $O(n)$ 的复杂度近似 $O(n^3)$ 的确定性等价估计,这是TD方法的一个重要优势。
\end{quote}

\end{document}

