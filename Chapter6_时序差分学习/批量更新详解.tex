\documentclass[12pt,a4paper]{article}
\usepackage[UTF8]{ctex}
\usepackage{amsmath}
\usepackage{amssymb}
\usepackage{amsthm}
\usepackage{geometry}
\usepackage{hyperref}
\usepackage{enumitem}
\usepackage{xcolor}
\usepackage{bm}
\usepackage{booktabs}
\usepackage{array}
\usepackage{algorithm}
\usepackage{algorithmic}

\geometry{margin=2.5cm}

\title{批量更新详解}
\subtitle{什么是批量更新?与普通更新有什么区别?}
\author{}
\date{}

\newtheorem{definition}{定义}
\newtheorem{theorem}{定理}
\newtheorem{proposition}{命题}
\newtheorem{example}{示例}
\newtheorem{remark}{注记}

\begin{document}

\maketitle

\tableofcontents
\newpage

\section{什么是批量更新?}

\subsection{基本定义}

\begin{definition}[批量更新(Batch Updating)]
\textbf{批量更新}是一种学习方式,其中:
\begin{enumerate}
    \item 给定有限的经验数据(如10个回合或100个时间步)
    \item 对每个时间步计算更新增量,但\textbf{不立即更新}价值函数
    \item 处理完所有数据后,将\textbf{所有增量的和}一次性应用到价值函数
    \item 用新的价值函数\textbf{重复处理}所有经验数据
    \item 重复直到价值函数收敛
\end{enumerate}
\end{definition}

\textbf{关键特征}:
\begin{itemize}
    \item 更新只在处理完\textbf{完整批次}的训练数据后进行
    \item 价值函数在处理批次期间\textbf{保持不变}
    \item 所有数据被\textbf{重复使用}直到收敛
\end{itemize}

\subsection{为什么叫"批量"?}

\textbf{"批量"的含义}:
\begin{itemize}
    \item 将经验数据组织成\textbf{批次}(batch)
    \item 每次处理一个完整的批次
    \item 更新只在批次处理完成后进行
    \item 类似于机器学习中的"批量梯度下降"
\end{itemize}

\section{批量更新 vs 普通更新}

\subsection{普通更新(Normal Updating / Online Updating)}

\textbf{过程}:
\begin{enumerate}
    \item 观察一个经验样本 $(S_t, A_t, R_{t+1}, S_{t+1})$
    \item \textbf{立即计算}更新增量
    \item \textbf{立即应用}更新到价值函数
    \item 继续下一个样本
\end{enumerate}

\textbf{特点}:
\begin{itemize}
    \item 每个样本处理后就更新
    \item 价值函数在每个时间步都改变
    \item 也称为\textbf{在线更新}(Online Updating)或\textbf{增量更新}(Incremental Updating)
\end{itemize}

\textbf{TD(0)的普通更新}:
\begin{equation}
V(S_t) \gets V(S_t) + \alpha [R_{t+1} + \gamma V(S_{t+1}) - V(S_t)]
\end{equation}

\textbf{蒙特卡洛的普通更新}:
\begin{equation}
V(S_t) \gets V(S_t) + \alpha [G_t - V(S_t)]
\end{equation}

\subsection{批量更新(Batch Updating)}

\textbf{过程}:
\begin{enumerate}
    \item 收集一批经验数据 $\mathcal{D} = \{(S_t, A_t, R_{t+1}, S_{t+1})\}$
    \item 对每个样本计算更新增量,但\textbf{不立即更新}
    \item 计算所有增量的\textbf{总和}
    \item \textbf{一次性应用}总和到价值函数
    \item 用新的价值函数\textbf{重复处理}同一批数据
    \item 重复直到收敛
\end{enumerate}

\textbf{特点}:
\begin{itemize}
    \item 更新只在批次处理完成后进行
    \item 价值函数在处理批次期间保持不变
    \item 所有数据被重复使用
\end{itemize}

\subsection{对比表格}

\begin{center}
\begin{tabular}{|l|c|c|}
\hline
\textbf{特性} & \textbf{普通更新} & \textbf{批量更新} \\
\hline
\textbf{更新时机} & 每个样本后立即更新 & 批次处理完成后更新 \\
\hline
\textbf{价值函数变化} & 每个时间步都改变 & 批次处理期间不变 \\
\hline
\textbf{数据使用} & 每个样本只用一次 & 数据重复使用直到收敛 \\
\hline
\textbf{收敛性} & 可能不收敛到确定值 & 确定性地收敛到唯一值 \\
\hline
\textbf{步长依赖} & 依赖步长参数 & 只要步长足够小,结果独立于步长 \\
\hline
\textbf{计算方式} & 增量式 & 批量式 \\
\hline
\end{tabular}
\end{center}

\section{批量更新的详细过程}

\subsection{算法伪代码}

\begin{algorithm}[H]
\caption{批量更新算法(以TD(0)为例)}
\begin{algorithmic}[1]
\REQUIRE 经验数据批次 $\mathcal{D} = \{(S_t, A_t, R_{t+1}, S_{t+1})\}$
\REQUIRE 初始价值函数 $V_0$
\REQUIRE 步长 $\alpha$(足够小)
\ENSURE 收敛的价值函数 $V$
\STATE $V \gets V_0$
\REPEAT
    \STATE $\Delta V(s) \gets 0$ 对所有 $s \in \mathcal{S}$(初始化增量)
    \FOR{每个经验样本 $(S_t, A_t, R_{t+1}, S_{t+1}) \in \mathcal{D}$}
        \STATE 计算增量:$\delta_t = R_{t+1} + \gamma V(S_{t+1}) - V(S_t)$
        \STATE 累加增量:$\Delta V(S_t) \gets \Delta V(S_t) + \delta_t$
    \ENDFOR
    \STATE 应用所有增量:$V(s) \gets V(s) + \alpha \cdot \Delta V(s)$ 对所有 $s$
\UNTIL{收敛($\Delta V(s)$ 接近0对所有 $s$)}
\RETURN $V$
\end{algorithmic}
\end{algorithm}

\subsection{逐步说明}

\textbf{步骤1:初始化}
\begin{itemize}
    \item 初始化价值函数 $V(s)$
    \item 初始化增量累加器 $\Delta V(s) = 0$ 对所有状态
\end{itemize}

\textbf{步骤2:处理批次}
\begin{itemize}
    \item 对批次中的每个经验样本:
    \begin{itemize}
        \item 使用\textbf{当前}价值函数 $V$ 计算更新增量
        \item 将增量累加到 $\Delta V(S_t)$
        \item \textbf{不更新} $V$(在处理批次期间保持不变)
    \end{itemize}
\end{itemize}

\textbf{步骤3:应用更新}
\begin{itemize}
    \item 处理完所有样本后,一次性应用所有增量:
    \begin{equation}
    V(s) \gets V(s) + \alpha \cdot \Delta V(s)
    \end{equation}
\end{itemize}

\textbf{步骤4:重复}
\begin{itemize}
    \item 用新的价值函数重新处理同一批数据
    \item 重复直到收敛
\end{itemize}

\section{具体例子}

\subsection{例子:简单的3状态MDP}

\textbf{设置}:
\begin{itemize}
    \item 状态:$s_1, s_2, s_3$($s_3$ 是终止状态)
    \item 折扣因子:$\gamma = 1$
    \item 步长:$\alpha = 0.1$
    \item 初始价值:$V(s_1) = 0, V(s_2) = 0$
\end{itemize}

\textbf{经验数据批次}(3个样本):
\begin{align}
&\text{样本1: } s_1 \to s_2, R = 1 \\
&\text{样本2: } s_2 \to s_3, R = 2 \\
&\text{样本3: } s_2 \to s_3, R = 3
\end{align}

\subsection{批量更新过程}

\textbf{第1轮迭代}:

\textbf{初始化}:
\begin{align}
V(s_1) &= 0 \\
V(s_2) &= 0 \\
\Delta V(s_1) &= 0 \\
\Delta V(s_2) &= 0
\end{align}

\textbf{处理样本1}:$s_1 \to s_2, R = 1$
\begin{align}
\delta_1 &= R + \gamma V(s_2) - V(s_1) = 1 + 1 \times 0 - 0 = 1 \\
\Delta V(s_1) &\gets \Delta V(s_1) + \delta_1 = 0 + 1 = 1
\end{align}

\textbf{处理样本2}:$s_2 \to s_3, R = 2$
\begin{align}
\delta_2 &= R + \gamma V(s_3) - V(s_2) = 2 + 1 \times 0 - 0 = 2 \\
\Delta V(s_2) &\gets \Delta V(s_2) + \delta_2 = 0 + 2 = 2
\end{align}

\textbf{处理样本3}:$s_2 \to s_3, R = 3$
\begin{align}
\delta_3 &= R + \gamma V(s_3) - V(s_2) = 3 + 1 \times 0 - 0 = 3 \\
\Delta V(s_2) &\gets \Delta V(s_2) + \delta_3 = 2 + 3 = 5
\end{align}

\textbf{应用更新}:
\begin{align}
V(s_1) &\gets V(s_1) + \alpha \cdot \Delta V(s_1) = 0 + 0.1 \times 1 = 0.1 \\
V(s_2) &\gets V(s_2) + \alpha \cdot \Delta V(s_2) = 0 + 0.1 \times 5 = 0.5
\end{align}

\textbf{第2轮迭代}:

\textbf{重置增量}:
\begin{align}
\Delta V(s_1) &= 0 \\
\Delta V(s_2) &= 0
\end{align}

\textbf{处理样本1}:$s_1 \to s_2, R = 1$(使用新的 $V$)
\begin{align}
\delta_1 &= R + \gamma V(s_2) - V(s_1) = 1 + 1 \times 0.5 - 0.1 = 1.4 \\
\Delta V(s_1) &\gets 0 + 1.4 = 1.4
\end{align}

\textbf{处理样本2}:$s_2 \to s_3, R = 2$
\begin{align}
\delta_2 &= R + \gamma V(s_3) - V(s_2) = 2 + 1 \times 0 - 0.5 = 1.5 \\
\Delta V(s_2) &\gets 0 + 1.5 = 1.5
\end{align}

\textbf{处理样本3}:$s_2 \to s_3, R = 3$
\begin{align}
\delta_3 &= R + \gamma V(s_3) - V(s_2) = 3 + 1 \times 0 - 0.5 = 2.5 \\
\Delta V(s_2) &\gets 1.5 + 2.5 = 4.0
\end{align}

\textbf{应用更新}:
\begin{align}
V(s_1) &\gets 0.1 + 0.1 \times 1.4 = 0.24 \\
V(s_2) &\gets 0.5 + 0.1 \times 4.0 = 0.9
\end{align}

\textbf{继续迭代}直到收敛...

\subsection{普通更新过程(对比)}

\textbf{处理样本1}:$s_1 \to s_2, R = 1$
\begin{align}
V(s_1) &\gets V(s_1) + \alpha [R + \gamma V(s_2) - V(s_1)] \\
       &= 0 + 0.1 \times [1 + 1 \times 0 - 0] = 0.1
\end{align}

\textbf{处理样本2}:$s_2 \to s_3, R = 2$(使用更新后的 $V$)
\begin{align}
V(s_2) &\gets V(s_2) + \alpha [R + \gamma V(s_3) - V(s_2)] \\
       &= 0 + 0.1 \times [2 + 1 \times 0 - 0] = 0.2
\end{align}

\textbf{处理样本3}:$s_2 \to s_3, R = 3$(使用更新后的 $V$)
\begin{align}
V(s_2) &\gets V(s_2) + \alpha [R + \gamma V(s_3) - V(s_2)] \\
       &= 0.2 + 0.1 \times [3 + 1 \times 0 - 0.2] = 0.48
\end{align}

\textbf{关键区别}:
\begin{itemize}
    \item 普通更新:每个样本处理后立即更新,后续样本使用更新后的 $V$
    \item 批量更新:所有样本使用相同的 $V$ 计算增量,然后一次性更新
\end{itemize}

\section{批量更新的特点}

\subsection{确定性收敛}

\textbf{定理}:在批量更新下,TD(0)和常数-$\alpha$ MC都确定性地收敛到唯一值。

\textbf{条件}:
\begin{itemize}
    \item 步长 $\alpha$ 足够小
    \item 数据被重复处理直到收敛
\end{itemize}

\textbf{意义}:
\begin{itemize}
    \item 不依赖步长的具体值(只要足够小)
    \item 收敛结果是确定的
    \item 可以用于理论分析
\end{itemize}

\subsection{与普通更新的关系}

\textbf{关键洞察}:
\begin{quote}
普通更新方法不会完全到达批量更新的答案,但在某种意义上朝着这个方向移动。
\end{quote}

\textbf{解释}:
\begin{itemize}
    \item 普通更新是增量式的,每一步都改变价值函数
    \item 批量更新是批量式的,处理完所有数据后才更新
    \item 普通更新朝着批量更新的方向移动,但可能不会完全到达
\end{itemize}

\subsection{为什么使用批量更新?}

\textbf{1. 理论分析}:
\begin{itemize}
    \item 批量更新有确定性的收敛结果
    \item 可以用于比较不同方法的最优性
    \item 可以用于理解方法的本质差异
\end{itemize}

\textbf{2. 理解方法差异}:
\begin{itemize}
    \item 批量更新下,TD(0)和MC收敛到不同的答案
    \item 这些答案揭示了方法的本质差异
    \item 有助于理解为什么TD方法在某些情况下更好
\end{itemize}

\textbf{3. 实际应用}:
\begin{itemize}
    \item 在某些场景下,批量更新可能更稳定
    \item 可以用于离线学习(offline learning)
    \item 可以用于经验回放(experience replay)
\end{itemize}

\section{批量更新下的收敛结果}

\subsection{TD(0)的收敛结果}

\textbf{定理}:批量TD(0)收敛到\textbf{确定性等价估计}(Certainty-Equivalence Estimate)。

\textbf{含义}:
\begin{itemize}
    \item 对最大似然马尔可夫过程模型完全正确的估计
    \item 利用状态转移结构
    \item 对未来数据泛化能力更强
\end{itemize}

\subsection{蒙特卡洛的收敛结果}

\textbf{定理}:批量蒙特卡洛方法收敛到在\textbf{训练数据上}最小化均方误差的估计。

\textbf{含义}:
\begin{itemize}
    \item 在训练数据上误差最小
    \item 不利用马尔可夫性质
    \item 对未来数据泛化能力有限
\end{itemize}

\subsection{为什么不同?}

\textbf{关键区别}:
\begin{itemize}
    \item TD(0)利用状态转移结构,基于模型进行估计
    \item MC直接使用观察到的回报,不利用结构
    \item 这导致了不同的收敛结果
\end{itemize}

\section{批量更新的实现细节}

\subsection{增量累加}

\textbf{方法1:累加所有增量}
\begin{itemize}
    \item 对每个状态 $s$,维护累加器 $\Delta V(s)$
    \item 对每个样本,计算增量并累加
    \item 处理完所有样本后,一次性应用
\end{itemize}

\textbf{方法2:平均增量}
\begin{itemize}
    \item 对每个状态 $s$,计算所有增量的平均值
    \item 应用平均增量
    \item 等价于方法1(如果步长相同)
\end{itemize}

\subsection{收敛判断}

\textbf{判断标准}:
\begin{itemize}
    \item 所有状态的增量都接近0
    \item 价值函数变化小于阈值
    \item 达到最大迭代次数
\end{itemize}

\textbf{实现}:
\begin{align}
\text{收敛} \Leftrightarrow \max_s |\Delta V(s)| < \epsilon
\end{align}

其中 $\epsilon$ 是收敛阈值。

\section{批量更新 vs 小批量更新}

\subsection{小批量更新(Mini-batch Updating)}

\textbf{定义}:将数据分成多个小批次,每个小批次处理完后更新。

\textbf{特点}:
\begin{itemize}
    \item 介于普通更新和批量更新之间
    \item 每个小批次处理完后更新
    \item 平衡了计算效率和收敛稳定性
\end{itemize}

\subsection{对比}

\begin{center}
\begin{tabular}{|l|c|c|c|}
\hline
\textbf{特性} & \textbf{普通更新} & \textbf{小批量更新} & \textbf{批量更新} \\
\hline
\textbf{批次大小} & 1 & 小(如32, 64) & 全部数据 \\
\hline
\textbf{更新频率} & 每个样本 & 每个小批次 & 整个数据集 \\
\hline
\textbf{计算效率} & 高 & 中等 & 低 \\
\hline
\textbf{收敛稳定性} & 低 & 中等 & 高 \\
\hline
\end{tabular}
\end{center}

\section{总结}

\subsection{关键要点}

\begin{enumerate}
    \item \textbf{定义}:批量更新是在处理完所有数据后才更新价值函数的方法
    
    \item \textbf{过程}:
    \begin{itemize}
        \item 收集一批经验数据
        \item 对每个样本计算增量(使用当前 $V$)
        \item 累加所有增量
        \item 一次性应用更新
        \item 重复直到收敛
    \end{itemize}
    
    \item \textbf{特点}:
    \begin{itemize}
        \item 确定性地收敛到唯一值
        \item 结果独立于步长(只要足够小)
        \item 价值函数在处理批次期间不变
    \end{itemize}
    
    \item \textbf{与普通更新的区别}:
    \begin{itemize}
        \item 普通更新:每个样本后立即更新
        \item 批量更新:批次处理完后更新
        \item 普通更新朝着批量更新的方向移动
    \end{itemize}
    
    \item \textbf{用途}:
    \begin{itemize}
        \item 理论分析和方法比较
        \item 理解方法本质差异
        \item 某些实际应用场景
    \end{itemize}
\end{enumerate}

\subsection{关键洞察}

\begin{quote}
\textbf{批量更新是一种理论工具,用于理解不同学习方法的本质差异。它通过重复处理数据直到收敛,揭示了TD方法和蒙特卡洛方法在最优性方面的根本区别。}
\end{quote}

\end{document}

