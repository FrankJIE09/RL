\documentclass[12pt,a4paper]{article}
\usepackage[UTF8]{ctex}
\usepackage{amsmath}
\usepackage{amssymb}
\usepackage{amsthm}
\usepackage{geometry}
\usepackage{hyperref}
\usepackage{enumitem}
\usepackage{xcolor}
\usepackage{bm}
\usepackage{booktabs}
\usepackage{array}

\geometry{margin=2.5cm}

\title{TD(0)详解}
\subtitle{什么是TD(0)?为什么叫"0"?}
\author{}
\date{}

\newtheorem{definition}{定义}
\newtheorem{theorem}{定理}
\newtheorem{proposition}{命题}
\newtheorem{example}{示例}
\newtheorem{remark}{注记}

\begin{document}

\maketitle

\tableofcontents
\newpage

\section{什么是TD(0)?}

\subsection{基本定义}

\begin{definition}[TD(0)]
\textbf{TD(0)}是最简单的时序差分学习方法,它使用\textbf{一步前瞻}来更新价值函数估计。更新公式为:
\begin{equation}
V(S_t) \gets V(S_t) + \alpha [R_{t+1} + \gamma V(S_{t+1}) - V(S_t)]
\label{eq:td0}
\end{equation}
其中:
\begin{itemize}
    \item $V(S_t)$ 是状态 $S_t$ 的当前价值估计
    \item $R_{t+1}$ 是即时奖励
    \item $V(S_{t+1})$ 是下一状态 $S_{t+1}$ 的当前价值估计
    \item $\alpha$ 是步长参数(学习率)
    \item $\gamma$ 是折扣因子
\end{itemize}
\end{definition}

\subsection{为什么叫"TD(0)"?}

\textbf{关键问题}:为什么叫"0"?这个"0"是什么意思?

\textbf{答案}:TD(0)中的"0"表示\textbf{向前看0步的真实奖励},即只使用一步转移,不使用未来的真实奖励。

\subsubsection{n步TD方法的一般形式}

为了理解"0"的含义,我们需要了解n步TD方法的一般形式:

\textbf{n步回报}:
\begin{equation}
G_{t:t+n} = R_{t+1} + \gamma R_{t+2} + \cdots + \gamma^{n-1} R_{t+n} + \gamma^n V(S_{t+n})
\label{eq:n_step_return}
\end{equation}

\textbf{n步TD更新}:
\begin{equation}
V(S_t) \gets V(S_t) + \alpha [G_{t:t+n} - V(S_t)]
\label{eq:n_step_td}
\end{equation}

\textbf{解释}:
\begin{itemize}
    \item $n$ 表示使用多少步的\textbf{真实奖励}
    \item $G_{t:t+n}$ 使用前 $n$ 步的真实奖励,然后用 $V(S_{t+n})$ 估计剩余部分
    \item 当 $n=0$ 时,就是TD(0)
\end{itemize}

\subsubsection{TD(0)的特殊情况}

\textbf{当 $n=0$ 时}:
\begin{align}
G_{t:t+0} &= \gamma^0 V(S_{t+0}) = V(S_t) \quad \text{(这不对)}
\end{align}

实际上,TD(0)对应的是 $n=1$ 的情况,但传统上称为TD(0)是因为它使用\textbf{0步的真实奖励}(只使用一步转移,不等待真实奖励)。

\textbf{更准确的理解}:
\begin{itemize}
    \item \textbf{TD(0)}:使用 $0$ 步的真实奖励,即立即使用 $R_{t+1} + \gamma V(S_{t+1})$
    \item \textbf{TD(1)}:使用 $1$ 步的真实奖励,即 $R_{t+1} + \gamma R_{t+2} + \gamma^2 V(S_{t+2})$
    \item \textbf{TD(n)}:使用 $n$ 步的真实奖励
    \item \textbf{蒙特卡洛}:使用所有步的真实奖励($n = \infty$)
\end{itemize}

\textbf{TD(0)的更新目标}:
\begin{equation}
\text{目标} = R_{t+1} + \gamma V(S_{t+1})
\end{equation}

\textbf{解释}:
\begin{itemize}
    \item 使用 $0$ 步的真实奖励(只使用 $R_{t+1}$,这是即时奖励,不是"未来"的奖励)
    \item 立即使用价值函数估计 $V(S_{t+1})$ 来估计剩余部分
    \item 因此称为TD(0)
\end{itemize}

\section{TD(0)的完整算法}

\subsection{算法伪代码}

\begin{algorithm}[H]
\caption{TD(0)算法(估计 $v_\pi$)}
\begin{algorithmic}[1]
\REQUIRE 要评估的策略 $\pi$
\ENSURE 状态价值函数 $v_\pi$ 的估计 $V$
\STATE \textbf{初始化}:$V(s) \in \mathbb{R}$ 任意初始化,对所有 $s \in \mathcal{S}$,但 $V(\text{终止}) = 0$
\STATE \textbf{算法参数}:步长 $\alpha \in (0, 1]$
\STATE
\REPEAT
    \STATE 初始化 $S$(回合的起始状态)
    \REPEAT
        \STATE $A \gets$ 策略 $\pi$ 给出的动作
        \STATE 执行动作 $A$,观察奖励 $R$ 和下一状态 $S'$
        \STATE $V(S) \gets V(S) + \alpha [R + \gamma V(S') - V(S)]$
        \STATE $S \gets S'$
    \UNTIL{$S$ 是终止状态}
\UNTIL{收敛}
\RETURN $V$
\end{algorithmic}
\end{algorithm}

\subsection{关键步骤}

\textbf{步骤1}:初始化价值函数 $V(s)$

\textbf{步骤2}:对每个回合:
\begin{enumerate}
    \item 从起始状态开始
    \item 对每个时间步:
    \begin{itemize}
        \item 根据策略选择动作
        \item 执行动作,观察奖励和下一状态
        \item 使用TD(0)更新当前状态的价值
        \item 转移到下一状态
    \end{itemize}
\end{enumerate}

\textbf{步骤3}:重复直到收敛

\section{TD(0) vs 其他方法}

\subsection{与蒙特卡洛方法对比}

\textbf{蒙特卡洛更新}:
\begin{equation}
V(S_t) \gets V(S_t) + \alpha [G_t - V(S_t)]
\end{equation}

其中 $G_t = R_{t+1} + \gamma R_{t+2} + \gamma^2 R_{t+3} + \cdots$ 是完整回报。

\textbf{对比}:

\begin{center}
\begin{tabular}{|l|c|c|}
\hline
\textbf{特性} & \textbf{蒙特卡洛} & \textbf{TD(0)} \\
\hline
更新目标 & $G_t$(完整回报) & $R_{t+1} + \gamma V(S_{t+1})$ \\
\hline
真实奖励步数 & 所有步($\infty$) & 0步(只使用即时奖励) \\
\hline
更新时机 & 回合结束后 & 每个时间步 \\
\hline
需要等待 & 是 & 否 \\
\hline
偏差 & 无偏 & 可能有偏 \\
\hline
方差 & 高方差 & 低方差 \\
\hline
\end{tabular}
\end{center}

\subsection{与动态规划对比}

\textbf{动态规划更新}:
\begin{equation}
V(s) \gets \sum_{a} \pi(a | s) \sum_{s', r} p(s', r | s, a) [r + \gamma V(s')]
\end{equation}

\textbf{对比}:

\begin{center}
\begin{tabular}{|l|c|c|}
\hline
\textbf{特性} & \textbf{动态规划} & \textbf{TD(0)} \\
\hline
需要环境模型 & 是 & 否 \\
\hline
更新方式 & 期望更新 & 采样更新 \\
\hline
更新目标 & $r + \gamma V(s')$(期望) & $R_{t+1} + \gamma V(S_{t+1})$(样本) \\
\hline
\end{tabular}
\end{center}

\subsection{与n步TD方法对比}

\textbf{TD(0)}:
\begin{equation}
\text{目标} = R_{t+1} + \gamma V(S_{t+1})
\end{equation}

\textbf{TD(1)}:
\begin{equation}
\text{目标} = R_{t+1} + \gamma R_{t+2} + \gamma^2 V(S_{t+2})
\end{equation}

\textbf{TD(2)}:
\begin{equation}
\text{目标} = R_{t+1} + \gamma R_{t+2} + \gamma^2 R_{t+3} + \gamma^3 V(S_{t+3})
\end{equation}

\textbf{蒙特卡洛}($n = \infty$):
\begin{equation}
\text{目标} = R_{t+1} + \gamma R_{t+2} + \gamma^2 R_{t+3} + \cdots
\end{equation}

\textbf{对比}:

\begin{center}
\begin{tabular}{|l|c|c|}
\hline
\textbf{方法} & \textbf{真实奖励步数} & \textbf{估计步数} \\
\hline
TD(0) & 0步(只即时奖励) & 1步 \\
\hline
TD(1) & 1步 & 1步 \\
\hline
TD(2) & 2步 & 1步 \\
\hline
蒙特卡洛 & 所有步 & 0步 \\
\hline
\end{tabular}
\end{center}

\section{TD(0)的特点}

\subsection{优点}

\begin{enumerate}
    \item \textbf{在线学习}:
    \begin{itemize}
        \item 每个时间步都可以更新
        \item 不需要等待回合结束
        \item 可以立即利用新信息
    \end{itemize}
    
    \item \textbf{无模型}:
    \begin{itemize}
        \item 不需要环境模型 $p(s', r | s, a)$
        \item 只需要经验样本
        \item 可以从实际交互中学习
    \end{itemize}
    
    \item \textbf{低方差}:
    \begin{itemize}
        \item 只依赖一步转移
        \item 比蒙特卡洛方法的方差小
        \item 通常收敛更快
    \end{itemize}
    
    \item \textbf{自举法}:
    \begin{itemize}
        \item 使用估计值更新估计值
        \item 信息传播快
        \item 数据效率高
    \end{itemize}
\end{enumerate}

\subsection{缺点}

\begin{enumerate}
    \item \textbf{可能有偏}:
    \begin{itemize}
        \item 如果 $V(S_{t+1})$ 不准确,TD目标也不准确
        \item 需要多次迭代才能收敛
    \end{itemize}
    
    \item \textbf{需要初始估计}:
    \begin{itemize}
        \item 需要初始化 $V(s)$
        \item 初始值可能影响收敛速度
    \end{itemize}
    
    \item \textbf{只使用一步信息}:
    \begin{itemize}
        \item 可能忽略长期依赖
        \item 在某些任务中可能不如n步TD方法
    \end{itemize}
\end{enumerate}

\section{TD误差}

\subsection{TD误差的定义}

\textbf{TD误差}:
\begin{equation}
\delta_t = R_{t+1} + \gamma V(S_{t+1}) - V(S_t)
\label{eq:td_error}
\end{equation}

\textbf{解释}:
\begin{itemize}
    \item $\delta_t$ 衡量当前估计 $V(S_t)$ 与更好估计 $R_{t+1} + \gamma V(S_{t+1})$ 之间的差异
    \item 如果 $\delta_t > 0$,说明当前估计过低,应该增加
    \item 如果 $\delta_t < 0$,说明当前估计过高,应该减少
    \item 如果 $\delta_t = 0$,说明当前估计正好
\end{itemize}

\subsection{TD更新用TD误差表示}

\begin{equation}
V(S_t) \gets V(S_t) + \alpha \delta_t
\end{equation}

\textbf{解释}:
\begin{itemize}
    \item 更新量与TD误差成正比
    \item 步长 $\alpha$ 控制更新幅度
    \item TD误差越大,更新幅度越大
\end{itemize}

\section{具体例子}

\subsection{Gridworld例子}

\textbf{设置}:
\begin{itemize}
    \item 状态:$s_5$(中心状态)
    \item 执行动作"上",观察到:$R_1 = -1$,$S_1 = s_2$
    \item 当前估计:$V(s_5) = -2.0$,$V(s_2) = -1.0$
    \item 折扣因子:$\gamma = 0.9$
    \item 步长:$\alpha = 0.1$
\end{itemize}

\textbf{TD(0)更新}:

\textbf{步骤1}:计算TD目标
\begin{align}
\text{目标} &= R_1 + \gamma V(S_1) \\
           &= -1 + 0.9 \times (-1.0) = -1.9
\end{align}

\textbf{步骤2}:计算TD误差
\begin{align}
\delta_0 &= R_1 + \gamma V(S_1) - V(S_0) \\
         &= -1.9 - (-2.0) = 0.1
\end{align}

\textbf{步骤3}:更新价值函数
\begin{align}
V(s_5) &\gets V(s_5) + \alpha \delta_0 \\
       &= -2.0 + 0.1 \times 0.1 = -1.99
\end{align}

\textbf{解释}:
\begin{itemize}
    \item TD目标 $-1.9$ 比当前估计 $-2.0$ 更好(更接近真实值)
    \item TD误差 $0.1$ 表示当前估计略低
    \item 更新后,$V(s_5)$ 从 $-2.0$ 增加到 $-1.99$
\end{itemize}

\section{为什么TD(0)有效?}

\subsection{理论基础:贝尔曼方程}

\textbf{状态价值函数的贝尔曼方程}:
\begin{equation}
v_\pi(s) = \mathbb{E}_\pi[R_{t+1} + \gamma v_\pi(S_{t+1}) | S_t = s]
\end{equation}

\textbf{关键洞察}:
\begin{quote}
根据贝尔曼方程,$v_\pi(S_t)$ 的真实值等于 $R_{t+1} + \gamma v_\pi(S_{t+1})$ 的期望值。TD(0)使用观察到的样本 $R_{t+1} + \gamma V(S_{t+1})$ 来估计这个期望值。
\end{quote}

\subsection{收敛性}

\begin{theorem}[TD(0)收敛性]
对于任何固定策略 $\pi$,TD(0)在以下条件下收敛到 $v_\pi$:
\begin{enumerate}
    \item 步长 $\alpha$ 满足 Robbins-Monro 条件:
    \begin{align}
    \sum_{t=0}^{\infty} \alpha_t &= \infty \\
    \sum_{t=0}^{\infty} \alpha_t^2 &< \infty
    \end{align}
    \item 每个状态被访问无限次
    \item 价值函数在回合期间不更新(或更新很小)
\end{enumerate}
\end{theorem}

\textbf{解释}:
\begin{itemize}
    \item 如果步长满足条件,TD(0)保证收敛
    \item 收敛到真实价值函数 $v_\pi$
    \item 这是TD(0)有效性的理论保证
\end{itemize}

\section{总结}

\subsection{TD(0)的核心特征}

\begin{enumerate}
    \item \textbf{定义}:使用一步前瞻 $R_{t+1} + \gamma V(S_{t+1})$ 更新价值函数
    
    \item \textbf{"0"的含义}:使用0步的真实奖励(只使用即时奖励),立即使用价值函数估计
    
    \item \textbf{更新公式}:
    \begin{equation}
    V(S_t) \gets V(S_t) + \alpha [R_{t+1} + \gamma V(S_{t+1}) - V(S_t)]
    \end{equation}
    
    \item \textbf{TD误差}:
    \begin{equation}
    \delta_t = R_{t+1} + \gamma V(S_{t+1}) - V(S_t)
    \end{equation}
    
    \item \textbf{特点}:
    \begin{itemize}
        \item 在线学习:每个时间步都可以更新
        \item 无模型:不需要环境模型
        \item 低方差:只依赖一步转移
        \item 自举法:使用估计值更新估计值
    \end{itemize}
\end{enumerate}

\subsection{关键洞察}

\begin{quote}
\textbf{TD(0)是最简单的时序差分学习方法,它通过结合即时奖励和下一状态的估计值来更新当前状态的价值。虽然它只使用一步信息,但通过自举法和在线学习,它能够高效地学习价值函数。}
\end{quote}

\subsection{与其他方法的关系}

\begin{itemize}
    \item \textbf{TD(0)} $\to$ \textbf{TD(n)}:增加使用的真实奖励步数
    \item \textbf{TD(0)} $\to$ \textbf{蒙特卡洛}:使用所有步的真实奖励
    \item \textbf{TD(0)} $\to$ \textbf{动态规划}:使用期望值而不是样本
\end{itemize}

TD(0)是时序差分学习的基础,理解TD(0)是理解更复杂TD方法的关键。

\end{document}

