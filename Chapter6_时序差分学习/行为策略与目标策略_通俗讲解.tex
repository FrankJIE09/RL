\documentclass[12pt,a4paper]{article}
\usepackage[UTF8]{ctex}
\usepackage{amsmath}
\usepackage{amssymb}
\usepackage{amsthm}
\usepackage{geometry}
\usepackage{hyperref}
\usepackage{enumitem}
\usepackage{booktabs}

\geometry{left=2.5cm,right=2.5cm,top=2.5cm,bottom=2.5cm}

\title{行为策略与目标策略:通俗讲解}
\subtitle{用生活例子理解On-policy和Off-policy}
\author{强化学习笔记}
\date{\today}

\begin{document}

\maketitle

\tableofcontents
\newpage

\section{引言}

本文用通俗易懂的方式,通过生活化的例子来解释"行为策略 = 目标策略"和"使用实际选择的下一动作"的含义及其关系。

\section{生活类比:学开车}

\subsection{类比设置}

想象你在学开车:

\begin{itemize}
    \item \textbf{行为策略}:你\textbf{实际怎么开车}(你的驾驶方式)
    \item \textbf{目标策略}:你\textbf{想要学会的驾驶方式}(理想的驾驶方式)
    \item \textbf{动作}:方向盘、刹车、油门等操作
    \item \textbf{状态}:路况、车速、位置等
    \item \textbf{奖励}:安全到达、舒适度等
\end{itemize}

\subsection{On-policy:边学边用}

\textbf{场景}:你正在学习如何开车,同时用你学到的知识来开车。

\textbf{特点}:
\begin{itemize}
    \item \textbf{行为策略 = 目标策略}:你实际怎么开车 = 你想要学会的驾驶方式
    \item 你\textbf{实际做的动作}(如"向左转")就是你\textbf{正在学习的策略}选择的动作
    \item 你从自己的驾驶经验中学习
    \item 学习的是"你当前怎么开车"的价值
\end{itemize}

\textbf{例子}:
\begin{enumerate}
    \item 你学会了"在路口应该减速"
    \item 你\textbf{实际在路口减速}(行为策略)
    \item 你\textbf{学习的是"在路口减速"这个策略}(目标策略)
    \item 行为策略 = 目标策略(同一个策略)
    \item 你观察自己减速后的结果,学习这个策略的好坏
\end{enumerate}

\subsection{Off-policy:看别人开车来学习}

\textbf{场景}:你观察教练或其他人开车,学习他们的驾驶方式。

\textbf{特点}:
\begin{itemize}
    \item \textbf{行为策略 $\neq$ 目标策略}:别人怎么开车 $\neq$ 你想要学会的驾驶方式
    \item 你观察\textbf{别人的动作}(如"教练向右转"),但学习的是\textbf{最优的驾驶方式}
    \item 你可以从任何人的驾驶经验中学习
    \item 学习的是"最优驾驶方式"的价值,而不是"别人怎么开车"的价值
\end{itemize}

\textbf{例子}:
\begin{enumerate}
    \item 教练在路口选择了"向右转"(行为策略:教练的策略)
    \item 但你学习的是"最优动作"(目标策略:最优策略)
    \item 即使教练向右转,你学习的是"在路口应该选择最优方向"
    \item 行为策略 $\neq$ 目标策略(不同策略)
    \item 你从教练的经验中学习,但目标是学会最优驾驶方式
\end{enumerate}

\section{通俗解释:行为策略 = 目标策略}

\subsection{简单理解}

\textbf{"行为策略 = 目标策略"就像}:

\begin{quote}
\textbf{"你实际怎么做 = 你想要学会怎么做"}
\end{quote}

\textbf{具体例子}:

\textbf{例子1:学做菜}
\begin{itemize}
    \item \textbf{行为策略}:你实际怎么做菜(放多少盐、火候多大)
    \item \textbf{目标策略}:你想要学会的做菜方式
    \item \textbf{On-policy}:你实际怎么做 = 你想要学会怎么做
    \item 你从自己的做菜经验中学习,改进自己的做菜方式
\end{itemize}

\textbf{例子2:学下棋}
\begin{itemize}
    \item \textbf{行为策略}:你实际怎么下棋(你的下棋策略)
    \item \textbf{目标策略}:你想要学会的下棋方式
    \item \textbf{On-policy}:你实际怎么下 = 你想要学会怎么下
    \item 你从自己的下棋经验中学习,改进自己的下棋策略
\end{itemize}

\subsection{在算法中的体现}

\textbf{SARSA(On-policy)}:
\begin{itemize}
    \item 智能体使用\textbf{当前策略} $\pi$ 选择动作
    \item 智能体学习的是\textbf{当前策略} $\pi$ 的价值
    \item 行为策略 = 目标策略 = $\pi$(同一个策略)
\end{itemize}

\textbf{类比}:
\begin{itemize}
    \item 就像你用自己的方式做事,同时学习这种方式的好坏
    \item 你做的动作 = 你正在学习的策略选择的动作
    \item 你从自己的经验中学习
\end{itemize}

\section{通俗解释:使用实际选择的下一动作}

\subsection{简单理解}

\textbf{"使用实际选择的下一动作"就像}:

\begin{quote}
\textbf{"你实际做了什么,就用这个结果来学习"}
\end{quote}

\textbf{具体例子}:

\textbf{例子1:学做菜}
\begin{itemize}
    \item 你在做菜时,\textbf{实际选择了}"放一勺盐"
    \item 你观察放一勺盐后的结果(菜的味道)
    \item 你\textbf{用这个结果}来学习"放一勺盐"这个动作的好坏
    \item 这就是"使用实际选择的动作"
\end{itemize}

\textbf{例子2:学下棋}
\begin{itemize}
    \item 你在下棋时,\textbf{实际选择了}"走马"
    \item 你观察走马后的结果(棋局变化)
    \item 你\textbf{用这个结果}来学习"走马"这个动作的好坏
    \item 这就是"使用实际选择的动作"
\end{itemize}

\subsection{在算法中的体现}

\textbf{SARSA}:
\begin{itemize}
    \item 在状态 $S_{t+1}$,智能体\textbf{实际选择了}动作 $A_{t+1}$
    \item 更新时使用 $Q(S_{t+1}, A_{t+1})$(实际选择的动作的价值)
    \item 学习的是"选择动作 $A_{t+1}$"的价值
\end{itemize}

\textbf{类比}:
\begin{itemize}
    \item 就像你实际做了什么,就用这个结果来学习
    \item 你做的动作 = 你用来学习的动作
    \item 你从自己的实际行为中学习
\end{itemize}

\section{两者的关系}

\subsection{关系链(通俗版)}

\textbf{第1步:行为策略 = 目标策略}
\begin{quote}
"你实际怎么做 = 你想要学会怎么做"
\end{quote}

\textbf{第2步:所有动作都由当前策略选择}
\begin{quote}
"因为你实际怎么做 = 你想要学会怎么做,所以你的所有动作都按照你想要学会的方式来做"
\end{quote}

\textbf{第3步:下一动作必须由当前策略选择}
\begin{quote}
"你的下一个动作也必须按照你想要学会的方式来做"
\end{quote}

\textbf{第4步:使用实际选择的动作}
\begin{quote}
"你实际做了什么,就用这个结果来学习"
\end{quote}

\textbf{第5步:学习当前策略的价值}
\begin{quote}
"你学习的是'你想要学会的方式'的好坏"
\end{quote}

\subsection{完整例子:学做菜}

\textbf{场景}:你正在学习如何做一道菜。

\textbf{第1步}:行为策略 = 目标策略
\begin{itemize}
    \item 你实际怎么做菜 = 你想要学会的做菜方式
    \item 你边学边做,做的就是你正在学的
\end{itemize}

\textbf{第2步}:选择动作
\begin{itemize}
    \item 你\textbf{实际选择了}"放一勺盐"(按照你正在学的策略)
    \item 这个动作\textbf{必须}由你正在学的策略选择
\end{itemize}

\textbf{第3步}:观察结果
\begin{itemize}
    \item 你观察放一勺盐后的结果(菜的味道)
    \item 你\textbf{用这个结果}来学习
\end{itemize}

\textbf{第4步}:学习
\begin{itemize}
    \item 你学习的是"你正在学的做菜方式"中"放一勺盐"这个动作的好坏
    \item 你从自己的实际行为中学习
\end{itemize}

\section{对比:On-policy vs Off-policy}

\subsection{On-policy:边学边用(SARSA)}

\textbf{类比}:你用自己的方式做事,同时学习这种方式的好坏。

\textbf{特点}:
\begin{itemize}
    \item 你实际怎么做 = 你想要学会怎么做
    \item 你做的动作 = 你正在学习的策略选择的动作
    \item 你从自己的经验中学习
    \item 你实际选择了什么,就用什么来学习
\end{itemize}

\textbf{例子}:
\begin{enumerate}
    \item 你学会了"在路口应该减速"
    \item 你\textbf{实际在路口减速}(行为策略)
    \item 你\textbf{学习的是"在路口减速"这个策略}(目标策略)
    \item 你观察减速后的结果,学习这个策略的好坏
    \item 你实际选择了"减速",就用"减速"的结果来学习
\end{enumerate}

\subsection{Off-policy:看别人学(Q-learning)}

\textbf{类比}:你观察别人怎么做,但学习的是最优方式。

\textbf{特点}:
\begin{itemize}
    \item 别人怎么做 $\neq$ 你想要学会的方式
    \item 你观察别人的动作,但学习的是最优动作
    \item 你可以从任何人的经验中学习
    \item 即使别人选择了某个动作,你学习的是最优动作
\end{itemize}

\textbf{例子}:
\begin{enumerate}
    \item 教练在路口选择了"向右转"(行为策略:教练的策略)
    \item 但你学习的是"在路口应该选择最优方向"(目标策略:最优策略)
    \item 即使教练向右转,你学习的是"最优方向"(可能是左转)
    \item 你从教练的经验中学习,但目标是学会最优方式
    \item 教练实际选择了"右转",但你学习的是"最优方向"的价值
\end{enumerate}

\section{为什么必须由当前策略选择?}

\subsection{通俗解释}

\textbf{问题}:为什么在On-policy方法中,$A_{t+1}$ 必须由当前策略选择?

\textbf{答案}:

\textbf{类比1:学做菜}
\begin{itemize}
    \item 如果你想要学习"你的做菜方式",你必须\textbf{按照你的方式}来做菜
    \item 如果你按照别人的方式做菜,你学习的就是"别人的做菜方式",而不是"你的做菜方式"
    \item 所以,你的所有动作(包括下一步)都必须按照"你的方式"来做
\end{itemize}

\textbf{类比2:学下棋}
\begin{itemize}
    \item 如果你想要学习"你的下棋策略",你必须\textbf{按照你的策略}来下棋
    \item 如果你按照别人的策略下棋,你学习的就是"别人的下棋策略",而不是"你的下棋策略"
    \item 所以,你的所有动作(包括下一步)都必须按照"你的策略"来做
\end{itemize}

\subsection{数学原因}

\textbf{原因1:学习目标的一致性}
\begin{itemize}
    \item 你想要学习策略 $\pi$ 的价值函数
    \item 要学习策略 $\pi$ 的价值,必须使用策略 $\pi$ 生成的数据
    \item 如果使用其他策略的数据,学习的就是其他策略的价值
\end{itemize}

\textbf{原因2:数据分布匹配}
\begin{itemize}
    \item 你的学习目标是策略 $\pi$ 的价值函数
    \item 你的数据必须来自策略 $\pi$
    \item 所以,所有动作(包括 $A_{t+1}$)都必须由策略 $\pi$ 选择
\end{itemize}

\section{具体例子:SARSA vs Q-learning}

\subsection{场景设置}

\textbf{场景}:在Gridworld中,智能体在状态 $s_5$,执行动作"上",转移到状态 $s_2$。

\textbf{假设}:
\begin{itemize}
    \item $Q(s_2, \text{上}) = 0$
    \item $Q(s_2, \text{下}) = 0$
    \item $Q(s_2, \text{左}) = -0.1$
    \item $Q(s_2, \text{右}) = 0$
    \item 当前策略 $\pi$ 在 $s_2$ 倾向于选择"左"
\end{itemize}

\subsection{SARSA(On-policy):边学边用}

\textbf{行为策略 = 目标策略}:
\begin{quote}
"你实际怎么选择动作 = 你想要学会的选择方式"
\end{quote}

\textbf{过程}:
\begin{enumerate}
    \item 在状态 $s_2$,你\textbf{按照当前策略} $\pi$ 选择动作
    \item 你\textbf{实际选择了}"左"(按照你正在学的策略)
    \item 你\textbf{用"左"的结果}来学习
    \item 更新:$Q(s_5, \text{上}) \gets Q(s_5, \text{上}) + \alpha [R + \gamma Q(s_2, \text{左}) - Q(s_5, \text{上})]$
    \item 你学习的是"你当前策略"中"选择左"的价值
\end{enumerate}

\textbf{关键}:
\begin{itemize}
    \item 你实际选择了"左",就用"左"的价值来学习
    \item 你学习的是"你当前策略"的价值
    \item 行为策略 = 目标策略 = 你当前策略
\end{itemize}

\subsection{Q-learning(Off-policy):看别人学}

\textbf{行为策略 $\neq$ 目标策略}:
\begin{quote}
"你实际怎么选择动作 $\neq$ 你想要学会的选择方式"
\end{quote}

\textbf{过程}:
\begin{enumerate}
    \item 在状态 $s_2$,你\textbf{按照行为策略} $b$ 选择动作(用于探索)
    \item 你\textbf{实际选择了}"右"(行为策略选择,用于探索)
    \item 但你\textbf{学习的是"最优动作"的价值},而不是"右"的价值
    \item 更新:$Q(s_5, \text{上}) \gets Q(s_5, \text{上}) + \alpha [R + \gamma \max_{a} Q(s_2, a) - Q(s_5, \text{上})]$
    \item 你学习的是"最优策略"的价值,即使你实际选择了"右"
\end{enumerate}

\textbf{关键}:
\begin{itemize}
    \item 你实际选择了"右",但学习的是"最优动作"(可能是"上"或"下")的价值
    \item 你学习的是"最优策略"的价值,而不是"你实际选择的动作"的价值
    \item 行为策略 $\neq$ 目标策略
\end{itemize}

\section{Off-policy的更多通俗例子}

\subsection{例子1:看视频学做菜}

\textbf{场景}:你在看美食博主的做菜视频,学习如何做菜。

\textbf{行为策略}:美食博主怎么做菜(博主的做菜方式)
\begin{itemize}
    \item 博主在视频中选择了"放两勺盐"(行为策略:博主的策略)
    \item 博主实际做了这个动作
\end{itemize}

\textbf{目标策略}:你想要学会的做菜方式(最优的做菜方式)
\begin{itemize}
    \item 你想要学会"放多少盐最合适"(目标策略:最优策略)
    \item 可能最优是"放一勺半盐",而不是"两勺"
\end{itemize}

\textbf{Off-policy特点}:
\begin{itemize}
    \item 行为策略 $\neq$ 目标策略:博主怎么做 $\neq$ 你想要学会的方式
    \item 你观察博主的动作(放两勺盐),但学习的是最优动作(放一勺半盐)
    \item 即使博主放了两勺盐,你学习的是"最优盐量"的价值
    \item 你可以从多个博主的视频中学习,但目标是学会最优方式
\end{itemize}

\textbf{类比到Q-learning}:
\begin{itemize}
    \item 博主选择了"放两勺盐"(行为策略选择)
    \item 但你学习的是"最优盐量"(可能是"一勺半")的价值
    \item 你使用 $\max$ 操作:$\max\{\text{一勺的价值}, \text{一勺半的价值}, \text{两勺的价值}\}$
    \item 即使博主放了两勺,你学习的是最优盐量的价值
\end{itemize}

\subsection{例子2:观察别人下棋来学习}

\textbf{场景}:你观察别人下棋,学习最优下棋策略。

\textbf{行为策略}:别人怎么下棋(观察对象的下棋方式)
\begin{itemize}
    \item 你观察的棋手在某个位置选择了"走马"(行为策略:棋手的策略)
    \item 棋手实际走了这个动作
\end{itemize}

\textbf{目标策略}:你想要学会的最优下棋方式
\begin{itemize}
    \item 你想要学会"在这个位置应该走什么"(目标策略:最优策略)
    \item 可能最优是"走车",而不是"走马"
\end{itemize}

\textbf{Off-policy特点}:
\begin{itemize}
    \item 行为策略 $\neq$ 目标策略:棋手怎么下 $\neq$ 你想要学会的方式
    \item 你观察棋手的动作(走马),但学习的是最优动作(走车)的价值
    \item 即使棋手走了马,你学习的是"最优走法"的价值
    \item 你可以观察多个棋手下棋,但目标是学会最优策略
\end{itemize}

\textbf{类比到Q-learning}:
\begin{itemize}
    \item 棋手选择了"走马"(行为策略选择)
    \item 但你学习的是"最优走法"(可能是"走车")的价值
    \item 你使用 $\max$ 操作:$\max\{\text{走马的价值}, \text{走车的价值}, \text{走兵的价值}\}$
    \item 即使棋手走了马,你学习的是最优走法的价值
\end{itemize}

\subsection{例子3:看别人开车来学习}

\textbf{场景}:你坐在副驾驶,观察司机开车,学习最优驾驶方式。

\textbf{行为策略}:司机怎么开车(司机的驾驶方式)
\begin{itemize}
    \item 司机在路口选择了"向右转"(行为策略:司机的策略)
    \item 司机实际做了这个动作
\end{itemize}

\textbf{目标策略}:你想要学会的最优驾驶方式
\begin{itemize}
    \item 你想要学会"在路口应该选择什么方向"(目标策略:最优策略)
    \item 可能最优是"直行",而不是"右转"
\end{itemize}

\textbf{Off-policy特点}:
\begin{itemize}
    \item 行为策略 $\neq$ 目标策略:司机怎么开 $\neq$ 你想要学会的方式
    \item 你观察司机的动作(右转),但学习的是最优动作(直行)的价值
    \item 即使司机右转了,你学习的是"最优方向"的价值
    \item 你可以观察多个司机开车,但目标是学会最优驾驶方式
\end{itemize}

\textbf{类比到Q-learning}:
\begin{itemize}
    \item 司机选择了"右转"(行为策略选择)
    \item 但你学习的是"最优方向"(可能是"直行")的价值
    \item 你使用 $\max$ 操作:$\max\{\text{左转的价值}, \text{直行的价值}, \text{右转的价值}\}$
    \item 即使司机右转了,你学习的是最优方向的价值
\end{itemize}

\subsection{例子4:看别人玩游戏来学习}

\textbf{场景}:你观看游戏高手的直播,学习最优游戏策略。

\textbf{行为策略}:游戏高手怎么玩(高手的游戏策略)
\begin{itemize}
    \item 高手在某个关卡选择了"使用技能A"(行为策略:高手的策略)
    \item 高手实际使用了这个技能
\end{itemize}

\textbf{目标策略}:你想要学会的最优游戏策略
\begin{itemize}
    \item 你想要学会"在这个关卡应该使用什么技能"(目标策略:最优策略)
    \item 可能最优是"使用技能B",而不是"技能A"
\end{itemize}

\textbf{Off-policy特点}:
\begin{itemize}
    \item 行为策略 $\neq$ 目标策略:高手怎么玩 $\neq$ 你想要学会的方式
    \item 你观察高手的动作(使用技能A),但学习的是最优动作(使用技能B)的价值
    \item 即使高手用了技能A,你学习的是"最优技能"的价值
    \item 你可以观看多个高手的直播,但目标是学会最优策略
\end{itemize}

\textbf{类比到Q-learning}:
\begin{itemize}
    \item 高手选择了"技能A"(行为策略选择)
    \item 但你学习的是"最优技能"(可能是"技能B")的价值
    \item 你使用 $\max$ 操作:$\max\{\text{技能A的价值}, \text{技能B的价值}, \text{技能C的价值}\}$
    \item 即使高手用了技能A,你学习的是最优技能的价值
\end{itemize}

\subsection{例子5:看别人投资来学习}

\textbf{场景}:你观察别人的投资行为,学习最优投资策略。

\textbf{行为策略}:别人怎么投资(投资者的投资方式)
\begin{itemize}
    \item 投资者选择了"买入股票A"(行为策略:投资者的策略)
    \item 投资者实际做了这个投资决策
\end{itemize}

\textbf{目标策略}:你想要学会的最优投资策略
\begin{itemize}
    \item 你想要学会"应该投资什么"(目标策略:最优策略)
    \item 可能最优是"买入股票B",而不是"股票A"
\end{itemize}

\textbf{Off-policy特点}:
\begin{itemize}
    \item 行为策略 $\neq$ 目标策略:别人怎么投资 $\neq$ 你想要学会的方式
    \item 你观察投资者的动作(买入股票A),但学习的是最优动作(买入股票B)的价值
    \item 即使投资者买了股票A,你学习的是"最优投资"的价值
    \item 你可以观察多个投资者的行为,但目标是学会最优投资策略
\end{itemize}

\textbf{类比到Q-learning}:
\begin{itemize}
    \item 投资者选择了"股票A"(行为策略选择)
    \item 但你学习的是"最优投资"(可能是"股票B")的价值
    \item 你使用 $\max$ 操作:$\max\{\text{股票A的价值}, \text{股票B的价值}, \text{股票C的价值}\}$
    \item 即使投资者买了股票A,你学习的是最优投资的价值
\end{itemize}

\subsection{例子6:看别人学习来学习}

\textbf{场景}:你观察别人的学习方法,学习最优学习方法。

\textbf{行为策略}:别人怎么学习(观察对象的学习方式)
\begin{itemize}
    \item 别人选择了"先看视频再做题"(行为策略:别人的学习策略)
    \item 别人实际用了这个方法
\end{itemize}

\textbf{目标策略}:你想要学会的最优学习方法
\begin{itemize}
    \item 你想要学会"应该用什么方法学习"(目标策略:最优策略)
    \item 可能最优是"先做题再看视频",而不是"先看视频再做题"
\end{itemize}

\textbf{Off-policy特点}:
\begin{itemize}
    \item 行为策略 $\neq$ 目标策略:别人怎么学 $\neq$ 你想要学会的方式
    \item 你观察别人的动作(先看视频再做题),但学习的是最优动作(先做题再看视频)的价值
    \item 即使别人用了"先看视频再做题",你学习的是"最优学习方法"的价值
    \item 你可以观察多个人的学习方法,但目标是学会最优方法
\end{itemize}

\textbf{类比到Q-learning}:
\begin{itemize}
    \item 别人选择了"先看视频再做题"(行为策略选择)
    \item 但你学习的是"最优方法"(可能是"先做题再看视频")的价值
    \item 你使用 $\max$ 操作:$\max\{\text{方法A的价值}, \text{方法B的价值}, \text{方法C的价值}\}$
    \item 即使别人用了方法A,你学习的是最优方法的价值
\end{itemize}

\section{Off-policy的核心特点总结}

\subsection{共同特征}

所有Off-policy例子都有以下共同特征:

\begin{enumerate}
    \item \textbf{观察者 vs 行动者}:
    \begin{itemize}
        \item 有一个\textbf{行动者}(别人)按照行为策略行动
        \item 有一个\textbf{观察者}(你)观察行动者的行为
        \item 观察者学习的是目标策略,而不是行为策略
    \end{itemize}
    
    \item \textbf{行为策略 $\neq$ 目标策略}:
    \begin{itemize}
        \item 行动者使用的策略 $\neq$ 观察者想要学习的策略
        \item 行动者可能选择次优动作(用于探索)
        \item 观察者学习的是最优动作的价值
    \end{itemize}
    
    \item \textbf{使用最优值而不是实际值}:
    \begin{itemize}
        \item 即使行动者选择了某个动作,观察者学习的是最优动作的价值
        \item 这就像Q-learning使用 $\max_{a} Q(S_{t+1}, a)$ 而不是 $Q(S_{t+1}, A_{t+1})$
    \end{itemize}
    
    \item \textbf{可以从多个来源学习}:
    \begin{itemize}
        \item 可以观察多个行动者的行为
        \item 可以从历史数据中学习
        \item 样本效率高
    \end{itemize}
\end{enumerate}

\subsection{与On-policy的对比}

\begin{center}
\begin{tabular}{|l|l|l|}
\hline
\textbf{特征} & \textbf{On-policy} & \textbf{Off-policy} \\
\hline
\textbf{类比} & 边学边用 & 看别人学 \\
\hline
\textbf{行为策略} & 你自己的方式 & 别人的方式 \\
\hline
\textbf{目标策略} & 你自己的方式 & 最优方式 \\
\hline
\textbf{关系} & 行为策略 = 目标策略 & 行为策略 $\neq$ 目标策略 \\
\hline
\textbf{学习来源} & 自己的经验 & 别人的经验或历史数据 \\
\hline
\textbf{使用动作} & 实际选择的动作 & 最优动作(不是实际选择的) \\
\hline
\textbf{样本效率} & 较低(必须用自己的数据) & 较高(可以用历史数据) \\
\hline
\end{tabular}
\end{center}

\section{常见问题解答}

\subsection{问题1:为什么On-policy必须使用实际选择的动作?}

\textbf{通俗回答}:

想象你在学做菜:
\begin{itemize}
    \item 你想要学习"你的做菜方式"的好坏
    \item 你必须\textbf{按照你的方式}来做菜
    \item 你\textbf{实际做了什么},就用\textbf{这个结果}来学习
    \item 如果你按照别人的方式做菜,你学习的就是"别人的做菜方式",而不是"你的做菜方式"
\end{itemize}

\textbf{数学回答}:
\begin{itemize}
    \item 因为行为策略 = 目标策略,所以所有动作都由当前策略选择
    \item $A_{t+1}$ 必须由当前策略选择
    \item 更新时使用 $Q(S_{t+1}, A_{t+1})$,学习的是当前策略的价值
\end{itemize}

\subsection{问题2:行为策略 = 目标策略 和 使用实际选择的动作 是什么关系?}

\textbf{关系链}:

\textbf{第1层:定义}
\begin{itemize}
    \item 行为策略 = 目标策略:你实际怎么做 = 你想要学会怎么做
\end{itemize}

\textbf{第2层:结果}
\begin{itemize}
    \item 因为行为策略 = 目标策略,所以所有动作都由当前策略选择
    \item $A_{t+1}$ 必须由当前策略选择
\end{itemize}

\textbf{第3层:使用}
\begin{itemize}
    \item 因为 $A_{t+1}$ 由当前策略选择,所以更新时使用 $Q(S_{t+1}, A_{t+1})$
    \item 使用实际选择的动作来学习
\end{itemize}

\textbf{第4层:目标}
\begin{itemize}
    \item 学习的是当前策略的价值函数
\end{itemize}

\subsection{问题3:能不能不用实际选择的动作?}

\textbf{On-policy方法}:
\begin{itemize}
    \item \textbf{不能}:因为行为策略 = 目标策略,所以必须使用实际选择的动作
    \item 如果不用实际选择的动作,学习的就是其他策略的价值
\end{itemize}

\textbf{Off-policy方法}:
\begin{itemize}
    \item \textbf{可以}:因为行为策略 $\neq$ 目标策略,所以可以使用最优动作
    \item Q-learning 就是使用 $\max_{a} Q(S_{t+1}, a)$,而不是实际选择的动作
\end{itemize}

\section{总结}

\subsection{核心概念(通俗版)}

\begin{enumerate}
    \item \textbf{行为策略 = 目标策略}:
    \begin{quote}
    "你实际怎么做 = 你想要学会怎么做"
    \end{quote}
    
    \item \textbf{使用实际选择的下一动作}:
    \begin{quote}
    "你实际做了什么,就用这个结果来学习"
    \end{quote}
    
    \item \textbf{关系}:
    \begin{quote}
    "因为你实际怎么做 = 你想要学会怎么做,所以你的所有动作都按照你想要学会的方式来做,你实际选择了什么,就用什么来学习"
    \end{quote}
    
    \item \textbf{为什么必须由当前策略选择?}:
    \begin{quote}
    "如果你想要学习'你的方式',你必须按照'你的方式'来做,否则你学习的就是'别人的方式'"
    \end{quote}
\end{enumerate}

\subsection{关键公式(简化版)}

\textbf{On-policy条件}:
\begin{equation}
\text{你实际怎么做} = \text{你想要学会怎么做}
\end{equation}

\textbf{动作选择}:
\begin{equation}
\text{你的下一个动作} = \text{按照你想要学会的方式选择}
\end{equation}

\textbf{学习}:
\begin{equation}
\text{你实际选择了什么} \to \text{用这个结果来学习} \to \text{学习你想要学会的方式的好坏}
\end{equation}

\subsection{记忆技巧}

\begin{itemize}
    \item \textbf{On-policy}:边学边用,做的就是你学的
    \item \textbf{Off-policy}:看别人学,但学的是最优的
    \item \textbf{使用实际动作}:你实际做了什么,就用什么来学
    \item \textbf{必须由当前策略}:因为你学的就是你的方式,所以必须按你的方式做
\end{itemize}

\end{document}

