\documentclass[12pt,a4paper]{article}
\usepackage[UTF8]{ctex}
\usepackage{amsmath}
\usepackage{amssymb}
\usepackage{amsthm}
\usepackage{algorithm}
\usepackage{algorithmic}
\usepackage{graphicx}
\usepackage{hyperref}
\usepackage{geometry}
\geometry{left=2.5cm,right=2.5cm,top=2.5cm,bottom=2.5cm}

\title{On-Policy 与 Off-Policy 方法详解}
\author{强化学习笔记}
\date{\today}

\begin{document}

\maketitle

\tableofcontents
\newpage

\section{引言}

在强化学习中,\textbf{On-policy} 和 \textbf{Off-policy} 是两种重要的学习方法分类。理解这两种方法的区别对于选择合适的算法至关重要。本章将详细讨论这两种方法的核心概念、区别、优缺点以及具体应用。

\section{基本定义}

\subsection{On-Policy 方法}

\textbf{定义}:On-policy 方法是指\textbf{评估和改进的是同一个策略}。也就是说,用于选择动作的策略(行为策略)和正在学习的策略(目标策略)是同一个策略。

\textbf{关键特征}:
\begin{itemize}
    \item 行为策略 = 目标策略 = $\pi$
    \item 学习的是当前策略的价值函数 $v_\pi(s)$ 或 $q_\pi(s, a)$
    \item 策略改进基于当前策略的价值估计
    \item 必须使用当前策略生成的数据
\end{itemize}

\subsection{Off-Policy 方法}

\textbf{定义}:Off-policy 方法是指\textbf{评估一个策略(目标策略 $\pi$),但使用另一个策略(行为策略 $b$)生成的数据}。

\textbf{关键特征}:
\begin{itemize}
    \item 行为策略 $b$ $\neq$ 目标策略 $\pi$
    \item 学习的是目标策略的价值函数 $v_\pi(s)$ 或 $q_\pi(s, a)$
    \item 可以使用任何策略(行为策略 $b$)收集数据
    \item 需要重要性采样等技术来修正数据分布差异
\end{itemize}

\section{核心区别}

\subsection{策略关系}

\textbf{On-Policy}:
\begin{equation}
\text{行为策略} = \text{目标策略} = \pi
\end{equation}

\textbf{Off-Policy}:
\begin{equation}
\text{行为策略} = b \neq \text{目标策略} = \pi
\end{equation}

\subsection{数据使用}

\textbf{On-Policy}:
\begin{itemize}
    \item 必须使用当前策略 $\pi$ 生成的数据
    \item 数据收集和学习使用同一个策略
    \item 数据用完即弃,不能重用历史数据
\end{itemize}

\textbf{Off-Policy}:
\begin{itemize}
    \item 可以使用任何策略 $b$ 生成数据
    \item 数据收集和学习使用不同策略
    \item 可以重用历史数据(经验回放)
    \item 可以从其他智能体或人类专家的经验中学习
\end{itemize}

\section{Off-Policy 的详细讲解}

\subsection{"学习最优动作价值函数,但可以使用任何策略收集数据"的含义}

这是 Q-learning 等 Off-policy 方法的核心特征。让我们详细分解这句话:

\subsubsection{第一部分:学习最优动作价值函数}

\textbf{最优动作价值函数 $q_*(s, a)$}:
\begin{equation}
q_*(s, a) = \max_\pi q_\pi(s, a) = \mathbb{E}[R_{t+1} + \gamma \max_{a'} q_*(S_{t+1}, a') | S_t = s, A_t = a]
\end{equation}

\textbf{Q-learning 的更新公式}:
\begin{equation}
Q(S_t, A_t) \gets Q(S_t, A_t) + \alpha [R_{t+1} + \gamma \max_{a} Q(S_{t+1}, a) - Q(S_t, A_t)]
\label{eq:qlearning}
\end{equation}

\textbf{关键点}:
\begin{itemize}
    \item 使用 $\max_{a} Q(S_{t+1}, a)$ 而不是 $Q(S_{t+1}, A_{t+1})$
    \item 这意味着无论实际选择的动作 $A_{t+1}$ 是什么,都使用最优动作的价值
    \item 直接学习最优策略 $\pi_*$ 的动作价值函数
    \item 不需要显式的策略改进步骤
\end{itemize}

\subsubsection{第二部分:可以使用任何策略收集数据}

\textbf{行为策略 $b$ 的作用}:

在 Q-learning 中,虽然我们学习的是最优动作价值函数 $q_*$,但收集数据时可以使用任何策略 $b$(行为策略)。

\textbf{Q-learning 算法流程}:
\begin{algorithm}[H]
\caption{Q-learning(Off-policy TD控制)}
\begin{algorithmic}[1]
\REQUIRE 所有状态-动作对都有非零概率被访问(通过行为策略 $b$)
\ENSURE 最优动作价值函数 $q_*$ 和最优策略 $\pi_*$
\STATE \textbf{初始化}:$Q(s, a) \in \mathbb{R}$ 任意初始化
\STATE
\REPEAT
    \STATE 初始化 $S$(回合的起始状态)
    \REPEAT
        \STATE $A \gets$ \textbf{行为策略 $b$} 给出的动作(如 $\varepsilon$-贪婪,基于 $Q$)
        \STATE 执行动作 $A$,观察奖励 $R$ 和下一状态 $S'$
        \STATE $Q(S, A) \gets Q(S, A) + \alpha [R + \gamma \max_{a} Q(S', a) - Q(S, A)]$
        \STATE $S \gets S'$
    \UNTIL{$S$ 是终止状态}
\UNTIL{收敛}
\STATE \textbf{提取最优策略}:
\FOR{每个状态 $s \in \mathcal{S}$}
    \STATE $\pi_*(s) \gets \arg\max_{a} Q(s, a)$
\ENDFOR
\RETURN $q_* = Q$,$\pi_*$
\end{algorithmic}
\end{algorithm}

\textbf{关键观察}:
\begin{itemize}
    \item \textbf{第 5 行}:使用行为策略 $b$ 选择动作(用于探索)
    \item \textbf{第 7 行}:使用 $\max_{a} Q(S', a)$ 更新(学习最优价值)
    \item 行为策略 $b$ 可以是 $\varepsilon$-贪婪、随机策略,甚至人类专家的策略
    \item 目标策略 $\pi_*$ 是贪婪策略:$\pi_*(s) = \arg\max_{a} Q(s, a)$
\end{itemize}

\subsection{为什么 Off-Policy 可以做到这一点?}

\textbf{核心原因}:Q-learning 的更新公式不依赖于实际选择的动作 $A_{t+1}$。

\textbf{对比 SARSA(On-policy)}:
\begin{equation}
Q(S_t, A_t) \gets Q(S_t, A_t) + \alpha [R_{t+1} + \gamma Q(S_{t+1}, A_{t+1}) - Q(S_t, A_t)]
\label{eq:sarsa}
\end{equation}

\textbf{关键区别}:
\begin{itemize}
    \item \textbf{SARSA}:使用实际选择的下一动作 $A_{t+1}$(必须由当前策略选择)
    \item \textbf{Q-learning}:使用最优动作 $\max_{a} Q(S_{t+1}, a)$(不依赖实际选择的动作)
\end{itemize}

\textbf{例子说明}:

假设在状态 $s_5$ 执行动作"上",转移到状态 $s_2$:

\textbf{SARSA(On-policy)}:
\begin{itemize}
    \item 当前策略在 $s_2$ 选择动作:$A_{t+1} = \text{左}$(可能是探索性动作)
    \item 更新目标:$R_{t+1} + \gamma Q(s_2, \text{左})$
    \item 学习的是当前策略(包含探索)的价值函数
\end{itemize}

\textbf{Q-learning(Off-policy)}:
\begin{itemize}
    \item 行为策略在 $s_2$ 选择动作:$A_{t+1} = \text{右}$(可能是探索性动作)
    \item 更新目标:$R_{t+1} + \gamma \max_{a} Q(s_2, a)$(使用最优动作,不是实际选择的动作)
    \item 学习的是最优策略的价值函数,不受行为策略影响
\end{itemize}

\subsection{Off-Policy 的优势}

\textbf{1. 样本效率高}:
\begin{itemize}
    \item 可以使用经验回放(Experience Replay)
    \item 可以重用历史数据
    \item 不需要每次都用最新策略生成新数据
\end{itemize}

\textbf{2. 灵活性}:
\begin{itemize}
    \item 可以从其他智能体的经验中学习
    \item 可以从人类专家的演示中学习
    \item 可以使用任何探索策略收集数据
\end{itemize}

\textbf{3. 学习最优策略}:
\begin{itemize}
    \item 直接学习最优动作价值函数 $q_*$
    \item 不需要在探索和利用之间妥协
    \item 可以使用贪婪的目标策略,同时保持探索性的行为策略
\end{itemize}

\section{具体算法对比}

\subsection{SARSA vs Q-learning}

\subsubsection{SARSA(On-policy)}

\textbf{更新公式}:
\begin{equation}
Q(S_t, A_t) \gets Q(S_t, A_t) + \alpha [R_{t+1} + \gamma Q(S_{t+1}, A_{t+1}) - Q(S_t, A_t)]
\end{equation}

\textbf{特点}:
\begin{itemize}
    \item 使用当前策略选择动作 $A_{t+1}$
    \item 学习当前策略的动作价值函数 $q_\pi(s, a)$
    \item 策略改进基于当前策略的价值估计
    \item 更保守,考虑探索带来的风险
\end{itemize}

\textbf{算法流程}:
\begin{algorithm}[H]
\caption{SARSA(On-policy TD控制)}
\begin{algorithmic}[1]
\STATE \textbf{初始化}:$Q(s, a)$ 任意初始化
\STATE $\pi$ 为 $\varepsilon$-贪婪策略,基于 $Q$
\REPEAT
    \STATE 初始化 $S$
    \STATE $A \gets$ 策略 $\pi$ 给出的动作(基于 $Q$)
    \REPEAT
        \STATE 执行动作 $A$,观察奖励 $R$ 和下一状态 $S'$
        \STATE $A' \gets$ 策略 $\pi$ 给出的动作(基于 $Q$)
        \STATE $Q(S, A) \gets Q(S, A) + \alpha [R + \gamma Q(S', A') - Q(S, A)]$
        \STATE $\pi(S) \gets \varepsilon$-贪婪策略,基于 $Q(S, \cdot)$
        \STATE $S \gets S'$,$A \gets A'$
    \UNTIL{$S$ 是终止状态}
\UNTIL{策略稳定}
\end{algorithmic}
\end{algorithm}

\subsubsection{Q-learning(Off-policy)}

\textbf{更新公式}:
\begin{equation}
Q(S_t, A_t) \gets Q(S_t, A_t) + \alpha [R_{t+1} + \gamma \max_{a} Q(S_{t+1}, a) - Q(S_t, A_t)]
\end{equation}

\textbf{特点}:
\begin{itemize}
    \item 使用行为策略选择动作,但更新时不依赖实际选择的动作
    \item 直接学习最优动作价值函数 $q_*(s, a)$
    \item 不需要显式策略改进步骤
    \item 更激进,直接学习最优策略
\end{itemize}

\subsection{关键区别总结}

\begin{table}[H]
\centering
\caption{SARSA vs Q-learning 对比}
\begin{tabular}{|l|l|l|}
\hline
\textbf{特征} & \textbf{SARSA(On-policy)} & \textbf{Q-learning(Off-policy)} \\
\hline
更新公式 & $Q(S', A')$ & $\max_{a} Q(S', a)$ \\
\hline
策略关系 & 行为策略 = 目标策略 & 行为策略 $\neq$ 目标策略 \\
\hline
学习目标 & 当前策略的价值函数 & 最优策略的价值函数 \\
\hline
数据使用 & 必须用当前策略生成 & 可以用任何策略生成 \\
\hline
探索考虑 & 考虑探索风险 & 不考虑探索风险 \\
\hline
适用场景 & 需要安全的场景 & 追求最优性能的场景 \\
\hline
\end{tabular}
\end{table}

\section{行为策略与目标策略}

\subsection{概念定义}

\textbf{行为策略(Behavior Policy)$b$}:
\begin{itemize}
    \item 用于生成数据的策略
    \item 决定智能体如何与环境交互
    \item 通常是探索性的(如 $\varepsilon$-贪婪策略)
    \item 在 Off-policy 方法中,$b \neq \pi$
\end{itemize}

\textbf{目标策略(Target Policy)$\pi$}:
\begin{itemize}
    \item 正在学习的策略
    \item 决定我们想要评估或改进的策略
    \item 在控制问题中,通常是贪婪策略
    \item 在 On-policy 方法中,$\pi = b$
\end{itemize}

\subsection{覆盖假设(Coverage Assumption)}

\textbf{定义}:为了使用行为策略 $b$ 的数据来学习目标策略 $\pi$,必须满足:

\begin{equation}
\pi(a|s) > 0 \implies b(a|s) > 0
\end{equation}

\textbf{含义}:
\begin{itemize}
    \item 目标策略 $\pi$ 可能选择的所有动作,行为策略 $b$ 也必须有可能选择
    \item 这确保了我们可以从行为策略的数据中学习目标策略
    \item 如果某个动作在目标策略中可能出现,但在行为策略中从不出现,就无法学习该动作的价值
\end{itemize}

\subsection{在 Q-learning 中的应用}

在 Q-learning 中:
\begin{itemize}
    \item \textbf{行为策略 $b$}:$\varepsilon$-贪婪策略(用于探索)
    \item \textbf{目标策略 $\pi_*$}:贪婪策略 $\pi_*(s) = \arg\max_{a} Q(s, a)$(最优策略)
    \item 覆盖假设:由于 $\varepsilon$-贪婪策略对所有动作都有非零概率,满足覆盖假设
\end{itemize}

\section{优缺点对比}

\subsection{On-Policy 方法}

\textbf{优点}:
\begin{itemize}
    \item \textbf{简单}:不需要重要性采样等复杂技术
    \item \textbf{稳定}:学习的是实际使用的策略,更符合实际行为
    \item \textbf{方差小}:数据分布与学习目标一致,方差较小
    \item \textbf{收敛快}:在简单问题上通常收敛更快
\end{itemize}

\textbf{缺点}:
\begin{itemize}
    \item \textbf{样本效率低}:必须使用当前策略生成新数据,不能重用历史数据
    \item \textbf{探索受限}:必须在学习策略中保持探索,可能影响性能
    \item \textbf{灵活性差}:无法从其他来源的数据中学习
\end{itemize}

\subsection{Off-Policy 方法}

\textbf{优点}:
\begin{itemize}
    \item \textbf{样本效率高}:可以重用历史数据(经验回放)
    \item \textbf{灵活性高}:可以从任何策略的数据中学习
    \item \textbf{学习最优策略}:可以直接学习最优策略,不受探索影响
    \item \textbf{应用广泛}:可以从人类演示、其他智能体等学习
\end{itemize}

\textbf{缺点}:
\begin{itemize}
    \item \textbf{复杂}:需要重要性采样等技术(某些方法如 Q-learning 不需要)
    \item \textbf{方差大}:数据分布与学习目标不一致,可能导致高方差
    \item \textbf{收敛慢}:在某些情况下可能收敛较慢
    \item \textbf{不稳定}:在函数逼近情况下可能不稳定
\end{itemize}

\section{应用场景}

\subsection{On-Policy 适用场景}

\begin{itemize}
    \item \textbf{需要安全的场景}:如悬崖行走问题,需要考虑探索风险
    \item \textbf{简单问题}:状态空间较小,样本生成成本低
    \item \textbf{在线学习}:需要实时学习和决策
    \item \textbf{策略梯度方法}:如 REINFORCE、PPO 等
\end{itemize}

\subsection{Off-Policy 适用场景}

\begin{itemize}
    \item \textbf{样本成本高}:如机器人控制,每次交互成本高
    \item \textbf{需要重用数据}:如深度强化学习中的经验回放
    \item \textbf{从演示学习}:从人类专家或其他智能体的经验中学习
    \item \textbf{多任务学习}:从一个任务的数据中学习另一个任务
    \item \textbf{Q-learning 系列}:DQN、DDPG、SAC 等
\end{itemize}

\section{深度强化学习中的应用}

\subsection{On-Policy 深度强化学习}

\textbf{代表性算法}:
\begin{itemize}
    \item \textbf{PPO(Proximal Policy Optimization)}:使用当前策略收集数据,然后更新策略
    \item \textbf{A3C(Asynchronous Advantage Actor-Critic)}:多个智能体并行收集数据
    \item \textbf{TRPO(Trust Region Policy Optimization)}:限制策略更新幅度
\end{itemize}

\textbf{特点}:
\begin{itemize}
    \item 数据收集后立即使用,用完即弃
    \item 策略更新后需要重新收集数据
    \item 样本效率相对较低,但更稳定
\end{itemize}

\subsection{Off-Policy 深度强化学习}

\textbf{代表性算法}:
\begin{itemize}
    \item \textbf{DQN(Deep Q-Network)}:使用经验回放,从历史数据中学习
    \item \textbf{DDPG(Deep Deterministic Policy Gradient)}:Actor-Critic 架构,使用经验回放
    \item \textbf{SAC(Soft Actor-Critic)}:最大熵强化学习,使用经验回放
\end{itemize}

\textbf{特点}:
\begin{itemize}
    \item 使用经验回放缓冲区存储历史数据
    \item 可以从旧策略的数据中学习新策略
    \item 样本效率高,但需要处理分布不匹配问题
\end{itemize}

\section{总结}

\subsection{核心要点}

\begin{enumerate}
    \item \textbf{On-policy}:行为策略 = 目标策略,学习当前策略的价值函数
    \item \textbf{Off-policy}:行为策略 $\neq$ 目标策略,可以使用任何策略收集数据
    \item \textbf{Q-learning} 的特殊性:不需要重要性采样,直接学习最优价值函数
    \item \textbf{选择原则}:根据问题特点、样本成本、稳定性要求选择合适的方法
\end{enumerate}

\subsection{关键公式}

\textbf{SARSA(On-policy)}:
\begin{equation}
Q(S_t, A_t) \gets Q(S_t, A_t) + \alpha [R_{t+1} + \gamma Q(S_{t+1}, A_{t+1}) - Q(S_t, A_t)]
\end{equation}

\textbf{Q-learning(Off-policy)}:
\begin{equation}
Q(S_t, A_t) \gets Q(S_t, A_t) + \alpha [R_{t+1} + \gamma \max_{a} Q(S_{t+1}, a) - Q(S_t, A_t)]
\end{equation}

\textbf{覆盖假设}:
\begin{equation}
\pi(a|s) > 0 \implies b(a|s) > 0
\end{equation}

\subsection{理解 Off-Policy 的关键}

\textbf{"学习最优动作价值函数,但可以使用任何策略收集数据"} 这句话的含义:

\begin{enumerate}
    \item \textbf{学习最优动作价值函数}:通过使用 $\max_{a} Q(S_{t+1}, a)$,直接学习 $q_*$,不依赖实际选择的动作
    \item \textbf{可以使用任何策略收集数据}:行为策略 $b$ 可以是任何策略($\varepsilon$-贪婪、随机、人类专家等),只要满足覆盖假设
    \item \textbf{核心机制}:更新公式不依赖实际选择的动作 $A_{t+1}$,只依赖状态 $S_{t+1}$ 和所有可能动作的价值
    \item \textbf{优势}:可以在保持探索性的同时,直接学习最优策略,提高样本效率
\end{enumerate}

\end{document}

