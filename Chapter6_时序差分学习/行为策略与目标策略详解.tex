\documentclass[12pt,a4paper]{article}
\usepackage[UTF8]{ctex}
\usepackage{amsmath}
\usepackage{amssymb}
\usepackage{amsthm}
\usepackage{geometry}
\usepackage{hyperref}
\usepackage{enumitem}
\usepackage{booktabs}

\geometry{left=2.5cm,right=2.5cm,top=2.5cm,bottom=2.5cm}

\title{行为策略与目标策略详解}
\subtitle{On-policy 和 Off-policy 的核心概念}
\author{强化学习笔记}
\date{\today}

\newtheorem{definition}{定义}
\newtheorem{theorem}{定理}
\newtheorem{example}{示例}
\newtheorem{remark}{注记}

\begin{document}

\maketitle

\tableofcontents
\newpage

\section{引言}

理解\textbf{行为策略}(Behavior Policy)和\textbf{目标策略}(Target Policy)的区别,以及它们与"使用实际选择的下一动作"的关系,是掌握 On-policy 和 Off-policy 方法的关键。

\section{基本定义}

\subsection{行为策略(Behavior Policy)}

\begin{definition}[行为策略 $b$]
\textbf{行为策略} $b$ 是用于\textbf{生成数据}(选择动作与环境交互)的策略。
\begin{equation}
b(a|s) = \Pr\{A_t = a | S_t = s\}
\end{equation}
\end{definition}

\textbf{作用}:
\begin{itemize}
    \item 决定智能体如何与环境交互
    \item 用于收集经验数据
    \item 通常是探索性的(如 $\varepsilon$-贪婪策略)
\end{itemize}

\subsection{目标策略(Target Policy)}

\begin{definition}[目标策略 $\pi$]
\textbf{目标策略} $\pi$ 是\textbf{正在学习}或\textbf{正在评估}的策略。
\begin{equation}
\pi(a|s) = \Pr\{A_t = a | S_t = s\}
\end{equation}
\end{definition}

\textbf{作用}:
\begin{itemize}
    \item 决定我们想要学习或评估的策略
    \item 在控制问题中,通常是贪婪策略或最优策略
    \item 是我们最终想要得到的策略
\end{itemize}

\section{On-policy vs Off-policy}

\subsection{On-policy:行为策略 = 目标策略}

\begin{definition}[On-policy]
当\textbf{行为策略等于目标策略}时,我们称方法为 \textbf{On-policy}:
\begin{equation}
b = \pi
\end{equation}
\end{definition}

\textbf{含义}:
\begin{itemize}
    \item 用于生成数据的策略 = 正在学习的策略
    \item 智能体使用当前策略与环境交互
    \item 学习的是当前策略的价值函数
\end{itemize}

\subsection{Off-policy:行为策略 $\neq$ 目标策略}

\begin{definition}[Off-policy]
当\textbf{行为策略不等于目标策略}时,我们称方法为 \textbf{Off-policy}:
\begin{equation}
b \neq \pi
\end{equation}
\end{definition}

\textbf{含义}:
\begin{itemize}
    \item 用于生成数据的策略 $\neq$ 正在学习的策略
    \item 可以使用任何策略收集数据
    \item 学习的是目标策略的价值函数
\end{itemize}

\section{"使用实际选择的下一动作"的含义}

\subsection{在SARSA中的体现}

\textbf{SARSA更新公式}:
\begin{equation}
Q(S_t, A_t) \gets Q(S_t, A_t) + \alpha [R_{t+1} + \gamma Q(S_{t+1}, A_{t+1}) - Q(S_t, A_t)]
\label{eq:sarsa}
\end{equation}

\textbf{关键点}:
\begin{itemize}
    \item $A_{t+1}$ 是\textbf{实际选择的下一动作}
    \item $A_{t+1}$ 必须由\textbf{当前策略}选择
    \item 在SARSA中,当前策略 = 目标策略 = 行为策略
\end{itemize}

\textbf{算法流程}:
\begin{enumerate}
    \item 在状态 $S_t$,使用策略 $\pi$ 选择动作 $A_t$
    \item 执行动作 $A_t$,观察 $R_{t+1}, S_{t+1}$
    \item 在状态 $S_{t+1}$,使用策略 $\pi$ 选择动作 $A_{t+1}$(\textbf{实际选择的动作})
    \item 使用 $Q(S_{t+1}, A_{t+1})$ 更新 $Q(S_t, A_t)$
\end{enumerate}

\subsection{为什么必须由当前策略选择?}

\textbf{原因1:On-policy的要求}

在On-policy方法中:
\begin{itemize}
    \item 行为策略 = 目标策略 = $\pi$
    \item 所有动作都必须由策略 $\pi$ 选择
    \item $A_{t+1}$ 必须由策略 $\pi$ 选择,才能保证学习的是策略 $\pi$ 的价值函数
\end{itemize}

\textbf{原因2:学习当前策略的价值函数}

SARSA学习的是\textbf{当前策略} $\pi$ 的动作价值函数 $q_\pi(s, a)$:
\begin{equation}
q_\pi(s, a) = \mathbb{E}_\pi[G_t | S_t = s, A_t = a]
\end{equation}

要学习 $q_\pi$,必须使用策略 $\pi$ 选择的动作,因为:
\begin{itemize}
    \item $q_\pi(s, a)$ 的定义依赖于策略 $\pi$
    \item 如果使用其他策略选择的动作,学习的就是其他策略的价值函数
\end{itemize}

\section{行为策略 = 目标策略 的含义}

\subsection{数学表达}

\textbf{On-policy条件}:
\begin{equation}
b(a|s) = \pi(a|s) \quad \text{对所有 } s \in \mathcal{S}, a \in \mathcal{A}(s)
\end{equation}

\textbf{等价表述}:
\begin{equation}
b = \pi
\end{equation}

\subsection{实际含义}

\textbf{1. 选择动作时}:
\begin{itemize}
    \item 在状态 $S_t$,使用策略 $\pi$ 选择动作 $A_t$
    \item 在状态 $S_{t+1}$,使用策略 $\pi$ 选择动作 $A_{t+1}$
    \item 所有动作都由同一个策略 $\pi$ 选择
\end{itemize}

\textbf{2. 学习目标时}:
\begin{itemize}
    \item 学习的是策略 $\pi$ 的价值函数 $q_\pi(s, a)$
    \item 评估的是策略 $\pi$ 的性能
    \item 改进的是策略 $\pi$
\end{itemize}

\textbf{3. 数据使用}:
\begin{itemize}
    \item 只能使用策略 $\pi$ 生成的数据
    \item 不能使用其他策略生成的数据
    \item 数据收集和学习使用同一个策略
\end{itemize}

\section{关系链:从策略到动作选择}

\subsection{On-policy方法的完整链条}

\textbf{步骤1:策略定义}
\begin{equation}
\pi(a|s) = \text{当前策略在状态 } s \text{ 选择动作 } a \text{ 的概率}
\end{equation}

\textbf{步骤2:动作选择}
\begin{equation}
A_t \sim \pi(\cdot|S_t) \quad \text{(根据策略 } \pi \text{ 选择动作)}
\end{equation}

\textbf{步骤3:下一动作选择}
\begin{equation}
A_{t+1} \sim \pi(\cdot|S_{t+1}) \quad \text{(根据策略 } \pi \text{ 选择下一动作)}
\end{equation}

\textbf{步骤4:更新使用}
\begin{equation}
Q(S_t, A_t) \gets Q(S_t, A_t) + \alpha [R_{t+1} + \gamma Q(S_{t+1}, A_{t+1}) - Q(S_t, A_t)]
\end{equation}

其中 $A_{t+1}$ 是\textbf{实际选择的动作},由\textbf{当前策略} $\pi$ 选择。

\subsection{关键关系}

\begin{center}
\begin{tabular}{|c|c|}
\hline
\textbf{概念} & \textbf{关系} \\
\hline
行为策略 = 目标策略 & $b = \pi$ \\
\hline
当前策略 & $\pi$ \\
\hline
实际选择的动作 & $A_{t+1} \sim \pi(\cdot|S_{t+1})$ \\
\hline
更新中使用 & $Q(S_{t+1}, A_{t+1})$ \\
\hline
学习目标 & $q_\pi(s, a)$ \\
\hline
\end{tabular}
\end{center}

\section{具体例子:SARSA}

\subsection{SARSA算法流程}

\textbf{初始化}:
\begin{itemize}
    \item 策略 $\pi$:$\varepsilon$-贪婪策略,基于 $Q$
    \item 行为策略 $b = \pi$(On-policy)
    \item 目标策略 $\pi = \pi$(同一个策略)
\end{itemize}

\textbf{每个时间步}:

\textbf{时间步 $t$}:
\begin{enumerate}
    \item 在状态 $S_t$,使用策略 $\pi$ 选择动作 $A_t$
    \begin{equation}
    A_t \sim \pi(\cdot|S_t)
    \end{equation}
    
    \item 执行动作 $A_t$,观察 $R_{t+1}, S_{t+1}$
    
    \item 在状态 $S_{t+1}$,使用策略 $\pi$ 选择动作 $A_{t+1}$
    \begin{equation}
    A_{t+1} \sim \pi(\cdot|S_{t+1})
    \end{equation}
    
    \item 更新动作价值函数:
    \begin{equation}
    Q(S_t, A_t) \gets Q(S_t, A_t) + \alpha [R_{t+1} + \gamma Q(S_{t+1}, A_{t+1}) - Q(S_t, A_t)]
    \end{equation}
    
    \item 更新策略 $\pi$(基于新的 $Q$)
\end{enumerate}

\subsection{关键观察}

\textbf{1. 行为策略 = 目标策略}:
\begin{itemize}
    \item 用于选择动作的策略 = 正在学习的策略
    \item $b = \pi$
\end{itemize}

\textbf{2. 使用实际选择的动作}:
\begin{itemize}
    \item $A_{t+1}$ 是实际选择的动作
    \item $A_{t+1}$ 由当前策略 $\pi$ 选择
    \item 更新时使用 $Q(S_{t+1}, A_{t+1})$
\end{itemize}

\textbf{3. 为什么必须由当前策略选择?}:
\begin{itemize}
    \item 因为 $b = \pi$,所以所有动作都必须由 $\pi$ 选择
    \item 这样才能学习策略 $\pi$ 的价值函数
    \item 如果使用其他策略选择的动作,学习的就是其他策略的价值函数
\end{itemize}

\section{对比:Q-learning(Off-policy)}

\subsection{Q-learning中的情况}

\textbf{Q-learning更新公式}:
\begin{equation}
Q(S_t, A_t) \gets Q(S_t, A_t) + \alpha [R_{t+1} + \gamma \max_{a} Q(S_{t+1}, a) - Q(S_t, A_t)]
\label{eq:qlearning}
\end{equation}

\textbf{关键点}:
\begin{itemize}
    \item 行为策略 $b$:$\varepsilon$-贪婪策略(用于选择动作)
    \item 目标策略 $\pi_*$:贪婪策略(正在学习的策略)
    \item $b \neq \pi_*$(Off-policy)
\end{itemize}

\textbf{算法流程}:
\begin{enumerate}
    \item 在状态 $S_t$,使用行为策略 $b$ 选择动作 $A_t$
    \begin{equation}
    A_t \sim b(\cdot|S_t) \quad \text{(行为策略选择)}
    \end{equation}
    
    \item 执行动作 $A_t$,观察 $R_{t+1}, S_{t+1}$
    
    \item 在状态 $S_{t+1}$,使用行为策略 $b$ 选择动作 $A_{t+1}$
    \begin{equation}
    A_{t+1} \sim b(\cdot|S_{t+1}) \quad \text{(行为策略选择,可能与最优动作不同)}
    \end{equation}
    
    \item 更新动作价值函数:
    \begin{equation}
    Q(S_t, A_t) \gets Q(S_t, A_t) + \alpha [R_{t+1} + \gamma \max_{a} Q(S_{t+1}, a) - Q(S_t, A_t)]
    \end{equation}
    
    \item \textbf{关键}:更新时使用 $\max_{a} Q(S_{t+1}, a)$,而不是 $Q(S_{t+1}, A_{t+1})$
\end{enumerate}

\subsection{关键区别}

\textbf{SARSA(On-policy)}:
\begin{itemize}
    \item 行为策略 = 目标策略:$b = \pi$
    \item 使用实际选择的动作:$Q(S_{t+1}, A_{t+1})$
    \item $A_{t+1}$ 必须由当前策略 $\pi$ 选择
    \item 学习当前策略的价值函数
\end{itemize}

\textbf{Q-learning(Off-policy)}:
\begin{itemize}
    \item 行为策略 $\neq$ 目标策略:$b \neq \pi_*$
    \item 不使用实际选择的动作:使用 $\max_{a} Q(S_{t+1}, a)$
    \item $A_{t+1}$ 由行为策略 $b$ 选择,但更新时不使用
    \item 学习最优策略的价值函数
\end{itemize}

\section{为什么On-policy必须使用实际选择的动作?}

\subsection{理论原因}

\textbf{策略梯度定理}(对于On-policy):
\begin{equation}
\nabla_\theta J(\theta) \propto \sum_{s} \mu(s) \sum_{a} q_\pi(s, a) \nabla_\theta \pi(a|s, \theta)
\end{equation}

其中 $\mu(s)$ 是在策略 $\pi$ 下的状态分布。

\textbf{关键点}:
\begin{itemize}
    \item 要学习 $q_\pi(s, a)$,必须使用策略 $\pi$ 生成的数据
    \item 如果使用其他策略的数据,学习的就是其他策略的价值函数
    \item 因此,$A_{t+1}$ 必须由策略 $\pi$ 选择
\end{itemize}

\subsection{实际原因}

\textbf{数据分布匹配}:
\begin{itemize}
    \item On-policy方法要求数据分布与学习目标匹配
    \item 如果 $A_{t+1}$ 由策略 $\pi$ 选择,数据分布就是在策略 $\pi$ 下的分布
    \item 这样才能准确估计 $q_\pi(s, a)$
\end{itemize}

\textbf{一致性要求}:
\begin{itemize}
    \item 学习的是策略 $\pi$ 的价值函数
    \item 使用的数据必须来自策略 $\pi$
    \item 因此,所有动作(包括 $A_{t+1}$)都必须由策略 $\pi$ 选择
\end{itemize}

\section{具体例子对比}

\subsection{例子:Gridworld中的状态 $s_5$}

\textbf{场景}:在状态 $s_5$,执行动作"上",转移到状态 $s_2$。

\textbf{假设}:
\begin{itemize}
    \item $Q(s_2, \text{上}) = 0$
    \item $Q(s_2, \text{下}) = 0$
    \item $Q(s_2, \text{左}) = -0.1$
    \item $Q(s_2, \text{右}) = 0$
    \item 当前策略 $\pi$ 在 $s_2$ 选择"左"的概率最高
\end{itemize}

\textbf{SARSA(On-policy)}:

\textbf{行为策略 = 目标策略}:
\begin{equation}
b = \pi
\end{equation}

\textbf{动作选择}:
\begin{itemize}
    \item 在 $s_2$,策略 $\pi$ 选择动作 $A_{t+1} = \text{左}$(实际选择的动作)
    \item 这个动作\textbf{必须}由当前策略 $\pi$ 选择
\end{itemize}

\textbf{更新}:
\begin{equation}
Q(s_5, \text{上}) \gets Q(s_5, \text{上}) + \alpha [R_{t+1} + \gamma Q(s_2, \text{左}) - Q(s_5, \text{上})]
\end{equation}

\textbf{关键}:使用实际选择的动作"左"的价值 $Q(s_2, \text{左}) = -0.1$。

\textbf{Q-learning(Off-policy)}:

\textbf{行为策略 $\neq$ 目标策略}:
\begin{equation}
b \neq \pi_*
\end{equation}

\textbf{动作选择}:
\begin{itemize}
    \item 在 $s_2$,行为策略 $b$ 选择动作 $A_{t+1} = \text{右}$(实际选择的动作,用于探索)
    \item 但目标策略 $\pi_*$ 会选择 $\arg\max_{a} Q(s_2, a) = \text{上}$(或下、右,因为都是0)
\end{itemize}

\textbf{更新}:
\begin{equation}
Q(s_5, \text{上}) \gets Q(s_5, \text{上}) + \alpha [R_{t+1} + \gamma \max_{a} Q(s_2, a) - Q(s_5, \text{上})]
\end{equation}

\textbf{关键}:
\begin{itemize}
    \item 不使用实际选择的动作"右"的价值
    \item 使用所有动作中的最大值:$\max_{a} Q(s_2, a) = 0$
    \item 即使实际选择了"右",更新时使用的是最优动作的价值
\end{itemize}

\section{关系总结}

\subsection{On-policy方法的完整关系链}

\begin{center}
\begin{tabular}{|c|c|}
\hline
\textbf{概念} & \textbf{关系} \\
\hline
行为策略 & $b$ \\
\hline
目标策略 & $\pi$ \\
\hline
On-policy条件 & $b = \pi$ \\
\hline
当前策略 & $\pi$(与行为策略和目标策略相同) \\
\hline
动作 $A_t$ 的选择 & $A_t \sim \pi(\cdot|S_t)$ \\
\hline
动作 $A_{t+1}$ 的选择 & $A_{t+1} \sim \pi(\cdot|S_{t+1})$(\textbf{必须}由当前策略选择) \\
\hline
更新中使用 & $Q(S_{t+1}, A_{t+1})$(使用实际选择的动作) \\
\hline
学习目标 & $q_\pi(s, a)$(当前策略的价值函数) \\
\hline
\end{tabular}
\end{center}

\subsection{关键逻辑链}

\textbf{逻辑1:行为策略 = 目标策略 $\implies$ 使用当前策略选择动作}
\begin{itemize}
    \item 因为 $b = \pi$,所以所有动作都由策略 $\pi$ 选择
    \item $A_{t+1}$ 必须由当前策略 $\pi$ 选择
\end{itemize}

\textbf{逻辑2:使用当前策略选择动作 $\implies$ 使用实际选择的动作}
\begin{itemize}
    \item $A_{t+1}$ 是实际选择的动作
    \item 因为 $A_{t+1}$ 由当前策略选择,所以更新时使用 $Q(S_{t+1}, A_{t+1})$
\end{itemize}

\textbf{逻辑3:使用实际选择的动作 $\implies$ 学习当前策略的价值函数}
\begin{itemize}
    \item 使用策略 $\pi$ 选择的动作更新
    \item 学习的是策略 $\pi$ 的价值函数 $q_\pi(s, a)$
\end{itemize}

\section{常见误解}

\subsection{误解1:On-policy就是使用实际选择的动作}

\textbf{误解}:On-policy 就是使用实际选择的动作 $A_{t+1}$。

\textbf{正确理解}:
\begin{itemize}
    \item On-policy 的核心是:行为策略 = 目标策略
    \item 使用实际选择的动作是 On-policy 的\textbf{结果},不是定义
    \item 因为 $b = \pi$,所以 $A_{t+1}$ 由策略 $\pi$ 选择,更新时使用 $Q(S_{t+1}, A_{t+1})$
\end{itemize}

\subsection{误解2:必须使用实际选择的动作就是On-policy}

\textbf{误解}:只要使用实际选择的动作就是 On-policy。

\textbf{正确理解}:
\begin{itemize}
    \item On-policy 的定义是:$b = \pi$
    \item 在 On-policy 方法中,$A_{t+1}$ 必须由当前策略选择
    \item 但使用实际选择的动作不一定是 On-policy(可能是巧合)
\end{itemize}

\section{总结}

\subsection{核心要点}

\begin{enumerate}
    \item \textbf{行为策略 = 目标策略}($b = \pi$)是 On-policy 的定义
    
    \item \textbf{在On-policy方法中}:
    \begin{itemize}
        \item 所有动作(包括 $A_{t+1}$)都必须由当前策略 $\pi$ 选择
        \item 更新时使用实际选择的动作 $A_{t+1}$
        \item 学习的是当前策略 $\pi$ 的价值函数
    \end{itemize}
    
    \item \textbf{关系链}:
    \begin{equation}
    b = \pi \implies A_{t+1} \sim \pi(\cdot|S_{t+1}) \implies \text{使用 } Q(S_{t+1}, A_{t+1}) \implies \text{学习 } q_\pi(s, a)
    \end{equation}
    
    \item \textbf{为什么必须由当前策略选择?}
    \begin{itemize}
        \item 因为 $b = \pi$,所以所有动作都由 $\pi$ 选择
        \item 这样才能学习策略 $\pi$ 的价值函数
        \item 数据分布必须与学习目标匹配
    \end{itemize}
\end{enumerate}

\subsection{关键公式}

\textbf{On-policy条件}:
\begin{equation}
b = \pi
\end{equation}

\textbf{动作选择}:
\begin{equation}
A_{t+1} \sim \pi(\cdot|S_{t+1}) \quad \text{(必须由当前策略选择)}
\end{equation}

\textbf{SARSA更新}:
\begin{equation}
Q(S_t, A_t) \gets Q(S_t, A_t) + \alpha [R_{t+1} + \gamma Q(S_{t+1}, A_{t+1}) - Q(S_t, A_t)]
\end{equation}

其中 $A_{t+1}$ 是实际选择的动作,由当前策略 $\pi$ 选择。

\subsection{直观理解}

\begin{quote}
\textbf{"行为策略 = 目标策略"}意味着:用于选择动作的策略和正在学习的策略是同一个策略。

\textbf{"使用实际选择的下一动作"}意味着:更新时使用实际选择的动作 $A_{t+1}$ 的价值。

\textbf{关系}:因为行为策略 = 目标策略,所以 $A_{t+1}$ 必须由当前策略选择,更新时使用 $Q(S_{t+1}, A_{t+1})$,学习的是当前策略的价值函数。
\end{quote}

\end{document}

