\documentclass[12pt,a4paper]{article}
\usepackage[UTF8]{ctex}
\usepackage{amsmath}
\usepackage{amssymb}
\usepackage{amsthm}
\usepackage{geometry}
\usepackage{hyperref}
\usepackage{enumitem}
\usepackage{booktabs}

\geometry{left=2.5cm,right=2.5cm,top=2.5cm,bottom=2.5cm}

\title{Q-learning通俗讲解}
\subtitle{用生活例子理解Off-policy TD控制}
\author{强化学习笔记}
\date{\today}

\begin{document}

\maketitle

\tableofcontents
\newpage

\section{引言}

Q-learning是强化学习中最重要和最常用的算法之一。本文用通俗易懂的方式,通过生活化的例子来解释Q-learning的核心概念和工作原理。

\section{什么是Q-learning?}

\subsection{通俗理解}

\textbf{Q-learning就像}:

\begin{quote}
\textbf{"看别人怎么做,但学习的是最优方式,即使别人没选最优动作,你也学习最优动作的价值"}
\end{quote}

\textbf{核心特点}:
\begin{itemize}
    \item \textbf{Off-policy}:可以使用任何策略收集数据,但学习的是最优策略
    \item \textbf{使用max操作}:学习最优动作的价值,而不是实际选择的动作的价值
    \item \textbf{直接学习最优策略}:不需要显式策略改进步骤
\end{itemize}

\subsection{与SARSA的对比}

\textbf{SARSA(On-policy)}:
\begin{quote}
"你实际怎么做,就学习这个方式的好坏"
\end{quote}

\textbf{Q-learning(Off-policy)}:
\begin{quote}
"看别人怎么做,但学习的是最优方式的好坏"
\end{quote}

\section{核心概念:Off-policy}

\subsection{行为策略 vs 目标策略}

\textbf{行为策略}(Behavior Policy)$b$:
\begin{itemize}
    \item 用于生成数据(选择动作与环境交互)的策略
    \item 通常是探索性的(如 $\varepsilon$-贪婪策略)
    \item 决定你\textbf{实际怎么做}
\end{itemize}

\textbf{目标策略}(Target Policy)$\pi_*$:
\begin{itemize}
    \item 正在学习的策略(最优策略)
    \item 决定你\textbf{想要学会的方式}
    \item 通常是贪婪策略
\end{itemize}

\textbf{Off-policy条件}:
\begin{equation}
b \neq \pi_*
\end{equation}

\textbf{通俗理解}:
\begin{itemize}
    \item 你实际怎么做 $\neq$ 你想要学会的方式
    \item 你可以用任何方式收集数据,但学习的是最优方式
\end{itemize}

\subsection{生活例子:看别人做菜来学习}

\textbf{场景}:你观察别人做菜,学习最优做菜方式。

\textbf{行为策略}:别人怎么做菜(如"放两勺盐")
\begin{itemize}
    \item 别人实际选择了"放两勺盐"
    \item 这是别人使用的策略
\end{itemize}

\textbf{目标策略}:你想要学会的最优做菜方式(如"放一勺半盐")
\begin{itemize}
    \item 你想要学会"应该放多少盐"(最优策略)
    \item 可能最优是"放一勺半盐",而不是"两勺"
\end{itemize}

\textbf{Q-learning的特点}:
\begin{itemize}
    \item 你观察别人放了两勺盐(行为策略选择)
    \item 但你学习的是"最优盐量"(可能是"一勺半")的价值
    \item 即使别人放了两勺,你学习的是最优盐量的价值
\end{itemize}

\section{Q-learning更新公式}

\subsection{公式}

\textbf{Q-learning更新公式}:
\begin{equation}
Q(S_t, A_t) \gets Q(S_t, A_t) + \alpha [R_{t+1} + \gamma \max_{a} Q(S_{t+1}, a) - Q(S_t, A_t)]
\label{eq:qlearning}
\end{equation}

\textbf{关键组成部分}:
\begin{itemize}
    \item $Q(S_t, A_t)$:当前状态-动作对的价值估计
    \item $R_{t+1}$:立即奖励
    \item $\max_{a} Q(S_{t+1}, a)$:下一状态所有动作中的最大价值(\textbf{关键!})
    \item $\alpha$:学习率
    \item $\gamma$:折扣因子
\end{itemize}

\subsection{关键点:使用max操作}

\textbf{关键区别}:

\textbf{SARSA}:
\begin{equation}
Q(S_t, A_t) \gets Q(S_t, A_t) + \alpha [R_{t+1} + \gamma Q(S_{t+1}, A_{t+1}) - Q(S_t, A_t)]
\end{equation}
\begin{itemize}
    \item 使用实际选择的动作 $A_{t+1}$ 的价值
    \item $A_{t+1}$ 由当前策略选择
\end{itemize}

\textbf{Q-learning}:
\begin{equation}
Q(S_t, A_t) \gets Q(S_t, A_t) + \alpha [R_{t+1} + \gamma \max_{a} Q(S_{t+1}, a) - Q(S_t, A_t)]
\end{equation}
\begin{itemize}
    \item 使用所有动作中的最大价值 $\max_{a} Q(S_{t+1}, a)$
    \item 不关心实际选择了什么动作
    \item 学习的是最优动作的价值
\end{itemize}

\subsection{通俗解释}

\textbf{类比:学做菜}

\textbf{场景}:在状态"菜还没放盐",你执行动作"放一勺盐",转移到状态"菜已经放了一勺盐"。

\textbf{假设}:
\begin{itemize}
    \item $Q(\text{菜还没放盐}, \text{放一勺盐}) = 5$
    \item $Q(\text{菜已经放了一勺盐}, \text{放一勺盐}) = 3$
    \item $Q(\text{菜已经放了一勺盐}, \text{放两勺盐}) = 8$(最优)
    \item $Q(\text{菜已经放了一勺盐}, \text{放三勺盐}) = 2$
    \item 实际选择的动作:$A_{t+1} = \text{放一勺盐}$(行为策略选择,用于探索)
\end{itemize}

\textbf{SARSA更新}:
\begin{itemize}
    \item 使用实际选择的动作"放一勺盐"的价值:$Q(S_{t+1}, A_{t+1}) = 3$
    \item 更新:$Q(\text{菜还没放盐}, \text{放一勺盐}) \gets 5 + \alpha [R + 0.9 \times 3 - 5]$
    \item 学习的是"实际选择的动作"的价值
\end{itemize}

\textbf{Q-learning更新}:
\begin{itemize}
    \item 使用所有动作中的最大价值:$\max_{a} Q(S_{t+1}, a) = \max\{3, 8, 2\} = 8$
    \item 更新:$Q(\text{菜还没放盐}, \text{放一勺盐}) \gets 5 + \alpha [R + 0.9 \times 8 - 5]$
    \item 学习的是"最优动作"(放两勺盐)的价值,即使实际选择了"放一勺盐"
\end{itemize}

\section{为什么使用max操作?}

\subsection{理论原因}

\textbf{目标}:学习最优动作价值函数 $q_*(s, a)$

\textbf{最优动作价值函数的贝尔曼方程}:
\begin{equation}
q_*(s, a) = \sum_{s', r} p(s', r | s, a) \left[r + \gamma \max_{a'} q_*(s', a')\right]
\end{equation}

\textbf{关键}:
\begin{itemize}
    \item 最优动作价值函数使用 $\max_{a'} q_*(s', a')$
    \item Q-learning直接学习最优动作价值函数
    \item 因此必须使用 $\max$ 操作
\end{itemize}

\subsection{通俗原因}

\textbf{类比:学做菜}

\textbf{问题}:为什么即使别人选择了"放一勺盐",你学习的是"放两勺盐"的价值?

\textbf{答案}:
\begin{itemize}
    \item 你的目标是学会"最优做菜方式"
    \item 即使别人选择了次优动作,你也要学习最优动作的价值
    \item 这样你才能知道"最优方式"是什么
    \item 如果学习实际选择的动作,你学到的可能是次优方式
\end{itemize}

\textbf{例子}:
\begin{itemize}
    \item 别人选择了"放一勺盐"(次优,价值3)
    \item 但最优是"放两勺盐"(价值8)
    \item 如果你学习"放一勺盐"的价值,你学到的就是次优方式
    \item 如果你学习"放两勺盐"的价值,你学到的就是最优方式
    \item 所以Q-learning使用 $\max$ 操作,学习最优动作的价值
\end{itemize}

\section{Q-learning算法流程}

\subsection{完整算法}

\textbf{初始化}:
\begin{itemize}
    \item 初始化 $Q(s, a)$(可以任意,通常设为0)
    \item 初始化行为策略 $b$(如 $\varepsilon$-贪婪策略)
\end{itemize}

\textbf{每个时间步}:
\begin{enumerate}
    \item 在状态 $S_t$,使用行为策略 $b$ 选择动作 $A_t$
    \begin{equation}
    A_t \sim b(\cdot|S_t)
    \end{equation}
    
    \item 执行动作 $A_t$,观察 $R_{t+1}, S_{t+1}$
    
    \item 在状态 $S_{t+1}$,使用行为策略 $b$ 选择动作 $A_{t+1}$(用于下一步,但更新时不使用)
    \begin{equation}
    A_{t+1} \sim b(\cdot|S_{t+1})
    \end{equation}
    
    \item 更新动作价值函数:
    \begin{equation}
    Q(S_t, A_t) \gets Q(S_t, A_t) + \alpha [R_{t+1} + \gamma \max_{a} Q(S_{t+1}, a) - Q(S_t, A_t)]
    \end{equation}
    
    \item \textbf{关键}:更新时使用 $\max_{a} Q(S_{t+1}, a)$,而不是 $Q(S_{t+1}, A_{t+1})$
    
    \item 更新行为策略 $b$(基于新的 $Q$,如 $\varepsilon$-贪婪策略)
\end{enumerate}

\textbf{提取最优策略}:
\begin{equation}
\pi_*(s) = \arg\max_{a} Q(s, a)
\end{equation}

\subsection{通俗例子:学做菜}

\textbf{场景}:学习如何做一道菜,需要决定放多少盐。

\textbf{状态}:
\begin{itemize}
    \item $s_1$:菜还没放盐
    \item $s_2$:菜已经放了一勺盐
    \item $s_3$:菜已经放了两勺盐
\end{itemize}

\textbf{动作}:
\begin{itemize}
    \item $a_1$:放一勺盐
    \item $a_2$:放两勺盐
    \item $a_3$:放三勺盐
\end{itemize}

\textbf{第1次做菜}:
\begin{enumerate}
    \item 在状态 $s_1$,使用行为策略 $b$($\varepsilon$-贪婪)选择动作
    \item 你选择了 $a_2$(放两勺盐,用于探索)
    \item 执行动作,转移到状态 $s_3$,获得奖励 $R = 2$
    \item 在状态 $s_3$,使用行为策略 $b$ 选择动作 $a_1$(放一勺盐,实际选择的动作)
    \item 更新:
    \begin{align}
    Q(s_1, a_2) &\gets Q(s_1, a_2) + \alpha [R + \gamma \max_{a} Q(s_3, a) - Q(s_1, a_2)] \\
                &\gets 0 + 0.1 [2 + 0.9 \times \max\{Q(s_3, a_1), Q(s_3, a_2), Q(s_3, a_3)\} - 0]
    \end{align}
    \item 假设 $\max_{a} Q(s_3, a) = 5$(最优动作的价值)
    \item 更新:$Q(s_1, a_2) \gets 0 + 0.1 [2 + 0.9 \times 5 - 0] = 0.65$
    \item \textbf{关键}:即使实际选择了 $a_1$,你学习的是最优动作的价值(5)
\end{enumerate}

\textbf{第2次做菜}:
\begin{enumerate}
    \item 在状态 $s_1$,使用行为策略 $b$ 选择动作
    \item 你选择了 $a_1$(放一勺盐)
    \item 执行动作,转移到状态 $s_2$,获得奖励 $R = 1$
    \item 在状态 $s_2$,使用行为策略 $b$ 选择动作 $a_3$(放三勺盐,实际选择的动作)
    \item 更新:
    \begin{align}
    Q(s_1, a_1) &\gets Q(s_1, a_1) + \alpha [R + \gamma \max_{a} Q(s_2, a) - Q(s_1, a_1)] \\
                &\gets 0 + 0.1 [1 + 0.9 \times \max_{a} Q(s_2, a) - 0]
    \end{align}
    \item 假设 $\max_{a} Q(s_2, a) = 4$(最优动作的价值)
    \item 更新:$Q(s_1, a_1) \gets 0 + 0.1 [1 + 0.9 \times 4 - 0] = 0.46$
    \item \textbf{关键}:即使实际选择了 $a_3$,你学习的是最优动作的价值(4)
\end{enumerate}

\textbf{继续学习}:
\begin{itemize}
    \item 你继续做菜,根据结果更新 $Q$ 值
    \item 每次更新都使用最优动作的价值,而不是实际选择的动作的价值
    \item 最终,你学会了最优做菜方式
\end{itemize}

\section{Q-learning vs SARSA}

\subsection{核心区别}

\begin{center}
\begin{tabular}{|l|l|l|}
\hline
\textbf{特征} & \textbf{SARSA} & \textbf{Q-learning} \\
\hline
\textbf{策略类型} & On-policy & Off-policy \\
\hline
\textbf{行为策略} & 当前策略 $\pi$ & 行为策略 $b$($\varepsilon$-贪婪) \\
\hline
\textbf{目标策略} & 当前策略 $\pi$ & 最优策略 $\pi_*$ \\
\hline
\textbf{更新目标} & $Q(S_{t+1}, A_{t+1})$ & $\max_{a} Q(S_{t+1}, a)$ \\
\hline
\textbf{使用实际动作} & 是 & 否 \\
\hline
\textbf{学习目标} & 当前策略的价值函数 & 最优策略的价值函数 \\
\hline
\textbf{探索风险} & 考虑(保守) & 不考虑(激进) \\
\hline
\end{tabular}
\end{center}

\subsection{通俗对比}

\textbf{SARSA:边学边用}
\begin{itemize}
    \item 你实际怎么做,就学习这个方式的好坏
    \item 你实际选择了"放一勺盐",就学习"放一勺盐"的价值
    \item 你考虑探索的风险(如果探索导致不好的结果,你会学到)
    \item 你学习的是"你当前怎么做"的价值
\end{itemize}

\textbf{Q-learning:看别人学最优}
\begin{itemize}
    \item 你观察别人怎么做,但学习的是最优方式的好坏
    \item 即使别人选择了"放一勺盐",你学习的是"最优盐量"的价值
    \item 你不考虑探索的风险(总是学习最优动作的价值)
    \item 你学习的是"最优方式"的价值
\end{itemize}

\subsection{具体例子对比}

\textbf{场景}:在Gridworld中,智能体在状态 $s_5$,执行动作"上",转移到状态 $s_2$。

\textbf{假设}:
\begin{itemize}
    \item $Q(s_2, \text{上}) = 0$
    \item $Q(s_2, \text{下}) = 0$
    \item $Q(s_2, \text{左}) = -0.1$
    \item $Q(s_2, \text{右}) = 0$
    \item 实际选择的动作:$A_{t+1} = \text{左}$(行为策略选择,用于探索)
\end{itemize}

\textbf{SARSA更新}:
\begin{equation}
Q(s_5, \text{上}) \gets Q(s_5, \text{上}) + \alpha [R + \gamma Q(s_2, \text{左}) - Q(s_5, \text{上})]
\end{equation}
\begin{itemize}
    \item 使用实际选择的动作"左"的价值:$Q(s_2, \text{左}) = -0.1$
    \item 学习的是"实际选择的动作"的价值
    \item 如果探索导致不好的结果,SARSA会学到这个风险
\end{itemize}

\textbf{Q-learning更新}:
\begin{equation}
Q(s_5, \text{上}) \gets Q(s_5, \text{上}) + \alpha [R + \gamma \max_{a} Q(s_2, a) - Q(s_5, \text{上})]
\end{equation}
\begin{itemize}
    \item 使用所有动作中的最大价值:$\max_{a} Q(s_2, a) = \max\{0, 0, -0.1, 0\} = 0$
    \item 学习的是"最优动作"(上、下、右)的价值
    \item 即使实际选择了"左"(不好的动作),也学习最优动作的价值
    \item 不考虑探索的风险
\end{itemize}

\section{Q-learning的优势}

\subsection{理论优势}

\textbf{1. 直接学习最优策略}
\begin{itemize}
    \item Q-learning直接学习最优动作价值函数 $q_*(s, a)$
    \item 不需要显式策略改进步骤
    \item 最终策略就是最优策略
\end{itemize}

\textbf{2. Off-policy学习}
\begin{itemize}
    \item 可以使用任何策略收集数据
    \item 可以从历史数据中学习
    \item 样本效率高
\end{itemize}

\textbf{3. 简单高效}
\begin{itemize}
    \item 算法简单,易于实现
    \item 更新规则清晰
    \item 收敛性好
\end{itemize}

\subsection{实际优势}

\textbf{1. 可以重用历史数据}
\begin{itemize}
    \item 可以使用之前收集的数据
    \item 不需要重新收集数据
    \item 样本效率高
\end{itemize}

\textbf{2. 可以并行学习}
\begin{itemize}
    \item 多个智能体可以共享经验
    \item 可以从其他智能体的经验中学习
    \item 加速学习过程
\end{itemize}

\textbf{3. 适合在线学习}
\begin{itemize}
    \item 可以立即更新
    \item 不需要等待回合结束
    \item 适合实时应用
\end{itemize}

\section{Q-learning的局限性}

\subsection{理论局限性}

\textbf{1. 不考虑探索风险}
\begin{itemize}
    \item Q-learning总是学习最优动作的价值
    \item 如果环境有风险(如悬崖),Q-learning可能不够保守
    \item SARSA在这种情况下可能更好
\end{itemize}

\textbf{2. 需要充分探索}
\begin{itemize}
    \item 需要探索所有状态-动作对
    \item 如果探索不足,可能无法找到最优策略
    \item 需要合适的探索策略(如 $\varepsilon$-贪婪)
\end{itemize}

\textbf{3. 函数逼近的挑战}
\begin{itemize}
    \item 在大状态空间中,需要使用函数逼近
    \item 函数逼近可能导致不收敛
    \item 需要特殊的技术(如经验回放、目标网络)
\end{itemize}

\subsection{实际局限性}

\textbf{1. 高估问题}
\begin{itemize}
    \item 使用 $\max$ 操作可能导致高估
    \item 需要技术来减少高估(如Double Q-learning)
\end{itemize}

\textbf{2. 样本效率}
\begin{itemize}
    \item 在某些情况下,样本效率可能不如On-policy方法
    \item 需要大量数据才能收敛
\end{itemize}

\section{总结}

\subsection{核心要点(通俗版)}

\begin{enumerate}
    \item \textbf{Q-learning}:看别人怎么做,但学习的是最优方式的好坏
    
    \item \textbf{Off-policy}:你实际怎么做 $\neq$ 你想要学会的方式
    
    \item \textbf{使用max操作}:学习最优动作的价值,而不是实际选择的动作的价值
    
    \item \textbf{直接学习最优策略}:不需要显式策略改进步骤
    
    \item \textbf{可以重用数据}:可以使用历史数据,样本效率高
\end{enumerate}

\subsection{关键公式(简化版)}

\textbf{Q-learning更新}:
\begin{equation}
\text{新价值} = \text{旧价值} + \alpha \times [\text{立即奖励} + \gamma \times \text{最优未来价值} - \text{旧价值}]
\end{equation}

\textbf{关键}:
\begin{equation}
\text{最优未来价值} = \max_{a} Q(\text{下一状态}, a)
\end{equation}

\textbf{提取最优策略}:
\begin{equation}
\text{最优动作} = \arg\max_{a} Q(\text{当前状态}, a)
\end{equation}

\subsection{记忆技巧}

\begin{itemize}
    \item \textbf{Q-learning}:看别人学,但学最优的
    \item \textbf{使用max}:总是学最优动作的价值
    \item \textbf{Off-policy}:实际做的 $\neq$ 想学的
    \item \textbf{直接最优}:不需要策略改进,直接学最优策略
    \item \textbf{可以重用}:可以用历史数据,效率高
\end{itemize}

\subsection{适用场景}

\textbf{Q-learning适合}:
\begin{itemize}
    \item 需要学习最优策略的问题
    \item 可以重用历史数据的情况
    \item 环境相对安全,探索风险不大的情况
    \item 需要在线学习的情况
\end{itemize}

\textbf{Q-learning不适合}:
\begin{itemize}
    \item 环境有高风险的情况(如悬崖问题)
    \item 需要保守策略的情况
    \item 探索成本很高的情况
\end{itemize}

\end{document}

