\documentclass[12pt,a4paper]{article}
\usepackage[UTF8]{ctex}
\usepackage{amsmath}
\usepackage{amssymb}
\usepackage{amsthm}
\usepackage{geometry}
\usepackage{hyperref}
\usepackage{enumitem}
\usepackage{xcolor}
\usepackage{bm}
\usepackage{booktabs}
\usepackage{array}

\geometry{margin=2.5cm}

\title{回报与价值函数关系详解}
\subtitle{$G_t$、$G_{t+1}$ 和 $v_\pi(S_{t+1})$ 的关系}
\author{}
\date{}

\newtheorem{definition}{定义}
\newtheorem{theorem}{定理}
\newtheorem{proposition}{命题}
\newtheorem{example}{示例}
\newtheorem{remark}{注记}

\begin{document}

\maketitle

\tableofcontents
\newpage

\section{引言}

在强化学习中,回报(Return)$G_t$ 和价值函数 $v_\pi(s)$ 是两个核心概念。理解它们之间的关系对于理解时序差分学习、蒙特卡洛方法等算法至关重要。

\textbf{核心问题}:
\begin{itemize}
    \item $G_t$ 和 $v_\pi(S_{t+1})$ 是什么关系?
    \item 为什么可以用 $v_\pi(S_{t+1})$ 来估计 $G_{t+1}$?
    \item 这种关系如何体现在TD学习中?
\end{itemize}

\section{基本定义}

\subsection{回报(Return)$G_t$}

\begin{definition}[回报]
从时刻 $t$ 开始的\textbf{折扣回报}定义为:
\begin{equation}
G_t = R_{t+1} + \gamma R_{t+2} + \gamma^2 R_{t+3} + \cdots = \sum_{k=0}^{\infty} \gamma^k R_{t+k+1}
\label{eq:return}
\end{equation}
其中 $\gamma \in [0, 1]$ 是折扣因子。
\end{definition}

\textbf{关键特性}:
\begin{itemize}
    \item $G_t$ 是一个\textbf{随机变量},依赖于从时刻 $t$ 开始的完整轨迹
    \item $G_t$ 的值只有在回合结束后才能确定
    \item $G_t$ 是未来所有奖励的加权和
\end{itemize}

\subsection{下一时刻的回报 $G_{t+1}$}

\begin{definition}[下一时刻的回报]
从时刻 $t+1$ 开始的\textbf{折扣回报}定义为:
\begin{equation}
G_{t+1} = R_{t+2} + \gamma R_{t+3} + \gamma^2 R_{t+4} + \cdots = \sum_{k=0}^{\infty} \gamma^k R_{t+1+k+1} = \sum_{k=0}^{\infty} \gamma^k R_{t+k+2}
\label{eq:return_next}
\end{equation}
\end{definition}

\textbf{关键特性}:
\begin{itemize}
    \item $G_{t+1}$ 是从时刻 $t+1$ 开始的回报
    \item $G_{t+1}$ 也是一个随机变量,依赖于从时刻 $t+1$ 开始的轨迹
    \item $G_{t+1}$ 的值只有在回合结束后才能确定
\end{itemize}

\subsection{回报的递归关系}

\textbf{重要关系}:
\begin{equation}
G_t = R_{t+1} + \gamma G_{t+1}
\label{eq:return_recursive}
\end{equation}

\textbf{推导}:
\begin{align}
G_t &= R_{t+1} + \gamma R_{t+2} + \gamma^2 R_{t+3} + \gamma^3 R_{t+4} + \cdots \\
    &= R_{t+1} + \gamma (R_{t+2} + \gamma R_{t+3} + \gamma^2 R_{t+4} + \cdots) \\
    &= R_{t+1} + \gamma G_{t+1}
\end{align}

\textbf{解释}:
\begin{itemize}
    \item 时刻 $t$ 的回报 = 即时奖励 $R_{t+1}$ + 折扣后的未来回报 $\gamma G_{t+1}$
    \item 这是回报的\textbf{递归定义}
    \item 这个关系是贝尔曼方程的基础
\end{itemize}

\subsection{状态价值函数 $v_\pi(s)$}

\begin{definition}[状态价值函数]
在策略 $\pi$ 下,状态 $s$ 的\textbf{价值函数}定义为回报的期望:
\begin{equation}
v_\pi(s) = \mathbb{E}_\pi[G_t | S_t = s]
\label{eq:value_function}
\end{equation}
\end{definition}

\textbf{关键特性}:
\begin{itemize}
    \item $v_\pi(s)$ 是一个\textbf{确定的值}(期望值),不是随机变量
    \item $v_\pi(s)$ 表示在策略 $\pi$ 下,从状态 $s$ 开始的\textbf{期望回报}
    \item $v_\pi(s)$ 是 $G_t$ 在给定 $S_t = s$ 条件下的期望值
\end{itemize}

\section{$G_t$ 和 $v_\pi(S_{t+1})$ 的关系}

\subsection{核心关系}

\textbf{关键洞察}:
\begin{quote}
$v_\pi(S_{t+1})$ 是 $G_{t+1}$ 的\textbf{期望值}(在给定 $S_{t+1}$ 的条件下)。
\end{quote}

\textbf{数学表达}:
\begin{equation}
v_\pi(S_{t+1}) = \mathbb{E}_\pi[G_{t+1} | S_{t+1}]
\label{eq:core_relation}
\end{equation}

\textbf{解释}:
\begin{itemize}
    \item $G_{t+1}$ 是从时刻 $t+1$ 开始的回报(随机变量)
    \item $v_\pi(S_{t+1})$ 是在策略 $\pi$ 下,从状态 $S_{t+1}$ 开始的期望回报
    \item 因此,$v_\pi(S_{t+1})$ 是 $G_{t+1}$ 的条件期望
\end{itemize}

\subsection{为什么这个关系重要?}

\textbf{原因1:贝尔曼方程的基础}:

从回报的递归关系 $G_t = R_{t+1} + \gamma G_{t+1}$ 出发,对两边取期望:

\begin{align}
v_\pi(S_t) &= \mathbb{E}_\pi[G_t | S_t] \\
           &= \mathbb{E}_\pi[R_{t+1} + \gamma G_{t+1} | S_t] \\
           &= \mathbb{E}_\pi[R_{t+1} | S_t] + \gamma \mathbb{E}_\pi[G_{t+1} | S_t]
\end{align}

\textbf{关键步骤}:使用全期望公式展开 $\mathbb{E}_\pi[G_{t+1} | S_t]$:

\begin{align}
\mathbb{E}_\pi[G_{t+1} | S_t] &= \mathbb{E}_\pi[\mathbb{E}_\pi[G_{t+1} | S_t, S_{t+1}] | S_t] \\
                              &= \mathbb{E}_\pi[v_\pi(S_{t+1}) | S_t] \\
                              &= \sum_{s'} \Pr\{S_{t+1} = s' | S_t\} \cdot v_\pi(s')
\end{align}

因此:
\begin{equation}
v_\pi(S_t) = \mathbb{E}_\pi[R_{t+1} | S_t] + \gamma \mathbb{E}_\pi[v_\pi(S_{t+1}) | S_t]
\end{equation}

这就是\textbf{贝尔曼方程}的雏形。

\textbf{原因2:TD学习的基础}:

在TD学习中,我们使用:
\begin{equation}
R_{t+1} + \gamma V(S_{t+1}) \approx R_{t+1} + \gamma v_\pi(S_{t+1}) \approx R_{t+1} + \gamma G_{t+1} = G_t
\end{equation}

\textbf{解释}:
\begin{itemize}
    \item $G_t = R_{t+1} + \gamma G_{t+1}$(回报的递归关系)
    \item $v_\pi(S_{t+1}) = \mathbb{E}_\pi[G_{t+1} | S_{t+1}]$(价值函数的定义)
    \item 因此,$R_{t+1} + \gamma v_\pi(S_{t+1})$ 是 $G_t$ 的一个估计
\end{itemize}

\section{详细推导}

\subsection{从定义出发}

\textbf{步骤1:回报的递归关系}

\begin{align}
G_t &= R_{t+1} + \gamma R_{t+2} + \gamma^2 R_{t+3} + \cdots \\
    &= R_{t+1} + \gamma (R_{t+2} + \gamma R_{t+3} + \cdots) \\
    &= R_{t+1} + \gamma G_{t+1}
\end{align}

\textbf{步骤2:对两边取期望}

\begin{align}
\mathbb{E}_\pi[G_t | S_t = s] &= \mathbb{E}_\pi[R_{t+1} + \gamma G_{t+1} | S_t = s] \\
v_\pi(s) &= \mathbb{E}_\pi[R_{t+1} | S_t = s] + \gamma \mathbb{E}_\pi[G_{t+1} | S_t = s]
\end{align}

\textbf{步骤3:展开 $\mathbb{E}_\pi[G_{t+1} | S_t = s]$}

使用全期望公式:
\begin{align}
\mathbb{E}_\pi[G_{t+1} | S_t = s] &= \mathbb{E}_\pi[\mathbb{E}_\pi[G_{t+1} | S_t = s, S_{t+1}] | S_t = s] \\
                                  &= \sum_{s'} \Pr\{S_{t+1} = s' | S_t = s\} \cdot \mathbb{E}_\pi[G_{t+1} | S_t = s, S_{t+1} = s']
\end{align}

\textbf{步骤4:利用马尔可夫性质}

由于马尔可夫性质,给定 $S_{t+1} = s'$,$G_{t+1}$ 的分布只依赖于 $s'$ 和后续的策略选择,不依赖于 $S_t$:

\begin{align}
\mathbb{E}_\pi[G_{t+1} | S_t = s, S_{t+1} = s'] &= \mathbb{E}_\pi[G_{t+1} | S_{t+1} = s'] \\
                                                &= v_\pi(s')
\end{align}

因此:
\begin{align}
\mathbb{E}_\pi[G_{t+1} | S_t = s] &= \sum_{s'} \Pr\{S_{t+1} = s' | S_t = s\} \cdot v_\pi(s') \\
                                  &= \mathbb{E}_\pi[v_\pi(S_{t+1}) | S_t = s]
\end{align}

\textbf{步骤5:最终结果}

\begin{align}
v_\pi(s) &= \mathbb{E}_\pi[R_{t+1} | S_t = s] + \gamma \mathbb{E}_\pi[v_\pi(S_{t+1}) | S_t = s] \\
         &= \sum_{a} \pi(a | s) \sum_{s', r} p(s', r | s, a) [r + \gamma v_\pi(s')]
\end{align}

这就是\textbf{状态价值函数的贝尔曼方程}。

\subsection{关键关系总结}

\textbf{关系1:回报的递归关系}
\begin{equation}
G_t = R_{t+1} + \gamma G_{t+1}
\end{equation}

\textbf{关系2:价值函数的定义}
\begin{equation}
v_\pi(S_{t+1}) = \mathbb{E}_\pi[G_{t+1} | S_{t+1}]
\end{equation}

\textbf{关系3:贝尔曼方程}
\begin{equation}
v_\pi(S_t) = \mathbb{E}_\pi[R_{t+1} + \gamma v_\pi(S_{t+1}) | S_t]
\end{equation}

\textbf{关系4:TD目标}
\begin{equation}
R_{t+1} + \gamma v_\pi(S_{t+1}) \approx G_t
\end{equation}

\section{在TD学习中的应用}

\subsection{TD(0)更新}

\textbf{蒙特卡洛更新}:
\begin{equation}
V(S_t) \gets V(S_t) + \alpha [G_t - V(S_t)]
\end{equation}

\textbf{TD(0)更新}:
\begin{equation}
V(S_t) \gets V(S_t) + \alpha [R_{t+1} + \gamma V(S_{t+1}) - V(S_t)]
\end{equation}

\textbf{为什么可以用 $R_{t+1} + \gamma V(S_{t+1})$ 代替 $G_t$?}

\textbf{原因}:
\begin{itemize}
    \item $G_t = R_{t+1} + \gamma G_{t+1}$(回报的递归关系)
    \item $v_\pi(S_{t+1}) = \mathbb{E}_\pi[G_{t+1} | S_{t+1}]$(价值函数的定义)
    \item 如果 $V(S_{t+1}) \approx v_\pi(S_{t+1})$,那么:
    \begin{align}
    R_{t+1} + \gamma V(S_{t+1}) &\approx R_{t+1} + \gamma v_\pi(S_{t+1}) \\
                                &\approx R_{t+1} + \gamma \mathbb{E}_\pi[G_{t+1} | S_{t+1}] \\
                                &\approx \mathbb{E}_\pi[R_{t+1} + \gamma G_{t+1} | S_t, S_{t+1}] \\
                                &\approx \mathbb{E}_\pi[G_t | S_t, S_{t+1}]
    \end{align}
    \item 因此,$R_{t+1} + \gamma V(S_{t+1})$ 是 $G_t$ 的一个估计
\end{itemize}

\subsection{TD误差}

\textbf{TD误差}:
\begin{equation}
\delta_t = R_{t+1} + \gamma V(S_{t+1}) - V(S_t)
\end{equation}

\textbf{解释}:
\begin{itemize}
    \item $\delta_t$ 衡量当前估计 $V(S_t)$ 与更好估计 $R_{t+1} + \gamma V(S_{t+1})$ 之间的差异
    \item 如果 $V(S_{t+1}) = v_\pi(S_{t+1})$,那么:
    \begin{align}
    \mathbb{E}[\delta_t | S_t = s] &= \mathbb{E}[R_{t+1} + \gamma v_\pi(S_{t+1}) - V(S_t) | S_t = s] \\
                                   &= \mathbb{E}[R_{t+1} + \gamma v_\pi(S_{t+1}) | S_t = s] - V(S_t) \\
                                   &= v_\pi(s) - V(S_t)
    \end{align}
    \item 因此,TD误差的期望等于真实价值与当前估计的差异
\end{itemize}

\section{具体例子}

\subsection{Gridworld例子}

\textbf{设置}:
\begin{itemize}
    \item 状态:$s_5$(中心状态)
    \item 执行动作"上",观察到:$R_1 = -1$,$S_1 = s_2$
    \item 折扣因子:$\gamma = 0.9$
\end{itemize}

\textbf{完整回报 $G_0$}:

假设从 $s_5$ 开始的完整轨迹是:
\begin{align}
S_0 &= s_5, \quad A_0 = \text{上}, \quad R_1 = -1, \quad S_1 = s_2 \\
S_1 &= s_2, \quad A_1 = \text{右}, \quad R_2 = -1, \quad S_2 = s_3 \\
S_2 &= s_3, \quad A_2 = \text{下}, \quad R_3 = -1, \quad S_3 = \text{终止}
\end{align}

那么:
\begin{align}
G_0 &= R_1 + \gamma R_2 + \gamma^2 R_3 \\
    &= -1 + 0.9 \times (-1) + 0.9^2 \times (-1) \\
    &= -1 - 0.9 - 0.81 = -2.71
\end{align}

\textbf{下一时刻的回报 $G_1$}:

从 $s_2$ 开始的回报:
\begin{align}
G_1 &= R_2 + \gamma R_3 \\
    &= -1 + 0.9 \times (-1) = -1.9
\end{align}

\textbf{回报的递归关系验证}:

\begin{align}
G_0 &= R_1 + \gamma G_1 \\
    &= -1 + 0.9 \times (-1.9) \\
    &= -1 - 1.71 = -2.71 \quad \checkmark
\end{align}

\textbf{价值函数 $v_\pi(s_2)$}:

假设 $v_\pi(s_2) = -1.8$(这是从 $s_2$ 开始的期望回报)。

\textbf{关系验证}:

\begin{itemize}
    \item $G_1 = -1.9$(这是从 $s_2$ 开始的一个具体轨迹的回报)
    \item $v_\pi(s_2) = -1.8 = \mathbb{E}_\pi[G_1 | S_1 = s_2]$(这是所有可能轨迹的期望回报)
    \item 因此,$v_\pi(s_2)$ 是 $G_1$ 的期望值
\end{itemize}

\textbf{TD目标}:

\begin{align}
R_1 + \gamma v_\pi(S_1) &= -1 + 0.9 \times (-1.8) \\
                        &= -1 - 1.62 = -2.62
\end{align}

\textbf{对比}:
\begin{itemize}
    \item $G_0 = -2.71$(完整回报,具体轨迹)
    \item $R_1 + \gamma v_\pi(S_1) = -2.62$(TD目标,期望值)
    \item 差异:$-2.62 - (-2.71) = 0.09$(由于 $v_\pi(s_2)$ 是期望值,而 $G_1$ 是具体值)
\end{itemize}

\section{总结}

\subsection{核心关系}

\begin{enumerate}
    \item \textbf{回报的递归关系}:
    \begin{equation}
    G_t = R_{t+1} + \gamma G_{t+1}
    \end{equation}
    
    \item \textbf{价值函数的定义}:
    \begin{equation}
    v_\pi(S_{t+1}) = \mathbb{E}_\pi[G_{t+1} | S_{t+1}]
    \end{equation}
    
    \item \textbf{贝尔曼方程}:
    \begin{equation}
    v_\pi(S_t) = \mathbb{E}_\pi[R_{t+1} + \gamma v_\pi(S_{t+1}) | S_t]
    \end{equation}
    
    \item \textbf{TD目标}:
    \begin{equation}
    R_{t+1} + \gamma v_\pi(S_{t+1}) \approx G_t
    \end{equation}
\end{enumerate}

\subsection{关键洞察}

\begin{quote}
\textbf{$v_\pi(S_{t+1})$ 是 $G_{t+1}$ 的期望值}。这个关系使得我们可以用价值函数来估计回报,从而在TD学习中实现一步更新,而不需要等待完整回合结束。
\end{quote}

\subsection{在算法中的应用}

\begin{itemize}
    \item \textbf{蒙特卡洛方法}:使用完整回报 $G_t$(需要等待回合结束)
    \item \textbf{TD方法}:使用 $R_{t+1} + \gamma V(S_{t+1})$(可以立即更新)
    \item \textbf{动态规划}:使用 $r + \gamma v_\pi(s')$(需要环境模型)
\end{itemize}

三种方法都基于同一个关系:$v_\pi(S_{t+1}) = \mathbb{E}_\pi[G_{t+1} | S_{t+1}]$,只是实现方式不同。

\end{document}

