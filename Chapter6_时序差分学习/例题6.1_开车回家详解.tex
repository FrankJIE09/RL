\documentclass[12pt,a4paper]{article}
\usepackage[UTF8]{ctex}
\usepackage{amsmath}
\usepackage{amssymb}
\usepackage{amsthm}
\usepackage{geometry}
\usepackage{hyperref}
\usepackage{enumitem}
\usepackage{xcolor}
\usepackage{bm}
\usepackage{booktabs}
\usepackage{array}
\usepackage{tabularx}

\geometry{margin=2.5cm}

\title{例题6.1:开车回家详解}
\subtitle{蒙特卡洛方法与TD方法的对比}
\author{}
\date{}

\newtheorem{definition}{定义}
\newtheorem{theorem}{定理}
\newtheorem{proposition}{命题}
\newtheorem{example}{示例}
\newtheorem{remark}{注记}

\begin{document}

\maketitle

\tableofcontents
\newpage

\section{问题描述}

\subsection{场景设置}

每天下班开车回家时,你试图预测到家需要多长时间。当你离开办公室时,你会注意时间、星期几、天气等相关因素。

\textbf{具体场景}:
\begin{itemize}
    \item 星期五下午6点离开办公室
    \item 初始估计:30分钟到家
    \item 实际到家时间:43分钟
\end{itemize}

\subsection{状态序列}

整个旅程的状态序列如下:

\begin{center}
\begin{tabular}{|l|c|c|c|}
\hline
\textbf{状态} & \textbf{已用时间(分钟)} & \textbf{预测剩余时间} & \textbf{预测总时间} \\
\hline
离开办公室,星期五6点 & 0 & 30 & 30 \\
\hline
到达汽车,开始下雨 & 5 & 35 & 40 \\
\hline
离开高速公路 & 20 & 15 & 35 \\
\hline
二级公路,跟在卡车后面 & 30 & 10 & 40 \\
\hline
进入家附近的街道 & 40 & 3 & 43 \\
\hline
到家 & 43 & 0 & 43 \\
\hline
\end{tabular}
\end{center}

\subsection{关键参数}

\begin{itemize}
    \item \textbf{奖励}:每段路程的耗时(已用时间)
    \item \textbf{折扣因子}:$\gamma = 1$(无折扣)
    \item \textbf{回报}:从每个状态到家的实际剩余时间
    \item \textbf{价值函数}:每个状态的期望剩余时间
\end{itemize}

\section{问题建模}

\subsection{状态定义}

\begin{itemize}
    \item $S_0$:离开办公室,星期五6点
    \item $S_1$:到达汽车,开始下雨
    \item $S_2$:离开高速公路
    \item $S_3$:二级公路,跟在卡车后面
    \item $S_4$:进入家附近的街道
    \item $S_5$:到家(终止状态)
\end{itemize}

\subsection{奖励序列}

\begin{itemize}
    \item $R_1 = 5$:从办公室到汽车(5分钟)
    \item $R_2 = 15$:从汽车到离开高速公路(15分钟)
    \item $R_3 = 10$:从高速公路到二级公路(10分钟)
    \item $R_4 = 10$:从二级公路到进入家附近街道(10分钟)
    \item $R_5 = 3$:从家附近街道到家(3分钟)
\end{itemize}

\subsection{初始价值估计}

\begin{itemize}
    \item $V(S_0) = 30$:初始估计30分钟
    \item $V(S_1) = 35$:到达汽车后估计35分钟
    \item $V(S_2) = 15$:离开高速公路后估计15分钟
    \item $V(S_3) = 10$:二级公路上估计10分钟
    \item $V(S_4) = 3$:进入家附近街道后估计3分钟
\end{itemize}

\subsection{实际回报}

由于 $\gamma = 1$,从每个状态开始的回报就是实际剩余时间:

\begin{itemize}
    \item $G_0 = R_1 + R_2 + R_3 + R_4 + R_5 = 5 + 15 + 10 + 10 + 3 = 43$
    \item $G_1 = R_2 + R_3 + R_4 + R_5 = 15 + 10 + 10 + 3 = 38$
    \item $G_2 = R_3 + R_4 + R_5 = 10 + 10 + 3 = 23$
    \item $G_3 = R_4 + R_5 = 10 + 3 = 13$
    \item $G_4 = R_5 = 3$
\end{itemize}

\section{蒙特卡洛方法}

\subsection{蒙特卡洛更新规则}

\textbf{更新公式}:
\begin{equation}
V(S_t) \gets V(S_t) + \alpha [G_t - V(S_t)]
\label{eq:mc_update}
\end{equation}

其中:
\begin{itemize}
    \item $G_t$ 是从状态 $S_t$ 开始的完整回报
    \item $\alpha$ 是步长参数
    \item 更新只能在回合结束后进行
\end{itemize}

\subsection{蒙特卡洛更新过程}

\textbf{假设 $\alpha = 1$}(完全更新):

\textbf{状态 $S_0$(离开办公室)}:
\begin{align}
V(S_0) &\gets V(S_0) + \alpha [G_0 - V(S_0)] \\
       &= 30 + 1 \times [43 - 30] = 43
\end{align}

\textbf{状态 $S_1$(到达汽车)}:
\begin{align}
V(S_1) &\gets V(S_1) + \alpha [G_1 - V(S_1)] \\
       &= 35 + 1 \times [38 - 35] = 38
\end{align}

\textbf{状态 $S_2$(离开高速公路)}:
\begin{align}
V(S_2) &\gets V(S_2) + \alpha [G_2 - V(S_2)] \\
       &= 15 + 1 \times [23 - 15] = 23
\end{align}

\textbf{状态 $S_3$(二级公路)}:
\begin{align}
V(S_3) &\gets V(S_3) + \alpha [G_3 - V(S_3)] \\
       &= 10 + 1 \times [13 - 10] = 13
\end{align}

\textbf{状态 $S_4$(进入家附近街道)}:
\begin{align}
V(S_4) &\gets V(S_4) + \alpha [G_4 - V(S_4)] \\
       &= 3 + 1 \times [3 - 3] = 3
\end{align}

\subsection{蒙特卡洛方法的特点}

\begin{itemize}
    \item \textbf{更新时机}:必须等待到家后才能更新所有状态
    \item \textbf{更新目标}:使用完整回报 $G_t$
    \item \textbf{误差}:$G_t - V(S_t)$ 是实际剩余时间与预测剩余时间的差异
    \item \textbf{缺点}:无法在途中学习,必须等待最终结果
\end{itemize}

\subsection{蒙特卡洛更新表格}

\begin{center}
\begin{tabular}{|l|c|c|c|c|}
\hline
\textbf{状态} & \textbf{初始估计} & \textbf{实际回报} & \textbf{误差} & \textbf{更新后估计} \\
\hline
$S_0$(离开办公室) & 30 & 43 & +13 & 43 \\
\hline
$S_1$(到达汽车) & 35 & 38 & +3 & 38 \\
\hline
$S_2$(离开高速公路) & 15 & 23 & +8 & 23 \\
\hline
$S_3$(二级公路) & 10 & 13 & +3 & 13 \\
\hline
$S_4$(进入家附近街道) & 3 & 3 & 0 & 3 \\
\hline
\end{tabular}
\end{center}

\textbf{关键观察}:
\begin{itemize}
    \item 离开高速公路时,你预测15分钟,但实际需要23分钟(误差+8分钟)
    \item 如果 $\alpha = 1/2$,更新后估计为:$15 + 0.5 \times 8 = 19$ 分钟
    \item 但所有这些更新都必须在到家后才能进行
\end{itemize}

\section{时序差分(TD)方法}

\subsection{TD更新规则}

\textbf{更新公式}:
\begin{equation}
V(S_t) \gets V(S_t) + \alpha [R_{t+1} + \gamma V(S_{t+1}) - V(S_t)]
\label{eq:td_update}
\end{equation}

其中:
\begin{itemize}
    \item $R_{t+1}$ 是即时奖励
    \item $V(S_{t+1})$ 是下一状态的当前估计
    \item $\gamma = 1$(无折扣)
    \item 更新可以在每个时间步立即进行
\end{itemize}

\subsection{TD更新过程}

\textbf{假设 $\alpha = 1$}(完全更新):

\textbf{时间 $t=0$(离开办公室 $\to$ 到达汽车)}:
\begin{align}
\delta_0 &= R_1 + \gamma V(S_1) - V(S_0) \\
         &= 5 + 1 \times 35 - 30 = 10 \\
V(S_0) &\gets V(S_0) + \alpha \delta_0 = 30 + 1 \times 10 = 40
\end{align}

\textbf{时间 $t=1$(到达汽车 $\to$ 离开高速公路)}:
\begin{align}
\delta_1 &= R_2 + \gamma V(S_2) - V(S_1) \\
         &= 15 + 1 \times 15 - 35 = -5 \\
V(S_1) &\gets V(S_1) + \alpha \delta_1 = 35 + 1 \times (-5) = 30
\end{align}

\textbf{时间 $t=2$(离开高速公路 $\to$ 二级公路)}:
\begin{align}
\delta_2 &= R_3 + \gamma V(S_3) - V(S_2) \\
         &= 10 + 1 \times 10 - 15 = 5 \\
V(S_2) &\gets V(S_2) + \alpha \delta_2 = 15 + 1 \times 5 = 20
\end{align}

\textbf{时间 $t=3$(二级公路 $\to$ 进入家附近街道)}:
\begin{align}
\delta_3 &= R_4 + \gamma V(S_4) - V(S_3) \\
         &= 10 + 1 \times 3 - 10 = 3 \\
V(S_3) &\gets V(S_3) + \alpha \delta_3 = 10 + 1 \times 3 = 13
\end{align}

\textbf{时间 $t=4$(进入家附近街道 $\to$ 到家)}:
\begin{align}
\delta_4 &= R_5 + \gamma V(S_5) - V(S_4) \\
         &= 3 + 1 \times 0 - 3 = 0 \\
V(S_4) &\gets V(S_4) + \alpha \delta_4 = 3 + 1 \times 0 = 3
\end{align}

\subsection{TD方法的特点}

\begin{itemize}
    \item \textbf{更新时机}:每个时间步都可以立即更新
    \item \textbf{更新目标}:使用 $R_{t+1} + \gamma V(S_{t+1})$(一步前瞻)
    \item \textbf{TD误差}:$\delta_t = R_{t+1} + \gamma V(S_{t+1}) - V(S_t)$
    \item \textbf{优点}:可以在途中学习,不需要等待最终结果
\end{itemize}

\subsection{TD更新表格}

\begin{center}
\begin{tabular}{|l|c|c|c|c|c|}
\hline
\textbf{时间} & \textbf{状态转移} & \textbf{奖励} & \textbf{TD误差} & \textbf{更新前估计} & \textbf{更新后估计} \\
\hline
$t=0$ & $S_0 \to S_1$ & $R_1=5$ & $\delta_0=10$ & $V(S_0)=30$ & $V(S_0)=40$ \\
\hline
$t=1$ & $S_1 \to S_2$ & $R_2=15$ & $\delta_1=-5$ & $V(S_1)=35$ & $V(S_1)=30$ \\
\hline
$t=2$ & $S_2 \to S_3$ & $R_3=10$ & $\delta_2=5$ & $V(S_2)=15$ & $V(S_2)=20$ \\
\hline
$t=3$ & $S_3 \to S_4$ & $R_4=10$ & $\delta_3=3$ & $V(S_3)=10$ & $V(S_3)=13$ \\
\hline
$t=4$ & $S_4 \to S_5$ & $R_5=3$ & $\delta_4=0$ & $V(S_4)=3$ & $V(S_4)=3$ \\
\hline
\end{tabular}
\end{center}

\section{两种方法的对比}

\subsection{更新时机对比}

\textbf{蒙特卡洛方法}:
\begin{itemize}
    \item 必须等待到家后才能更新
    \item 所有更新都在回合结束后进行
    \item 无法在途中学习
\end{itemize}

\textbf{TD方法}:
\begin{itemize}
    \item 每个时间步都可以立即更新
    \item 可以在途中学习
    \item 信息传播更快
\end{itemize}

\subsection{更新目标对比}

\textbf{蒙特卡洛方法}:
\begin{equation}
\text{目标} = G_t = R_{t+1} + R_{t+2} + \cdots + R_T
\end{equation}
\begin{itemize}
    \item 使用完整回报
    \item 无偏估计
    \item 高方差
\end{itemize}

\textbf{TD方法}:
\begin{equation}
\text{目标} = R_{t+1} + \gamma V(S_{t+1})
\end{equation}
\begin{itemize}
    \item 使用一步前瞻
    \item 可能有偏(如果 $V$ 不准确)
    \item 低方差
\end{itemize}

\subsection{具体例子:交通堵塞场景}

\textbf{场景}:另一天,你再次估计30分钟到家,但在高速公路上遇到严重交通堵塞。离开办公室25分钟后,你仍然在高速公路上堵车。你现在估计还需要25分钟到家,总时间50分钟。

\textbf{蒙特卡洛方法}:
\begin{itemize}
    \item 必须等待到家后才能更新初始状态 $S_0$ 的估计
    \item 即使你已经知道初始估计30分钟太乐观,也无法立即更新
\end{itemize}

\textbf{TD方法}:
\begin{itemize}
    \item 可以立即更新初始状态 $S_0$ 的估计
    \item 在时间 $t=0$(离开办公室 $\to$ 到达汽车):
    \begin{align}
    \delta_0 &= R_1 + \gamma V(S_1) - V(S_0) \\
             &= 5 + 1 \times 45 - 30 = 20 \\
    V(S_0) &\gets 30 + \alpha \times 20
    \end{align}
    \item 初始估计会立即向50分钟调整
    \item 每个估计都会向紧随其后的估计调整
\end{itemize}

\section{TD误差的直观理解}

\subsection{TD误差的含义}

\textbf{TD误差}:
\begin{equation}
\delta_t = R_{t+1} + \gamma V(S_{t+1}) - V(S_t)
\end{equation}

\textbf{解释}:
\begin{itemize}
    \item $\delta_t > 0$:当前估计 $V(S_t)$ 过低,应该增加
    \item $\delta_t < 0$:当前估计 $V(S_t)$ 过高,应该减少
    \item $\delta_t = 0$:当前估计正好
\end{itemize}

\subsection{时序差异}

\textbf{关键洞察}:
\begin{quote}
TD误差衡量的是\textbf{预测的时间差异}(temporal differences in predictions)。每个误差与预测随时间的变化成正比。
\end{quote}

\textbf{例子}:
\begin{itemize}
    \item 在时间 $t=0$,你预测总时间30分钟
    \item 在时间 $t=1$,你预测总时间40分钟
    \item 这个10分钟的变化就是TD误差的基础
    \item TD方法利用这个变化来更新估计
\end{itemize}

\section{数值验证}

\subsection{验证TD误差之和}

根据方程(6.6),当 $V$ 不变时:
\begin{equation}
G_t - V(S_t) = \sum_{k=t}^{T-1} \gamma^{k-t} \delta_k
\end{equation}

\textbf{验证 $G_0 - V(S_0)$}:

\begin{align}
G_0 - V(S_0) &= 43 - 30 = 13 \\
\sum_{k=0}^{4} \gamma^{k} \delta_k &= \delta_0 + \delta_1 + \delta_2 + \delta_3 + \delta_4 \\
                                   &= 10 + (-5) + 5 + 3 + 0 = 13 \quad \checkmark
\end{align}

\textbf{验证 $G_2 - V(S_2)$}:

\begin{align}
G_2 - V(S_2) &= 23 - 15 = 8 \\
\sum_{k=2}^{4} \gamma^{k-2} \delta_k &= \delta_2 + \delta_3 + \delta_4 \\
                                     &= 5 + 3 + 0 = 8 \quad \checkmark
\end{align}

\section{总结}

\subsection{关键要点}

\begin{enumerate}
    \item \textbf{蒙特卡洛方法}:
    \begin{itemize}
        \item 使用完整回报 $G_t$
        \item 必须等待回合结束
        \item 无偏估计,但高方差
    \end{itemize}
    
    \item \textbf{TD方法}:
    \begin{itemize}
        \item 使用一步前瞻 $R_{t+1} + \gamma V(S_{t+1})$
        \item 可以立即更新
        \item 可能有偏,但低方差
    \end{itemize}
    
    \item \textbf{TD方法的优势}:
    \begin{itemize}
        \item 在线学习:不需要等待最终结果
        \item 信息传播更快
        \item 特别适合长回合或连续任务
    \end{itemize}
    
    \item \textbf{实际应用}:
    \begin{itemize}
        \item 在交通堵塞时,TD方法可以立即更新估计
        \item 蒙特卡洛方法必须等待到家后才能更新
        \item TD方法提供了"在等待时有事可做"的学习机会
    \end{itemize}
\end{enumerate}

\subsection{关键洞察}

\begin{quote}
\textbf{TD方法的核心优势在于它能够基于当前预测进行学习,而不需要等待最终结果。这使得TD方法特别适合在线学习和实时应用。}
\end{quote}

\end{document}

