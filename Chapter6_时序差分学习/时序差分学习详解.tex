\documentclass[12pt,a4paper]{article}
\usepackage[UTF8]{ctex}
\usepackage{amsmath}
\usepackage{amssymb}
\usepackage{amsthm}
\usepackage{geometry}
\usepackage{hyperref}
\usepackage{enumitem}
\usepackage{xcolor}
\usepackage{bm}
\usepackage{tikz}
\usepackage{booktabs}
\usepackage{array}
\usepackage{algorithm}
\usepackage{algorithmic}

\geometry{margin=2.5cm}

\title{时序差分学习详解}
\subtitle{第6章:结合蒙特卡洛和动态规划的无模型学习方法}
\author{}
\date{}

\newtheorem{definition}{定义}
\newtheorem{theorem}{定理}
\newtheorem{proposition}{命题}
\newtheorem{example}{示例}
\newtheorem{remark}{注记}

\begin{document}

\maketitle

\tableofcontents
\newpage

\section{引言}

\subsection{什么是时序差分学习?}

\begin{definition}[时序差分学习]
\textbf{时序差分学习}(Temporal-Difference Learning, TD)结合了蒙特卡洛方法和动态规划的思想。像蒙特卡洛方法一样,TD方法可以直接从原始经验中学习,不需要环境模型。像动态规划一样,TD方法基于其他学习到的估计值进行更新,不需要等待最终结果(使用自举法)。
\end{definition}

\textbf{关键特征}:
\begin{itemize}
    \item \textbf{无模型}:不需要环境模型,只需要经验
    \item \textbf{自举法}:使用估计值更新估计值
    \item \textbf{在线学习}:每个时间步都可以更新
    \item \textbf{结合优势}:结合了蒙特卡洛和动态规划的优势
\end{itemize}

\subsection{时序差分学习的核心思想}

\textbf{三种方法的对比}:

\begin{center}
\begin{tabular}{|l|c|c|c|}
\hline
\textbf{特性} & \textbf{动态规划} & \textbf{蒙特卡洛} & \textbf{时序差分} \\
\hline
\textbf{需要模型} & 是 & 否 & 否 \\
\hline
\textbf{自举法} & 是 & 否 & 是 \\
\hline
\textbf{更新时机} & 可以立即 & 等待回合结束 & 可以立即 \\
\hline
\textbf{更新目标} & $r + \gamma v(s')$ & $G_t$ & $r + \gamma v(s')$ \\
\hline
\textbf{采样} & 否(期望) & 是(完整回报) & 是(一步采样) \\
\hline
\end{tabular}
\end{center}

\section{TD预测(TD Prediction)}

\subsection{TD(0)算法}

\textbf{蒙特卡洛更新}:
\begin{equation}
V(S_t) \gets V(S_t) + \alpha [G_t - V(S_t)]
\label{eq:mc_update}
\end{equation}

其中 $G_t$ 是完整的真实回报。

\textbf{TD(0)更新}:
\begin{equation}
V(S_t) \gets V(S_t) + \alpha [R_{t+1} + \gamma V(S_{t+1}) - V(S_t)]
\label{eq:td0_update}
\end{equation}

\textbf{关键区别}:
\begin{itemize}
    \item \textbf{蒙特卡洛}:目标 = $G_t$(完整回报,需要等待回合结束)
    \item \textbf{TD(0)}:目标 = $R_{t+1} + \gamma V(S_{t+1})$(一步前瞻,可以立即更新)
\end{itemize}

\subsection{TD(0)算法伪代码}

\begin{algorithm}[H]
\caption{TD(0)算法(估计 $v_\pi$)}
\begin{algorithmic}[1]
\REQUIRE 要评估的策略 $\pi$
\ENSURE 状态价值函数 $v_\pi$ 的估计 $V$
\STATE \textbf{初始化}:$V(s) \in \mathbb{R}$ 任意初始化,对所有 $s \in \mathcal{S}$,但 $V(\text{终止}) = 0$
\STATE \textbf{算法参数}:步长 $\alpha \in (0, 1]$
\STATE
\REPEAT
    \STATE 初始化 $S$(回合的起始状态)
    \REPEAT
        \STATE $A \gets$ 策略 $\pi$ 给出的动作
        \STATE 执行动作 $A$,观察奖励 $R$ 和下一状态 $S'$
        \STATE $V(S) \gets V(S) + \alpha [R + \gamma V(S') - V(S)]$
        \STATE $S \gets S'$
    \UNTIL{$S$ 是终止状态}
\UNTIL{收敛}
\RETURN $V$
\end{algorithmic}
\end{algorithm}

\subsection{TD误差}

\textbf{TD误差的定义}:
\begin{equation}
\delta_t = R_{t+1} + \gamma V(S_{t+1}) - V(S_t)
\label{eq:td_error}
\end{equation}

\textbf{解释}:
\begin{itemize}
    \item TD误差衡量当前估计 $V(S_t)$ 与更好估计 $R_{t+1} + \gamma V(S_{t+1})$ 之间的差异
    \item 如果 $\delta_t > 0$,说明当前估计过低,应该增加
    \item 如果 $\delta_t < 0$,说明当前估计过高,应该减少
    \item TD误差在强化学习中以各种形式出现
\end{itemize}

\textbf{TD更新用TD误差表示}:
\begin{equation}
V(S_t) \gets V(S_t) + \alpha \delta_t
\end{equation}

\subsection{TD vs 蒙特卡洛 vs 动态规划}

\textbf{更新目标的来源}:

\textbf{贝尔曼方程}:
\begin{equation}
v_\pi(s) = \mathbb{E}_\pi[G_t | S_t = s] = \mathbb{E}_\pi[R_{t+1} + \gamma v_\pi(S_{t+1}) | S_t = s]
\end{equation}

\textbf{三种方法的更新目标}:

\begin{enumerate}
    \item \textbf{蒙特卡洛方法}:
    \begin{itemize}
        \item 使用估计:$\mathbb{E}_\pi[G_t | S_t = s]$ 的样本
        \item 目标:$G_t$(完整回报的样本)
        \item 原因:期望值未知,使用样本回报
    \end{itemize}
    
    \item \textbf{动态规划}:
    \begin{itemize}
        \item 使用估计:$\mathbb{E}_\pi[R_{t+1} + \gamma v_\pi(S_{t+1}) | S_t = s]$ 的期望
        \item 目标:$r + \gamma v_\pi(s')$(期望值,但 $v_\pi(s')$ 未知)
        \item 原因:期望值由环境模型提供,但使用当前估计 $V(s')$ 代替真实值
    \end{itemize}
    
    \item \textbf{时序差分}:
    \begin{itemize}
        \item 使用估计:$\mathbb{E}_\pi[R_{t+1} + \gamma v_\pi(S_{t+1}) | S_t = s]$ 的样本
        \item 目标:$R_{t+1} + \gamma V(S_{t+1})$(采样期望值,使用当前估计)
        \item 原因:既采样期望值,又使用当前估计 $V$ 代替真实值 $v_\pi$
    \end{itemize}
\end{enumerate}

\textbf{关键洞察}:
\begin{quote}
TD方法结合了蒙特卡洛的采样和动态规划的自举法。通过精心设计和想象,这可以让我们在获得蒙特卡洛和动态规划的优势方面走得很远。
\end{quote}

\section{TD预测的优势}

\subsection{1. 不需要环境模型}

\textbf{与动态规划对比}:
\begin{itemize}
    \item 动态规划需要完整的环境模型 $p(s', r | s, a)$
    \item TD方法只需要经验样本
    \item 可以从实际交互中学习
\end{itemize}

\subsection{2. 在线、完全增量式学习}

\textbf{与蒙特卡洛对比}:
\begin{itemize}
    \item 蒙特卡洛方法必须等待回合结束才能更新
    \item TD方法只需要等待一个时间步
    \item 可以立即学习,不需要等待最终结果
\end{itemize}

\textbf{优势场景}:
\begin{itemize}
    \item 非常长的回合:延迟所有学习直到回合结束太慢
    \item 持续任务:没有回合的概念
    \item 实验性动作:蒙特卡洛方法必须忽略或折扣包含实验性动作的回合
\end{itemize}

\subsection{3. 收敛性}

\begin{theorem}[TD(0)收敛性]
对于任何固定策略 $\pi$,TD(0)在以下条件下收敛到 $v_\pi$:
\begin{enumerate}
    \item 如果步长参数 $\alpha$ 足够小,在均值意义下收敛
    \item 如果步长参数按照通常的随机逼近条件(2.7)递减,以概率1收敛
\end{enumerate}
\end{theorem}

\textbf{收敛性证明}:
\begin{itemize}
    \item 大多数收敛证明适用于表格情况
    \item 一些也适用于一般线性函数逼近
    \item 在第9章中讨论更一般的情况
\end{itemize}

\subsection{4. 学习速度}

\textbf{实践中的观察}:
\begin{itemize}
    \item TD方法通常比常数-$\alpha$蒙特卡洛方法收敛更快
    \item 在随机任务上,TD方法通常更高效地使用有限数据
    \item 数学上尚未证明哪个方法收敛更快
\end{itemize}

\section{SARSA:On-policy TD控制}

\subsection{问题:从状态价值到动作价值}

\textbf{为什么需要动作价值函数?}:
\begin{itemize}
    \item 如果只有状态价值函数 $v_\pi(s)$,需要环境模型来改进策略
    \item 但TD方法的目标是\textbf{无模型}学习
    \item 解决方案:直接估计动作价值函数 $q_\pi(s, a)$
\end{itemize}

\subsection{SARSA算法}

\textbf{名称来源}:
\begin{itemize}
    \item \textbf{S}:状态 $S_t$
    \item \textbf{A}:动作 $A_t$
    \item \textbf{R}:奖励 $R_{t+1}$
    \item \textbf{S}:下一状态 $S_{t+1}$
    \item \textbf{A}:下一动作 $A_{t+1}$
\end{itemize}

\textbf{更新公式}:
\begin{equation}
Q(S_t, A_t) \gets Q(S_t, A_t) + \alpha [R_{t+1} + \gamma Q(S_{t+1}, A_{t+1}) - Q(S_t, A_t)]
\label{eq:sarsa_update}
\end{equation}

\textbf{关键特征}:
\begin{itemize}
    \item \textbf{On-policy}:使用当前策略选择动作
    \item \textbf{五元组}:$(S_t, A_t, R_{t+1}, S_{t+1}, A_{t+1})$
    \item \textbf{自举法}:使用 $Q(S_{t+1}, A_{t+1})$ 更新 $Q(S_t, A_t)$
\end{itemize}

\subsection{SARSA算法伪代码}

\begin{algorithm}[H]
\caption{SARSA算法(On-policy TD控制)}
\begin{algorithmic}[1]
\REQUIRE 所有状态-动作对都有非零概率被访问(通过 $\varepsilon$-贪婪策略)
\ENSURE 最优动作价值函数 $q_*$ 和最优策略 $\pi_*$
\STATE \textbf{初始化}:$Q(s, a) \in \mathbb{R}$ 任意初始化,对所有 $s \in \mathcal{S}$,$a \in \mathcal{A}(s)$
\STATE $\pi$ 为 $\varepsilon$-贪婪策略,基于 $Q$
\STATE
\REPEAT
    \STATE 初始化 $S$(回合的起始状态)
    \STATE $A \gets$ 策略 $\pi$ 给出的动作(基于 $Q$)
    \REPEAT
        \STATE 执行动作 $A$,观察奖励 $R$ 和下一状态 $S'$
        \STATE $A' \gets$ 策略 $\pi$ 给出的动作(基于 $Q$)
        \STATE $Q(S, A) \gets Q(S, A) + \alpha [R + \gamma Q(S', A') - Q(S, A)]$
        \STATE $\pi(S) \gets \varepsilon$-贪婪策略,基于 $Q(S, \cdot)$
        \STATE $S \gets S'$,$A \gets A'$
    \UNTIL{$S$ 是终止状态}
\UNTIL{策略稳定}
\RETURN $q_* = Q$,$\pi_* = \pi$
\end{algorithmic}
\end{algorithm}

\subsection{SARSA的特点}

\textbf{On-policy特征}:
\begin{itemize}
    \item 使用当前策略选择动作
    \item 学习的是当前策略的动作价值函数
    \item 策略改进基于当前策略的价值函数
\end{itemize}

\textbf{与蒙特卡洛控制的对比}:
\begin{itemize}
    \item 蒙特卡洛:需要等待回合结束
    \item SARSA:每个时间步都可以更新
    \item 学习速度更快
\end{itemize}

\section{Q-learning:Off-policy TD控制}

\subsection{Q-learning算法}

\textbf{更新公式}:
\begin{equation}
Q(S_t, A_t) \gets Q(S_t, A_t) + \alpha [R_{t+1} + \gamma \max_{a} Q(S_{t+1}, a) - Q(S_t, A_t)]
\label{eq:qlearning_update}
\end{equation}

\textbf{关键特征}:
\begin{itemize}
    \item \textbf{Off-policy}:学习最优动作价值函数,但可以使用任何策略收集数据
    \item \textbf{最大化操作}:使用 $\max_{a} Q(S_{t+1}, a)$ 而不是 $Q(S_{t+1}, A_{t+1})$
    \item \textbf{直接学习最优策略}:不需要显式策略改进步骤
\end{itemize}

\subsection{Q-learning vs SARSA}

\textbf{更新公式对比}:

\textbf{SARSA}(On-policy):
\begin{equation}
Q(S_t, A_t) \gets Q(S_t, A_t) + \alpha [R_{t+1} + \gamma Q(S_{t+1}, A_{t+1}) - Q(S_t, A_t)]
\end{equation}

\textbf{Q-learning}(Off-policy):
\begin{equation}
Q(S_t, A_t) \gets Q(S_t, A_t) + \alpha [R_{t+1} + \gamma \max_{a} Q(S_{t+1}, a) - Q(S_t, A_t)]
\end{equation}

\textbf{关键区别}:
\begin{itemize}
    \item \textbf{SARSA}:使用实际选择的下一动作 $A_{t+1}$(遵循当前策略)
    \item \textbf{Q-learning}:使用最优动作 $\max_{a} Q(S_{t+1}, a)$(不依赖实际选择的动作)
\end{itemize}

\subsection{Q-learning算法伪代码}

\begin{algorithm}[H]
\caption{Q-learning算法(Off-policy TD控制)}
\begin{algorithmic}[1]
\REQUIRE 所有状态-动作对都有非零概率被访问(通过行为策略)
\ENSURE 最优动作价值函数 $q_*$ 和最优策略 $\pi_*$
\STATE \textbf{初始化}:$Q(s, a) \in \mathbb{R}$ 任意初始化,对所有 $s \in \mathcal{S}$,$a \in \mathcal{A}(s)$
\STATE
\REPEAT
    \STATE 初始化 $S$(回合的起始状态)
    \REPEAT
        \STATE $A \gets$ 行为策略给出的动作(如 $\varepsilon$-贪婪,基于 $Q$)
        \STATE 执行动作 $A$,观察奖励 $R$ 和下一状态 $S'$
        \STATE $Q(S, A) \gets Q(S, A) + \alpha [R + \gamma \max_{a} Q(S', a) - Q(S, A)]$
        \STATE $S \gets S'$
    \UNTIL{$S$ 是终止状态}
\UNTIL{收敛}
\STATE \textbf{提取策略}:
\FOR{每个状态 $s \in \mathcal{S}$}
    \STATE $\pi_*(s) \gets \arg\max_{a} Q(s, a)$
\ENDFOR
\RETURN $q_* = Q$,$\pi_*$
\end{algorithmic}
\end{algorithm}

\subsection{Q-learning的特点}

\textbf{Off-policy特征}:
\begin{itemize}
    \item 可以使用任何策略(行为策略)收集数据
    \item 学习的是最优动作价值函数 $q_*$
    \item 不依赖数据收集时的策略
\end{itemize}

\textbf{优势}:
\begin{itemize}
    \item 可以使用历史数据(经验回放)
    \item 可以从其他智能体的经验中学习
    \item 样本效率高
\end{itemize}

\section{TD方法的统一视角}

\subsection{更新目标的统一形式}

\textbf{所有TD方法的更新都可以写成}:
\begin{equation}
V(S_t) \gets V(S_t) + \alpha [\text{目标} - V(S_t)]
\end{equation}

\textbf{不同方法的目标}:
\begin{itemize}
    \item \textbf{蒙特卡洛}:目标 = $G_t$(完整回报)
    \item \textbf{TD(0)}:目标 = $R_{t+1} + \gamma V(S_{t+1})$(一步前瞻)
    \item \textbf{动态规划}:目标 = $\mathbb{E}[R_{t+1} + \gamma V(S_{t+1})]$(期望值)
\end{itemize}

\subsection{自举法的应用}

\textbf{TD方法使用自举法}:
\begin{itemize}
    \item 使用估计值 $V(S_{t+1})$ 来更新估计值 $V(S_t)$
    \item 不等待完整的真实回报
    \item 可以立即更新
\end{itemize}

\textbf{与蒙特卡洛的对比}:
\begin{itemize}
    \item 蒙特卡洛:使用完整的真实回报 $G_t$(不使用自举法)
    \item TD:使用一步前瞻 $R_{t+1} + \gamma V(S_{t+1})$(使用自举法)
\end{itemize}

\section{TD误差与蒙特卡洛误差的关系}

\subsection{TD误差的分解}

\textbf{如果价值函数在回合中不改变},蒙特卡洛误差可以写成TD误差的和:

\begin{align}
G_t - V(S_t) &= R_{t+1} + \gamma G_{t+1} - V(S_t) + \gamma V(S_{t+1}) - \gamma V(S_{t+1}) \\
             &= [R_{t+1} + \gamma V(S_{t+1}) - V(S_t)] + \gamma [G_{t+1} - V(S_{t+1})] \\
             &= \delta_t + \gamma [G_{t+1} - V(S_{t+1})] \\
             &= \delta_t + \gamma \delta_{t+1} + \gamma^2 [G_{t+2} - V(S_{t+2})] \\
             &= \delta_t + \gamma \delta_{t+1} + \gamma^2 \delta_{t+2} + \cdots + \gamma^{T-t-1} \delta_{T-1}
\end{align}

\textbf{关键洞察}:
\begin{quote}
蒙特卡洛误差等于从当前时间步到回合结束的所有TD误差的折扣和。
\end{quote}

\textbf{注意}:
\begin{itemize}
    \item 这个等式在价值函数在回合中不改变时是精确的
    \item 在TD(0)中,价值函数在回合中会改变
    \item 如果步长很小,这个等式仍然近似成立
\end{itemize}

\section{总结}

\subsection{核心要点}

\begin{enumerate}
    \item \textbf{定义}:时序差分学习结合了蒙特卡洛的采样和动态规划的自举法
    
    \item \textbf{关键特征}:
    \begin{itemize}
        \item 无模型:只需要经验样本
        \item 自举法:使用估计值更新估计值
        \item 在线学习:每个时间步都可以更新
    \end{itemize}
    
    \item \textbf{主要算法}:
    \begin{itemize}
        \item TD(0):状态价值函数预测
        \item SARSA:On-policy TD控制
        \item Q-learning:Off-policy TD控制
    \end{itemize}
    
    \item \textbf{优势}:
    \begin{itemize}
        \item 不需要环境模型
        \item 在线、增量式学习
        \item 通常比蒙特卡洛方法收敛更快
    \end{itemize}
    
    \item \textbf{TD误差}:
    \begin{itemize}
        \item $\delta_t = R_{t+1} + \gamma V(S_{t+1}) - V(S_t)$
        \item 衡量当前估计与更好估计之间的差异
        \item 蒙特卡洛误差可以写成TD误差的折扣和
    \end{itemize}
    
    \item \textbf{与动态规划和蒙特卡洛的关系}:
    \begin{itemize}
        \item 动态规划:需要模型,使用期望值
        \item 蒙特卡洛:不需要模型,使用完整回报
        \item TD:不需要模型,使用一步前瞻(结合两者优势)
    \end{itemize}
\end{enumerate}

\subsection{关键洞察}

\begin{quote}
\textbf{时序差分学习是强化学习的核心创新。它通过结合蒙特卡洛的采样能力和动态规划的自举法,实现了无模型的在线学习,为现代强化学习算法(如DQN、SAC等)奠定了基础。}
\end{quote}

\vspace{1cm}

\textbf{参考文献}:
\begin{itemize}
    \item Sutton, R. S., \& Barto, A. G. (2018). \textit{Reinforcement Learning: An Introduction} (2nd Edition). MIT Press, Chapter 6.
\end{itemize}

\end{document}

