\documentclass[12pt,a4paper]{article}
\usepackage[UTF8]{ctex}
\usepackage{amsmath}
\usepackage{amssymb}
\usepackage{amsthm}
\usepackage{geometry}
\usepackage{hyperref}
\usepackage{enumitem}
\usepackage{xcolor}
\usepackage{bm}
\usepackage{tikz}
\usepackage{booktabs}

\geometry{margin=2.5cm}

\title{期望更新详解}
\subtitle{动态规划中的核心概念}
\author{}
\date{}

\newtheorem{definition}{定义}
\newtheorem{theorem}{定理}
\newtheorem{proposition}{命题}
\newtheorem{example}{示例}
\newtheorem{remark}{注记}

\begin{document}

\maketitle

\tableofcontents
\newpage

\section{引言}

在动态规划中,"期望更新"(Expected Update)是一个核心概念。理解期望更新对于掌握动态规划算法至关重要。本文档将详细解释什么是期望更新,它与采样更新的区别,以及为什么它在动态规划中如此重要。

\section{什么是期望更新?}

\subsection{定义}

\begin{definition}[期望更新]
\textbf{期望更新}是指基于\textbf{所有可能}的下一状态和奖励的\textbf{期望值}来更新价值函数,而不是基于单个样本。它使用环境的完整动态函数 $p(s', r | s, a)$ 来计算期望。
\end{definition}

\textbf{关键特征}:
\begin{itemize}
    \item \textbf{使用完整模型}:需要知道 $p(s', r | s, a)$
    \item \textbf{计算期望}:对所有可能的 $(s', r)$ 求期望
    \item \textbf{精确计算}:结果是精确的,不是估计
    \item \textbf{不需要采样}:不需要实际与环境交互
\end{itemize}

\subsection{数学表达}

对于状态价值函数的期望更新:

\begin{equation}
v_{k+1}(s) = \sum_{a} \pi(a | s) \sum_{s', r} p(s', r | s, a) [r + \gamma v_k(s')]
\label{eq:expected_update_v}
\end{equation}

对于动作价值函数的期望更新:

\begin{equation}
q_{k+1}(s, a) = \sum_{s', r} p(s', r | s, a) \left[r + \gamma \sum_{a'} \pi(a' | s') q_k(s', a')\right]
\label{eq:expected_update_q}
\end{equation}

\textbf{关键观察}:
\begin{itemize}
    \item 使用求和 $\sum_{s', r}$ 对所有可能的 $(s', r)$ 求期望
    \item 使用概率 $p(s', r | s, a)$ 作为权重
    \item 结果是精确的期望值
\end{itemize}

\section{期望更新 vs 采样更新}

\subsection{对比}

\begin{center}
\begin{tabular}{|l|c|c|}
\hline
\textbf{特性} & \textbf{期望更新} & \textbf{采样更新} \\
\hline
需要模型 & 是($p(s', r | s, a)$) & 否 \\
\hline
计算方式 & 对所有可能结果求期望 & 使用单个样本 \\
\hline
精确性 & 精确 & 估计(有方差) \\
\hline
计算成本 & 高(需要遍历所有可能) & 低(只需一个样本) \\
\hline
适用场景 & 动态规划 & 强化学习(无模型) \\
\hline
\end{tabular}
\end{center}

\subsection{期望更新的示例}

\textbf{状态价值函数的期望更新}:

假设在状态 $s$,策略 $\pi$ 选择动作 $a_1$ 和 $a_2$ 的概率分别为 $0.6$ 和 $0.4$。

对于动作 $a_1$,可能的结果:
\begin{itemize}
    \item 转移到 $s_1'$,奖励 $r_1 = 1$,概率 $p(s_1', 1 | s, a_1) = 0.7$
    \item 转移到 $s_2'$,奖励 $r_2 = 0$,概率 $p(s_2', 0 | s, a_1) = 0.3$
\end{itemize}

对于动作 $a_2$,可能的结果:
\begin{itemize}
    \item 转移到 $s_3'$,奖励 $r_3 = 2$,概率 $p(s_3', 2 | s, a_2) = 1.0$
\end{itemize}

\textbf{期望更新计算}:

\begin{align}
v_{k+1}(s) &= \sum_{a} \pi(a | s) \sum_{s', r} p(s', r | s, a) [r + \gamma v_k(s')] \\
           &= 0.6 \times [0.7 \times (1 + \gamma v_k(s_1')) + 0.3 \times (0 + \gamma v_k(s_2'))] \\
           &\quad + 0.4 \times [1.0 \times (2 + \gamma v_k(s_3'))] \\
           &= 0.6 \times [0.7 + 0.7\gamma v_k(s_1') + 0.3\gamma v_k(s_2')] \\
           &\quad + 0.4 \times [2 + \gamma v_k(s_3')] \\
           &= 0.42 + 0.42\gamma v_k(s_1') + 0.18\gamma v_k(s_2') + 0.8 + 0.4\gamma v_k(s_3') \\
           &= 1.22 + 0.42\gamma v_k(s_1') + 0.18\gamma v_k(s_2') + 0.4\gamma v_k(s_3')
\end{align}

\textbf{关键点}:
\begin{itemize}
    \item 考虑了\textbf{所有可能}的结果
    \item 使用概率作为权重
    \item 结果是精确的期望值
\end{itemize}

\subsection{采样更新的示例}

\textbf{同样的状态,使用采样更新}:

假设我们实际执行动作 $a_1$,观察到转移到 $s_1'$,奖励 $r_1 = 1$。

\textbf{采样更新计算}:

\begin{align}
v_{k+1}(s) &= v_k(s) + \alpha [r_1 + \gamma v_k(s_1') - v_k(s)] \\
           &= v_k(s) + \alpha [1 + \gamma v_k(s_1') - v_k(s)]
\end{align}

其中 $\alpha$ 是学习率。

\textbf{关键点}:
\begin{itemize}
    \item 只使用\textbf{一个样本}
    \item 结果有\textbf{方差}(可能不准确)
    \item 但计算成本低
\end{itemize}

\section{期望更新在动态规划中的应用}

\subsection{策略评估中的期望更新}

策略评估使用期望更新来精确计算策略的价值函数:

\begin{equation}
v_\pi^{k+1}(s) = \sum_{a} \pi(a | s) \sum_{s', r} p(s', r | s, a) [r + \gamma v_\pi^k(s')]
\end{equation}

\textbf{特点}:
\begin{itemize}
    \item 每次更新都考虑所有可能的动作和结果
    \item 使用完整的概率分布
    \item 保证收敛到精确的价值函数
\end{itemize}

\subsection{价值迭代中的期望更新}

价值迭代也使用期望更新,但选择最优动作:

\begin{equation}
v_{k+1}(s) = \max_{a} \sum_{s', r} p(s', r | s, a) [r + \gamma v_k(s')]
\end{equation}

\textbf{特点}:
\begin{itemize}
    \item 对每个动作计算期望回报
    \item 选择使期望回报最大的动作
    \item 保证收敛到最优价值函数
\end{itemize}

\subsection{策略迭代中的期望更新}

策略迭代结合了策略评估和策略改进:

\textbf{策略评估阶段}:
\begin{equation}
v_\pi^{k+1}(s) = \sum_{a} \pi(a | s) \sum_{s', r} p(s', r | s, a) [r + \gamma v_\pi^k(s')]
\end{equation}

\textbf{策略改进阶段}:
\begin{equation}
\pi'(s) = \arg\max_{a} \sum_{s', r} p(s', r | s, a) [r + \gamma v_\pi(s')]
\end{equation}

两个阶段都使用期望更新。

\section{为什么叫"期望"更新?}

\subsection{期望的数学含义}

\textbf{期望(Expectation)}是概率论中的概念:

\begin{equation}
\mathbb{E}[X] = \sum_{x} x \cdot \Pr\{X = x\}
\end{equation}

\textbf{在价值更新中的应用}:

价值更新计算的是\textbf{期望回报}:

\begin{align}
v_{k+1}(s) &= \mathbb{E}_\pi[R_{t+1} + \gamma v_k(S_{t+1}) | S_t = s] \\
           &= \sum_{a} \pi(a | s) \sum_{s', r} p(s', r | s, a) [r + \gamma v_k(s')]
\end{align}

\textbf{解释}:
\begin{itemize}
    \item $R_{t+1} + \gamma v_k(S_{t+1})$ 是随机变量(因为 $S_{t+1}$ 和 $R_{t+1}$ 是随机的)
    \item 我们计算这个随机变量的\textbf{期望值}
    \item 期望值 = 所有可能值 × 对应概率 的求和
\end{itemize}

\subsection{与采样更新的对比}

\textbf{期望更新}:
\begin{itemize}
    \item 计算 $\mathbb{E}[R_{t+1} + \gamma v_k(S_{t+1}) | S_t = s]$
    \item 使用所有可能的 $(s', r)$ 及其概率
    \item 结果是精确的期望值
\end{itemize}

\textbf{采样更新}:
\begin{itemize}
    \item 使用单个样本 $(s', r)$
    \item 估计期望值:$r + \gamma v_k(s') \approx \mathbb{E}[R_{t+1} + \gamma v_k(S_{t+1}) | S_t = s]$
    \item 结果有方差,但通过多次更新可以收敛
\end{itemize}

\section{期望更新的计算复杂度}

\subsection{单次更新的复杂度}

对于状态 $s$ 的期望更新:

\begin{equation}
v_{k+1}(s) = \sum_{a} \pi(a | s) \sum_{s', r} p(s', r | s, a) [r + \gamma v_k(s')]
\end{equation}

\textbf{计算步骤}:
\begin{enumerate}
    \item 对每个动作 $a$:$O(|\mathcal{A}|)$
    \item 对每个可能的 $(s', r)$:$O(|\mathcal{S}| \times |\mathcal{R}|)$
    \item 总复杂度:$O(|\mathcal{A}| \times |\mathcal{S}| \times |\mathcal{R}|)$
\end{enumerate}

\textbf{对于所有状态}:
\begin{equation}
O(|\mathcal{S}| \times |\mathcal{A}| \times |\mathcal{S}| \times |\mathcal{R}|) = O(|\mathcal{S}|^2 \times |\mathcal{A}| \times |\mathcal{R}|)
\end{equation}

\subsection{与采样更新的对比}

\textbf{采样更新}:
\begin{itemize}
    \item 单次更新:$O(1)$(只需一个样本)
    \item 但需要多次更新才能收敛
    \item 总计算量取决于样本数量
\end{itemize}

\textbf{期望更新}:
\begin{itemize}
    \item 单次更新:$O(|\mathcal{S}| \times |\mathcal{A}| \times |\mathcal{R}|)$
    \item 但每次更新都是精确的
    \item 通常需要更少的迭代次数
\end{itemize}

\section{期望更新的优势与劣势}

\subsection{优势}

\begin{enumerate}
    \item \textbf{精确性}:
    \begin{itemize}
        \item 结果是精确的期望值
        \item 没有方差
        \item 保证收敛到正确值
    \end{itemize}
    
    \item \textbf{效率(小状态空间)}:
    \begin{itemize}
        \item 对于小状态空间,期望更新很快
        \item 不需要实际与环境交互
        \item 可以离线计算
    \end{itemize}
    
    \item \textbf{理论保证}:
    \begin{itemize}
        \item 有收敛性保证
        \item 可以证明最优性
        \item 数学上更优雅
    \end{itemize}
\end{enumerate}

\subsection{劣势}

\begin{enumerate}
    \item \textbf{需要模型}:
    \begin{itemize}
        \item 必须知道 $p(s', r | s, a)$
        \item 模型可能不准确
        \item 模型可能难以获得
    \end{itemize}
    
    \item \textbf{计算复杂度}:
    \begin{itemize}
        \item 对于大状态空间,计算成本高
        \item 需要遍历所有可能的结果
        \item 可能不可行
    \end{itemize}
    
    \item \textbf{不适用于无模型场景}:
    \begin{itemize}
        \item 不知道环境动态时无法使用
        \item 必须使用采样更新
    \end{itemize}
\end{enumerate}

\section{期望更新的具体例子}

\subsection{Gridworld中的期望更新}

考虑Gridworld中的状态 $s$(中心状态,第3行第3列)。

\textbf{环境动态}:
\begin{itemize}
    \item 动作"北":转移到 $s'$(第2行第3列),奖励 $0$,概率 $1.0$
    \item 动作"南":转移到 $s''$(第4行第3列),奖励 $0$,概率 $1.0$
    \item 动作"东":转移到 $s'''$(第3行第4列),奖励 $0$,概率 $1.0$
    \item 动作"西":转移到 $s''''$(第3行第2列),奖励 $0$,概率 $1.0$
\end{itemize}

\textbf{策略}:均匀随机策略,$\pi(a | s) = 0.25$ 对所有动作。

\textbf{期望更新计算}:

\begin{align}
v_\pi^{k+1}(s) &= \sum_{a} \pi(a | s) \sum_{s', r} p(s', r | s, a) [r + \gamma v_\pi^k(s')] \\
               &= 0.25 \times [1.0 \times (0 + \gamma v_\pi^k(s'))] \\
               &\quad + 0.25 \times [1.0 \times (0 + \gamma v_\pi^k(s''))] \\
               &\quad + 0.25 \times [1.0 \times (0 + \gamma v_\pi^k(s'''))] \\
               &\quad + 0.25 \times [1.0 \times (0 + \gamma v_\pi^k(s''''))] \\
               &= 0.25 \gamma [v_\pi^k(s') + v_\pi^k(s'') + v_\pi^k(s''') + v_\pi^k(s'''')]
\end{align}

\textbf{关键点}:
\begin{itemize}
    \item 考虑了所有4个动作
    \item 考虑了每个动作的确定结果
    \item 使用策略概率作为权重
    \item 结果是精确的期望值
\end{itemize}

\subsection{随机环境中的期望更新}

考虑一个随机环境,在状态 $s$ 采取动作 $a$:

\textbf{可能的结果}:
\begin{itemize}
    \item 转移到 $s_1'$,奖励 $r_1 = 10$,概率 $p(s_1', 10 | s, a) = 0.6$
    \item 转移到 $s_2'$,奖励 $r_2 = 1$,概率 $p(s_2', 1 | s, a) = 0.3$
    \item 转移到 $s_3'$,奖励 $r_3 = -5$,概率 $p(s_3', -5 | s, a) = 0.1$
\end{itemize}

\textbf{期望更新计算}:

\begin{align}
q_{k+1}(s, a) &= \sum_{s', r} p(s', r | s, a) [r + \gamma \max_{a'} q_k(s', a')] \\
              &= 0.6 \times [10 + \gamma \max_{a'} q_k(s_1', a')] \\
              &\quad + 0.3 \times [1 + \gamma \max_{a'} q_k(s_2', a')] \\
              &\quad + 0.1 \times [-5 + \gamma \max_{a'} q_k(s_3', a')] \\
              &= 6.0 + 0.6\gamma \max_{a'} q_k(s_1', a') \\
              &\quad + 0.3 + 0.3\gamma \max_{a'} q_k(s_2', a') \\
              &\quad - 0.5 + 0.1\gamma \max_{a'} q_k(s_3', a') \\
              &= 5.8 + 0.6\gamma \max_{a'} q_k(s_1', a') \\
              &\quad + 0.3\gamma \max_{a'} q_k(s_2', a') \\
              &\quad + 0.1\gamma \max_{a'} q_k(s_3', a')
\end{align}

\textbf{关键点}:
\begin{itemize}
    \item 考虑了所有3种可能的结果
    \item 使用概率作为权重
    \item 结果是精确的期望值(考虑了随机性)
\end{itemize}

\section{期望更新与贝尔曼方程}

\subsection{关系}

期望更新与贝尔曼方程密切相关:

\begin{quote}
\textbf{期望更新本质上是贝尔曼方程的迭代形式。}
\end{quote}

\textbf{贝尔曼方程}(策略评估):
\begin{equation}
v_\pi(s) = \sum_{a} \pi(a | s) \sum_{s', r} p(s', r | s, a) [r + \gamma v_\pi(s')]
\end{equation}

\textbf{期望更新}(迭代求解):
\begin{equation}
v_\pi^{k+1}(s) = \sum_{a} \pi(a | s) \sum_{s', r} p(s', r | s, a) [r + \gamma v_\pi^k(s')]
\end{equation}

\textbf{区别}:
\begin{itemize}
    \item 贝尔曼方程:$v_\pi$ 是方程的解(不动点)
    \item 期望更新:$v_\pi^k$ 是迭代过程中的近似值
    \item 当 $k \to \infty$ 时,$v_\pi^k \to v_\pi$(收敛到贝尔曼方程的解)
\end{itemize}

\subsection{四种期望更新}

根据Sutton和Barto的教材,有四种对应的贝尔曼方程和四种对应的期望更新:

\begin{enumerate}
    \item \textbf{状态价值函数的贝尔曼方程}:
    \begin{equation}
    v_\pi(s) = \sum_{a} \pi(a | s) \sum_{s', r} p(s', r | s, a) [r + \gamma v_\pi(s')]
    \end{equation}
    
    \item \textbf{动作价值函数的贝尔曼方程}:
    \begin{equation}
    q_\pi(s, a) = \sum_{s', r} p(s', r | s, a) \left[r + \gamma \sum_{a'} \pi(a' | s') q_\pi(s', a')\right]
    \end{equation}
    
    \item \textbf{最优状态价值函数的贝尔曼方程}:
    \begin{equation}
    v_*(s) = \max_{a} \sum_{s', r} p(s', r | s, a) [r + \gamma v_*(s')]
    \end{equation}
    
    \item \textbf{最优动作价值函数的贝尔曼方程}:
    \begin{equation}
    q_*(s, a) = \sum_{s', r} p(s', r | s, a) \left[r + \gamma \max_{a'} q_*(s', a')\right]
    \end{equation}
\end{enumerate}

每种贝尔曼方程都有对应的期望更新形式。

\section{期望更新的实现}

\subsection{伪代码}

\begin{algorithm}[H]
\caption{期望更新(策略评估)}
\begin{algorithmic}[1]
\REQUIRE 策略 $\pi$,环境动态 $p(s', r | s, a)$,当前价值函数 $v_k$
\ENSURE 更新后的价值函数 $v_{k+1}$
\FOR{每个状态 $s \in \mathcal{S}$}
    \STATE $v_{k+1}(s) \gets 0$
    \FOR{每个动作 $a \in \mathcal{A}(s)$}
        \STATE $q \gets 0$
        \FOR{每个 $(s', r)$ 使得 $p(s', r | s, a) > 0$}
            \STATE $q \gets q + p(s', r | s, a) \times [r + \gamma v_k(s')]$
        \ENDFOR
        \STATE $v_{k+1}(s) \gets v_{k+1}(s) + \pi(a | s) \times q$
    \ENDFOR
\ENDFOR
\RETURN $v_{k+1}$
\end{algorithmic}
\end{algorithmic}

\subsection{关键步骤}

\textbf{步骤1}:对每个状态 $s$,初始化 $v_{k+1}(s) = 0$

\textbf{步骤2}:对每个动作 $a$,计算动作价值:
\begin{equation}
q(s, a) = \sum_{s', r} p(s', r | s, a) [r + \gamma v_k(s')]
\end{equation}

\textbf{步骤3}:对动作求期望(按策略加权):
\begin{equation}
v_{k+1}(s) = \sum_{a} \pi(a | s) q(s, a)
\end{equation}

\section{总结}

\subsection{核心要点}

\begin{enumerate}
    \item \textbf{定义}:期望更新是基于所有可能结果的期望值来更新价值函数
    
    \item \textbf{特点}:
    \begin{itemize}
        \item 需要完整的环境模型 $p(s', r | s, a)$
        \item 计算精确的期望值
        \item 不需要实际与环境交互
    \end{itemize}
    
    \item \textbf{与采样更新的区别}:
    \begin{itemize}
        \item 期望更新:使用所有可能结果,精确但计算成本高
        \item 采样更新:使用单个样本,估计但有方差,计算成本低
    \end{itemize}
    
    \item \textbf{应用}:
    \begin{itemize}
        \item 策略评估
        \item 价值迭代
        \item 策略迭代
        \item 所有动态规划方法
    \end{itemize}
    
    \item \textbf{与贝尔曼方程的关系}:
    \begin{itemize}
        \item 期望更新是贝尔曼方程的迭代形式
        \item 通过迭代,收敛到贝尔曼方程的解
    \end{itemize}
\end{enumerate}

\subsection{关键洞察}

\begin{quote}
\textbf{期望更新是动态规划的核心。它允许我们在"脑海中"模拟环境,精确计算价值函数,而不需要实际与环境交互。这使得动态规划方法既精确又高效(对于小状态空间)。}
\end{quote}

\vspace{1cm}

\textbf{参考文献}:
\begin{itemize}
    \item Sutton, R. S., \& Barto, A. G. (2018). \textit{Reinforcement Learning: An Introduction} (2nd Edition). MIT Press, Chapter 4, 8.5.
\end{itemize}

\end{document}

