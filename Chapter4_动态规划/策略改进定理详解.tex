\documentclass[12pt,a4paper]{article}
\usepackage[UTF8]{ctex}
\usepackage{amsmath}
\usepackage{amssymb}
\usepackage{amsthm}
\usepackage{geometry}
\usepackage{hyperref}
\usepackage{enumitem}
\usepackage{xcolor}
\usepackage{bm}
\usepackage{tikz}
\usepackage{booktabs}

\geometry{margin=2.5cm}

\title{策略改进定理详解}
\subtitle{为什么 $q_\pi(s, a) > v_\pi(s)$ 意味着策略可以改进?}
\author{}
\date{}

\newtheorem{definition}{定义}
\newtheorem{theorem}{定理}
\newtheorem{proposition}{命题}
\newtheorem{example}{示例}
\newtheorem{remark}{注记}

\begin{document}

\maketitle

\tableofcontents
\newpage

\section{引言}

策略改进(Policy Improvement)是动态规划中的核心概念。本文档将详细解释以下关键思想:

\begin{quote}
\textbf{如果在状态 $s$ 选择动作 $a$,然后遵循策略 $\pi$,这个行为的价值是 $q_\pi(s, a)$。关键标准是:这个值是否大于或小于 $v_\pi(s)$。}
\end{quote}

\section{问题设置}

\subsection{当前策略}

假设我们有一个确定性策略 $\pi$,其状态价值函数为 $v_\pi(s)$。

\textbf{问题}:我们想知道是否应该改变策略,在某个状态 $s$ 选择不同的动作 $a \neq \pi(s)$。

\subsection{两种行为方式}

\textbf{方式1:遵循当前策略 $\pi$}
\begin{itemize}
    \item 在状态 $s$,选择动作 $\pi(s)$
    \item 然后继续遵循策略 $\pi$
    \item 这个行为的价值:$v_\pi(s)$
\end{itemize}

\textbf{方式2:改变策略}
\begin{itemize}
    \item 在状态 $s$,选择动作 $a \neq \pi(s)$
    \item 然后继续遵循策略 $\pi$
    \item 这个行为的价值:$q_\pi(s, a)$
\end{itemize}

\section{关键公式}

\subsection{动作价值函数的定义}

如果在状态 $s$ 选择动作 $a$,然后遵循策略 $\pi$,这个行为的价值是:

\begin{equation}
q_\pi(s, a) = \mathbb{E}[R_{t+1} + \gamma v_\pi(S_{t+1}) | S_t = s, A_t = a]
\label{eq:q_pi_definition}
\end{equation}

\textbf{展开形式}:

\begin{equation}
q_\pi(s, a) = \sum_{s', r} p(s', r | s, a) [r + \gamma v_\pi(s')]
\label{eq:q_pi_expanded}
\end{equation}

\subsection{公式的含义}

\textbf{解释}:
\begin{itemize}
    \item $q_\pi(s, a)$ 表示:在状态 $s$ 采取动作 $a$,然后遵循策略 $\pi$ 的期望回报
    \item 即时奖励:$R_{t+1}$(期望值为 $r$)
    \item 未来回报:从下一状态 $S_{t+1}$ 开始,遵循策略 $\pi$ 的期望回报,即 $v_\pi(S_{t+1})$
    \item 总回报:即时奖励 + 折扣后的未来回报
\end{itemize}

\section{关键标准}

\subsection{比较 $q_\pi(s, a)$ 和 $v_\pi(s)$}

\textbf{关键问题}:$q_\pi(s, a)$ 是否大于 $v_\pi(s)$?

\begin{quote}
\textbf{关键标准是:这个值($q_\pi(s, a)$)是否大于或小于 $v_\pi(s)$。}
\end{quote}

\subsection{三种情况}

\textbf{情况1:$q_\pi(s, a) > v_\pi(s)$}
\begin{itemize}
    \item 在状态 $s$ 选择动作 $a$ 比遵循当前策略更好
    \item 应该改变策略:在状态 $s$ 选择动作 $a$
    \item 新策略会更好
\end{itemize}

\textbf{情况2:$q_\pi(s, a) = v_\pi(s)$}
\begin{itemize}
    \item 在状态 $s$ 选择动作 $a$ 与遵循当前策略一样好
    \item 可以改变策略,但不会改进
    \item 新策略至少不会更差
\end{itemize}

\textbf{情况3:$q_\pi(s, a) < v_\pi(s)$}
\begin{itemize}
    \item 在状态 $s$ 选择动作 $a$ 比遵循当前策略更差
    \item 不应该改变策略
    \item 保持当前策略更好
\end{itemize}

\section{为什么这个标准有效?}

\subsection{直观理解}

\textbf{为什么 $q_\pi(s, a) > v_\pi(s)$ 意味着策略可以改进?}

\begin{enumerate}
    \item \textbf{$v_\pi(s)$ 的含义}:
    \begin{itemize}
        \item 在状态 $s$ 遵循策略 $\pi$ 的期望回报
        \item 这是当前策略在状态 $s$ 的表现
    \end{itemize}
    
    \item \textbf{$q_\pi(s, a)$ 的含义}:
    \begin{itemize}
        \item 在状态 $s$ 选择动作 $a$,然后遵循策略 $\pi$ 的期望回报
        \item 这是在状态 $s$ 改变动作后的表现
    \end{itemize}
    
    \item \textbf{比较}:
    \begin{itemize}
        \item 如果 $q_\pi(s, a) > v_\pi(s)$,说明在状态 $s$ 选择动作 $a$ 比遵循当前策略更好
        \item 因此,改变策略(在状态 $s$ 选择动作 $a$)会改进策略
    \end{itemize}
\end{enumerate}

\subsection{数学证明思路}

\textbf{策略改进定理}:

如果 $q_\pi(s, a) \geq v_\pi(s)$ 对所有状态 $s$ 成立,且至少对一个状态严格大于,则新策略 $\pi'$(在状态 $s$ 选择动作 $a$)比策略 $\pi$ 更好。

\textbf{证明思路}:
\begin{enumerate}
    \item 定义新策略 $\pi'$:在状态 $s$ 选择动作 $a$,其他状态遵循 $\pi$
    
    \item 在状态 $s$:
    \begin{align}
    v_{\pi'}(s) &= q_\pi(s, a) \geq v_\pi(s)
    \end{align}
    
    \item 对于其他状态,由于后续遵循策略 $\pi$,且 $v_{\pi'}(s) \geq v_\pi(s)$,可以证明 $v_{\pi'}(s') \geq v_\pi(s')$ 对所有 $s'$ 成立
    
    \item 因此,新策略 $\pi'$ 至少与 $\pi$ 一样好,且至少在一个状态更好
\end{enumerate}

\section{具体例子}

\subsection{Gridworld示例}

考虑例题4.1的Gridworld,假设当前策略 $\pi$ 是等概率随机策略。

\textbf{状态7}:
\begin{itemize}
    \item 当前策略的价值:$v_\pi(7) = -20$
    \item 动作"下"的价值:$q_\pi(7, \text{down}) = -19$
    \item 因为 $-19 > -20$,所以在状态7选择"下"比随机策略更好
\end{itemize}

\textbf{状态11}:
\begin{itemize}
    \item 当前策略的价值:$v_\pi(11) = -18$
    \item 动作"下"的价值:$q_\pi(11, \text{down}) = -1$
    \item 因为 $-1 > -18$,所以在状态11选择"下"比随机策略好得多
\end{itemize}

\subsection{策略改进}

基于这个比较,我们可以改进策略:

\textbf{改进后的策略 $\pi'$}:
\begin{itemize}
    \item 在状态7:选择"下"(因为 $q_\pi(7, \text{down}) > v_\pi(7)$)
    \item 在状态11:选择"下"(因为 $q_\pi(11, \text{down}) > v_\pi(11)$)
    \item 在其他状态:可以类似地选择使 $q_\pi(s, a)$ 最大的动作
\end{itemize}

\section{公式推导}

\subsection{从定义到展开}

\textbf{步骤1:动作价值函数的定义}

\begin{equation}
q_\pi(s, a) = \mathbb{E}_\pi[G_t | S_t = s, A_t = a]
\end{equation}

\textbf{步骤2:回报分解}

\begin{align}
G_t &= R_{t+1} + \gamma G_{t+1} \\
q_\pi(s, a) &= \mathbb{E}_\pi[R_{t+1} + \gamma G_{t+1} | S_t = s, A_t = a]
\end{align}

\textbf{步骤3:分离即时奖励和未来回报}

\begin{align}
q_\pi(s, a) &= \mathbb{E}_\pi[R_{t+1} | S_t = s, A_t = a] + \gamma \mathbb{E}_\pi[G_{t+1} | S_t = s, A_t = a]
\end{align}

\textbf{步骤4:利用马尔可夫性质}

由于马尔可夫性质,给定 $S_{t+1} = s'$,$G_{t+1}$ 的分布只依赖于 $s'$ 和后续策略:

\begin{align}
\mathbb{E}_\pi[G_{t+1} | S_t = s, A_t = a, S_{t+1} = s'] &= \mathbb{E}_\pi[G_{t+1} | S_{t+1} = s'] \\
                                                          &= v_\pi(s')
\end{align}

\textbf{步骤5:全概率展开}

\begin{align}
q_\pi(s, a) &= \sum_{s', r} p(s', r | s, a) \mathbb{E}_\pi[R_{t+1} + \gamma G_{t+1} | S_t = s, A_t = a, S_{t+1} = s', R_{t+1} = r] \\
            &= \sum_{s', r} p(s', r | s, a) [r + \gamma v_\pi(s')]
\end{align}

这就是公式 \eqref{eq:q_pi_expanded}。

\section{关键洞察}

\subsection{为什么这个比较有效?}

\textbf{核心思想}:

\begin{quote}
\textbf{如果在状态 $s$ 选择动作 $a$ 的期望回报($q_\pi(s, a)$)大于遵循当前策略的期望回报($v_\pi(s)$),那么改变策略在状态 $s$ 选择动作 $a$ 会改进策略。}
\end{quote}

\textbf{原因}:
\begin{enumerate}
    \item \textbf{局部改进}:在状态 $s$ 选择更好的动作,立即获得更高的期望回报
    
    \item \textbf{全局影响}:由于后续仍然遵循策略 $\pi$,且 $v_\pi(s')$ 已经考虑了从 $s'$ 开始的全局回报,所以局部改进会传播到全局
    
    \item \textbf{单调性}:如果 $q_\pi(s, a) \geq v_\pi(s)$,则新策略的价值 $v_{\pi'}(s) \geq v_\pi(s)$
\end{enumerate}

\subsection{策略改进的过程}

\textbf{策略改进算法}:

\begin{enumerate}
    \item 对每个状态 $s$,计算所有动作的动作价值:$q_\pi(s, a)$ 对所有 $a$
    
    \item 比较 $q_\pi(s, a)$ 和 $v_\pi(s)$
    
    \item 如果存在 $a$ 使得 $q_\pi(s, a) > v_\pi(s)$,则改进策略:
    \begin{equation}
    \pi'(s) = \arg\max_{a} q_\pi(s, a)
    \end{equation}
    
    \item 新策略 $\pi'$ 至少与 $\pi$ 一样好,且至少在一个状态更好
\end{enumerate}

\section{数值示例}

\subsection{详细计算}

考虑一个简单的状态,当前策略 $\pi$ 的价值为 $v_\pi(s) = 10$。

\textbf{动作价值}:
\begin{itemize}
    \item $q_\pi(s, a_1) = 8$(小于 $v_\pi(s)$)
    \item $q_\pi(s, a_2) = 10$(等于 $v_\pi(s)$)
    \item $q_\pi(s, a_3) = 12$(大于 $v_\pi(s)$)
\end{itemize}

\textbf{分析}:
\begin{itemize}
    \item 动作 $a_1$:比当前策略差,不应该选择
    \item 动作 $a_2$:与当前策略一样好,可以选择
    \item 动作 $a_3$:比当前策略好,应该选择
\end{itemize}

\textbf{策略改进}:
\begin{itemize}
    \item 改进后的策略:在状态 $s$ 选择动作 $a_3$
    \item 新策略的价值:$v_{\pi'}(s) = q_\pi(s, a_3) = 12 > 10 = v_\pi(s)$
    \item 策略得到改进!
\end{itemize}

\section{策略改进定理}

\subsection{正式表述}

\begin{theorem}[策略改进定理]
设 $\pi$ 和 $\pi'$ 是一对确定性策略,如果对所有 $s \in \mathcal{S}$:
\begin{equation}
q_\pi(s, \pi'(s)) \geq v_\pi(s)
\end{equation}
则策略 $\pi'$ 至少与策略 $\pi$ 一样好,即对所有 $s \in \mathcal{S}$:
\begin{equation}
v_{\pi'}(s) \geq v_\pi(s)
\end{equation}
\end{theorem}

\textbf{证明思路}:

\begin{align}
v_\pi(s) &\leq q_\pi(s, \pi'(s)) \\
         &= \mathbb{E}[R_{t+1} + \gamma v_\pi(S_{t+1}) | S_t = s, A_t = \pi'(s)] \\
         &= \mathbb{E}_{\pi'}[R_{t+1} + \gamma v_\pi(S_{t+1}) | S_t = s] \\
         &\leq \mathbb{E}_{\pi'}[R_{t+1} + \gamma q_\pi(S_{t+1}, \pi'(S_{t+1})) | S_t = s] \\
         &= \mathbb{E}_{\pi'}[R_{t+1} + \gamma v_{\pi'}(S_{t+1}) | S_t = s] \\
         &= v_{\pi'}(s)
\end{align}

因此 $v_{\pi'}(s) \geq v_\pi(s)$ 对所有 $s$ 成立。

\subsection{贪婪策略}

\textbf{定义}:贪婪策略 $\pi'$ 定义为:

\begin{equation}
\pi'(s) = \arg\max_{a} q_\pi(s, a)
\end{equation}

\textbf{性质}:
\begin{itemize}
    \item 贪婪策略在每一步都选择使动作价值最大的动作
    \item 对于贪婪策略,$q_\pi(s, \pi'(s)) = \max_a q_\pi(s, a) \geq v_\pi(s)$
    \item 因此,贪婪策略至少与当前策略一样好
\end{itemize}

\section{总结}

\subsection{核心要点}

\begin{enumerate}
    \item \textbf{动作价值函数}:
    \begin{equation}
    q_\pi(s, a) = \sum_{s', r} p(s', r | s, a) [r + \gamma v_\pi(s')]
    \end{equation}
    表示在状态 $s$ 选择动作 $a$,然后遵循策略 $\pi$ 的期望回报。
    
    \item \textbf{关键标准}:
    \begin{itemize}
        \item 如果 $q_\pi(s, a) > v_\pi(s)$:改变策略会改进
        \item 如果 $q_\pi(s, a) = v_\pi(s)$:改变策略不会变差
        \item 如果 $q_\pi(s, a) < v_\pi(s)$:不应该改变策略
    \end{itemize}
    
    \item \textbf{策略改进}:
    \begin{itemize}
        \item 对每个状态,选择使 $q_\pi(s, a)$ 最大的动作
        \item 新策略至少与当前策略一样好
        \item 这是策略迭代算法的基础
    \end{itemize}
    
    \item \textbf{直观理解}:
    \begin{itemize}
        \item $v_\pi(s)$:遵循当前策略的期望回报
        \item $q_\pi(s, a)$:改变动作后的期望回报
        \item 比较两者,选择更好的
    \end{itemize}
\end{enumerate}

\subsection{关键洞察}

\begin{quote}
\textbf{策略改进的核心思想是:如果在某个状态选择不同的动作能获得更高的期望回报,那么改变策略就会改进策略。这个比较是通过 $q_\pi(s, a)$ 和 $v_\pi(s)$ 来实现的。}
\end{quote}

\vspace{1cm}

\textbf{参考文献}:
\begin{itemize}
    \item Sutton, R. S., \& Barto, A. G. (2018). \textit{Reinforcement Learning: An Introduction} (2nd Edition). MIT Press, Chapter 4.2.
\end{itemize}

\end{document}

