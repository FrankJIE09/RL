\documentclass[12pt,a4paper]{article}
\usepackage[UTF8]{ctex}
\usepackage{amsmath}
\usepackage{amssymb}
\usepackage{amsthm}
\usepackage{geometry}
\usepackage{hyperref}
\usepackage{enumitem}
\usepackage{xcolor}
\usepackage{bm}
\usepackage{tikz}
\usepackage{booktabs}

\geometry{margin=2.5cm}

\title{例题4.2:新增状态的价值函数计算}
\subtitle{Gridworld扩展后的策略评估}
\author{}
\date{}

\newtheorem{definition}{定义}
\newtheorem{theorem}{定理}
\newtheorem{proposition}{命题}
\newtheorem{example}{示例}
\newtheorem{remark}{注记}

\begin{document}

\maketitle

\tableofcontents
\newpage

\section{问题描述}

\subsection{原始问题}

在例题4.1的4×4 Gridworld基础上,考虑以下扩展:

\begin{enumerate}
    \item \textbf{情况1}:在状态13下方添加新状态15,其动作(左、上、右、下)分别将智能体转移到状态12、13、14、15。假设原始状态的转移不变。
    
    \item \textbf{情况2}:在情况1的基础上,同时改变状态13的动态,使得从状态13采取动作"下"会转移到新状态15。
\end{enumerate}

\textbf{问题}:在这两种情况下,等概率随机策略下状态15的价值函数 $v_\pi(15)$ 分别是多少?

\subsection{原始Gridworld布局}

原始4×4 Gridworld的状态布局:

\begin{center}
\begin{tabular}{|c|c|c|c|}
\hline
\textbf{终止} & 2 & 3 & 4 \\
\hline
5 & 6 & 7 & 8 \\
\hline
9 & 10 & 11 & 12 \\
\hline
13 & 14 & \textbf{终止} & \textbf{终止} \\
\hline
\end{tabular}
\end{center}

\textbf{已知状态价值函数}(等概率随机策略):
\begin{center}
\begin{tabular}{|c|c|c|c|}
\hline
\textbf{终止} & -14 & -20 & -22 \\
\hline
-14 & -18 & -20 & -20 \\
\hline
-20 & -20 & -18 & -14 \\
\hline
-22 & -20 & \textbf{终止} & \textbf{终止} \\
\hline
\end{tabular}
\end{center}

\section{情况1:仅添加新状态15}

\subsection{新的Gridworld布局}

在状态13下方添加新状态15:

\begin{center}
\begin{tabular}{|c|c|c|c|}
\hline
\textbf{终止} & 2 & 3 & 4 \\
\hline
5 & 6 & 7 & 8 \\
\hline
9 & 10 & 11 & 12 \\
\hline
13 & 14 & \textbf{终止} & \textbf{终止} \\
\hline
15 & & & \\
\hline
\end{tabular}
\end{center}

\textbf{状态15的位置}:第5行第1列(在状态13正下方)

\subsection{状态15的转移规则}

根据题目描述,状态15的动作转移为:
\begin{itemize}
    \item \textbf{左(left)}:转移到状态12(第3行第4列)
    \item \textbf{上(up)}:转移到状态13(第4行第1列)
    \item \textbf{右(right)}:转移到状态14(第4行第2列)
    \item \textbf{下(down)}:转移到状态15(自身,因为下方没有状态)
\end{itemize}

\textbf{注意}:题目说"its actions, left, up, right, and down, take the agent to states 12, 13, 14, and 15, respectively",这意味着:
\begin{itemize}
    \item 左 → 状态12
    \item 上 → 状态13
    \item 右 → 状态14
    \item 下 → 状态15(自身)
\end{itemize}

\subsection{计算 $v_\pi(15)$}

使用状态价值函数的贝尔曼方程:

\begin{equation}
v_\pi(s) = \sum_{a} \pi(a | s) \sum_{s', r} p(s', r | s, a) [r + \gamma v_\pi(s')]
\end{equation}

对于无折扣任务($\gamma = 1$)和等概率随机策略($\pi(a | s) = 0.25$):

\begin{equation}
v_\pi(15) = \frac{1}{4} \sum_{a} \sum_{s', r} p(s', r | 15, a) [r + v_\pi(s')]
\end{equation}

\textbf{展开计算}:

\begin{align}
v_\pi(15) &= \frac{1}{4} \left[ p(12, -1 | 15, \text{left}) \times [-1 + v_\pi(12)] \right. \\
          &\quad + p(13, -1 | 15, \text{up}) \times [-1 + v_\pi(13)] \\
          &\quad + p(14, -1 | 15, \text{right}) \times [-1 + v_\pi(14)] \\
          &\quad + \left. p(15, -1 | 15, \text{down}) \times [-1 + v_\pi(15)] \right] \\
          &= \frac{1}{4} \left[ 1 \times [-1 + (-14)] + 1 \times [-1 + (-22)] \right. \\
          &\quad + 1 \times [-1 + (-20)] + \left. 1 \times [-1 + v_\pi(15)] \right] \\
          &= \frac{1}{4} \left[ -15 + (-23) + (-21) + (-1 + v_\pi(15)) \right] \\
          &= \frac{1}{4} \left[ -60 + v_\pi(15) \right] \\
          &= -15 + \frac{1}{4} v_\pi(15)
\end{align}

\textbf{求解 $v_\pi(15)$}:

\begin{align}
v_\pi(15) &= -15 + \frac{1}{4} v_\pi(15) \\
v_\pi(15) - \frac{1}{4} v_\pi(15) &= -15 \\
\frac{3}{4} v_\pi(15) &= -15 \\
v_\pi(15) &= -15 \times \frac{4}{3} = -20
\end{align}

\textbf{答案1}:$v_\pi(15) = -20$

\subsection{验证}

让我们验证这个结果是否合理:

\begin{align}
v_\pi(15) &= \frac{1}{4} \left[ -1 + v_\pi(12) + -1 + v_\pi(13) + -1 + v_\pi(14) + -1 + v_\pi(15) \right] \\
          &= \frac{1}{4} \left[ -1 + (-14) + -1 + (-22) + -1 + (-20) + -1 + (-20) \right] \\
          &= \frac{1}{4} \left[ -15 - 23 - 21 - 21 \right] \\
          &= \frac{1}{4} \times (-80) = -20
\end{align}

验证通过!

\section{情况2:同时改变状态13的动态}

\subsection{新的转移规则}

在情况1的基础上,状态13的动态也改变了:
\begin{itemize}
    \item 从状态13采取动作"下"现在会转移到状态15(而不是撞墙)
\end{itemize}

\textbf{状态13的新转移规则}:
\begin{itemize}
    \item 上:转移到状态9(第3行第1列)
    \item 下:转移到状态15(新状态,第5行第1列)
    \item 右:转移到状态14(第4行第2列)
    \item 左:撞墙(状态不变)
\end{itemize}

\subsection{重新计算状态13的价值}

首先需要重新计算状态13的价值,因为它的转移规则改变了。

\textbf{状态13的贝尔曼方程}:

\begin{align}
v_\pi(13) &= \frac{1}{4} \left[ p(9, -1 | 13, \text{up}) \times [-1 + v_\pi(9)] \right. \\
          &\quad + p(15, -1 | 13, \text{down}) \times [-1 + v_\pi(15)] \\
          &\quad + p(14, -1 | 13, \text{right}) \times [-1 + v_\pi(14)] \\
          &\quad + \left. p(13, -1 | 13, \text{left}) \times [-1 + v_\pi(13)] \right] \\
          &= \frac{1}{4} \left[ 1 \times [-1 + (-20)] + 1 \times [-1 + v_\pi(15)] \right. \\
          &\quad + 1 \times [-1 + (-20)] + \left. 1 \times [-1 + v_\pi(13)] \right] \\
          &= \frac{1}{4} \left[ -21 + (-1 + v_\pi(15)) + (-21) + (-1 + v_\pi(13)) \right] \\
          &= \frac{1}{4} \left[ -44 + v_\pi(15) + v_\pi(13) \right] \\
          &= -11 + \frac{1}{4} v_\pi(15) + \frac{1}{4} v_\pi(13)
\end{align}

\subsection{状态15的贝尔曼方程(情况2)}

状态15的转移规则与情况1相同:

\begin{align}
v_\pi(15) &= \frac{1}{4} \left[ -1 + v_\pi(12) + -1 + v_\pi(13) + -1 + v_\pi(14) + -1 + v_\pi(15) \right] \\
          &= \frac{1}{4} \left[ -1 + (-14) + -1 + v_\pi(13) + -1 + (-20) + -1 + v_\pi(15) \right] \\
          &= \frac{1}{4} \left[ -15 + v_\pi(13) + (-21) + (-1 + v_\pi(15)) \right] \\
          &= \frac{1}{4} \left[ -37 + v_\pi(13) + v_\pi(15) \right] \\
          &= -\frac{37}{4} + \frac{1}{4} v_\pi(13) + \frac{1}{4} v_\pi(15) \\
          &= -9.25 + \frac{1}{4} v_\pi(13) + \frac{1}{4} v_\pi(15)
\end{align}

\subsection{联立求解}

我们有两个方程:

\begin{align}
v_\pi(13) &= -11 + \frac{1}{4} v_\pi(15) + \frac{1}{4} v_\pi(13) \label{eq:state13} \\
v_\pi(15) &= -9.25 + \frac{1}{4} v_\pi(13) + \frac{1}{4} v_\pi(15) \label{eq:state15}
\end{align}

\textbf{求解方程 \eqref{eq:state13}}:

\begin{align}
v_\pi(13) &= -11 + \frac{1}{4} v_\pi(15) + \frac{1}{4} v_\pi(13) \\
v_\pi(13) - \frac{1}{4} v_\pi(13) &= -11 + \frac{1}{4} v_\pi(15) \\
\frac{3}{4} v_\pi(13) &= -11 + \frac{1}{4} v_\pi(15) \\
v_\pi(13) &= -\frac{44}{3} + \frac{1}{3} v_\pi(15) \label{eq:state13_solved}
\end{align}

\textbf{将 \eqref{eq:state13_solved} 代入 \eqref{eq:state15}}:

\begin{align}
v_\pi(15) &= -9.25 + \frac{1}{4} \left[ -\frac{44}{3} + \frac{1}{3} v_\pi(15) \right] + \frac{1}{4} v_\pi(15) \\
          &= -9.25 - \frac{11}{3} + \frac{1}{12} v_\pi(15) + \frac{1}{4} v_\pi(15) \\
          &= -9.25 - \frac{11}{3} + \frac{1}{12} v_\pi(15) + \frac{3}{12} v_\pi(15) \\
          &= -9.25 - \frac{11}{3} + \frac{4}{12} v_\pi(15) \\
          &= -9.25 - \frac{11}{3} + \frac{1}{3} v_\pi(15)
\end{align}

继续计算:

\begin{align}
v_\pi(15) &= -9.25 - \frac{11}{3} + \frac{1}{3} v_\pi(15) \\
          &= -\frac{27.75}{3} - \frac{11}{3} + \frac{1}{3} v_\pi(15) \\
          &= -\frac{38.75}{3} + \frac{1}{3} v_\pi(15) \\
v_\pi(15) - \frac{1}{3} v_\pi(15) &= -\frac{38.75}{3} \\
\frac{2}{3} v_\pi(15) &= -\frac{38.75}{3} \\
v_\pi(15) &= -\frac{38.75}{2} = -19.375
\end{align}

\textbf{答案2}:$v_\pi(15) \approx -19.375$

\subsection{精确计算}

让我们用分数精确计算:

\begin{align}
v_\pi(15) &= -9.25 + \frac{1}{4} v_\pi(13) + \frac{1}{4} v_\pi(15) \\
          &= -\frac{37}{4} + \frac{1}{4} v_\pi(13) + \frac{1}{4} v_\pi(15)
\end{align}

从方程 \eqref{eq:state13_solved}:
\begin{align}
v_\pi(13) &= -\frac{44}{3} + \frac{1}{3} v_\pi(15)
\end{align}

代入:
\begin{align}
v_\pi(15) &= -\frac{37}{4} + \frac{1}{4} \left[ -\frac{44}{3} + \frac{1}{3} v_\pi(15) \right] + \frac{1}{4} v_\pi(15) \\
          &= -\frac{37}{4} - \frac{11}{3} + \frac{1}{12} v_\pi(15) + \frac{1}{4} v_\pi(15) \\
          &= -\frac{37}{4} - \frac{11}{3} + \frac{1}{12} v_\pi(15) + \frac{3}{12} v_\pi(15) \\
          &= -\frac{37}{4} - \frac{11}{3} + \frac{4}{12} v_\pi(15) \\
          &= -\frac{111}{12} - \frac{44}{12} + \frac{1}{3} v_\pi(15) \\
          &= -\frac{155}{12} + \frac{1}{3} v_\pi(15)
\end{align}

继续:
\begin{align}
v_\pi(15) - \frac{1}{3} v_\pi(15) &= -\frac{155}{12} \\
\frac{2}{3} v_\pi(15) &= -\frac{155}{12} \\
v_\pi(15) &= -\frac{155}{12} \times \frac{3}{2} = -\frac{155}{8} = -19.375
\end{align}

\textbf{精确答案2}:$v_\pi(15) = -\frac{155}{8} = -19.375$

\section{结果总结}

\subsection{答案}

\begin{enumerate}
    \item \textbf{情况1}:$v_\pi(15) = -20$
    \item \textbf{情况2}:$v_\pi(15) = -\frac{155}{8} = -19.375$
\end{enumerate}

\subsection{解释}

\textbf{情况1}:
\begin{itemize}
    \item 状态15只能通过自己的动作"下"回到自身
    \item 其他三个动作分别转移到状态12、13、14
    \item 由于状态13的价值是 $-22$(原始值),状态15的价值是 $-20$
    \item 这个值反映了从状态15到终止状态的期望步数
\end{itemize}

\textbf{情况2}:
\begin{itemize}
    \item 状态13现在可以向下转移到状态15
    \item 这创建了状态13和状态15之间的双向连接
    \item 状态13的价值会改变(因为现在有更好的路径)
    \item 状态15的价值也会改变(因为状态13的价值改变了)
    \item 状态15的价值变为 $-19.375$,比情况1略好(因为状态13的价值改善了)
\end{itemize}

\section{关键洞察}

\subsection{价值函数的相互依赖}

\textbf{重要观察}:
\begin{itemize}
    \item 状态价值函数是相互依赖的
    \item 改变一个状态的转移规则会影响其他状态的价值
    \item 需要同时求解所有状态的贝尔曼方程
\end{itemize}

\subsection{求解方法}

\textbf{方法1}:迭代策略评估
\begin{itemize}
    \item 初始化所有状态的价值
    \item 迭代更新直到收敛
    \item 适用于所有情况
\end{itemize}

\textbf{方法2}:解析求解
\begin{itemize}
    \item 对于小规模问题,可以联立求解线性方程组
    \item 如情况2中,需要同时求解 $v_\pi(13)$ 和 $v_\pi(15)$
\end{itemize}

\section{验证计算}

\subsection{情况1的验证}

已知:
\begin{itemize}
    \item $v_\pi(12) = -14$
    \item $v_\pi(13) = -22$(原始值,未改变)
    \item $v_\pi(14) = -20$
    \item $v_\pi(15) = -20$(计算值)
\end{itemize}

验证状态15:
\begin{align}
v_\pi(15) &= \frac{1}{4} \left[ -1 + (-14) + -1 + (-22) + -1 + (-20) + -1 + (-20) \right] \\
          &= \frac{1}{4} \times (-80) = -20 \quad \checkmark
\end{align}

\subsection{情况2的验证}

计算 $v_\pi(13)$:
\begin{align}
v_\pi(13) &= -\frac{44}{3} + \frac{1}{3} \times (-19.375) \\
          &= -\frac{44}{3} - \frac{19.375}{3} \\
          &= -\frac{63.375}{3} = -21.125
\end{align}

验证状态13:
\begin{align}
v_\pi(13) &= \frac{1}{4} \left[ -1 + (-20) + -1 + (-19.375) + -1 + (-20) + -1 + (-21.125) \right] \\
          &= \frac{1}{4} \times (-84.5) = -21.125 \quad \checkmark
\end{align}

验证状态15:
\begin{align}
v_\pi(15) &= \frac{1}{4} \left[ -1 + (-14) + -1 + (-21.125) + -1 + (-20) + -1 + (-19.375) \right] \\
          &= \frac{1}{4} \times (-77.5) = -19.375 \quad \checkmark
\end{align}

验证通过!

\section{总结}

\subsection{核心要点}

\begin{enumerate}
    \item \textbf{状态价值函数的计算}:使用贝尔曼方程,考虑所有可能的动作和转移
    
    \item \textbf{相互依赖}:状态价值函数是相互依赖的,改变一个状态的转移会影响其他状态
    
    \item \textbf{求解方法}:
    \begin{itemize}
        \item 迭代策略评估(适用于所有情况)
        \item 解析求解(适用于小规模问题)
    \end{itemize}
    
    \item \textbf{结果}:
    \begin{itemize}
        \item 情况1:$v_\pi(15) = -20$
        \item 情况2:$v_\pi(15) = -19.375$(略好,因为状态13的路径改善了)
    \end{itemize}
\end{enumerate}

\subsection{学习价值}

例题4.2展示了:
\begin{itemize}
    \item 如何计算新添加状态的价值函数
    \item 状态价值函数的相互依赖性
    \item 改变转移规则对价值函数的影响
    \item 联立求解多个状态的贝尔曼方程
\end{itemize}

\vspace{1cm}

\textbf{参考文献}:
\begin{itemize}
    \item Sutton, R. S., \& Barto, A. G. (2018). \textit{Reinforcement Learning: An Introduction} (2nd Edition). MIT Press, Chapter 4, Exercise 4.2.
\end{itemize}

\end{document}

