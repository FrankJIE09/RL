\documentclass[12pt,a4paper]{article}
\usepackage[UTF8]{ctex}
\usepackage{amsmath}
\usepackage{amssymb}
\usepackage{amsthm}
\usepackage{geometry}
\usepackage{hyperref}
\usepackage{enumitem}
\usepackage{xcolor}
\usepackage{bm}
\usepackage{tikz}
\usepackage{booktabs}
\usepackage{array}
\usepackage{algorithm}
\usepackage{algorithmic}

\geometry{margin=2.5cm}

\title{自举法详解}
\subtitle{强化学习中的核心概念:基于估计值更新估计值}
\author{}
\date{}

\newtheorem{definition}{定义}
\newtheorem{theorem}{定理}
\newtheorem{proposition}{命题}
\newtheorem{example}{示例}
\newtheorem{remark}{注记}

\begin{document}

\maketitle

\tableofcontents
\newpage

\section{引言}

\subsection{什么是自举法?}

\textbf{自举法}(Bootstrapping)是强化学习中的一个核心概念,指的是:

\begin{quote}
\textbf{基于其他估计值来更新估计值。}
\end{quote}

在动态规划中,我们使用后继状态的\textbf{估计值}来更新当前状态的估计值,而不是使用完整的真实回报。

\subsection{自举法的定义}

\begin{definition}[自举法]
\textbf{自举法}(Bootstrapping)是一种更新估计值的方法,其中:
\begin{itemize}
    \item 使用\textbf{其他状态的估计值}来更新当前状态的估计值
    \item 不等待完整的真实回报
    \item 基于部分信息进行更新
\end{itemize}
\end{definition}

\section{自举法 vs 非自举法}

\subsection{对比表格}

\begin{center}
\begin{tabular}{|l|c|c|}
\hline
\textbf{特性} & \textbf{自举法} & \textbf{非自举法} \\
\hline
更新依据 & 其他状态的估计值 & 完整的真实回报 \\
\hline
需要完整轨迹 & 否 & 是 \\
\hline
可以立即更新 & 是 & 否(需等待回合结束) \\
\hline
典型方法 & 动态规划、TD学习 & 蒙特卡洛方法 \\
\hline
更新公式 & $v(s) \gets r + \gamma v(s')$ & $v(s) \gets G_t$ \\
\hline
\end{tabular}
\end{center}

\subsection{自举法的更新公式}

\textbf{动态规划中的自举更新}:

\begin{equation}
v_{k+1}(s) = \sum_{a} \pi(a | s) \sum_{s', r} p(s', r | s, a) [r + \gamma v_k(s')]
\label{eq:bootstrap_update}
\end{equation}

\textbf{关键观察}:
\begin{itemize}
    \item 使用 $v_k(s')$(后继状态的\textbf{估计值})
    \item 不等待完整的回报 $G_t$
    \item 基于一步前瞻进行更新
\end{itemize}

\subsection{非自举法的更新公式}

\textbf{蒙特卡洛方法中的非自举更新}:

\begin{equation}
v(s) \gets v(s) + \alpha [G_t - v(s)]
\label{eq:non_bootstrap_update}
\end{equation}

其中 $G_t = R_{t+1} + \gamma R_{t+2} + \gamma^2 R_{t+3} + \cdots$ 是\textbf{完整的真实回报}。

\textbf{关键观察}:
\begin{itemize}
    \item 使用 $G_t$(完整的真实回报)
    \item 需要等待回合结束
    \item 基于完整轨迹进行更新
\end{itemize}

\section{动态规划中的自举法}

\subsection{策略评估中的自举}

\textbf{策略评估更新}:

\begin{equation}
v_\pi^{k+1}(s) = \sum_{a} \pi(a | s) \sum_{s', r} p(s', r | s, a) [r + \gamma v_\pi^k(s')]
\end{equation}

\textbf{自举特征}:
\begin{itemize}
    \item 使用 $v_\pi^k(s')$(后继状态的估计值)
    \item 不等待完整的回报
    \item 基于一步前瞻更新
\end{itemize}

\subsection{价值迭代中的自举}

\textbf{价值迭代更新}:

\begin{equation}
v_{k+1}(s) = \max_{a} \sum_{s', r} p(s', r | s, a) [r + \gamma v_k(s')]
\end{equation}

\textbf{自举特征}:
\begin{itemize}
    \item 使用 $v_k(s')$(后继状态的估计值)
    \item 选择最优动作
    \item 基于一步前瞻更新
\end{itemize}

\section{为什么叫"自举"?}

\subsection{词源}

\textbf{Bootstrapping} 这个词来自英语短语 "pull oneself up by one's bootstraps"(拉着自己的靴带把自己拉起来),这是一个不可能完成的任务,但用来比喻"靠自己努力成功"。

\textbf{在强化学习中的含义}:
\begin{itemize}
    \item 使用\textbf{估计值}来改进\textbf{估计值}
    \item 就像"拉着自己的靴带把自己拉起来"
    \item 看似不可能,但在数学上是可以收敛的
\end{itemize}

\subsection{自举的可行性}

\textbf{为什么自举法可以工作?}

\begin{enumerate}
    \item \textbf{初始估计}:从某个初始估计开始(如 $v_0(s) = 0$)
    
    \item \textbf{迭代改进}:每次迭代使用更准确的估计值
    
    \item \textbf{收敛性}:在适当的条件下,估计值会收敛到真实值
    
    \item \textbf{数学保证}:贝尔曼方程保证收敛性
\end{enumerate}

\textbf{关键洞察}:
\begin{quote}
虽然我们使用估计值来更新估计值,但通过迭代,这些估计值会逐渐变得更准确,最终收敛到真实值。
\end{quote}

\section{自举法的具体例子}

\subsection{Gridworld例子}

考虑一个简单的Gridworld:

\begin{center}
\begin{tabular}{|c|c|c|}
\hline
$s_1$ & $s_2$ & \textbf{终止} \\
\hline
$s_3$ & $s_4$ & $s_5$ \\
\hline
\end{tabular}
\end{center}

\textbf{环境设置}:
\begin{itemize}
    \item 所有转移的奖励都是 $-1$
    \item 折扣因子 $\gamma = 0.9$
    \item 终止状态价值为 $0$
\end{itemize}

\subsection{自举更新过程}

\textbf{初始化}:
\begin{equation}
v_0(s_1) = v_0(s_2) = v_0(s_3) = v_0(s_4) = v_0(s_5) = 0
\end{equation}

\textbf{第1次迭代(自举更新)}:

对于状态 $s_2$(假设只能向右移动到终止状态):
\begin{align}
v_1(s_2) &= -1 + 0.9 \times v_0(\text{终止}) \\
         &= -1 + 0.9 \times 0 \\
         &= -1.0
\end{align}

\textbf{关键点}:
\begin{itemize}
    \item 使用 $v_0(\text{终止}) = 0$(估计值)
    \item 不等待完整的回报
    \item 立即更新
\end{itemize}

\textbf{第2次迭代}:

对于状态 $s_1$(假设可以向右移动到 $s_2$):
\begin{align}
v_2(s_1) &= -1 + 0.9 \times v_1(s_2) \\
         &= -1 + 0.9 \times (-1.0) \\
         &= -1.9
\end{align}

\textbf{关键点}:
\begin{itemize}
    \item 使用 $v_1(s_2) = -1.0$(第1次迭代的估计值)
    \item 基于估计值更新估计值
    \item 这就是自举法
\end{itemize}

\subsection{非自举更新(对比)}

\textbf{蒙特卡洛方法}:

假设我们从状态 $s_1$ 开始,执行一个完整的回合:
\begin{itemize}
    \item $s_1 \to s_2$:奖励 $-1$
    \item $s_2 \to \text{终止}$:奖励 $-1$
    \item 完整回报:$G_1 = -1 + 0.9 \times (-1) = -1.9$
\end{itemize}

\textbf{非自举更新}:
\begin{equation}
v(s_1) \gets v(s_1) + \alpha [G_1 - v(s_1)]
\end{equation}

\textbf{关键点}:
\begin{itemize}
    \item 使用 $G_1 = -1.9$(完整的真实回报)
    \item 需要等待回合结束
    \item 不使用估计值
\end{itemize}

\section{自举法的优势}

\subsection{可以立即更新}

\textbf{自举法}:
\begin{itemize}
    \item 不需要等待完整的轨迹
    \item 可以立即更新
    \item 学习速度快
\end{itemize}

\textbf{非自举法}:
\begin{itemize}
    \item 需要等待回合结束
    \item 在回合结束前无法更新
    \item 学习速度较慢
\end{itemize}

\subsection{数据效率}

\textbf{自举法}:
\begin{itemize}
    \item 每个转移都可以用于更新
    \item 数据利用率高
    \item 适合在线学习
\end{itemize}

\textbf{非自举法}:
\begin{itemize}
    \item 需要完整的轨迹
    \item 数据利用率较低
    \item 适合离线学习
\end{itemize}

\subsection{适用场景}

\textbf{自举法适用于}:
\begin{itemize}
    \item 需要快速学习的场景
    \item 在线学习场景
    \item 状态空间大的问题
    \item 需要利用部分信息的场景
\end{itemize}

\section{自举法的局限性}

\subsection{需要模型或估计}

\textbf{自举法}:
\begin{itemize}
    \item 需要知道或估计后继状态的价值
    \item 在动态规划中需要环境模型
    \item 在TD学习中需要估计值函数
\end{itemize}

\subsection{可能引入偏差}

\textbf{自举法}:
\begin{itemize}
    \item 如果初始估计不准确,可能引入偏差
    \item 需要多次迭代才能收敛
    \item 在某些情况下可能不收敛
\end{itemize}

\section{自举法在不同算法中的应用}

\subsection{动态规划}

\textbf{策略评估}:
\begin{equation}
v_\pi^{k+1}(s) = \sum_{a} \pi(a | s) \sum_{s', r} p(s', r | s, a) [r + \gamma v_\pi^k(s')]
\end{equation}

\textbf{自举特征}:使用 $v_\pi^k(s')$ 更新 $v_\pi^{k+1}(s)$

\subsection{时序差分学习(TD)}

\textbf{TD(0)更新}:
\begin{equation}
v(s_t) \gets v(s_t) + \alpha [r_{t+1} + \gamma v(s_{t+1}) - v(s_t)]
\end{equation}

\textbf{自举特征}:使用 $v(s_{t+1})$(估计值)更新 $v(s_t)$

\subsection{Q学习}

\textbf{Q学习更新}:
\begin{equation}
Q(s_t, a_t) \gets Q(s_t, a_t) + \alpha [r_{t+1} + \gamma \max_{a} Q(s_{t+1}, a) - Q(s_t, a_t)]
\end{equation}

\textbf{自举特征}:使用 $\max_{a} Q(s_{t+1}, a)$(估计值)更新 $Q(s_t, a_t)$

\subsection{n步自举}

\textbf{n步回报}:
\begin{equation}
G_{t:t+n} = R_{t+1} + \gamma R_{t+2} + \cdots + \gamma^{n-1} R_{t+n} + \gamma^n v(S_{t+n})
\end{equation}

\textbf{自举特征}:使用 $v(S_{t+n})$(估计值)来估计 $n$ 步后的回报

\section{自举法的数学基础}

\subsection{贝尔曼方程}

\textbf{自举法的数学基础是贝尔曼方程}:

\begin{equation}
v_\pi(s) = \sum_{a} \pi(a | s) \sum_{s', r} p(s', r | s, a) [r + \gamma v_\pi(s')]
\end{equation}

\textbf{关键点}:
\begin{itemize}
    \item 贝尔曼方程表明:$v_\pi(s)$ 可以用 $v_\pi(s')$ 表示
    \item 这为自举法提供了数学基础
    \item 自举更新是贝尔曼方程的迭代形式
\end{itemize}

\subsection{不动点理论}

\textbf{自举法的收敛性}:

\begin{theorem}[自举法收敛性]
在适当的条件下(如 $\gamma < 1$),自举更新序列 $\{v_k\}$ 收敛到贝尔曼方程的唯一不动点 $v_*$。
\end{theorem}

\textbf{证明思路}:
\begin{itemize}
    \item 贝尔曼算子是一个压缩映射
    \item 压缩映射有唯一不动点
    \item 自举更新收敛到该不动点
\end{itemize}

\section{自举法 vs 非自举法的详细对比}

\subsection{更新时机}

\textbf{自举法(TD学习)}:
\begin{itemize}
    \item 每个时间步都可以更新
    \item 不需要等待回合结束
    \item 在线学习
\end{itemize}

\textbf{非自举法(蒙特卡洛)}:
\begin{itemize}
    \item 必须等待回合结束
    \item 需要完整的回报
    \item 离线学习
\end{itemize}

\subsection{方差 vs 偏差}

\textbf{自举法}:
\begin{itemize}
    \item \textbf{偏差}:可能引入偏差(如果估计不准确)
    \item \textbf{方差}:方差较小(因为使用估计值,不依赖随机采样)
    \item \textbf{权衡}:偏差-方差权衡
\end{itemize}

\textbf{非自举法}:
\begin{itemize}
    \item \textbf{偏差}:无偏差(使用真实回报)
    \item \textbf{方差}:方差较大(因为依赖随机采样)
    \item \textbf{权衡}:无偏差但高方差
\end{itemize}

\subsection{适用问题}

\textbf{自举法适用于}:
\begin{itemize}
    \item 需要快速学习的任务
    \item 在线学习场景
    \item 状态空间大的问题
    \item 需要利用部分信息的场景
\end{itemize}

\textbf{非自举法适用于}:
\begin{itemize}
    \item 需要无偏差估计的任务
    \item 离线学习场景
    \item 回合制任务
    \item 需要完整轨迹信息的场景
\end{itemize}

\section{自举法的直观理解}

\subsection{类比:学习地图}

\textbf{自举法}:
\begin{itemize}
    \item 就像使用\textbf{部分地图}来绘制\textbf{完整地图}
    \item 从已知区域推断未知区域
    \item 逐步完善地图
\end{itemize}

\textbf{非自举法}:
\begin{itemize}
    \item 就像\textbf{实际探索}每个区域
    \item 需要完整的信息
    \item 更准确但更慢
\end{itemize}

\subsection{类比:拼图游戏}

\textbf{自举法}:
\begin{itemize}
    \item 使用\textbf{已拼好的部分}来推断\textbf{未拼的部分}
    \item 基于部分信息进行推理
    \item 逐步完成拼图
\end{itemize}

\textbf{非自举法}:
\begin{itemize}
    \item 需要\textbf{完整的图片}作为参考
    \item 不依赖推理
    \item 更直接但需要更多信息
\end{itemize}

\section{总结}

\subsection{核心要点}

\begin{enumerate}
    \item \textbf{定义}:自举法是指基于其他估计值来更新估计值
    
    \item \textbf{特征}:
    \begin{itemize}
        \item 使用后继状态的估计值
        \item 不等待完整的真实回报
        \item 可以立即更新
    \end{itemize}
    
    \item \textbf{应用}:
    \begin{itemize}
        \item 动态规划
        \item 时序差分学习
        \item Q学习
        \item n步自举方法
    \end{itemize}
    
    \item \textbf{优势}:
    \begin{itemize}
        \item 可以立即更新
        \item 数据效率高
        \item 适合在线学习
    \end{itemize}
    
    \item \textbf{局限性}:
    \begin{itemize}
        \item 需要模型或估计
        \item 可能引入偏差
        \item 需要多次迭代才能收敛
    \end{itemize}
    
    \item \textbf{数学基础}:
    \begin{itemize}
        \item 贝尔曼方程
        \item 不动点理论
        \item 压缩映射
    \end{itemize}
\end{enumerate}

\subsection{关键洞察}

\begin{quote}
\textbf{自举法允许我们使用部分信息(估计值)来改进估计值,通过迭代逐渐提高准确性,最终收敛到真实值。这使得我们可以在不等待完整轨迹的情况下进行学习,大大提高了学习效率。}
\end{quote}

\vspace{1cm}

\textbf{参考文献}:
\begin{itemize}
    \item Sutton, R. S., \& Barto, A. G. (2018). \textit{Reinforcement Learning: An Introduction} (2nd Edition). MIT Press, Chapter 4, 6, 7.
\end{itemize}

\end{document}

