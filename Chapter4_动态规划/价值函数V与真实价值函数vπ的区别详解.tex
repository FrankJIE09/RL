\documentclass[12pt,a4paper]{article}
\usepackage[UTF8]{ctex}
\usepackage{amsmath}
\usepackage{amssymb}
\usepackage{amsthm}
\usepackage{geometry}
\usepackage{hyperref}
\usepackage{enumitem}
\usepackage{booktabs}

\geometry{left=2.5cm,right=2.5cm,top=2.5cm,bottom=2.5cm}

\title{价值函数 $V$ 与真实价值函数 $v_\pi$ 的区别详解}
\author{强化学习笔记}
\date{\today}

\newtheorem{definition}{定义}
\newtheorem{theorem}{定理}
\newtheorem{example}{示例}
\newtheorem{remark}{注记}

\begin{document}

\maketitle

\tableofcontents
\newpage

\section{引言}

在强化学习中,\textbf{价值函数 $V$} 和 \textbf{真实价值函数 $v_\pi$} 是两个容易混淆但非常重要的概念。理解它们的区别对于掌握强化学习算法至关重要。

\section{基本定义}

\subsection{真实价值函数 $v_\pi$}

\begin{definition}[真实价值函数 $v_\pi$]
\textbf{真实价值函数} $v_\pi(s)$ 是策略 $\pi$ 的\textbf{理论上的真实价值},定义为从状态 $s$ 开始,遵循策略 $\pi$ 的期望回报:
\begin{equation}
v_\pi(s) = \mathbb{E}_\pi[G_t | S_t = s]
\label{eq:true_value_function}
\end{equation}
\end{definition}

\textbf{特点}:
\begin{itemize}
    \item $v_\pi$ 是\textbf{理论值},是策略 $\pi$ 的"真实"价值
    \item $v_\pi$ 满足策略 $\pi$ 的贝尔曼方程
    \item $v_\pi$ 是唯一的,由策略 $\pi$ 和环境动态 $p(s', r | s, a)$ 唯一确定
    \item 在算法中,我们通常\textbf{无法直接计算} $v_\pi$(除非有完整的环境模型)
\end{itemize}

\subsection{价值函数 $V$}

\begin{definition}[价值函数 $V$]
\textbf{价值函数} $V(s)$ 是我们\textbf{估计}或\textbf{计算}的状态价值,是算法中实际使用的价值函数。
\end{definition}

\textbf{特点}:
\begin{itemize}
    \item $V$ 是\textbf{估计值}或\textbf{近似值}
    \item $V$ 是算法中\textbf{实际存储和更新}的值
    \item $V$ 可能等于 $v_\pi$(当一致时),也可能不等于(当不一致时)
    \item $V$ 在算法迭代过程中不断更新,逐渐接近 $v_\pi$
\end{itemize}

\section{关键区别}

\subsection{概念层面}

\begin{center}
\begin{tabular}{|l|l|l|}
\hline
\textbf{特征} & \textbf{真实价值函数 $v_\pi$} & \textbf{价值函数 $V$} \\
\hline
\textbf{性质} & 理论值、真实值 & 估计值、近似值 \\
\hline
\textbf{存在性} & 总是存在(理论上) & 算法中的变量 \\
\hline
\textbf{唯一性} & 由策略 $\pi$ 唯一确定 & 可能不唯一(取决于初始化和算法) \\
\hline
\textbf{可计算性} & 需要完整环境模型 & 可以通过算法估计 \\
\hline
\textbf{在算法中} & 目标值 & 实际使用的值 \\
\hline
\end{tabular}
\end{center}

\subsection{数学关系}

\textbf{目标}:
\begin{equation}
V \to v_\pi
\end{equation}

即:通过算法使价值函数 $V$ 逐渐接近真实价值函数 $v_\pi$。

\textbf{一致性条件}:
\begin{equation}
V(s) = v_\pi(s) \quad \text{对所有 } s \in \mathcal{S}
\end{equation}

当这个条件满足时,我们说"价值函数 $V$ 与策略 $\pi$ 一致"。

\section{具体例子}

\subsection{例子1:策略评估过程}

\textbf{初始状态}:
\begin{itemize}
    \item 策略 $\pi$:等概率随机策略
    \item 价值函数 $V^0(s) = 0$(初始估计,可能不等于 $v_\pi$)
\end{itemize}

\textbf{迭代过程}:

\textbf{第1次迭代}:
\begin{equation}
V^1(s) = \sum_{a} \pi(a|s) \sum_{s', r} p(s', r | s, a) [r + \gamma V^0(s')]
\end{equation}

此时 $V^1 \neq v_\pi$(还没有收敛)

\textbf{第2次迭代}:
\begin{equation}
V^2(s) = \sum_{a} \pi(a|s) \sum_{s', r} p(s', r | s, a) [r + \gamma V^1(s')]
\end{equation}

\textbf{继续迭代},直到:
\begin{equation}
V^k(s) = V^{k+1}(s) \quad \text{对所有 } s
\end{equation}

\textbf{收敛后}:
\begin{equation}
V^k = v_\pi
\end{equation}

此时,价值函数 $V$ 等于真实价值函数 $v_\pi$。

\subsection{例子2:策略迭代中的变化}

\textbf{第1次迭代}:

\textbf{策略评估}:
\begin{itemize}
    \item 给定策略 $\pi_0$
    \item 计算 $V = v_{\pi_0}$(使 $V$ 与 $\pi_0$ 一致)
    \item 此时 $V = v_{\pi_0}$(一致)
\end{itemize}

\textbf{策略改进}:
\begin{itemize}
    \item 基于 $V$ 改进策略,得到 $\pi_1$
    \item 但 $V$ 还没有更新
    \item 此时 $V = v_{\pi_0} \neq v_{\pi_1}$(不一致!)
\end{itemize}

\textbf{第2次迭代}:

\textbf{策略评估}:
\begin{itemize}
    \item 给定策略 $\pi_1$
    \item 重新计算 $V = v_{\pi_1}$(使 $V$ 与 $\pi_1$ 一致)
    \item 此时 $V = v_{\pi_1}$(再次一致)
\end{itemize}

\section{在不同方法中的体现}

\subsection{动态规划}

\textbf{策略评估}:
\begin{itemize}
    \item 使用迭代更新:$V^{k+1}(s) = \sum_{a} \pi(a|s) \sum_{s', r} p(s', r | s, a) [r + \gamma V^k(s')]$
    \item $V^k$ 是迭代过程中的估计值
    \item 当 $k \to \infty$ 时,$V^k \to v_\pi$
    \item 收敛后,$V = v_\pi$
\end{itemize}

\subsection{蒙特卡洛方法}

\textbf{策略评估}:
\begin{itemize}
    \item 使用样本回报估计:$V(s) \approx \frac{1}{n} \sum_{i=1}^{n} G_i$
    \item $V(s)$ 是样本平均,是对 $v_\pi(s)$ 的估计
    \item 当样本数 $n \to \infty$ 时,$V(s) \to v_\pi(s)$(大数定律)
    \item 在有限样本下,$V(s) \approx v_\pi(s)$(近似相等)
\end{itemize}

\subsection{时序差分学习}

\textbf{策略评估}:
\begin{itemize}
    \item 使用TD更新:$V(S_t) \gets V(S_t) + \alpha [R_{t+1} + \gamma V(S_{t+1}) - V(S_t)]$
    \item $V$ 是逐步更新的估计值
    \item 在适当条件下,$V \to v_\pi$(收敛)
    \item 在收敛前,$V \approx v_\pi$(近似)
\end{itemize}

\section{符号约定}

\subsection{标准符号}

\textbf{真实价值函数}:
\begin{itemize}
    \item $v_\pi(s)$:策略 $\pi$ 的状态价值函数(真实值)
    \item $q_\pi(s, a)$:策略 $\pi$ 的动作价值函数(真实值)
    \item $v_*(s)$:最优状态价值函数(真实值)
    \item $q_*(s, a)$:最优动作价值函数(真实值)
\end{itemize}

\textbf{估计价值函数}:
\begin{itemize}
    \item $V(s)$:状态价值函数的估计
    \item $Q(s, a)$:动作价值函数的估计
    \item $\hat{v}(s, w)$:参数化的状态价值函数估计($w$ 是参数)
    \item $\hat{q}(s, a, w)$:参数化的动作价值函数估计
\end{itemize}

\subsection{在算法中的使用}

\textbf{策略评估的目标}:
\begin{equation}
V(s) \to v_\pi(s) \quad \text{对所有 } s
\end{equation}

\textbf{一致性条件}:
\begin{equation}
V(s) = v_\pi(s) \quad \text{对所有 } s
\end{equation}

\textbf{等价表述}:
\begin{equation}
V = v_\pi
\end{equation}

\section{为什么需要区分?}

\subsection{算法设计}

\textbf{理解算法目标}:
\begin{itemize}
    \item 算法的目标是使 $V \to v_\pi$
    \item 需要知道 $V$ 是估计值,$v_\pi$ 是目标值
    \item 算法通过更新 $V$ 使其接近 $v_\pi$
\end{itemize}

\textbf{理解收敛性}:
\begin{itemize}
    \item 收敛意味着 $V = v_\pi$
    \item 在收敛前,$V \neq v_\pi$(不一致)
    \item 算法通过迭代使它们逐渐接近
\end{itemize}

\subsection{理解GPI}

\textbf{策略评估}:
\begin{itemize}
    \item 目标:使 $V \to v_\pi$(使价值函数与策略一致)
    \item 过程:更新 $V$,使其满足策略 $\pi$ 的贝尔曼方程
    \item 结果:当 $V = v_\pi$ 时,评估完成
\end{itemize}

\textbf{策略改进}:
\begin{itemize}
    \item 输入:价值函数 $V$(可能是 $v_\pi$,也可能不是)
    \item 过程:基于 $V$ 改进策略
    \item 结果:新策略 $\pi'$
\end{itemize}

\section{具体计算例子}

\subsection{Gridworld例子}

考虑一个简单的Gridworld,策略 $\pi$ 是等概率随机策略。

\textbf{真实价值函数 $v_\pi$}(理论值):

假设通过解析计算或无限次迭代,我们得到:
\begin{align}
v_\pi(s_2) &= -10.5 \\
v_\pi(s_5) &= -12.3 \\
v_\pi(s_8) &= -11.8
\end{align}

这些是策略 $\pi$ 的\textbf{真实价值}。

\textbf{价值函数 $V$}(算法中的估计值):

\textbf{第0次迭代}:
\begin{align}
V^0(s_2) &= 0 \quad \text{(初始值,不等于 } v_\pi(s_2) = -10.5 \text{)} \\
V^0(s_5) &= 0 \quad \text{(初始值,不等于 } v_\pi(s_5) = -12.3 \text{)} \\
V^0(s_8) &= 0 \quad \text{(初始值,不等于 } v_\pi(s_8) = -11.8 \text{)}
\end{align}

\textbf{第1次迭代}:
\begin{align}
V^1(s_2) &= -1 \quad \text{(更接近 } v_\pi(s_2) = -10.5 \text{,但仍不相等)} \\
V^1(s_5) &= -1 \quad \text{(更接近 } v_\pi(s_5) = -12.3 \text{,但仍不相等)} \\
V^1(s_8) &= -1 \quad \text{(更接近 } v_\pi(s_8) = -11.8 \text{,但仍不相等)}
\end{align}

\textbf{继续迭代},逐渐接近真实值。

\textbf{收敛后}(假设第100次迭代):
\begin{align}
V^{100}(s_2) &= -10.5 = v_\pi(s_2) \quad \text{(相等!)} \\
V^{100}(s_5) &= -12.3 = v_\pi(s_5) \quad \text{(相等!)} \\
V^{100}(s_8) &= -11.8 = v_\pi(s_8) \quad \text{(相等!)}
\end{align}

此时 $V^{100} = v_\pi$,价值函数与策略一致。

\section{总结}

\subsection{核心区别}

\begin{enumerate}
    \item \textbf{$v_\pi$}:策略 $\pi$ 的\textbf{真实价值函数},理论值,由策略和环境唯一确定
    
    \item \textbf{$V$}:算法中\textbf{估计的价值函数},实际使用的值,通过算法迭代更新
    
    \item \textbf{关系}:$V$ 是 $v_\pi$ 的估计,算法的目标是使 $V \to v_\pi$
    
    \item \textbf{一致性}:当 $V = v_\pi$ 时,我们说价值函数与策略一致
\end{enumerate}

\subsection{关键公式}

\textbf{真实价值函数定义}:
\begin{equation}
v_\pi(s) = \mathbb{E}_\pi[G_t | S_t = s]
\end{equation}

\textbf{真实价值函数的贝尔曼方程}:
\begin{equation}
v_\pi(s) = \sum_{a} \pi(a|s) \sum_{s', r} p(s', r | s, a) [r + \gamma v_\pi(s')]
\end{equation}

\textbf{价值函数的迭代更新}:
\begin{equation}
V^{k+1}(s) = \sum_{a} \pi(a|s) \sum_{s', r} p(s', r | s, a) [r + \gamma V^k(s')]
\end{equation}

\textbf{收敛条件}:
\begin{equation}
V^k = V^{k+1} \implies V^k = v_\pi
\end{equation}

\subsection{直观理解}

\begin{quote}
\textbf{$v_\pi$} 是策略 $\pi$ 的"标准答案",是理论上应该得到的价值。

\textbf{$V$} 是我们算法中"正在计算的答案",通过迭代逐渐接近"标准答案"。

当 $V = v_\pi$ 时,我们的"答案"正确了,价值函数与策略一致。
\end{quote}

\subsection{类比}

\begin{itemize}
    \item \textbf{$v_\pi$}:就像数学题的"标准答案"
    \item \textbf{$V$}:就像我们"正在计算的答案"
    \item \textbf{策略评估}:就像"检查计算过程,使答案接近标准答案"
    \item \textbf{一致性}:就像"我们的答案等于标准答案"
\end{itemize}

\end{document}

