\documentclass[12pt,a4paper]{article}
\usepackage[UTF8]{ctex}
\usepackage{amsmath}
\usepackage{amssymb}
\usepackage{amsthm}
\usepackage{geometry}
\usepackage{hyperref}
\usepackage{enumitem}
\usepackage{xcolor}
\usepackage{bm}
\usepackage{tikz}
\usepackage{booktabs}
\usepackage{graphicx}

\geometry{margin=2.5cm}

\title{例题4.2:Jack的租车问题详解}
\subtitle{策略迭代的实际应用}
\author{}
\date{}

\newtheorem{definition}{定义}
\newtheorem{theorem}{定理}
\newtheorem{proposition}{命题}
\newtheorem{example}{示例}
\newtheorem{remark}{注记}

\begin{document}

\maketitle

\tableofcontents
\newpage

\section{问题描述}

\subsection{问题背景}

Jack管理一家全国性租车公司的两个地点。每天,每个地点都有一些顾客来租车。

\textbf{业务规则}:
\begin{itemize}
    \item 如果Jack有车可用,他租出去,全国公司给他 \$10 的信用
    \item 如果某个地点没有车,则失去这笔业务
    \item 车在归还后的第二天才能再次出租
    \item 为了确保车在需要的地方可用,Jack可以在夜间在两个地点之间移动车辆,每移动一辆车成本为 \$2
\end{itemize}

\subsection{随机过程}

\textbf{租车请求和归还数量}:
\begin{itemize}
    \item 每个地点的租车请求和归还数量是\textbf{泊松随机变量}
    \item 泊松分布:$\Pr\{N = n\} = \frac{\lambda^n}{n!} e^{-\lambda}$
    \item 其中 $\lambda$ 是期望数量
\end{itemize}

\textbf{参数设置}:
\begin{itemize}
    \item 地点1的租车请求:$\lambda = 3$
    \item 地点2的租车请求:$\lambda = 4$
    \item 地点1的归还:$\lambda = 3$
    \item 地点2的归还:$\lambda = 2$
\end{itemize}

\subsection{约束条件}

\begin{itemize}
    \item \textbf{最大车辆数}:每个地点最多20辆车(超出部分返回全国公司,从问题中消失)
    \item \textbf{最大移动数}:每晚最多可以在两个地点之间移动5辆车
    \item \textbf{折扣率}:$\gamma = 0.9$
    \item \textbf{任务类型}:持续有限MDP(continuing finite MDP)
    \item \textbf{时间步}:天
\end{itemize}

\section{MDP建模}

\subsection{状态空间}

\textbf{状态}:一天结束时每个地点的车辆数

\begin{equation}
s = (n_1, n_2)
\end{equation}

其中:
\begin{itemize}
    \item $n_1$:地点1的车辆数,$n_1 \in \{0, 1, 2, \ldots, 20\}$
    \item $n_2$:地点2的车辆数,$n_2 \in \{0, 1, 2, \ldots, 20\}$
\end{itemize}

\textbf{状态空间大小}:$21 \times 21 = 441$ 个状态

\subsection{动作空间}

\textbf{动作}:夜间在两个地点之间移动的净车辆数

\begin{equation}
a \in \{-5, -4, -3, -2, -1, 0, 1, 2, 3, 4, 5\}
\end{equation}

其中:
\begin{itemize}
    \item $a > 0$:从地点1移动到地点2 $a$ 辆车
    \item $a < 0$:从地点2移动到地点1 $|a|$ 辆车
    \item $a = 0$:不移动车辆
\end{itemize}

\textbf{约束}:
\begin{itemize}
    \item 移动后,每个地点的车辆数必须在 $[0, 20]$ 范围内
    \item 即:$0 \leq n_1 - a \leq 20$ 且 $0 \leq n_2 + a \leq 20$
\end{itemize}

\subsection{奖励函数}

\textbf{即时奖励}:

\begin{equation}
R_{t+1} = \text{租车收入} - \text{移动成本}
\end{equation}

详细计算:
\begin{itemize}
    \item \textbf{租车收入}:每个地点租出的车辆数 $\times \$10$
    \item \textbf{移动成本}:移动的车辆数 $\times \$2$
\end{itemize}

\textbf{数学表达}:

\begin{equation}
R_{t+1} = 10 \times (\text{地点1租出数} + \text{地点2租出数}) - 2 \times |a|
\end{equation}

\subsection{状态转移}

\textbf{转移过程}:

\begin{enumerate}
    \item \textbf{夜间移动}:根据动作 $a$,车辆从地点1移动到地点2(或相反)
    \begin{align}
    n_1' &= n_1 - a \\
    n_2' &= n_2 + a
    \end{align}
    
    \item \textbf{第二天}:
    \begin{itemize}
        \item 归还:每个地点归还的车辆数(泊松分布)
        \item 租车:每个地点的租车请求数(泊松分布)
    \end{itemize}
    
    \item \textbf{新状态}:
    \begin{align}
    n_1^{\text{new}} &= \min(20, n_1' + \text{归还数}_1 - \text{租出数}_1) \\
    n_2^{\text{new}} &= \min(20, n_2' + \text{归还数}_2 - \text{租出数}_2)
    \end{align}
\end{enumerate}

\section{策略迭代算法}

\subsection{算法流程}

策略迭代算法包含两个步骤:

\begin{enumerate}
    \item \textbf{策略评估}:计算当前策略 $\pi$ 的价值函数 $v_\pi$
    \item \textbf{策略改进}:基于 $v_\pi$ 改进策略,得到新策略 $\pi'$
    \item 重复步骤1和2,直到策略不再改变
\end{enumerate}

\subsection{策略评估}

对于策略 $\pi$,使用迭代策略评估:

\begin{equation}
v_\pi^{k+1}(s) = \sum_{a} \pi(a | s) \sum_{s', r} p(s', r | s, a) [r + \gamma v_\pi^k(s')]
\end{equation}

对于确定性策略,$\pi(a | s) = 1$ 对于 $a = \pi(s)$,否则为0:

\begin{equation}
v_\pi^{k+1}(s) = \sum_{s', r} p(s', r | s, \pi(s)) [r + \gamma v_\pi^k(s')]
\end{equation}

\subsection{策略改进}

基于当前价值函数 $v_\pi$,计算新策略:

\begin{equation}
\pi'(s) = \arg\max_{a} \sum_{s', r} p(s', r | s, a) [r + \gamma v_\pi(s')]
\end{equation}

即:在每个状态选择使期望回报最大的动作。

\section{问题求解}

\subsection{初始策略}

\textbf{策略 $\pi_0$}:从不移动任何车辆($a = 0$ 对所有状态)

\subsection{策略迭代过程}

根据图4.2,策略迭代过程如下:

\textbf{策略 $\pi_0$}(初始策略):
\begin{itemize}
    \item 所有状态都不移动车辆
    \item 这是最保守的策略
\end{itemize}

\textbf{策略 $\pi_1$}(第1次改进):
\begin{itemize}
    \item 在某些状态下开始移动车辆
    \item 策略得到改进
\end{itemize}

\textbf{策略 $\pi_2$}(第2次改进):
\begin{itemize}
    \item 进一步优化车辆移动
    \item 策略继续改进
\end{itemize}

\textbf{策略 $\pi_3$}(第3次改进):
\begin{itemize}
    \item 更精细的车辆分配
    \item 策略继续改进
\end{itemize}

\textbf{策略 $\pi_4$}(第4次改进,最优策略):
\begin{itemize}
    \item 策略不再改变
    \item 这是最优策略
\end{itemize}

\subsection{最优策略的特征}

从图4.2可以看出最优策略 $\pi_4$ 的特征:

\begin{itemize}
    \item \textbf{车辆分配模式}:
    \begin{itemize}
        \item 当地点1车辆多、地点2车辆少时,从地点1移动到地点2
        \item 当地点2车辆多、地点1车辆少时,从地点2移动到地点1
        \item 移动数量取决于两个地点的车辆差异
    \end{itemize}
    
    \item \textbf{边界行为}:
    \begin{itemize}
        \item 当某个地点车辆数接近20时,会移动车辆到另一个地点
        \item 当某个地点车辆数为0时,会从另一个地点移动车辆过来
    \end{itemize}
    
    \item \textbf{最优性}:
    \begin{itemize}
        \item 策略在4次迭代后收敛
        \item 每个后续策略都严格优于前一个策略
        \item 最终策略是最优的
    \end{itemize}
\end{itemize}

\section{价值函数}

\subsection{最优价值函数 $v_*$}

图4.2显示了最优价值函数 $v_*$ 的等高线图。

\textbf{特征}:
\begin{itemize}
    \item 价值函数随两个地点的车辆数变化
    \item 当两个地点的车辆数平衡时,价值较高
    \item 当某个地点车辆数过多或过少时,价值较低
\end{itemize}

\section{关键洞察}

\subsection{策略迭代的收敛性}

\textbf{观察}:
\begin{itemize}
    \item 策略迭代在4次迭代后收敛
    \item 这比预期的要快
    \item 说明策略迭代算法非常高效
\end{itemize}

\textbf{原因}:
\begin{itemize}
    \item 策略改进定理保证每次迭代都严格改进策略
    \item 状态空间虽然大(441个状态),但策略空间相对简单
    \item 最优策略具有清晰的结构
\end{itemize}

\subsection{问题的复杂性}

\textbf{挑战}:
\begin{itemize}
    \item \textbf{状态空间大}:441个状态
    \item \textbf{随机性}:租车请求和归还是随机的(泊松分布)
    \item \textbf{转移概率复杂}:需要考虑所有可能的租车和归还组合
    \item \textbf{动作空间}:每个状态有11个可能的动作(考虑约束后可能更少)
\end{itemize}

\textbf{动态规划的优势}:
\begin{itemize}
    \item 可以处理随机性
    \item 可以处理复杂的转移概率
    \item 保证找到最优策略
    \item 计算效率高(相对于枚举所有策略)
\end{itemize}

\section{具体计算示例}

\subsection{示例:计算状态(10, 10)的价值函数}

让我们详细计算状态 $s = (10, 10)$(地点1有10辆车,地点2有10辆车)在初始策略 $\pi_0$(不移动车辆,$a = 0$)下的价值函数。

\subsection{步骤1:确定动作}

在策略 $\pi_0$ 下,状态 $(10, 10)$ 的动作是 $a = 0$(不移动车辆)。

\subsection{步骤2:夜间移动后的车辆数}

由于 $a = 0$,车辆数不变:
\begin{align}
n_1' &= 10 - 0 = 10 \\
n_2' &= 10 + 0 = 10
\end{align}

\subsection{步骤3:计算可能的租车和归还组合}

我们需要考虑所有可能的租车请求和归还组合。由于泊松分布有无限多个可能值,我们只考虑概率较大的情况。

\textbf{地点1}:
\begin{itemize}
    \item 归还数 $r_1$:泊松分布,$\lambda = 3$
    \item 租车请求数 $d_1$:泊松分布,$\lambda = 3$
\end{itemize}

\textbf{地点2}:
\begin{itemize}
    \item 归还数 $r_2$:泊松分布,$\lambda = 2$
    \item 租车请求数 $d_2$:泊松分布,$\lambda = 4$
\end{itemize}

\textbf{概率计算}(泊松分布):
\begin{equation}
\Pr\{X = k\} = \frac{\lambda^k}{k!} e^{-\lambda}
\end{equation}

\subsection{步骤4:具体数值计算}

让我们计算几个高概率的组合:

\textbf{组合1}:$r_1 = 3, r_2 = 2, d_1 = 3, d_2 = 4$

\textbf{概率}:
\begin{align}
\Pr\{R_1 = 3\} &= \frac{3^3}{3!} e^{-3} = \frac{27}{6} \times 0.0498 \approx 0.224 \\
\Pr\{R_2 = 2\} &= \frac{2^2}{2!} e^{-2} = \frac{4}{2} \times 0.1353 \approx 0.271 \\
\Pr\{D_1 = 3\} &= \frac{3^3}{3!} e^{-3} = \frac{27}{6} \times 0.0498 \approx 0.224 \\
\Pr\{D_2 = 4\} &= \frac{4^4}{4!} e^{-4} = \frac{256}{24} \times 0.0183 \approx 0.195
\end{align}

\textbf{联合概率}:
\begin{align}
p_1 &= \Pr\{R_1 = 3\} \times \Pr\{R_2 = 2\} \times \Pr\{D_1 = 3\} \times \Pr\{D_2 = 4\} \\
    &= 0.224 \times 0.271 \times 0.224 \times 0.195 \\
    &\approx 0.00265
\end{align}

\textbf{新状态计算}:
\begin{itemize}
    \item 地点1:$n_1' + r_1 = 10 + 3 = 13$,租车请求 $d_1 = 3$
    \item 租出数:$\min(13, 3) = 3$
    \item 新车辆数:$n_1^{\text{new}} = \min(20, 13 - 3) = 10$
    
    \item 地点2:$n_2' + r_2 = 10 + 2 = 12$,租车请求 $d_2 = 4$
    \item 租出数:$\min(12, 4) = 4$
    \item 新车辆数:$n_2^{\text{new}} = \min(20, 12 - 4) = 8$
\end{itemize}

\textbf{新状态}:$s' = (10, 8)$

\textbf{奖励}:
\begin{align}
r_1 &= 10 \times (\text{租出数}_1 + \text{租出数}_2) - 2 \times |a| \\
    &= 10 \times (3 + 4) - 2 \times 0 \\
    &= 70
\end{align}

\textbf{贡献}:
\begin{align}
\text{贡献}_1 &= p_1 \times [r_1 + \gamma v_\pi^k(10, 8)] \\
            &\approx 0.00265 \times [70 + 0.9 \times v_\pi^k(10, 8)]
\end{align}

\textbf{组合2}:$r_1 = 2, r_2 = 1, d_1 = 2, d_2 = 3$

\textbf{概率}:
\begin{align}
\Pr\{R_1 = 2\} &= \frac{3^2}{2!} e^{-3} = \frac{9}{2} \times 0.0498 \approx 0.224 \\
\Pr\{R_2 = 1\} &= \frac{2^1}{1!} e^{-2} = 2 \times 0.1353 \approx 0.271 \\
\Pr\{D_1 = 2\} &= \frac{3^2}{2!} e^{-3} = \frac{9}{2} \times 0.0498 \approx 0.224 \\
\Pr\{D_2 = 3\} &= \frac{4^3}{3!} e^{-4} = \frac{64}{6} \times 0.0183 \approx 0.195
\end{align}

\textbf{联合概率}:
\begin{align}
p_2 &\approx 0.224 \times 0.271 \times 0.224 \times 0.195 \\
    &\approx 0.00265
\end{align}

\textbf{新状态计算}:
\begin{itemize}
    \item 地点1:$10 + 2 = 12$,租车请求 $2$,租出 $2$,新状态 $10$
    \item 地点2:$10 + 1 = 11$,租车请求 $3$,租出 $3$,新状态 $8$
\end{itemize}

\textbf{新状态}:$s' = (10, 8)$

\textbf{奖励}:
\begin{align}
r_2 &= 10 \times (2 + 3) - 0 = 50
\end{align}

\textbf{贡献}:
\begin{align}
\text{贡献}_2 &\approx 0.00265 \times [50 + 0.9 \times v_\pi^k(10, 8)]
\end{align}

\subsection{步骤5:完整计算}

为了计算 $v_\pi(10, 10)$,我们需要对所有可能的 $(r_1, r_2, d_1, d_2)$ 组合求和。

\textbf{迭代策略评估公式}:

\begin{align}
v_\pi^{k+1}(10, 10) &= \sum_{r_1=0}^{\infty} \sum_{r_2=0}^{\infty} \sum_{d_1=0}^{\infty} \sum_{d_2=0}^{\infty} \\
&\quad \Pr\{R_1 = r_1\} \times \Pr\{R_2 = r_2\} \times \Pr\{D_1 = d_1\} \times \Pr\{D_2 = d_2\} \\
&\quad \times [r(r_1, r_2, d_1, d_2) + \gamma v_\pi^k(s'(r_1, r_2, d_1, d_2))]
\end{align}

\textbf{数值近似}:

由于泊松分布的尾部概率很小,我们可以截断求和。通常考虑 $k \leq 10$ 就足够了。

\textbf{第1次迭代}($k = 0$,$v_\pi^0(s) = 0$ 对所有 $s$):

\begin{align}
v_\pi^1(10, 10) &\approx \sum_{r_1=0}^{10} \sum_{r_2=0}^{10} \sum_{d_1=0}^{10} \sum_{d_2=0}^{10} \\
&\quad \Pr\{R_1 = r_1\} \times \Pr\{R_2 = r_2\} \times \Pr\{D_1 = d_1\} \times \Pr\{D_2 = d_2\} \\
&\quad \times r(r_1, r_2, d_1, d_2)
\end{align}

\textbf{期望奖励计算}:

\begin{align}
\mathbb{E}[R | s = (10, 10), a = 0] &= 10 \times (\mathbb{E}[\text{租出数}_1] + \mathbb{E}[\text{租出数}_2]) \\
                                    &= 10 \times (\mathbb{E}[\min(10 + R_1, D_1)] + \mathbb{E}[\min(10 + R_2, D_2)])
\end{align}

由于 $R_1 \sim \text{Poisson}(3)$,$R_2 \sim \text{Poisson}(2)$,$D_1 \sim \text{Poisson}(3)$,$D_2 \sim \text{Poisson}(4)$:

\begin{align}
\mathbb{E}[\min(10 + R_1, D_1)] &\approx \mathbb{E}[\min(10 + 3, 3)] = 3 \\
\mathbb{E}[\min(10 + R_2, D_2)] &\approx \mathbb{E}[\min(10 + 2, 4)] = 4
\end{align}

因此:
\begin{align}
v_\pi^1(10, 10) &\approx 10 \times (3 + 4) = 70
\end{align}

\subsection{步骤6:策略改进示例}

现在让我们展示如何对状态 $(10, 10)$ 进行策略改进。

\textbf{计算每个动作的动作价值}:

对于动作 $a = 0$(不移动):
\begin{align}
q_\pi(10, 10, 0) &\approx 70 + 0.9 \times v_\pi(10, 10) \\
                 &\approx 70 + 0.9 \times 70 = 133
\end{align}

对于动作 $a = 1$(从地点1移动1辆车到地点2):
\begin{itemize}
    \item 移动后:$n_1' = 9$,$n_2' = 11$
    \item 移动成本:$2 \times 1 = 2$
    \item 期望收入会改变(因为车辆分配改变了)
\end{itemize}

假设 $q_\pi(10, 10, 1) \approx 135$(需要完整计算)

对于动作 $a = -1$(从地点2移动1辆车到地点1):
\begin{itemize}
    \item 移动后:$n_1' = 11$,$n_2' = 9$
    \item 移动成本:$2 \times 1 = 2$
\end{itemize}

假设 $q_\pi(10, 10, -1) \approx 132$

\textbf{策略改进}:

\begin{align}
\pi'(10, 10) &= \arg\max_{a} q_\pi(10, 10, a) \\
             &= \arg\max\{133, 135, 132, \ldots\} \\
             &= 1
\end{align}

因此,在状态 $(10, 10)$,改进后的策略会选择动作 $a = 1$(从地点1移动1辆车到地点2)。

\subsection{完整数值示例:状态(5, 5)的详细计算}

让我们更详细地计算状态 $s = (5, 5)$ 在策略 $\pi_0$($a = 0$)下的价值函数,展示完整的计算过程。

\textbf{初始状态}:$n_1 = 5$,$n_2 = 5$

\textbf{动作}:$a = 0$(不移动)

\textbf{移动后}:$n_1' = 5$,$n_2' = 5$

\textbf{计算几个具体的转移}:

\textbf{转移1}:$r_1 = 3, r_2 = 2, d_1 = 3, d_2 = 4$

\textbf{概率计算}:
\begin{align}
\Pr\{R_1 = 3\} &= \frac{3^3}{3!} e^{-3} = \frac{27}{6} \times 0.049787 = 0.224042 \\
\Pr\{R_2 = 2\} &= \frac{2^2}{2!} e^{-2} = \frac{4}{2} \times 0.135335 = 0.270671 \\
\Pr\{D_1 = 3\} &= \frac{3^3}{3!} e^{-3} = 0.224042 \\
\Pr\{D_2 = 4\} &= \frac{4^4}{4!} e^{-4} = \frac{256}{24} \times 0.018316 = 0.195367
\end{align}

\textbf{联合概率}:
\begin{align}
p_1 &= 0.224042 \times 0.270671 \times 0.224042 \times 0.195367 \\
    &= 0.002653
\end{align}

\textbf{新状态计算}:
\begin{itemize}
    \item 地点1:可用车辆 $5 + 3 = 8$,请求 $3$,租出 $\min(8, 3) = 3$,剩余 $8 - 3 = 5$
    \item 地点2:可用车辆 $5 + 2 = 7$,请求 $4$,租出 $\min(7, 4) = 4$,剩余 $7 - 4 = 3$
    \item 新状态:$s' = (5, 3)$
\end{itemize}

\textbf{奖励}:
\begin{align}
r_1 &= 10 \times (3 + 4) - 2 \times 0 = 70
\end{align}

\textbf{转移2}:$r_1 = 2, r_2 = 1, d_1 = 2, d_2 = 3$

\textbf{概率}:
\begin{align}
\Pr\{R_1 = 2\} &= \frac{3^2}{2!} e^{-3} = \frac{9}{2} \times 0.049787 = 0.224042 \\
\Pr\{R_2 = 1\} &= \frac{2^1}{1!} e^{-2} = 2 \times 0.135335 = 0.270671 \\
\Pr\{D_1 = 2\} &= \frac{3^2}{2!} e^{-3} = 0.224042 \\
\Pr\{D_2 = 3\} &= \frac{4^3}{3!} e^{-4} = \frac{64}{6} \times 0.018316 = 0.195367
\end{align}

\textbf{联合概率}:
\begin{align}
p_2 &= 0.224042 \times 0.270671 \times 0.224042 \times 0.195367 \\
    &= 0.002653
\end{align}

\textbf{新状态}:
\begin{itemize}
    \item 地点1:$5 + 2 = 7$,请求 $2$,租出 $2$,剩余 $5$
    \item 地点2:$5 + 1 = 6$,请求 $3$,租出 $3$,剩余 $3$
    \item 新状态:$s' = (5, 3)$
\end{itemize}

\textbf{奖励}:
\begin{align}
r_2 &= 10 \times (2 + 3) - 0 = 50
\end{align}

\textbf{转移3}:$r_1 = 4, r_2 = 3, d_1 = 4, d_2 = 5$

\textbf{概率}:
\begin{align}
\Pr\{R_1 = 4\} &= \frac{3^4}{4!} e^{-3} = \frac{81}{24} \times 0.049787 = 0.168031 \\
\Pr\{R_2 = 3\} &= \frac{2^3}{3!} e^{-2} = \frac{8}{6} \times 0.135335 = 0.180447 \\
\Pr\{D_1 = 4\} &= \frac{3^4}{4!} e^{-3} = 0.168031 \\
\Pr\{D_2 = 5\} &= \frac{4^5}{5!} e^{-4} = \frac{1024}{120} \times 0.018316 = 0.156293
\end{align}

\textbf{联合概率}:
\begin{align}
p_3 &= 0.168031 \times 0.180447 \times 0.168031 \times 0.156293 \\
    &= 0.000797
\end{align}

\textbf{新状态}:
\begin{itemize}
    \item 地点1:$5 + 4 = 9$,请求 $4$,租出 $4$,剩余 $5$
    \item 地点2:$5 + 3 = 8$,请求 $5$,租出 $5$,剩余 $3$
    \item 新状态:$s' = (5, 3)$
\end{itemize}

\textbf{奖励}:
\begin{align}
r_3 &= 10 \times (4 + 5) - 0 = 90
\end{align}

\textbf{第1次迭代的价值函数}(假设 $v_\pi^0(s') = 0$):

\begin{align}
v_\pi^1(5, 5) &\approx p_1 \times r_1 + p_2 \times r_2 + p_3 \times r_3 + \cdots \\
             &\approx 0.002653 \times 70 + 0.002653 \times 50 + 0.000797 \times 90 + \cdots
\end{align}

由于需要对所有可能的组合求和,实际计算中会使用截断的泊松分布(通常 $k \leq 10$ 或 $k \leq 15$)。

\textbf{近似计算}:

使用期望值近似:
\begin{align}
\mathbb{E}[\text{租出数}_1] &\approx \mathbb{E}[\min(5 + R_1, D_1)] \\
                           &\approx \min(5 + 3, 3) = 3 \\
\mathbb{E}[\text{租出数}_2] &\approx \mathbb{E}[\min(5 + R_2, D_2)] \\
                           &\approx \min(5 + 2, 4) = 4
\end{align}

因此:
\begin{align}
v_\pi^1(5, 5) &\approx 10 \times (3 + 4) = 70
\end{align}

\textbf{第2次迭代}($k = 1$):

假设 $v_\pi^1(5, 3) \approx 65$,$v_\pi^1(5, 5) \approx 70$,则:

\begin{align}
v_\pi^2(5, 5) &\approx \sum_{r_1, r_2, d_1, d_2} p(r_1, r_2, d_1, d_2) \times [r + 0.9 \times v_\pi^1(s')] \\
             &\approx 70 + 0.9 \times \mathbb{E}[v_\pi^1(S')] \\
             &\approx 70 + 0.9 \times 67.5 \\
             &\approx 70 + 60.75 = 130.75
\end{align}

继续迭代直到收敛。

\section{计算细节}

\subsection{转移概率的计算}

对于状态 $s = (n_1, n_2)$ 和动作 $a$:

\textbf{步骤1:夜间移动后的车辆数}:
\begin{align}
n_1' &= \max(0, \min(20, n_1 - a)) \\
n_2' &= \max(0, \min(20, n_2 + a))
\end{align}

\textbf{步骤2:计算所有可能的租车和归还组合}:

对于每个可能的组合 $(r_1, r_2, d_1, d_2)$:
\begin{itemize}
    \item $r_1$:地点1的归还数(泊松分布,$\lambda = 3$)
    \item $r_2$:地点2的归还数(泊松分布,$\lambda = 2$)
    \item $d_1$:地点1的租车请求数(泊松分布,$\lambda = 3$)
    \item $d_2$:地点2的租车请求数(泊松分布,$\lambda = 4$)
\end{itemize}

\textbf{步骤3:计算新状态}:
\begin{align}
n_1^{\text{new}} &= \min(20, n_1' + r_1 - \min(n_1' + r_1, d_1)) \\
n_2^{\text{new}} &= \min(20, n_2' + r_2 - \min(n_2' + r_2, d_2))
\end{align}

其中租出数 = $\min(\text{可用车辆数}, \text{租车请求数})$

\textbf{步骤4:计算概率和奖励}:
\begin{align}
p((n_1^{\text{new}}, n_2^{\text{new}}), r | (n_1, n_2), a) &= \Pr\{R_1 = r_1\} \times \Pr\{R_2 = r_2\} \\
&\quad \times \Pr\{D_1 = d_1\} \times \Pr\{D_2 = d_2\}
\end{align}

其中:
\begin{align}
\Pr\{R_i = r_i\} &= \frac{\lambda_{r_i}^{r_i}}{r_i!} e^{-\lambda_{r_i}} \\
\Pr\{D_i = d_i\} &= \frac{\lambda_{d_i}^{d_i}}{d_i!} e^{-\lambda_{d_i}}
\end{align}

奖励:
\begin{equation}
r = 10 \times (\min(n_1' + r_1, d_1) + \min(n_2' + r_2, d_2)) - 2 \times |a|
\end{equation}

\subsection{策略评估的实现}

\textbf{迭代策略评估}:

\begin{algorithm}[H]
\caption{策略评估(Jack的租车问题)}
\begin{algorithmic}[1]
\REQUIRE 策略 $\pi$,环境动态 $p(s', r | s, a)$
\ENSURE 价值函数 $v_\pi$
\STATE 初始化 $v_\pi^0(s) = 0$ 对所有状态 $s$
\REPEAT
    \FOR{每个状态 $s = (n_1, n_2)$}
        \STATE $v_{\text{new}}(s) \gets 0$
        \STATE $a \gets \pi(s)$
        \FOR{所有可能的 $(r_1, r_2, d_1, d_2)$}
            \STATE 计算新状态 $s'$ 和奖励 $r$
            \STATE $v_{\text{new}}(s) \gets v_{\text{new}}(s) + p(s', r | s, a) \times [r + \gamma v_\pi(s')]$
        \ENDFOR
        \STATE $v_\pi(s) \gets v_{\text{new}}(s)$
    \ENDFOR
\UNTIL{收敛}
\RETURN $v_\pi$
\end{algorithmic}
\end{algorithmic}

\section{总结}

\subsection{核心要点}

\begin{enumerate}
    \item \textbf{问题建模}:
    \begin{itemize}
        \item 将租车问题建模为持续有限MDP
        \item 状态:两个地点的车辆数
        \item 动作:夜间移动的车辆数
        \item 奖励:租车收入减去移动成本
    \end{itemize}
    
    \item \textbf{随机性}:
    \begin{itemize}
        \item 租车请求和归还是泊松随机变量
        \item 需要计算所有可能的组合
        \item 转移概率是多个泊松分布的乘积
    \end{itemize}
    
    \item \textbf{策略迭代}:
    \begin{itemize}
        \item 从"不移动车辆"策略开始
        \item 通过策略评估和策略改进逐步优化
        \item 在4次迭代后收敛到最优策略
    \end{itemize}
    
    \item \textbf{最优策略}:
    \begin{itemize}
        \item 根据两个地点的车辆数动态调整
        \item 平衡两个地点的车辆分配
        \item 最大化长期期望回报
    \end{itemize}
\end{enumerate}

\subsection{学习价值}

例题4.2展示了:
\begin{itemize}
    \item \textbf{实际应用}:如何将实际问题建模为MDP
    \item \textbf{策略迭代}:策略迭代算法在实际问题中的应用
    \item \textbf{收敛性}:策略迭代通常收敛很快
    \item \textbf{最优性}:动态规划保证找到最优策略
    \item \textbf{复杂性}:可以处理大规模状态空间和随机性
\end{itemize}

\vspace{1cm}

\textbf{参考文献}:
\begin{itemize}
    \item Sutton, R. S., \& Barto, A. G. (2018). \textit{Reinforcement Learning: An Introduction} (2nd Edition). MIT Press, Chapter 4, Example 4.2.
\end{itemize}

\end{document}

