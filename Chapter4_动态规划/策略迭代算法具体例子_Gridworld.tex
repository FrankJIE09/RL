\documentclass[12pt,a4paper]{article}
\usepackage[UTF8]{ctex}
\usepackage{amsmath}
\usepackage{amssymb}
\usepackage{amsthm}
\usepackage{geometry}
\usepackage{hyperref}
\usepackage{enumitem}
\usepackage{xcolor}
\usepackage{bm}
\usepackage{tikz}
\usepackage{booktabs}
\usepackage{array}

\geometry{margin=2.5cm}

\title{策略迭代算法具体例子:Gridworld}
\subtitle{完整的迭代过程演示}
\author{}
\date{}

\newtheorem{definition}{定义}
\newtheorem{theorem}{定理}
\newtheorem{proposition}{命题}
\newtheorem{example}{示例}
\newtheorem{remark}{注记}

\begin{document}

\maketitle

\tableofcontents
\newpage

\section{问题设置}

\subsection{Gridworld环境}

我们考虑一个简化的3×3 Gridworld:

\begin{center}
\begin{tabular}{|c|c|c|}
\hline
\textbf{终止} & $s_2$ & $s_3$ \\
\hline
$s_4$ & $s_5$ & $s_6$ \\
\hline
$s_7$ & $s_8$ & \textbf{终止} \\
\hline
\end{tabular}
\end{center}

\textbf{环境设置}:
\begin{itemize}
    \item \textbf{非终止状态}:$\mathcal{S} = \{s_2, s_3, s_4, s_5, s_6, s_7, s_8\}$(共7个状态)
    \item \textbf{终止状态}:左上角和右下角
    \item \textbf{动作空间}:每个状态有4个动作:上、下、左、右
    \item \textbf{奖励}:所有转移的奖励都是 $-1$,直到到达终止状态
    \item \textbf{折扣因子}:$\gamma = 0.9$
    \item \textbf{任务类型}:无折扣的回合任务(但为了演示,我们使用 $\gamma = 0.9$)
\end{itemize}

\subsection{状态转移规则}

\textbf{正常移动}:
\begin{itemize}
    \item 动作确定性地使智能体移动到相邻格子
    \item 奖励为 $-1$
\end{itemize}

\textbf{边界处理}:
\begin{itemize}
    \item 试图移出网格时,状态不变
    \item 奖励仍为 $-1$
\end{itemize}

\textbf{终止状态}:
\begin{itemize}
    \item 到达终止状态后,episode结束
    \item 终止状态的价值为 $0$
\end{itemize}

\section{策略迭代算法执行}

\subsection{初始化}

\textbf{初始策略 $\pi_0$}:等概率随机策略(每个动作概率 $0.25$)

\textbf{初始价值函数 $V_0$}:
\begin{equation}
V_0(s) = 0 \quad \text{对所有 } s \in \mathcal{S}
\end{equation}

\subsection{第1次迭代}

\subsubsection{步骤1:策略评估}

\textbf{目标}:计算策略 $\pi_0$ 的价值函数 $v_{\pi_0}$。

\textbf{迭代策略评估}:

\textbf{第1次评估迭代}($k=1$):

对每个状态 $s$:
\begin{equation}
V_1(s) = \sum_{a} \pi_0(a | s) \sum_{s', r} p(s', r | s, a) [r + \gamma V_0(s')]
\end{equation}

由于 $V_0(s') = 0$ 对所有 $s'$,且 $r = -1$:
\begin{equation}
V_1(s) = \sum_{a} \frac{1}{4} \sum_{s', r} p(s', r | s, a) \times (-1) = -1
\end{equation}

因此:
\begin{equation}
V_1(s) = -1 \quad \text{对所有 } s \in \mathcal{S}
\end{equation}

\textbf{第2次评估迭代}($k=2$):

对状态 $s_5$(中心状态):
\begin{align}
V_2(s_5) &= \sum_{a} \frac{1}{4} \sum_{s', r} p(s', r | s_5, a) [r + \gamma V_1(s')] \\
         &= \frac{1}{4} \times [-1 + 0.9 \times (-1)] \times 4 \\
         &= -1 - 0.9 = -1.9
\end{align}

对状态 $s_2$(第1行第2列,接近终止状态):
\begin{itemize}
    \item 上:转移到终止状态,$V_1(\text{终止}) = 0$
    \item 下:转移到 $s_5$,$V_1(s_5) = -1$
    \item 左:撞墙,状态不变,$V_1(s_2) = -1$
    \item 右:转移到 $s_3$,$V_1(s_3) = -1$
\end{itemize}

\begin{align}
V_2(s_2) &= \frac{1}{4} \times [-1 + 0.9 \times 0] + \frac{1}{4} \times [-1 + 0.9 \times (-1)] \\
         &\quad + \frac{1}{4} \times [-1 + 0.9 \times (-1)] + \frac{1}{4} \times [-1 + 0.9 \times (-1)] \\
         &= \frac{1}{4} \times [-1 + -1.9 + -1.9 + -1.9] \\
         &= \frac{1}{4} \times (-6.7) = -1.675
\end{align}

\textbf{继续迭代直到收敛}:

假设经过多次迭代后,策略 $\pi_0$ 的价值函数收敛到:
\begin{center}
\begin{tabular}{|c|c|c|}
\hline
-1.9 & -1.7 & -1.9 \\
\hline
-1.9 & -2.0 & -1.9 \\
\hline
-1.9 & -1.9 & -1.7 \\
\hline
\end{tabular}
\end{center}

即:
\begin{align}
v_{\pi_0}(s_2) &= -1.7 \\
v_{\pi_0}(s_3) &= -1.9 \\
v_{\pi_0}(s_4) &= -1.9 \\
v_{\pi_0}(s_5) &= -2.0 \\
v_{\pi_0}(s_6) &= -1.9 \\
v_{\pi_0}(s_7) &= -1.9 \\
v_{\pi_0}(s_8) &= -1.7
\end{align}

\subsubsection{步骤2:策略改进}

\textbf{目标}:基于 $v_{\pi_0}$ 改进策略,得到 $\pi_1$。

\textbf{对每个状态计算动作价值并更新策略}:

\textbf{状态 $s_2$}:

计算所有动作的动作价值:
\begin{align}
q(s_2, \text{上}) &= -1 + 0.9 \times 0 = -1.0 \quad \text{(转移到终止状态)} \\
q(s_2, \text{下}) &= -1 + 0.9 \times (-2.0) = -1 - 1.8 = -2.8 \\
q(s_2, \text{左}) &= -1 + 0.9 \times (-1.7) = -1 - 1.53 = -2.53 \\
q(s_2, \text{右}) &= -1 + 0.9 \times (-1.9) = -1 - 1.71 = -2.71
\end{align}

最大值:$\max\{-1.0, -2.8, -2.53, -2.71\} = -1.0$,对应动作"上"。

\textbf{更新策略}:
\begin{equation}
\pi_1(s_2) = \text{上}
\end{equation}

\textbf{状态 $s_5$}(中心状态):

计算所有动作的动作价值:
\begin{align}
q(s_5, \text{上}) &= -1 + 0.9 \times (-1.7) = -2.53 \\
q(s_5, \text{下}) &= -1 + 0.9 \times (-1.7) = -2.53 \\
q(s_5, \text{左}) &= -1 + 0.9 \times (-1.9) = -2.71 \\
q(s_5, \text{右}) &= -1 + 0.9 \times (-1.9) = -2.71
\end{align}

最大值:$\max\{-2.53, -2.53, -2.71, -2.71\} = -2.53$,对应动作"上"或"下"。

假设选择"上"(平局时按固定顺序选择第一个)。

\textbf{更新策略}:
\begin{equation}
\pi_1(s_5) = \text{上}
\end{equation}

\textbf{状态 $s_8$}:

计算所有动作的动作价值:
\begin{align}
q(s_8, \text{上}) &= -1 + 0.9 \times (-1.9) = -2.71 \\
q(s_8, \text{下}) &= -1 + 0.9 \times 0 = -1.0 \quad \text{(转移到终止状态)} \\
q(s_8, \text{左}) &= -1 + 0.9 \times (-1.9) = -2.71 \\
q(s_8, \text{右}) &= -1 + 0.9 \times 0 = -1.0 \quad \text{(转移到终止状态)}
\end{align}

最大值:$\max\{-2.71, -1.0, -2.71, -1.0\} = -1.0$,对应动作"下"或"右"。

假设选择"下"。

\textbf{更新策略}:
\begin{equation}
\pi_1(s_8) = \text{下}
\end{equation}

\textbf{完整的新策略 $\pi_1$}:

\begin{center}
\begin{tabular}{|c|c|c|}
\hline
\textbf{终止} & $\pi_1(s_2) = \text{上}$ & $\pi_1(s_3) = \text{上}$ \\
\hline
$\pi_1(s_4) = \text{上}$ & $\pi_1(s_5) = \text{上}$ & $\pi_1(s_6) = \text{右}$ \\
\hline
$\pi_1(s_7) = \text{右}$ & $\pi_1(s_8) = \text{下}$ & \textbf{终止} \\
\hline
\end{tabular}
\end{center}

\textbf{策略变化}:
\begin{itemize}
    \item 状态 $s_2$:从随机策略改为"上"(直接到终止状态)
    \item 状态 $s_5$:从随机策略改为"上"
    \item 状态 $s_8$:从随机策略改为"下"(直接到终止状态)
    \item 其他状态也有相应改变
\end{itemize}

\subsection{第2次迭代}

\subsubsection{步骤1:策略评估}

\textbf{目标}:计算策略 $\pi_1$ 的价值函数 $v_{\pi_1}$。

\textbf{迭代策略评估}:

由于策略 $\pi_1$ 是确定性策略,策略评估公式简化为:
\begin{equation}
V(s) = \sum_{s', r} p(s', r | s, \pi_1(s)) [r + \gamma V(s')]
\end{equation}

\textbf{第1次评估迭代}:

对状态 $s_2$(策略选择"上"):
\begin{align}
V_1(s_2) &= p(\text{终止}, -1 | s_2, \text{上}) \times [-1 + 0.9 \times 0] \\
         &= 1 \times (-1) = -1
\end{align}

对状态 $s_5$(策略选择"上"):
\begin{align}
V_1(s_5) &= p(s_2, -1 | s_5, \text{上}) \times [-1 + 0.9 \times V_0(s_2)] \\
         &= 1 \times [-1 + 0.9 \times 0] = -1
\end{align}

\textbf{第2次评估迭代}:

对状态 $s_2$:
\begin{align}
V_2(s_2) &= -1 + 0.9 \times 0 = -1
\end{align}

对状态 $s_5$:
\begin{align}
V_2(s_5) &= -1 + 0.9 \times V_1(s_2) \\
         &= -1 + 0.9 \times (-1) = -1.9
\end{align}

\textbf{继续迭代直到收敛}:

假设经过多次迭代后,策略 $\pi_1$ 的价值函数收敛到:
\begin{center}
\begin{tabular}{|c|c|c|}
\hline
-1.0 & -1.0 & -1.0 \\
\hline
-1.9 & -1.9 & -1.0 \\
\hline
-1.9 & -1.0 & -1.0 \\
\hline
\end{tabular}
\end{center}

即:
\begin{align}
v_{\pi_1}(s_2) &= -1.0 \\
v_{\pi_1}(s_3) &= -1.0 \\
v_{\pi_1}(s_4) &= -1.9 \\
v_{\pi_1}(s_5) &= -1.9 \\
v_{\pi_1}(s_6) &= -1.0 \\
v_{\pi_1}(s_7) &= -1.9 \\
v_{\pi_1}(s_8) &= -1.0
\end{align}

\subsubsection{步骤2:策略改进}

\textbf{状态 $s_5$}:

计算所有动作的动作价值:
\begin{align}
q(s_5, \text{上}) &= -1 + 0.9 \times (-1.0) = -1.9 \\
q(s_5, \text{下}) &= -1 + 0.9 \times (-1.0) = -1.9 \\
q(s_5, \text{左}) &= -1 + 0.9 \times (-1.9) = -2.71 \\
q(s_5, \text{右}) &= -1 + 0.9 \times (-1.0) = -1.9
\end{align}

最大值:$\max\{-1.9, -1.9, -2.71, -1.9\} = -1.9$,对应动作"上"、"下"或"右"。

当前策略是"上",保持不变。

\textbf{检查所有状态}:

假设所有状态的策略都没有改变,则:
\begin{equation}
\pi_2 = \pi_1
\end{equation}

\textbf{策略稳定}:算法收敛!

\subsection{最终结果}

\textbf{最优策略 $\pi_* = \pi_1$}:

\begin{center}
\begin{tabular}{|c|c|c|}
\hline
\textbf{终止} & $\uparrow$ & $\uparrow$ \\
\hline
$\uparrow$ & $\uparrow$ & $\rightarrow$ \\
\hline
$\rightarrow$ & $\downarrow$ & \textbf{终止} \\
\hline
\end{tabular}
\end{center}

\textbf{最优价值函数 $v_* = v_{\pi_1}$}:

\begin{center}
\begin{tabular}{|c|c|c|}
\hline
0 & -1.0 & -1.0 \\
\hline
-1.9 & -1.9 & -1.0 \\
\hline
-1.9 & -1.0 & 0 \\
\hline
\end{tabular}
\end{center}

\section{详细计算:状态 $s_5$ 的完整过程}

\subsection{第1次迭代:策略评估}

\textbf{初始}:$V_0(s_5) = 0$

\textbf{第1次评估迭代}:
\begin{align}
V_1(s_5) &= \sum_{a} \frac{1}{4} \sum_{s', r} p(s', r | s_5, a) [r + \gamma V_0(s')] \\
         &= \frac{1}{4} \times [-1 + 0.9 \times 0] \times 4 = -1
\end{align}

\textbf{第2次评估迭代}:
\begin{align}
V_2(s_5) &= \sum_{a} \frac{1}{4} \sum_{s', r} p(s', r | s_5, a) [r + \gamma V_1(s')] \\
         &= \frac{1}{4} \times [-1 + 0.9 \times (-1)] \times 4 \\
         &= -1 - 0.9 = -1.9
\end{align}

\textbf{第3次评估迭代}:
\begin{align}
V_3(s_5) &= \sum_{a} \frac{1}{4} \sum_{s', r} p(s', r | s_5, a) [r + \gamma V_2(s')] \\
         &\approx -1.9 \quad \text{(收敛)}
\end{align}

\subsection{第1次迭代:策略改进}

\textbf{计算动作价值}(使用收敛后的 $v_{\pi_0}$):

假设 $v_{\pi_0}(s_2) = -1.7$,$v_{\pi_0}(s_4) = -1.9$,$v_{\pi_0}(s_6) = -1.9$,$v_{\pi_0}(s_8) = -1.7$。

\begin{align}
q(s_5, \text{上}) &= -1 + 0.9 \times (-1.7) = -2.53 \\
q(s_5, \text{下}) &= -1 + 0.9 \times (-1.7) = -2.53 \\
q(s_5, \text{左}) &= -1 + 0.9 \times (-1.9) = -2.71 \\
q(s_5, \text{右}) &= -1 + 0.9 \times (-1.9) = -2.71
\end{align}

\textbf{选择最优动作}:
\begin{equation}
\arg\max_{a} q(s_5, a) = \text{上} \quad \text{或} \quad \text{下}
\end{equation}

假设选择"上"。

\textbf{更新策略}:
\begin{equation}
\pi_1(s_5) = \text{上}
\end{equation}

\subsection{第2次迭代:策略评估}

\textbf{使用策略 $\pi_1$}(在 $s_5$ 选择"上"):

\textbf{第1次评估迭代}:
\begin{align}
V_1(s_5) &= p(s_2, -1 | s_5, \text{上}) \times [-1 + 0.9 \times V_0(s_2)] \\
         &= 1 \times [-1 + 0.9 \times 0] = -1
\end{align}

\textbf{第2次评估迭代}:
\begin{align}
V_2(s_5) &= -1 + 0.9 \times V_1(s_2) \\
         &= -1 + 0.9 \times (-1) = -1.9
\end{align}

\textbf{继续迭代},假设收敛到 $v_{\pi_1}(s_5) = -1.9$。

\subsection{第2次迭代:策略改进}

\textbf{计算动作价值}(使用 $v_{\pi_1}$):

假设 $v_{\pi_1}(s_2) = -1.0$,$v_{\pi_1}(s_4) = -1.9$,$v_{\pi_1}(s_6) = -1.0$,$v_{\pi_1}(s_8) = -1.0$。

\begin{align}
q(s_5, \text{上}) &= -1 + 0.9 \times (-1.0) = -1.9 \\
q(s_5, \text{下}) &= -1 + 0.9 \times (-1.0) = -1.9 \\
q(s_5, \text{左}) &= -1 + 0.9 \times (-1.9) = -2.71 \\
q(s_5, \text{右}) &= -1 + 0.9 \times (-1.0) = -1.9
\end{align}

\textbf{选择最优动作}:
\begin{equation}
\arg\max_{a} q(s_5, a) = \text{上} \quad \text{(或下、右)}
\end{equation}

当前策略是"上",保持不变。

\textbf{策略稳定}:$\pi_2(s_5) = \pi_1(s_5) = \text{上}$

\section{迭代过程总结}

\subsection{迭代序列}

\begin{center}
\begin{tabular}{|c|c|c|}
\hline
\textbf{迭代} & \textbf{策略} & \textbf{价值函数} \\
\hline
0 & $\pi_0$(随机) & $V_0 = 0$ \\
\hline
1 & $\pi_1$(改进后) & $v_{\pi_0}$(评估 $\pi_0$) \\
\hline
2 & $\pi_2 = \pi_1$(稳定) & $v_{\pi_1}$(评估 $\pi_1$) \\
\hline
\end{tabular}
\end{center}

\subsection{关键观察}

\begin{enumerate}
    \item \textbf{策略评估}:每次迭代都需要多次子迭代来收敛价值函数
    
    \item \textbf{策略改进}:基于当前价值函数,选择最优动作
    
    \item \textbf{收敛}:当策略不再改变时,算法收敛
    
    \item \textbf{效率}:在这个例子中,2次迭代就收敛了
\end{enumerate}

\section{完整算法流程的可视化}

\subsection{第1次迭代的完整流程}

\begin{enumerate}
    \item \textbf{初始化}:
    \begin{itemize}
        \item $\pi_0$:随机策略
        \item $V_0(s) = 0$ 对所有 $s$
    \end{itemize}
    
    \item \textbf{策略评估}(迭代直到收敛):
    \begin{itemize}
        \item $V_1(s) = -1$ 对所有 $s$
        \item $V_2(s_5) = -1.9$,$V_2(s_2) = -1.675$,等等
        \item 继续迭代...
        \item 收敛到 $v_{\pi_0}$
    \end{itemize}
    
    \item \textbf{策略改进}:
    \begin{itemize}
        \item 对每个状态 $s$:
        \begin{itemize}
            \item 计算 $q(s, a)$ 对所有 $a$
            \item 选择 $\arg\max_{a} q(s, a)$
            \item 更新 $\pi_1(s) = \arg\max_{a} q(s, a)$
        \end{itemize}
    \end{itemize}
    
    \item \textbf{检查}:策略是否改变?
    \begin{itemize}
        \item 是:继续迭代
        \item 否:算法收敛
    \end{itemize}
\end{enumerate}

\subsection{数值验证}

\textbf{验证策略改进的有效性}:

对于状态 $s_2$:
\begin{itemize}
    \item 旧策略(随机):$v_{\pi_0}(s_2) = -1.7$
    \item 新策略(选择"上"):$v_{\pi_1}(s_2) = -1.0$
    \item 改进:$-1.0 > -1.7$,策略确实改进了!
\end{itemize}

\section{总结}

\subsection{策略迭代的关键步骤}

\begin{enumerate}
    \item \textbf{策略评估}:
    \begin{itemize}
        \item 输入:策略 $\pi_k$
        \item 输出:价值函数 $v_{\pi_k}$
        \item 方法:迭代策略评估
    \end{itemize}
    
    \item \textbf{策略改进}:
    \begin{itemize}
        \item 输入:价值函数 $v_{\pi_k}$
        \item 输出:新策略 $\pi_{k+1}$
        \item 方法:$\pi_{k+1}(s) = \arg\max_{a} q_{\pi_k}(s, a)$
    \end{itemize}
    
    \item \textbf{重复}:直到策略不再改变
\end{enumerate}

\subsection{本例的特点}

\begin{itemize}
    \item \textbf{快速收敛}:2次迭代就收敛
    \item \textbf{策略改进明显}:从随机策略改进到确定性策略
    \item \textbf{价值函数改善}:每个状态的价值都提高了
\end{itemize}

\vspace{1cm}

\textbf{参考文献}:
\begin{itemize}
    \item Sutton, R. S., \& Barto, A. G. (2018). \textit{Reinforcement Learning: An Introduction} (2nd Edition). MIT Press, Chapter 4.
\end{itemize}

\end{document}

