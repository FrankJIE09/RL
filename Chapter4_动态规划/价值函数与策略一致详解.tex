\documentclass[12pt,a4paper]{article}
\usepackage[UTF8]{ctex}
\usepackage{amsmath}
\usepackage{amssymb}
\usepackage{amsthm}
\usepackage{geometry}
\usepackage{hyperref}
\usepackage{enumitem}
\usepackage{booktabs}

\geometry{left=2.5cm,right=2.5cm,top=2.5cm,bottom=2.5cm}

\title{价值函数与策略一致详解}
\author{强化学习笔记}
\date{\today}

\newtheorem{definition}{定义}
\newtheorem{theorem}{定理}
\newtheorem{example}{示例}

\begin{document}

\maketitle

\tableofcontents
\newpage

\section{引言}

"使价值函数与当前策略一致"是策略评估(Policy Evaluation)过程的核心目标,也是广义策略迭代(GPI)中两个关键过程之一。理解这个概念对于掌握强化学习算法至关重要。

\section{基本定义}

\subsection{什么是一致性?}

\begin{definition}[价值函数与策略一致]
价值函数 $V$ 与策略 $\pi$ \textbf{一致}(consistent),当且仅当 $V$ 等于策略 $\pi$ 的真实价值函数 $v_\pi$,即:
\begin{equation}
V = v_\pi
\label{eq:consistency}
\end{equation}
\end{definition}

\textbf{数学表达}:
\begin{equation}
V(s) = v_\pi(s) \quad \text{对所有 } s \in \mathcal{S}
\end{equation}

\subsection{策略的真实价值函数}

\textbf{策略 $\pi$ 的状态价值函数}:
\begin{equation}
v_\pi(s) = \mathbb{E}_\pi[G_t | S_t = s]
\end{equation}

即:从状态 $s$ 开始,遵循策略 $\pi$,获得的期望回报。

\textbf{贝尔曼方程}:
\begin{equation}
v_\pi(s) = \sum_{a} \pi(a|s) \sum_{s', r} p(s', r | s, a) [r + \gamma v_\pi(s')]
\label{eq:bellman_equation}
\end{equation}

\section{一致性的含义}

\subsection{直观理解}

\textbf{"使价值函数与当前策略一致"意味着}:

\begin{enumerate}
    \item \textbf{给定策略}:我们有一个固定的策略 $\pi$
    
    \item \textbf{计算价值}:计算在这个策略下,每个状态的真实价值
    
    \item \textbf{满足方程}:价值函数 $V$ 必须满足贝尔曼方程(\ref{eq:bellman_equation})
    
    \item \textbf{准确反映}:$V(s)$ 准确反映了在策略 $\pi$ 下从状态 $s$ 开始的期望回报
\end{enumerate}

\subsection{具体例子}

\textbf{例子1:简单Gridworld}

考虑一个简单的3×3 Gridworld,策略 $\pi$ 是等概率随机策略(每个动作概率 $0.25$)。

\textbf{状态 $s_5$(中心)}:
\begin{itemize}
    \item 策略 $\pi$:上、下、左、右各 $0.25$ 概率
    \item 如果价值函数 $V$ 与策略 $\pi$ 一致,则:
    \begin{equation}
    V(s_5) = 0.25 \times [r_{\text{上}} + \gamma V(s_2)] + 0.25 \times [r_{\text{下}} + \gamma V(s_8)] + 0.25 \times [r_{\text{左}} + \gamma V(s_4)] + 0.25 \times [r_{\text{右}} + \gamma V(s_6)]
    \end{equation}
    \item 这个等式对所有状态都成立
\end{itemize}

\textbf{例子2:不一致的情况}

假设我们有一个策略 $\pi$,但价值函数 $V$ 还没有更新到与 $\pi$ 一致:

\textbf{初始状态}:
\begin{itemize}
    \item 策略 $\pi$:在状态 $s$ 选择动作"上"的概率是 $0.8$,选择"下"的概率是 $0.2$
    \item 价值函数 $V(s) = 0$(初始值,可能不一致)
\end{itemize}

\textbf{策略评估过程}:
\begin{itemize}
    \item 迭代更新 $V$,使其满足:
    \begin{equation}
    V(s) = 0.8 \times [r_{\text{上}} + \gamma V(s_{\text{上}})] + 0.2 \times [r_{\text{下}} + \gamma V(s_{\text{下}})]
    \end{equation}
    \item 当 $V$ 不再改变时,$V = v_\pi$(价值函数与策略一致)
\end{itemize}

\section{策略评估过程}

\subsection{迭代策略评估}

\textbf{目标}:给定策略 $\pi$,计算 $v_\pi$

\textbf{方法}:迭代更新价值函数,直到收敛

\textbf{更新规则}:
\begin{equation}
V^{k+1}(s) = \sum_{a} \pi(a|s) \sum_{s', r} p(s', r | s, a) [r + \gamma V^k(s')]
\label{eq:iterative_evaluation}
\end{equation}

\textbf{收敛条件}:
\begin{equation}
V^k(s) = V^{k+1}(s) \quad \text{对所有 } s \in \mathcal{S}
\end{equation}

当收敛时,$V^k = v_\pi$,即价值函数与策略一致。

\subsection{具体计算过程}

\textbf{第0次迭代}(初始化):
\begin{equation}
V^0(s) = 0 \quad \text{对所有 } s
\end{equation}

\textbf{第1次迭代}:
\begin{equation}
V^1(s) = \sum_{a} \pi(a|s) \sum_{s', r} p(s', r | s, a) [r + \gamma V^0(s')]
\end{equation}

由于 $V^0(s') = 0$:
\begin{equation}
V^1(s) = \sum_{a} \pi(a|s) \sum_{s', r} p(s', r | s, a) \cdot r
\end{equation}

\textbf{第2次迭代}:
\begin{equation}
V^2(s) = \sum_{a} \pi(a|s) \sum_{s', r} p(s', r | s, a) [r + \gamma V^1(s')]
\end{equation}

\textbf{继续迭代},直到 $V^k(s) = V^{k+1}(s)$ 对所有 $s$ 成立。

\section{一致性的数学条件}

\subsection{贝尔曼方程}

价值函数 $V$ 与策略 $\pi$ 一致,当且仅当 $V$ 满足策略 $\pi$ 的贝尔曼方程:

\begin{equation}
V(s) = \sum_{a} \pi(a|s) \sum_{s', r} p(s', r | s, a) [r + \gamma V(s')] \quad \text{对所有 } s \in \mathcal{S}
\label{eq:bellman_consistency}
\end{equation}

\textbf{等价表述}:
\begin{equation}
V = \mathcal{T}_\pi V
\end{equation}

其中 $\mathcal{T}_\pi$ 是策略 $\pi$ 的贝尔曼算子(Bellman Operator)。

\subsection{不动点性质}

\textbf{关键洞察}:
\begin{itemize}
    \item 策略 $\pi$ 的价值函数 $v_\pi$ 是贝尔曼方程的唯一不动点
    \item 迭代策略评估收敛到这个不动点
    \item 当 $V = v_\pi$ 时,我们说价值函数与策略一致
\end{itemize}

\section{不一致的情况}

\subsection{什么是不一致?}

\textbf{不一致}意味着:
\begin{equation}
V(s) \neq v_\pi(s) \quad \text{对某些 } s \in \mathcal{S}
\end{equation}

\textbf{可能的原因}:
\begin{enumerate}
    \item \textbf{价值函数未收敛}:策略评估过程还没有完成
    \item \textbf{价值函数过时}:策略已经改变,但价值函数还没有更新
    \item \textbf{初始值}:价值函数是随机初始化的,还没有反映策略的真实价值
\end{enumerate}

\subsection{例子:策略改变后的不一致}

\textbf{初始状态}:
\begin{itemize}
    \item 策略 $\pi_0$:随机策略
    \item 价值函数 $V = v_{\pi_0}$(一致)
\end{itemize}

\textbf{策略改进后}:
\begin{itemize}
    \item 新策略 $\pi_1$:基于 $V$ 改进得到
    \item 价值函数仍然是 $V$(还没有更新)
    \item 此时 $V \neq v_{\pi_1}$(不一致!)
\end{itemize}

\textbf{需要重新评估}:
\begin{itemize}
    \item 使用策略评估计算 $v_{\pi_1}$
    \item 更新 $V \gets v_{\pi_1}$
    \item 现在 $V = v_{\pi_1}$(再次一致)
\end{itemize}

\section{在GPI中的作用}

\subsection{两个过程的交互}

\textbf{过程1:策略评估}(使价值函数与策略一致)
\begin{equation}
V \to v_\pi
\end{equation}

\textbf{过程2:策略改进}(使策略相对于价值函数贪婪)
\begin{equation}
\pi \to \text{greedy}(V)
\end{equation}

\subsection{竞争性}

\textbf{关键观察}:
\begin{itemize}
    \item 使策略贪婪通常会使价值函数对改变后的策略不正确
    \item 使价值函数与策略一致通常会使策略不再贪婪
    \item 两个过程"拉向相反的方向"
\end{itemize}

\textbf{例子}:
\begin{enumerate}
    \item \textbf{初始}:$V = v_{\pi_0}$,$\pi_0$ 是随机策略
    \item \textbf{策略改进}:$\pi_1 = \text{greedy}(V)$,现在 $\pi_1 \neq \text{greedy}(v_{\pi_1})$(不一致)
    \item \textbf{策略评估}:计算 $v_{\pi_1}$,更新 $V \gets v_{\pi_1}$,现在 $V = v_{\pi_1}$(一致)
    \item \textbf{但}:$\pi_1$ 可能不再是相对于 $V$ 的贪婪策略
    \item \textbf{继续迭代}:直到两个过程都稳定
\end{enumerate}

\subsection{收敛条件}

\textbf{同时稳定}:
\begin{align}
V &= v_\pi \quad \text{(价值函数与策略一致)} \\
\pi &= \text{greedy}(V) \quad \text{(策略相对于价值函数贪婪)}
\end{align}

\textbf{这意味着}:
\begin{equation}
\pi = \text{greedy}(v_\pi)
\end{equation}

这等价于贝尔曼最优性方程,因此 $\pi$ 和 $v_\pi$ 都是最优的。

\section{具体例子:Gridworld}

\subsection{问题设置}

考虑一个简单的3×3 Gridworld:
\begin{center}
\begin{tabular}{|c|c|c|}
\hline
终止 & $s_2$ & $s_3$ \\
\hline
$s_4$ & $s_5$ & $s_6$ \\
\hline
$s_7$ & $s_8$ & 终止 \\
\hline
\end{tabular}
\end{center}

\textbf{策略 $\pi$}:等概率随机策略
\begin{equation}
\pi(a|s) = \frac{1}{4} \quad \text{对所有 } s, a
\end{equation}

\textbf{奖励}:所有转移的奖励都是 $-1$

\textbf{折扣因子}:$\gamma = 0.9$

\subsection{策略评估过程}

\textbf{第0次迭代}:
\begin{equation}
V^0(s) = 0 \quad \text{对所有 } s
\end{equation}

\textbf{第1次迭代}:

对于状态 $s_5$(中心):
\begin{align}
V^1(s_5) &= \sum_{a} \frac{1}{4} \sum_{s', r} p(s', r | s_5, a) [r + \gamma V^0(s')] \\
         &= \frac{1}{4} \times [-1 + 0.9 \times 0] + \frac{1}{4} \times [-1 + 0.9 \times 0] + \frac{1}{4} \times [-1 + 0.9 \times 0] + \frac{1}{4} \times [-1 + 0.9 \times 0] \\
         &= -1
\end{align}

\textbf{第2次迭代}:

对于状态 $s_5$:
\begin{align}
V^2(s_5) &= \sum_{a} \frac{1}{4} \sum_{s', r} p(s', r | s_5, a) [r + \gamma V^1(s')] \\
         &= \frac{1}{4} \times [-1 + 0.9 \times V^1(s_2)] + \frac{1}{4} \times [-1 + 0.9 \times V^1(s_8)] + \frac{1}{4} \times [-1 + 0.9 \times V^1(s_4)] + \frac{1}{4} \times [-1 + 0.9 \times V^1(s_6)]
\end{align}

\textbf{继续迭代},直到 $V^k(s) = V^{k+1}(s)$ 对所有 $s$ 成立。

\textbf{收敛后}:
\begin{equation}
V^k = v_\pi
\end{equation}

此时,价值函数 $V$ 与策略 $\pi$ 一致。

\section{不同方法中的体现}

\subsection{动态规划}

\textbf{策略迭代}:
\begin{enumerate}
    \item \textbf{策略评估}:迭代直到 $V = v_\pi$
    \item \textbf{策略改进}:基于 $V$ 改进策略
\end{enumerate}

\subsection{蒙特卡洛方法}

\textbf{策略评估}:
\begin{itemize}
    \item 从样本回报估计 $v_\pi(s)$
    \item 当样本足够多时,$V(s) \approx v_\pi(s)$
    \item 使价值函数与策略一致
\end{itemize}

\subsection{时序差分学习}

\textbf{策略评估}:
\begin{itemize}
    \item 使用TD更新估计 $v_\pi(s)$
    \item 每个时间步更新:$V(S_t) \gets V(S_t) + \alpha [R_{t+1} + \gamma V(S_{t+1}) - V(S_t)]$
    \item 逐渐使价值函数与策略一致
\end{itemize}

\section{总结}

\subsection{核心要点}

\begin{enumerate}
    \item \textbf{定义}:价值函数 $V$ 与策略 $\pi$ 一致,当且仅当 $V = v_\pi$
    
    \item \textbf{数学条件}:$V$ 满足策略 $\pi$ 的贝尔曼方程
    
    \item \textbf{过程}:策略评估就是使价值函数与当前策略一致的过程
    
    \item \textbf{在GPI中}:这是两个核心过程之一,与策略改进过程交互
    
    \item \textbf{收敛}:当价值函数与策略一致,且策略相对于价值函数贪婪时,达到最优
\end{enumerate}

\subsection{关键公式}

\textbf{一致性条件}:
\begin{equation}
V(s) = \sum_{a} \pi(a|s) \sum_{s', r} p(s', r | s, a) [r + \gamma V(s')] \quad \text{对所有 } s
\end{equation}

\textbf{迭代更新}:
\begin{equation}
V^{k+1}(s) = \sum_{a} \pi(a|s) \sum_{s', r} p(s', r | s, a) [r + \gamma V^k(s')]
\end{equation}

\textbf{收敛条件}:
\begin{equation}
V^k = V^{k+1} \implies V^k = v_\pi
\end{equation}

\subsection{直观理解}

\begin{quote}
\textbf{"使价值函数与当前策略一致"}意味着:给定一个策略,计算这个策略下每个状态的\textbf{真实价值},使得价值函数准确反映在策略下从每个状态开始的期望回报。这通过迭代更新价值函数,使其满足该策略的贝尔曼方程来实现。
\end{quote}

\end{document}

