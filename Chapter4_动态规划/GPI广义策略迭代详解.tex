\documentclass[12pt,a4paper]{article}
\usepackage[UTF8]{ctex}
\usepackage{amsmath}
\usepackage{amssymb}
\usepackage{amsthm}
\usepackage{algorithm}
\usepackage{algorithmic}
\usepackage{graphicx}
\usepackage{hyperref}
\usepackage{geometry}
\usepackage{tikz}
\usepackage{booktabs}
\geometry{left=2.5cm,right=2.5cm,top=2.5cm,bottom=2.5cm}

\title{GPI(广义策略迭代)详解}
\author{强化学习笔记}
\date{\today}

\newtheorem{definition}{定义}
\newtheorem{theorem}{定理}
\newtheorem{proposition}{命题}
\newtheorem{example}{示例}
\newtheorem{remark}{注记}

\begin{document}

\maketitle

\tableofcontents
\newpage

\section{引言}

\textbf{广义策略迭代}(Generalized Policy Iteration, GPI)是强化学习中的一个核心概念,它统一了几乎所有强化学习方法的框架。理解 GPI 对于掌握强化学习的本质至关重要。

\section{什么是 GPI?}

\subsection{基本定义}

\begin{definition}[广义策略迭代(GPI)]
\textbf{广义策略迭代}(Generalized Policy Iteration, GPI)是指让\textbf{策略评估}(Policy Evaluation)和\textbf{策略改进}(Policy Improvement)两个过程交互的一般思想,独立于这两个过程的粒度和其他细节。
\end{definition}

\textbf{核心思想}:
\begin{itemize}
    \item GPI 包含两个相互作用的进程
    \item \textbf{策略评估}:使价值函数与当前策略一致
    \item \textbf{策略改进}:使策略相对于当前价值函数变得贪婪
    \item 这两个过程可以以任何方式交互,不要求完全分离
\end{itemize}

\subsection{两个核心过程}

\textbf{过程1:策略评估(Policy Evaluation)}
\begin{equation}
V \to v_\pi
\end{equation}
\begin{itemize}
    \item 给定策略 $\pi$,计算其价值函数 $v_\pi$
    \item 使价值函数与当前策略一致
    \item 满足贝尔曼方程:$v_\pi(s) = \sum_{a} \pi(a|s) \sum_{s', r} p(s', r | s, a) [r + \gamma v_\pi(s')]$
\end{itemize}

\textbf{过程2:策略改进(Policy Improvement)}
\begin{equation}
\pi \to \text{greedy}(V)
\end{equation}
\begin{itemize}
    \item 给定价值函数 $V$,改进策略 $\pi$
    \item 使策略相对于当前价值函数变得贪婪
    \item 对每个状态选择最优动作:$\pi'(s) = \arg\max_{a} q_\pi(s, a)$
\end{itemize}

\section{经典策略迭代 vs GPI}

\subsection{经典策略迭代}

\textbf{特点}:
\begin{itemize}
    \item 两个过程\textbf{交替}执行
    \item 每个过程\textbf{完全完成}后才开始另一个
    \item 策略评估:迭代直到收敛
    \item 策略改进:基于收敛的价值函数改进策略
\end{itemize}

\textbf{流程}:
\begin{equation}
\pi_0 \xrightarrow{\text{完整评估}} v_{\pi_0} \xrightarrow{\text{完整改进}} \pi_1 \xrightarrow{\text{完整评估}} v_{\pi_1} \xrightarrow{\text{完整改进}} \cdots
\end{equation}

\subsection{GPI 的灵活性}

\textbf{GPI 允许}:
\begin{itemize}
    \item 两个过程可以\textbf{以任何粒度}交互
    \item 不需要等待一个过程完全完成
    \item 可以只执行一次评估迭代就进行改进
    \item 可以异步更新,甚至单个状态更新
\end{itemize}

\textbf{不同实现方式}:

\begin{enumerate}
    \item \textbf{策略迭代}:完整评估 $\to$ 完整改进 $\to$ 完整评估 $\to$ ...
    
    \item \textbf{价值迭代}:一次评估迭代 $\to$ 改进(合并)$\to$ 一次评估迭代 $\to$ ...
    
    \item \textbf{异步动态规划}:单个状态评估 $\to$ 单个状态改进 $\to$ 单个状态评估 $\to$ ...
    
    \item \textbf{蒙特卡洛方法}:从样本学习价值 $\to$ 改进策略 $\to$ 从样本学习价值 $\to$ ...
    
    \item \textbf{时序差分学习}:每个时间步更新价值 $\to$ 改进策略 $\to$ 每个时间步更新价值 $\to$ ...
\end{enumerate}

\section{GPI 的收敛性}

\subsection{收敛条件}

\begin{theorem}[GPI 收敛性]
如果策略评估和策略改进两个过程都稳定(不再产生变化),那么价值函数和策略必须是最优的。
\end{theorem}

\textbf{稳定条件}:
\begin{itemize}
    \item \textbf{价值函数稳定}:当且仅当它与当前策略一致时
    \item \textbf{策略稳定}:当且仅当它相对于当前价值函数是贪婪的
    \item \textbf{同时稳定}:当且仅当策略相对于自己的评估函数是贪婪的
\end{itemize}

\textbf{数学表达}:
\begin{align}
V &= v_\pi \quad \text{(价值函数与策略一致)} \\
\pi &= \text{greedy}(V) \quad \text{(策略相对于价值函数贪婪)}
\end{align}

这意味着:
\begin{equation}
\pi = \text{greedy}(v_\pi)
\end{equation}

这等价于贝尔曼最优性方程:
\begin{equation}
v_\pi(s) = \max_{a} \sum_{s', r} p(s', r | s, a) [r + \gamma v_\pi(s')]
\end{equation}

因此,$\pi$ 和 $v_\pi$ 都是最优的。

\subsection{收敛过程}

\textbf{两个过程的竞争与合作}:

\textbf{竞争性}:
\begin{itemize}
    \item 使策略贪婪通常会使价值函数对改变后的策略不正确
    \item 使价值函数与策略一致通常会使策略不再贪婪
    \item 两个过程"拉向相反的方向"
\end{itemize}

\textbf{合作性}:
\begin{itemize}
    \item 长期来看,两个过程相互作用找到单一联合解
    \item 最终收敛到最优价值函数和最优策略
    \item 即使每个过程都不直接追求最优性,它们一起达到最优
\end{itemize}

\section{GPI 的几何解释}

\subsection{二维约束视角}

可以将 GPI 的两个过程视为两个约束或目标:

\textbf{约束1:价值函数与策略一致}
\begin{equation}
V = v_\pi
\end{equation}

\textbf{约束2:策略相对于价值函数贪婪}
\begin{equation}
\pi = \text{greedy}(V)
\end{equation}

\textbf{几何解释}:
\begin{itemize}
    \item 每个过程驱动价值函数或策略朝向代表其中一个目标的线
    \item 两个目标相互作用,因为两条线不是正交的
    \item 直接朝向一个目标会导致远离另一个目标
    \item 但联合过程最终会接近整体最优性目标
\end{itemize}

\subsection{收敛路径}

\textbf{策略迭代的路径}:
\begin{itemize}
    \item 每个箭头完全达到其中一个目标
    \item 完整评估 $\to$ 完整改进 $\to$ 完整评估 $\to$ ...
\end{itemize}

\textbf{GPI 的路径}:
\begin{itemize}
    \item 可以采取更小的、不完整的步骤朝向每个目标
    \item 两个过程一起达到整体最优性目标
    \item 即使每个过程都不直接追求最优性
\end{itemize}

\section{GPI 在不同方法中的体现}

\subsection{动态规划中的 GPI}

\subsubsection{策略迭代}

\textbf{实现方式}:
\begin{enumerate}
    \item \textbf{策略评估}:迭代直到收敛
    \begin{equation}
    v_\pi^{k+1}(s) = \sum_{a} \pi(a|s) \sum_{s', r} p(s', r | s, a) [r + \gamma v_\pi^k(s')]
    \end{equation}
    
    \item \textbf{策略改进}:基于收敛的价值函数
    \begin{equation}
    \pi'(s) = \arg\max_{a} \sum_{s', r} p(s', r | s, a) [r + \gamma v_\pi(s')]
    \end{equation}
\end{enumerate}

\textbf{特点}:
\begin{itemize}
    \item 两个过程完全分离
    \item 每个过程完全完成
    \item 收敛保证强
\end{itemize}

\subsubsection{价值迭代}

\textbf{实现方式}:
\begin{equation}
v_{k+1}(s) = \max_{a} \sum_{s', r} p(s', r | s, a) [r + \gamma v_k(s')]
\end{equation}

\textbf{特点}:
\begin{itemize}
    \item 策略评估和策略改进合并为一个操作
    \item 每次只执行一次评估迭代
    \item $\max$ 操作隐含地执行了策略改进
\end{itemize}

\textbf{为什么是 GPI?}
\begin{itemize}
    \item 仍然包含评估(计算每个动作的价值)和改进(选择最大值)
    \item 只是以更细的粒度执行
    \item 每次更新都同时进行部分评估和改进
\end{itemize}

\subsubsection{异步动态规划}

\textbf{实现方式}:
\begin{itemize}
    \item 可以按任意顺序更新状态
    \item 可以使用过时的信息
    \item 可以只更新部分状态
\end{itemize}

\textbf{特点}:
\begin{itemize}
    \item 粒度最细:单个状态更新
    \item 不需要完整扫描所有状态
    \item 更灵活,适合大规模问题
\end{itemize}

\subsection{蒙特卡洛方法中的 GPI}

\textbf{策略评估}:
\begin{itemize}
    \item 从样本回报学习价值函数
    \item 使用经验平均估计 $q_\pi(s, a)$
    \item 不需要环境模型
\end{itemize}

\textbf{策略改进}:
\begin{itemize}
    \item 基于估计的动作价值函数改进策略
    \item 使用 $\varepsilon$-贪婪策略平衡探索和利用
\end{itemize}

\textbf{GPI 框架}:
\begin{enumerate}
    \item 生成回合,估计 $q_\pi$
    \item 基于 $q_\pi$ 改进策略
    \item 重复,直到收敛
\end{enumerate}

\subsection{时序差分学习中的 GPI}

\textbf{策略评估}:
\begin{itemize}
    \item 使用 TD 更新估计价值函数
    \item 每个时间步都可以更新
    \item 使用自举法(bootstrapping)
\end{itemize}

\textbf{策略改进}:
\begin{itemize}
    \item 基于当前价值函数改进策略
    \item 在 SARSA 中,策略和价值函数同时更新
\end{itemize}

\textbf{GPI 框架}:
\begin{enumerate}
    \item 每个时间步更新 $Q(S_t, A_t)$
    \item 基于 $Q$ 更新策略 $\pi$
    \item 两个过程紧密交织
\end{enumerate}

\section{GPI 的统一视角}

\subsection{所有方法的共同点}

\textbf{几乎所有强化学习方法都可以用 GPI 描述}:
\begin{itemize}
    \item 都有可识别的策略和价值函数
    \item 策略总是相对于价值函数被改进
    \item 价值函数总是被驱动朝向策略的价值函数
\end{itemize}

\subsection{不同方法的区别}

\textbf{主要区别在于}:
\begin{enumerate}
    \item \textbf{策略评估的方式}:
    \begin{itemize}
        \item 动态规划:从模型计算
        \item 蒙特卡洛:从样本回报学习
        \item 时序差分:从一步前瞻学习
    \end{itemize}
    
    \item \textbf{策略改进的方式}:
    \begin{itemize}
        \item 贪婪策略
        \item $\varepsilon$-贪婪策略
        \item 软策略(如 Softmax)
    \end{itemize}
    
    \item \textbf{两个过程的交互粒度}:
    \begin{itemize}
        \item 策略迭代:完整过程
        \item 价值迭代:一次迭代
        \item 异步方法:单个状态
        \item TD 方法:每个时间步
    \end{itemize}
\end{enumerate}

\section{具体例子}

\subsection{例子1:策略迭代(完整 GPI)}

\textbf{初始策略}:$\pi_0$(随机策略)

\textbf{第1次迭代}:
\begin{enumerate}
    \item \textbf{策略评估}:计算 $v_{\pi_0}$(迭代直到收敛)
    \item \textbf{策略改进}:$\pi_1 = \text{greedy}(v_{\pi_0})$
\end{enumerate}

\textbf{第2次迭代}:
\begin{enumerate}
    \item \textbf{策略评估}:计算 $v_{\pi_1}$(迭代直到收敛)
    \item \textbf{策略改进}:$\pi_2 = \text{greedy}(v_{\pi_1})$
\end{enumerate}

\textbf{继续迭代},直到 $\pi_k = \pi_{k+1}$(策略不再改变)

\subsection{例子2:价值迭代(合并 GPI)}

\textbf{初始价值}:$v_0(s) = 0$ 对所有 $s$

\textbf{每次迭代}:
\begin{equation}
v_{k+1}(s) = \max_{a} \sum_{s', r} p(s', r | s, a) [r + \gamma v_k(s')]
\end{equation}

\textbf{特点}:
\begin{itemize}
    \item 每次更新同时进行部分评估和改进
    \item 不需要显式的策略表示
    \item 收敛后提取策略:$\pi_*(s) = \arg\max_{a} \sum_{s', r} p(s', r | s, a) [r + \gamma v_*(s')]$
\end{itemize}

\subsection{例子3:SARSA(在线 GPI)}

\textbf{每个时间步}:
\begin{enumerate}
    \item 执行动作 $A_t$,观察 $R_{t+1}, S_{t+1}$
    \item 选择下一动作 $A_{t+1}$(根据当前策略)
    \item \textbf{策略评估}:更新 $Q(S_t, A_t)$
    \begin{equation}
    Q(S_t, A_t) \gets Q(S_t, A_t) + \alpha [R_{t+1} + \gamma Q(S_{t+1}, A_{t+1}) - Q(S_t, A_t)]
    \end{equation}
    \item \textbf{策略改进}:更新策略 $\pi(S_t)$(基于 $Q(S_t, \cdot)$)
\end{enumerate}

\textbf{特点}:
\begin{itemize}
    \item 评估和改进在每个时间步都发生
    \item 两个过程紧密交织
    \item 在线学习,边交互边学习
\end{itemize}

\section{GPI 的优势}

\subsection{理论优势}

\textbf{1. 统一框架}:
\begin{itemize}
    \item 几乎所有强化学习方法都可以用 GPI 描述
    \item 提供统一的理论视角
    \item 便于理解和比较不同方法
\end{itemize}

\textbf{2. 灵活性}:
\begin{itemize}
    \item 不要求两个过程完全分离
    \item 可以根据问题特点选择交互方式
    \item 允许细粒度的更新
\end{itemize}

\textbf{3. 收敛保证}:
\begin{itemize}
    \item 在适当条件下,GPI 保证收敛到最优解
    \item 适用于多种实现方式
    \item 为算法设计提供指导
\end{itemize}

\subsection{实践优势}

\textbf{1. 算法设计}:
\begin{itemize}
    \item 提供清晰的算法设计框架
    \item 可以灵活选择评估和改进的方式
    \item 便于实现和调试
\end{itemize}

\textbf{2. 理解算法}:
\begin{itemize}
    \item 帮助理解不同算法的本质
    \item 识别算法的共同点和区别
    \item 便于学习新算法
\end{itemize}

\textbf{3. 算法改进}:
\begin{itemize}
    \item 可以独立改进评估或改进过程
    \item 可以尝试不同的交互方式
    \item 便于设计新算法
\end{itemize}

\section{总结}

\subsection{核心要点}

\begin{enumerate}
    \item \textbf{GPI 定义}:策略评估和策略改进两个过程的交互
    
    \item \textbf{两个过程}:
    \begin{itemize}
        \item 策略评估:$V \to v_\pi$
        \item 策略改进:$\pi \to \text{greedy}(V)$
    \end{itemize}
    
    \item \textbf{灵活性}:两个过程可以以任何粒度交互
    
    \item \textbf{收敛性}:当两个过程都稳定时,达到最优解
    
    \item \textbf{统一性}:几乎所有强化学习方法都是 GPI 的实例
\end{enumerate}

\subsection{关键公式}

\textbf{策略评估}(贝尔曼方程):
\begin{equation}
v_\pi(s) = \sum_{a} \pi(a|s) \sum_{s', r} p(s', r | s, a) [r + \gamma v_\pi(s')]
\end{equation}

\textbf{策略改进}:
\begin{equation}
\pi'(s) = \arg\max_{a} q_\pi(s, a) = \arg\max_{a} \sum_{s', r} p(s', r | s, a) [r + \gamma v_\pi(s')]
\end{equation}

\textbf{最优性条件}:
\begin{equation}
v_*(s) = \max_{a} \sum_{s', r} p(s', r | s, a) [r + \gamma v_*(s')]
\end{equation}

\subsection{理解 GPI 的关键}

\textbf{1. 两个过程的相互作用}:
\begin{itemize}
    \item 竞争性:拉向相反方向
    \item 合作性:最终达到共同目标
\end{itemize}

\textbf{2. 灵活的实现方式}:
\begin{itemize}
    \item 不需要完全分离两个过程
    \item 可以根据需要选择交互粒度
    \item 允许各种优化和变体
\end{itemize}

\textbf{3. 统一的理论框架}:
\begin{itemize}
    \item 几乎所有方法都遵循 GPI
    \item 区别在于实现细节
    \item 提供统一的理解视角
\end{itemize}

\section{进一步阅读}

\begin{itemize}
    \item 动态规划方法:策略迭代、价值迭代、异步动态规划
    \item 蒙特卡洛方法:On-policy 和 Off-policy 控制
    \item 时序差分学习:SARSA、Q-learning
    \item 函数逼近:线性函数逼近、神经网络
    \item 策略梯度方法:REINFORCE、Actor-Critic
\end{itemize}

\end{document}

