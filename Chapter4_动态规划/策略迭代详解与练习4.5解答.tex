\documentclass[12pt,a4paper]{article}
\usepackage[UTF8]{ctex}
\usepackage{amsmath}
\usepackage{amssymb}
\usepackage{amsthm}
\usepackage{geometry}
\usepackage{hyperref}
\usepackage{enumitem}
\usepackage{xcolor}
\usepackage{bm}
\usepackage{tikz}
\usepackage{booktabs}
\usepackage{algorithm}
\usepackage{algorithmic}

\geometry{margin=2.5cm}

\title{策略迭代详解与练习4.5解答}
\subtitle{策略迭代算法及其在动作价值函数中的应用}
\author{}
\date{}

\newtheorem{definition}{定义}
\newtheorem{theorem}{定理}
\newtheorem{proposition}{命题}
\newtheorem{example}{示例}
\newtheorem{remark}{注记}

\begin{document}

\maketitle

\tableofcontents
\newpage

\section{策略迭代算法详解}

\subsection{什么是策略迭代?}

\begin{definition}[策略迭代]
\textbf{策略迭代}(Policy Iteration)是一种动态规划算法,通过交替执行\textbf{策略评估}(Policy Evaluation)和\textbf{策略改进}(Policy Improvement)来找到最优策略。
\end{definition}

\textbf{核心思想}:
\begin{enumerate}
    \item 给定一个策略,计算它的价值函数(策略评估)
    \item 基于价值函数改进策略(策略改进)
    \item 重复上述过程,直到策略不再改变
\end{enumerate}

\subsection{策略迭代的过程}

策略迭代产生一个单调改进的策略和价值函数序列:

\begin{equation}
\pi_0 \xrightarrow{E} v_{\pi_0} \xrightarrow{I} \pi_1 \xrightarrow{E} v_{\pi_1} \xrightarrow{I} \pi_2 \xrightarrow{E} \cdots \xrightarrow{I} \pi_* \xrightarrow{E} v_*
\end{equation}

其中:
\begin{itemize}
    \item $\xrightarrow{E}$ 表示策略评估(Evaluation)
    \item $\xrightarrow{I}$ 表示策略改进(Improvement)
    \item $\pi_*$ 是最优策略
    \item $v_*$ 是最优价值函数
\end{itemize}

\subsection{策略迭代算法(基于状态价值函数)}

\begin{algorithm}[H]
\caption{策略迭代算法(计算 $v_*$ 和 $\pi_*$)}
\begin{algorithmic}[1]
\REQUIRE 环境动态 $p(s', r | s, a)$,折扣因子 $\gamma$,收敛阈值 $\theta$
\ENSURE 最优价值函数 $v_*$ 和最优策略 $\pi_*$
\STATE \textbf{初始化}:$V(s) \in \mathbb{R}$ 和 $\pi(s) \in \mathcal{A}(s)$ 任意初始化,对所有 $s \in \mathcal{S}$
\REPEAT
    \STATE \textbf{策略评估}:
    \REPEAT
        \STATE $\Delta \gets 0$
        \FOR{每个状态 $s \in \mathcal{S}$}
            \STATE $v \gets V(s)$
            \STATE $V(s) \gets \sum_{s', r} p(s', r | s, \pi(s)) [r + \gamma V(s')]$
            \STATE $\Delta \gets \max(\Delta, |v - V(s)|)$
        \ENDFOR
    \UNTIL{$\Delta < \theta$}
    \STATE \textbf{策略改进}:
    \STATE $\text{policy-stable} \gets \text{true}$
    \FOR{每个状态 $s \in \mathcal{S}$}
        \STATE $a_{\text{old}} \gets \pi(s)$
        \STATE $\pi(s) \gets \arg\max_{a} \sum_{s', r} p(s', r | s, a) [r + \gamma V(s')]$
        \IF{$a_{\text{old}} \neq \pi(s)$}
            \STATE $\text{policy-stable} \gets \text{false}$
        \ENDIF
    \ENDFOR
\UNTIL{$\text{policy-stable} = \text{true}$}
\RETURN $v_* = V$,$\pi_* = \pi$
\end{algorithmic}
\end{algorithmic}

\subsection{算法步骤详解}

\textbf{步骤1:初始化}
\begin{itemize}
    \item 初始化价值函数 $V(s)$(可以任意,通常设为0)
    \item 初始化策略 $\pi(s)$(可以任意,通常设为随机策略)
\end{itemize}

\textbf{步骤2:策略评估}
\begin{itemize}
    \item 使用迭代策略评估计算当前策略 $\pi$ 的价值函数 $v_\pi$
    \item 迭代更新直到收敛:
    \begin{equation}
    V(s) \gets \sum_{s', r} p(s', r | s, \pi(s)) [r + \gamma V(s')]
    \end{equation}
    \item 当价值函数变化小于阈值 $\theta$ 时停止
\end{itemize}

\textbf{步骤3:策略改进}
\begin{itemize}
    \item 对每个状态 $s$,计算所有动作的动作价值:
    \begin{equation}
    q(s, a) = \sum_{s', r} p(s', r | s, a) [r + \gamma V(s')]
    \end{equation}
    \item 选择使动作价值最大的动作:
    \begin{equation}
    \pi(s) \gets \arg\max_{a} q(s, a)
    \end{equation}
    \item 如果策略没有改变,算法收敛
\end{itemize}

\subsection{策略改进步骤的详细解释}

\textbf{问题}:$\pi(s) \gets \arg\max_{a} q(s, a)$ 这一步具体是怎么操作的?

\subsubsection{步骤1:计算所有动作的动作价值}

对于状态 $s$,对每个可能的动作 $a \in \mathcal{A}(s)$,计算动作价值:

\begin{equation}
q(s, a) = \sum_{s', r} p(s', r | s, a) [r + \gamma V(s')]
\end{equation}

\textbf{具体过程}:

\begin{enumerate}
    \item \textbf{遍历所有动作}:对每个 $a \in \mathcal{A}(s)$
    
    \item \textbf{计算动作价值}:
    \begin{itemize}
        \item 对于每个可能的 $(s', r)$,计算 $p(s', r | s, a) \times [r + \gamma V(s')]$
        \item 对所有 $(s', r)$ 求和
    \end{itemize}
    
    \item \textbf{存储结果}:得到 $q(s, a)$ 的值
\end{enumerate}

\subsubsection{步骤2:比较所有动作的动作价值}

计算完所有动作的动作价值后,得到一组值:
\begin{equation}
\{q(s, a_1), q(s, a_2), q(s, a_3), \ldots, q(s, a_n)\}
\end{equation}

其中 $n = |\mathcal{A}(s)|$ 是状态 $s$ 的可能动作数。

\subsubsection{步骤3:选择最大值对应的动作}

\textbf{$\arg\max$ 的含义}:

$\arg\max_{a} q(s, a)$ 表示使 $q(s, a)$ 达到最大值的动作 $a$。

\textbf{数学定义}:
\begin{equation}
\arg\max_{a} q(s, a) = a^* \quad \text{使得} \quad q(s, a^*) = \max_{a \in \mathcal{A}(s)} q(s, a)
\end{equation}

\textbf{具体操作}:
\begin{enumerate}
    \item \textbf{找到最大值}:
    \begin{equation}
    q_{\max} = \max\{q(s, a_1), q(s, a_2), \ldots, q(s, a_n)\}
    \end{equation}
    
    \item \textbf{找到对应的动作}:
    \begin{equation}
    a^* = \text{使得 } q(s, a^*) = q_{\max} \text{ 的动作}
    \end{equation}
    
    \item \textbf{更新策略}:
    \begin{equation}
    \pi(s) \gets a^*
    \end{equation}
\end{enumerate}

\subsubsection{具体数值例子}

\textbf{例子:Gridworld中的状态 $s$}

假设状态 $s$ 有4个动作:上、下、左、右。

\textbf{步骤1:计算动作价值}

假设当前价值函数 $V$ 的值:
\begin{itemize}
    \item $V(s_{\text{上}}) = 10$
    \item $V(s_{\text{下}}) = 15$
    \item $V(s_{\text{左}}) = 8$
    \item $V(s_{\text{右}}) = 12$
\end{itemize}

假设 $\gamma = 0.9$,所有转移的奖励 $r = 0$(普通移动)。

\textbf{计算每个动作的动作价值}:

\textbf{动作"上"}:
\begin{align}
q(s, \text{上}) &= \sum_{s', r} p(s', r | s, \text{上}) [r + \gamma V(s')] \\
                &= p(s_{\text{上}}, 0 | s, \text{上}) \times [0 + 0.9 \times 10] \\
                &= 1 \times 9 = 9
\end{align}

\textbf{动作"下"}:
\begin{align}
q(s, \text{下}) &= p(s_{\text{下}}, 0 | s, \text{下}) \times [0 + 0.9 \times 15] \\
                &= 1 \times 13.5 = 13.5
\end{align}

\textbf{动作"左"}:
\begin{align}
q(s, \text{左}) &= p(s_{\text{左}}, 0 | s, \text{左}) \times [0 + 0.9 \times 8] \\
                &= 1 \times 7.2 = 7.2
\end{align}

\textbf{动作"右"}:
\begin{align}
q(s, \text{右}) &= p(s_{\text{右}}, 0 | s, \text{右}) \times [0 + 0.9 \times 12] \\
                &= 1 \times 10.8 = 10.8
\end{align}

\textbf{步骤2:比较动作价值}

得到动作价值集合:
\begin{align}
q(s, \text{上}) &= 9.0 \\
q(s, \text{下}) &= 13.5 \\
q(s, \text{左}) &= 7.2 \\
q(s, \text{右}) &= 10.8
\end{align}

\textbf{步骤3:选择最大值}

\begin{align}
q_{\max} &= \max\{9.0, 13.5, 7.2, 10.8\} = 13.5 \\
a^* &= \text{下} \quad \text{(因为 } q(s, \text{下}) = 13.5 = q_{\max} \text{)}
\end{align}

\textbf{更新策略}:
\begin{equation}
\pi(s) \gets \text{下}
\end{equation}

\subsection{策略更新的具体实现}

\textbf{问题}:$\pi(s) \gets \text{下}$ 这一步具体是怎么更新策略的?

\subsubsection{策略的表示方式}

\textbf{确定性策略的表示}:

对于确定性策略,$\pi(s)$ 表示在状态 $s$ 下选择的动作。

\textbf{数据结构}:
\begin{itemize}
    \item \textbf{字典/映射}:$\pi: \mathcal{S} \to \mathcal{A}$
    \item \textbf{数组}:如果状态可以编号,可以用数组存储
    \item \textbf{函数}:策略可以表示为一个函数
\end{itemize}

\textbf{具体例子}:

假设有4个状态:$s_1, s_2, s_3, s_4$,每个状态有4个动作:上、下、左、右。

\textbf{策略的存储方式}:

\textbf{方式1:字典(Python风格)}:
\begin{verbatim}
π = {
    s_1: '上',
    s_2: '右',
    s_3: '下',
    s_4: '左'
}
\end{verbatim}

\textbf{方式2:数组(如果状态可以编号)}:
\begin{verbatim}
π = ['上', '右', '下', '左']  // π[0] = '上' 表示状态0选择'上'
\end{verbatim}

\textbf{方式3:函数}:
\begin{equation}
\pi(s) = \begin{cases}
\text{上}, & \text{如果 } s = s_1 \\
\text{右}, & \text{如果 } s = s_2 \\
\text{下}, & \text{如果 } s = s_3 \\
\text{左}, & \text{如果 } s = s_4
\end{cases}
\end{equation}

\subsubsection{策略更新的操作}

\textbf{更新操作的含义}:

$\pi(s) \gets \text{下}$ 表示:将状态 $s$ 的策略更新为选择动作"下"。

\textbf{具体实现}:

\textbf{使用字典}:
\begin{verbatim}
π[s] = '下'  // 更新状态s的策略为'下'
\end{verbatim}

\textbf{使用数组}:
\begin{verbatim}
π[state_index] = action_index  // 更新状态state_index的策略为action_index
\end{verbatim}

\textbf{使用函数}:
\begin{equation}
\pi_{\text{new}}(s') = \begin{cases}
\text{下}, & \text{如果 } s' = s \\
\pi_{\text{old}}(s'), & \text{如果 } s' \neq s
\end{cases}
\end{equation}

\subsubsection{完整示例:策略更新的过程}

\textbf{初始策略 $\pi_0$}:

假设初始策略是随机策略或某个初始策略:
\begin{center}
\begin{tabular}{|c|c|}
\hline
\textbf{状态} & \textbf{动作} \\
\hline
$s_1$ & 上 \\
\hline
$s_2$ & 左 \\
\hline
$s_3$ & 右 \\
\hline
$s_4$ & 上 \\
\hline
\end{tabular}
\end{center}

\textbf{策略评估后}:

假设经过策略评估,得到价值函数 $V$:
\begin{itemize}
    \item $V(s_1) = 10$
    \item $V(s_2) = 15$
    \item $V(s_3) = 12$
    \item $V(s_4) = 8$
\end{itemize}

\textbf{策略改进:对每个状态更新策略}

\textbf{状态 $s_1$}:

计算所有动作的动作价值:
\begin{align}
q(s_1, \text{上}) &= 9.0 \\
q(s_1, \text{下}) &= 13.5 \\
q(s_1, \text{左}) &= 7.2 \\
q(s_1, \text{右}) &= 10.8
\end{align}

最大值是 $13.5$,对应动作"下"。

\textbf{更新策略}:
\begin{verbatim}
π[s_1] = '下'  // 从'上'更新为'下'
\end{verbatim}

\textbf{状态 $s_2$}:

计算所有动作的动作价值:
\begin{align}
q(s_2, \text{上}) &= 12.0 \\
q(s_2, \text{下}) &= 11.5 \\
q(s_2, \text{左}) &= 10.2 \\
q(s_2, \text{右}) &= 14.8
\end{align}

最大值是 $14.8$,对应动作"右"。

\textbf{更新策略}:
\begin{verbatim}
π[s_2] = '右'  // 从'左'更新为'右'
\end{verbatim}

\textbf{状态 $s_3$}:

假设计算得到最大值对应的动作是"下"。

\textbf{更新策略}:
\begin{verbatim}
π[s_3] = '下'  // 从'右'更新为'下'
\end{verbatim}

\textbf{状态 $s_4$}:

假设计算得到最大值对应的动作仍然是"上"。

\textbf{更新策略}:
\begin{verbatim}
π[s_4] = '上'  // 保持不变
\end{verbatim}

\textbf{更新后的策略 $\pi_1$}:

\begin{center}
\begin{tabular}{|c|c|c|}
\hline
\textbf{状态} & \textbf{旧策略 $\pi_0$} & \textbf{新策略 $\pi_1$} \\
\hline
$s_1$ & 上 & \textcolor{red}{下}(改变) \\
\hline
$s_2$ & 左 & \textcolor{red}{右}(改变) \\
\hline
$s_3$ & 右 & \textcolor{red}{下}(改变) \\
\hline
$s_4$ & 上 & 上(不变) \\
\hline
\end{tabular}
\end{center}

\subsubsection{代码实现示例}

\textbf{Python伪代码}:

\begin{verbatim}
# 初始化策略(字典形式)
policy = {}
for state in all_states:
    policy[state] = random_action()  # 或初始策略

# 策略改进
def policy_improvement(policy, value_function):
    policy_stable = True
    
    for state in all_states:
        old_action = policy[state]  # 保存旧动作
        
        # 计算所有动作的动作价值
        action_values = {}
        for action in possible_actions(state):
            action_values[action] = compute_q_value(
                state, action, value_function
            )
        
        # 选择使动作价值最大的动作
        best_action = max(action_values, key=action_values.get)
        
        # 更新策略
        policy[state] = best_action  # 这就是 π(s) ← argmax_a q(s,a)
        
        # 检查策略是否改变
        if old_action != best_action:
            policy_stable = False
    
    return policy, policy_stable
\end{verbatim}

\textbf{关键代码行}:

\begin{verbatim}
policy[state] = best_action
\end{verbatim}

这就是 $\pi(s) \gets \arg\max_{a} q(s, a)$ 的具体实现。

\subsubsection{策略更新的内存表示}

\textbf{内存中的存储}:

策略在内存中通常存储为一个数组或字典:

\textbf{数组形式}(如果状态可以编号):
\begin{verbatim}
// 假设有4个状态,4个动作
// 动作编号:0=上, 1=下, 2=左, 3=右

int policy[4];  // 策略数组

// 初始策略
policy[0] = 0;  // 状态0选择动作0(上)
policy[1] = 2;  // 状态1选择动作2(左)
policy[2] = 3;  // 状态2选择动作3(右)
policy[3] = 0;  // 状态3选择动作0(上)

// 策略改进:更新状态0的策略
policy[0] = 1;  // 状态0现在选择动作1(下)
// 这就是 π(s_0) ← 下 的实现
\end{verbatim}

\textbf{字典形式}(更通用):
\begin{verbatim}
// Python风格
policy = {
    's1': 'up',      // 状态s1选择'up'
    's2': 'left',    // 状态s2选择'left'
    's3': 'right',   // 状态s3选择'right'
    's4': 'up'       // 状态s4选择'up'
}

// 策略改进:更新状态s1的策略
policy['s1'] = 'down'  // 这就是 π(s1) ← 下 的实现
\end{verbatim}

\subsubsection{策略更新的完整流程}

\textbf{步骤1:读取当前策略}

\begin{verbatim}
old_action = policy[state]  // 例如:old_action = '上'
\end{verbatim}

\textbf{步骤2:计算所有动作的动作价值}

\begin{verbatim}
q_values = {}
for action in ['上', '下', '左', '右']:
    q_values[action] = compute_q_value(state, action, V)
// 结果:q_values = {'上': 9.0, '下': 13.5, '左': 7.2, '右': 10.8}
\end{verbatim}

\textbf{步骤3:找到最大值对应的动作}

\begin{verbatim}
best_action = argmax(q_values)  // best_action = '下'
\end{verbatim}

\textbf{步骤4:更新策略}

\begin{verbatim}
policy[state] = best_action  // policy[state] = '下'
// 这就是 π(s) ← 下 的具体操作
\end{verbatim}

\textbf{步骤5:检查策略是否改变}

\begin{verbatim}
if old_action != best_action:
    policy_changed = True
// 如果 old_action = '上',best_action = '下'
// 则策略改变了
\end{verbatim}

\subsubsection{可视化:策略更新的前后对比}

\textbf{更新前}:

\begin{center}
\begin{tabular}{|c|c|c|}
\hline
\textbf{状态} & \textbf{策略 $\pi_0(s)$} & \textbf{动作价值} \\
\hline
$s$ & 上 & $q(s, \text{上}) = 9.0$ \\
\hline
\end{tabular}
\end{center}

\textbf{计算所有动作的动作价值}:

\begin{center}
\begin{tabular}{|c|c|}
\hline
\textbf{动作} & \textbf{动作价值 $q(s, a)$} \\
\hline
上 & 9.0 \\
\hline
下 & \textcolor{red}{13.5}(最大值) \\
\hline
左 & 7.2 \\
\hline
右 & 10.8 \\
\hline
\end{tabular}
\end{center}

\textbf{更新后}:

\begin{center}
\begin{tabular}{|c|c|c|}
\hline
\textbf{状态} & \textbf{策略 $\pi_1(s)$} & \textbf{变化} \\
\hline
$s$ & \textcolor{red}{下} & 从"上"改为"下" \\
\hline
\end{tabular}
\end{center}

\textbf{内存中的变化}:

\begin{verbatim}
// 更新前
policy[s] = '上'

// 执行更新操作
policy[s] = '下'  // 这就是 π(s) ← 下

// 更新后
policy[s] = '下'
\end{verbatim}

\subsubsection{关键要点}

\begin{enumerate}
    \item \textbf{策略是一个映射}:从状态到动作的映射
    
    \item \textbf{更新是赋值操作}:
    \begin{itemize}
        \item $\pi(s) \gets a$ 就是将状态 $s$ 对应的动作更新为 $a$
        \item 在代码中就是:$\text{policy}[s] = a$
    \end{itemize}
    
    \item \textbf{只更新当前状态}:
    \begin{itemize}
        \item 更新 $\pi(s)$ 不影响其他状态的策略
        \item 每个状态的策略独立更新
    \end{itemize}
    
    \item \textbf{策略改进是批量更新}:
    \begin{itemize}
        \item 对每个状态 $s$,都执行一次 $\pi(s) \gets \arg\max_{a} q(s, a)$
        \item 所有状态更新完成后,得到新策略 $\pi'$
    \end{itemize}
\end{enumerate}

\subsubsection{处理平局的情况}

\textbf{问题}:如果有多个动作的动作价值相同且都是最大值怎么办?

\textbf{例子}:
\begin{align}
q(s, \text{上}) &= 13.5 \\
q(s, \text{下}) &= 13.5 \\
q(s, \text{左}) &= 7.2 \\
q(s, \text{右}) &= 10.8
\end{align}

\textbf{处理方式}:
\begin{itemize}
    \item 可以任意选择一个(通常按动作的固定顺序选择第一个)
    \item 或者可以随机选择一个
    \item 在随机策略的情况下,可以给所有最大值动作相等的概率
\end{itemize}

\textbf{算法实现}:
\begin{algorithm}[H]
\caption{策略改进步骤的详细实现}
\begin{algorithmic}[1]
\FOR{每个状态 $s \in \mathcal{S}$}
    \STATE $q_{\max} \gets -\infty$
    \STATE $a_{\text{best}} \gets \text{None}$
    \FOR{每个动作 $a \in \mathcal{A}(s)$}
        \STATE $q(s, a) \gets 0$
        \FOR{每个 $(s', r)$ 使得 $p(s', r | s, a) > 0$}
            \STATE $q(s, a) \gets q(s, a) + p(s', r | s, a) \times [r + \gamma V(s')]$
        \ENDFOR
        \IF{$q(s, a) > q_{\max}$}
            \STATE $q_{\max} \gets q(s, a)$
            \STATE $a_{\text{best}} \gets a$
        \ELSIF{$q(s, a) = q_{\max}$ \AND 需要处理平局}
            \STATE 可以随机选择或按固定顺序选择
        \ENDIF
    \ENDFOR
    \STATE $\pi(s) \gets a_{\text{best}}$
\ENDFOR
\end{algorithmic}
\end{algorithmic}

\subsubsection{完整示例:Jack租车问题}

考虑Jack租车问题中的状态 $s = (10, 10)$(地点1有10辆车,地点2有10辆车)。

\textbf{可能的动作}:$a \in \{-5, -4, -3, -2, -1, 0, 1, 2, 3, 4, 5\}$(移动的车辆数)

\textbf{步骤1:计算每个动作的动作价值}

假设当前价值函数 $V$ 的值(部分):
\begin{itemize}
    \item $V(9, 11) = 100$
    \item $V(10, 10) = 95$
    \item $V(11, 9) = 90$
    \item 等等...
\end{itemize}

\textbf{动作 $a = 0$}(不移动):
\begin{align}
q((10, 10), 0) &= \sum_{s', r} p(s', r | (10, 10), 0) [r + \gamma V(s')] \\
               &\approx 95 \quad \text{(简化计算)}
\end{align}

\textbf{动作 $a = 1$}(从地点1移动1辆车到地点2):
\begin{align}
q((10, 10), 1) &= \sum_{s', r} p(s', r | (10, 10), 1) [r + \gamma V(s')] \\
               &\approx 98 \quad \text{(考虑移动成本和未来价值)}
\end{align}

\textbf{动作 $a = -1$}(从地点2移动1辆车到地点1):
\begin{align}
q((10, 10), -1) &\approx 92
\end{align}

\textbf{步骤2:比较所有动作价值}

假设计算得到:
\begin{align}
q((10, 10), -5) &= 85 \\
q((10, 10), -4) &= 88 \\
q((10, 10), -3) &= 90 \\
q((10, 10), -2) &= 91 \\
q((10, 10), -1) &= 92 \\
q((10, 10), 0) &= 95 \\
q((10, 10), 1) &= 98 \\
q((10, 10), 2) &= 96 \\
q((10, 10), 3) &= 93 \\
q((10, 10), 4) &= 89 \\
q((10, 10), 5) &= 85
\end{align}

\textbf{步骤3:选择最大值}

\begin{align}
q_{\max} &= \max\{85, 88, 90, 91, 92, 95, 98, 96, 93, 89, 85\} = 98 \\
a^* &= 1 \quad \text{(因为 } q((10, 10), 1) = 98 = q_{\max} \text{)}
\end{align}

\textbf{更新策略}:
\begin{equation}
\pi((10, 10)) \gets 1
\end{equation}

即:在状态 $(10, 10)$,最优动作是从地点1移动1辆车到地点2。

\subsubsection{关键要点总结}

\begin{enumerate}
    \item \textbf{计算阶段}:
    \begin{itemize}
        \item 对每个动作,计算 $q(s, a) = \sum_{s', r} p(s', r | s, a) [r + \gamma V(s')]$
        \item 这需要遍历所有可能的 $(s', r)$ 组合
    \end{itemize}
    
    \item \textbf{比较阶段}:
    \begin{itemize}
        \item 比较所有动作的动作价值
        \item 找到最大值 $q_{\max} = \max_a q(s, a)$
    \end{itemize}
    
    \item \textbf{选择阶段}:
    \begin{itemize}
        \item 找到使 $q(s, a) = q_{\max}$ 的动作 $a^*$
        \item 更新策略:$\pi(s) \gets a^*$
    \end{itemize}
    
    \item \textbf{时间复杂度}:
    \begin{itemize}
        \item 对每个状态:$O(|\mathcal{A}(s)| \times |\mathcal{S}| \times |\mathcal{R}|)$
        \item 对所有状态:$O(|\mathcal{S}| \times |\mathcal{A}| \times |\mathcal{S}| \times |\mathcal{R}|)$
    \end{itemize}
\end{enumerate}

\subsection{策略迭代的收敛性}

\begin{theorem}[策略迭代收敛性]
对于有限MDP,策略迭代算法在有限次迭代后收敛到最优策略 $\pi_*$ 和最优价值函数 $v_*$。
\end{theorem}

\textbf{证明思路}:
\begin{itemize}
    \item 每次策略改进都严格改进策略(除非已经最优)
    \item 有限MDP只有有限个策略
    \item 因此算法在有限次迭代后必须收敛
\end{itemize}

\subsection{重要澄清:$v_\pi$ 不是最优价值函数}

\textbf{常见误解}:在策略迭代过程中,每一步的策略改进是否都产生一个最优的价值函数?

\textbf{答案}:\textbf{不是的}!

\textbf{关键区别}:

\begin{center}
\begin{tabular}{|l|c|c|}
\hline
\textbf{概念} & \textbf{符号} & \textbf{含义} \\
\hline
策略 $\pi$ 的价值函数 & $v_\pi(s)$ & 遵循策略 $\pi$ 的期望回报(\textbf{不是最优}) \\
\hline
最优价值函数 & $v_*(s)$ & 所有策略中的最大期望回报(\textbf{最优}) \\
\hline
\end{tabular}
\end{center}

\textbf{策略迭代过程中的价值函数}:

在策略迭代过程中:
\begin{equation}
v_{\pi_0} \leq v_{\pi_1} \leq v_{\pi_2} \leq \cdots \leq v_{\pi_*} = v_*
\end{equation}

\textbf{关键点}:
\begin{itemize}
    \item $v_{\pi_k}$ 是策略 $\pi_k$ 的价值函数,\textbf{不是最优的}
    \item 每一步都是\textbf{改进},但不一定是最优
    \item 只有 $v_{\pi_*} = v_*$ 时才是最优
\end{itemize}

\textbf{策略改进定理的含义}:

策略改进定理说的是:
\begin{itemize}
    \item 如果 $q_\pi(s, \pi'(s)) \geq v_\pi(s)$,则 $v_{\pi'}(s) \geq v_\pi(s)$
    \item 即:新策略\textbf{至少和旧策略一样好}
    \item 但\textbf{不意味着} $v_\pi$ 或 $v_{\pi'}$ 是最优的
\end{itemize}

\textbf{具体例子}:

假设在某个状态 $s$:
\begin{itemize}
    \item $v_{\pi_0}(s) = 10$(初始策略的价值)
    \item $v_*(s) = 20$(最优价值)
\end{itemize}

\textbf{第1次策略改进后}:
\begin{itemize}
    \item $v_{\pi_1}(s) = 15$(比 $\pi_0$ 好,但\textbf{不是最优})
    \item $v_{\pi_1}(s) = 15 < 20 = v_*(s)$
\end{itemize}

\textbf{第2次策略改进后}:
\begin{itemize}
    \item $v_{\pi_2}(s) = 18$(继续改进,但仍\textbf{不是最优})
    \item $v_{\pi_2}(s) = 18 < 20 = v_*(s)$
\end{itemize}

\textbf{收敛后}:
\begin{itemize}
    \item $v_{\pi_*}(s) = 20 = v_*(s)$(此时才达到最优)
\end{itemize}

\textbf{为什么需要多次迭代?}

\begin{enumerate}
    \item \textbf{策略改进只保证局部改进}:
    \begin{itemize}
        \item 新策略至少和旧策略一样好
        \item 但可能还有更好的策略
    \end{itemize}
    
    \item \textbf{需要继续迭代}:
    \begin{itemize}
        \item 每次改进都使策略更好
        \item 但需要多次迭代才能找到最优策略
    \end{itemize}
    
    \item \textbf{收敛条件}:
    \begin{itemize}
        \item 只有当策略不再改变时,才说明找到了最优策略
        \item 此时 $v_{\pi_*} = v_*$
    \end{itemize}
\end{enumerate}

\textbf{数学表达}:

在策略迭代过程中:
\begin{align}
v_{\pi_0}(s) &\leq v_{\pi_1}(s) \leq v_{\pi_2}(s) \leq \cdots \leq v_{\pi_k}(s) \leq \cdots \leq v_*(s) \\
\pi_0 &\xrightarrow{I} \pi_1 \xrightarrow{I} \pi_2 \xrightarrow{I} \cdots \xrightarrow{I} \pi_*
\end{align}

\textbf{关键洞察}:
\begin{quote}
\textbf{策略改进产生的是改进后的策略和价值函数,不是最优的。只有收敛后(策略不再改变)才得到最优策略和最优价值函数。策略迭代是一个逐步改进的过程,不是一步到位。}
\end{quote}

\textbf{总结}:
\begin{itemize}
    \item \textbf{每一步的策略改进}:产生一个\textbf{更好的}策略,而不是最优策略
    \item \textbf{只有算法收敛时}:才得到最优策略 $\pi_*$ 和最优价值函数 $v_*$
    \item \textbf{策略迭代}:是一个\textbf{逐步改进}的过程
\end{itemize}

\subsection{策略迭代 vs 价值迭代}

\textbf{策略迭代}:
\begin{itemize}
    \item 每次迭代都完全评估当前策略
    \item 然后改进策略
    \item 通常收敛更快(迭代次数少)
    \item 但每次迭代计算量大
\end{itemize}

\textbf{价值迭代}:
\begin{itemize}
    \item 每次迭代只更新一次价值函数
    \item 不显式维护策略
    \item 迭代次数可能更多
    \item 但每次迭代计算量小
\end{itemize}

\section{练习4.5:基于动作价值的策略迭代}

\subsection{问题}

\textbf{练习4.5}:如何定义基于动作价值的策略迭代?给出计算 $q_*$ 的完整算法,类似于第80页计算 $v_*$ 的算法。

\subsection{关键思想}

策略迭代可以基于动作价值函数 $q_\pi(s, a)$ 而不是状态价值函数 $v_\pi(s)$ 来定义。

\textbf{优势}:
\begin{itemize}
    \item 不需要环境模型来计算动作价值
    \item 可以直接从动作价值函数得到策略
    \item 在某些情况下更直观
\end{itemize}

\subsection{动作价值函数的策略评估}

\textbf{动作价值函数的贝尔曼方程}:

\begin{equation}
q_\pi(s, a) = \sum_{s', r} p(s', r | s, a) \left[r + \gamma \sum_{a'} \pi(a' | s') q_\pi(s', a')\right]
\label{eq:q_bellman}
\end{equation}

\textbf{迭代策略评估}(动作价值函数):

\begin{equation}
q_\pi^{k+1}(s, a) = \sum_{s', r} p(s', r | s, a) \left[r + \gamma \sum_{a'} \pi(a' | s') q_\pi^k(s', a')\right]
\label{eq:q_policy_evaluation}
\end{equation}

\subsection{动作价值函数的策略改进}

\textbf{策略改进}:基于动作价值函数 $q_\pi$,改进策略:

\begin{equation}
\pi'(s) = \arg\max_{a} q_\pi(s, a)
\label{eq:q_policy_improvement}
\end{equation}

\textbf{策略改进定理}(动作价值函数版本):

如果 $q_\pi(s, \pi'(s)) \geq q_\pi(s, \pi(s))$ 对所有 $s$ 成立,则策略 $\pi'$ 至少与策略 $\pi$ 一样好。

\subsection{基于动作价值的策略迭代算法}

\begin{algorithm}[H]
\caption{策略迭代算法(基于动作价值函数,计算 $q_*$ 和 $\pi_*$)}
\begin{algorithmic}[1]
\REQUIRE 环境动态 $p(s', r | s, a)$,折扣因子 $\gamma$,收敛阈值 $\theta$
\ENSURE 最优动作价值函数 $q_*$ 和最优策略 $\pi_*$
\STATE \textbf{初始化}:$Q(s, a) \in \mathbb{R}$ 和 $\pi(s) \in \mathcal{A}(s)$ 任意初始化,对所有 $s \in \mathcal{S}$ 和 $a \in \mathcal{A}(s)$
\REPEAT
    \STATE \textbf{策略评估}(动作价值函数):
    \REPEAT
        \STATE $\Delta \gets 0$
        \FOR{每个状态 $s \in \mathcal{S}$}
            \FOR{每个动作 $a \in \mathcal{A}(s)$}
                \STATE $q \gets Q(s, a)$
                \STATE $Q(s, a) \gets \sum_{s', r} p(s', r | s, a) \left[r + \gamma \sum_{a'} \pi(a' | s') Q(s', a')\right]$
                \STATE $\Delta \gets \max(\Delta, |q - Q(s, a)|)$
            \ENDFOR
        \ENDFOR
    \UNTIL{$\Delta < \theta$}
    \STATE \textbf{策略改进}:
    \STATE $\text{policy-stable} \gets \text{true}$
    \FOR{每个状态 $s \in \mathcal{S}$}
        \STATE $a_{\text{old}} \gets \pi(s)$
        \STATE $\pi(s) \gets \arg\max_{a} Q(s, a)$
        \IF{$a_{\text{old}} \neq \pi(s)$}
            \STATE $\text{policy-stable} \gets \text{false}$
        \ENDIF
    \ENDFOR
\UNTIL{$\text{policy-stable} = \text{true}$}
\RETURN $q_* = Q$,$\pi_* = \pi$
\end{algorithmic}
\end{algorithmic}

\subsection{算法步骤详解}

\textbf{步骤1:初始化}
\begin{itemize}
    \item 初始化动作价值函数 $Q(s, a)$(可以任意,通常设为0)
    \item 初始化策略 $\pi(s)$(可以任意,通常设为随机策略)
\end{itemize}

\textbf{步骤2:策略评估(动作价值函数)}
\begin{itemize}
    \item 使用迭代策略评估计算当前策略 $\pi$ 的动作价值函数 $q_\pi$
    \item 迭代更新直到收敛:
    \begin{equation}
    Q(s, a) \gets \sum_{s', r} p(s', r | s, a) \left[r + \gamma \sum_{a'} \pi(a' | s') Q(s', a')\right]
    \end{equation}
    \item 当动作价值函数变化小于阈值 $\theta$ 时停止
\end{itemize}

\textbf{步骤3:策略改进}
\begin{itemize}
    \item 对每个状态 $s$,选择使动作价值最大的动作:
    \begin{equation}
    \pi(s) \gets \arg\max_{a} Q(s, a)
    \end{equation}
    \item 如果策略没有改变,算法收敛
\end{itemize}

\subsection{与基于状态价值的策略迭代的对比}

\begin{center}
\begin{tabular}{|l|c|c|}
\hline
\textbf{特性} & \textbf{基于 $v_\pi$} & \textbf{基于 $q_\pi$} \\
\hline
\textbf{评估对象} & 状态价值函数 $v_\pi(s)$ & 动作价值函数 $q_\pi(s, a)$ \\
\hline
\textbf{评估公式} & $v_\pi(s) = \sum_a \pi(a|s) \sum_{s',r} p(\cdot)[r + \gamma v_\pi(s')]$ & $q_\pi(s,a) = \sum_{s',r} p(\cdot)[r + \gamma \sum_{a'} \pi(a'|s') q_\pi(s',a')]$ \\
\hline
\textbf{改进公式} & $\pi'(s) = \arg\max_a \sum_{s',r} p(\cdot)[r + \gamma v_\pi(s')]$ & $\pi'(s) = \arg\max_a q_\pi(s, a)$ \\
\hline
\textbf{存储空间} & $|\mathcal{S}|$ & $|\mathcal{S}| \times |\mathcal{A}|$ \\
\hline
\textbf{计算复杂度} & 较低 & 较高 \\
\hline
\textbf{优势} & 存储空间小 & 策略改进更直接 \\
\hline
\end{tabular}
\end{center}

\section{数学推导}

\subsection{动作价值函数的贝尔曼方程推导}

从动作价值函数的定义:

\begin{align}
q_\pi(s, a) &= \mathbb{E}_\pi[G_t | S_t = s, A_t = a] \\
            &= \mathbb{E}_\pi[R_{t+1} + \gamma G_{t+1} | S_t = s, A_t = a]
\end{align}

分解为即时奖励和未来回报:

\begin{align}
q_\pi(s, a) &= \mathbb{E}_\pi[R_{t+1} | S_t = s, A_t = a] + \gamma \mathbb{E}_\pi[G_{t+1} | S_t = s, A_t = a]
\end{align}

第一项是期望即时奖励:

\begin{align}
\mathbb{E}_\pi[R_{t+1} | S_t = s, A_t = a] &= \sum_{s', r} p(s', r | s, a) \cdot r
\end{align}

第二项需要展开到下一状态和动作:

\begin{align}
\mathbb{E}_\pi[G_{t+1} | S_t = s, A_t = a] &= \sum_{s', r} p(s', r | s, a) \mathbb{E}_\pi[G_{t+1} | S_t = s, A_t = a, S_{t+1} = s', R_{t+1} = r]
\end{align}

由于马尔可夫性质,给定 $S_{t+1} = s'$,$G_{t+1}$ 的分布依赖于 $s'$ 和后续策略:

\begin{align}
\mathbb{E}_\pi[G_{t+1} | S_{t+1} = s'] &= \sum_{a'} \pi(a' | s') \mathbb{E}_\pi[G_{t+1} | S_{t+1} = s', A_{t+1} = a'] \\
                                      &= \sum_{a'} \pi(a' | s') q_\pi(s', a')
\end{align}

因此:

\begin{align}
q_\pi(s, a) &= \sum_{s', r} p(s', r | s, a) \left[r + \gamma \sum_{a'} \pi(a' | s') q_\pi(s', a')\right]
\end{align}

这就是公式 \eqref{eq:q_bellman}。

\subsection{最优动作价值函数的贝尔曼方程}

\textbf{最优动作价值函数}:

\begin{equation}
q_*(s, a) = \max_\pi q_\pi(s, a)
\end{equation}

\textbf{最优动作价值函数的贝尔曼方程}:

\begin{equation}
q_*(s, a) = \sum_{s', r} p(s', r | s, a) \left[r + \gamma \max_{a'} q_*(s', a')\right]
\label{eq:q_star_bellman}
\end{equation}

\textbf{推导}:

\begin{align}
q_*(s, a) &= \max_\pi q_\pi(s, a) \\
          &= \max_\pi \sum_{s', r} p(s', r | s, a) \left[r + \gamma \sum_{a'} \pi(a' | s') q_\pi(s', a')\right]
\end{align}

对于最优策略,在下一状态 $s'$ 应该选择最优动作:

\begin{align}
q_*(s, a) &= \sum_{s', r} p(s', r | s, a) \left[r + \gamma \max_{a'} q_*(s', a')\right]
\end{align}

\section{算法实现细节}

\subsection{策略评估的收敛性}

\textbf{定理}:迭代策略评估(动作价值函数)收敛到 $q_\pi$。

\textbf{证明思路}:
\begin{itemize}
    \item 动作价值函数的贝尔曼方程是压缩映射
    \item 迭代更新收敛到唯一不动点
    \item 不动点就是 $q_\pi$
\end{itemize}

\subsection{策略改进的有效性}

\textbf{策略改进定理}(动作价值函数版本):

如果 $q_\pi(s, \pi'(s)) \geq q_\pi(s, \pi(s))$ 对所有 $s$ 成立,则 $q_{\pi'}(s, a) \geq q_\pi(s, a)$ 对所有 $s, a$ 成立。

\textbf{证明思路}:
\begin{itemize}
    \item 在状态 $s$,新策略选择更好的动作
    \item 由于后续遵循策略 $\pi$,且 $q_\pi$ 已经考虑了全局回报
    \item 因此新策略至少与旧策略一样好
\end{itemize}

\section{具体例子}

\subsection{Gridworld示例}

考虑一个简单的Gridworld,使用基于动作价值的策略迭代:

\textbf{初始化}:
\begin{itemize}
    \item $Q(s, a) = 0$ 对所有 $s, a$
    \item $\pi(s)$ 为随机策略
\end{itemize}

\textbf{第1次迭代 - 策略评估}:

对状态 $s$ 和动作 $a$:
\begin{align}
Q(s, a) &= \sum_{s', r} p(s', r | s, a) \left[r + \gamma \sum_{a'} \pi(a' | s') Q(s', a')\right] \\
        &= \sum_{s', r} p(s', r | s, a) [r + \gamma \times 0] \\
        &= \sum_{s', r} p(s', r | s, a) \cdot r
\end{align}

\textbf{第1次迭代 - 策略改进}:

对每个状态 $s$:
\begin{equation}
\pi(s) = \arg\max_{a} Q(s, a) = \arg\max_{a} \sum_{s', r} p(s', r | s, a) \cdot r
\end{equation}

选择使期望即时奖励最大的动作。

\textbf{后续迭代}:

继续交替执行策略评估和策略改进,直到策略不再改变。

\section{策略迭代 vs 策略改进}

\subsection{关键区别}

\textbf{策略改进}(Policy Improvement)和\textbf{策略迭代}(Policy Iteration)是两个相关但不同的概念:

\begin{center}
\begin{tabular}{|l|c|c|}
\hline
\textbf{特性} & \textbf{策略改进} & \textbf{策略迭代} \\
\hline
\textbf{定义} & 单个步骤:基于价值函数改进策略 & 完整算法:交替执行评估和改进 \\
\hline
\textbf{输入} & 价值函数 $v_\pi$ 或 $q_\pi$ & 初始策略 $\pi_0$ \\
\hline
\textbf{输出} & 改进后的策略 $\pi'$ & 最优策略 $\pi_*$ 和最优价值函数 $v_*$ \\
\hline
\textbf{步骤数} & 1步 & 多步(直到收敛) \\
\hline
\textbf{包含内容} & 只包含策略改进步骤 & 包含策略评估 + 策略改进 \\
\hline
\textbf{关系} & 策略迭代的一个组成部分 & 包含策略改进作为子步骤 \\
\hline
\end{tabular}
\end{center}

\subsection{策略改进:单个步骤}

\textbf{策略改进}是一个\textbf{单步操作}:

\begin{quote}
给定策略 $\pi$ 的价值函数 $v_\pi$(或 $q_\pi$),策略改进生成一个新策略 $\pi'$,使得 $\pi'$ 至少与 $\pi$ 一样好。
\end{quote}

\textbf{策略改进的公式}:

基于状态价值函数:
\begin{equation}
\pi'(s) = \arg\max_{a} \sum_{s', r} p(s', r | s, a) [r + \gamma v_\pi(s')]
\end{equation}

基于动作价值函数:
\begin{equation}
\pi'(s) = \arg\max_{a} q_\pi(s, a)
\end{equation}

\textbf{特点}:
\begin{itemize}
    \item \textbf{输入}:价值函数 $v_\pi$ 或 $q_\pi$
    \item \textbf{输出}:改进后的策略 $\pi'$
    \item \textbf{操作}:对每个状态选择最优动作
    \item \textbf{保证}:$\pi'$ 至少与 $\pi$ 一样好
\end{itemize}

\subsection{策略迭代:完整算法}

\textbf{策略迭代}是一个\textbf{完整算法},包含多个步骤:

\begin{quote}
策略迭代通过交替执行策略评估和策略改进,逐步找到最优策略。
\end{quote}

\textbf{策略迭代的流程}:

\begin{enumerate}
    \item \textbf{策略评估}:计算当前策略 $\pi$ 的价值函数 $v_\pi$
    \item \textbf{策略改进}:基于 $v_\pi$ 改进策略,得到 $\pi'$
    \item \textbf{重复}:将 $\pi'$ 作为新的 $\pi$,重复步骤1和2
    \item \textbf{终止}:当策略不再改变时停止
\end{enumerate}

\textbf{特点}:
\begin{itemize}
    \item \textbf{输入}:初始策略 $\pi_0$
    \item \textbf{输出}:最优策略 $\pi_*$ 和最优价值函数 $v_*$
    \item \textbf{操作}:多次执行评估和改进
    \item \textbf{保证}:收敛到最优策略
\end{itemize}

\subsection{关系图示}

\textbf{策略改进}(单步):
\begin{equation}
v_\pi \xrightarrow{\text{策略改进}} \pi'
\end{equation}

\textbf{策略迭代}(多步循环):
\begin{equation}
\pi_0 \xrightarrow{\text{评估}} v_{\pi_0} \xrightarrow{\text{改进}} \pi_1 \xrightarrow{\text{评估}} v_{\pi_1} \xrightarrow{\text{改进}} \pi_2 \xrightarrow{\text{评估}} \cdots \xrightarrow{\text{改进}} \pi_*
\end{equation}

\subsection{具体例子}

\textbf{例子:Gridworld}

\textbf{策略改进}(单步):
\begin{itemize}
    \item 给定:当前策略 $\pi$ 的价值函数 $v_\pi$
    \item 操作:对每个状态 $s$,计算 $q_\pi(s, a)$ 对所有动作 $a$,选择最优动作
    \item 结果:得到改进后的策略 $\pi'$
    \item 时间:执行一次
\end{itemize}

\textbf{策略迭代}(完整算法):
\begin{itemize}
    \item 给定:初始策略 $\pi_0$(例如:随机策略)
    \item 第1次迭代:
    \begin{itemize}
        \item 评估:计算 $v_{\pi_0}$
        \item 改进:得到 $\pi_1$
    \end{itemize}
    \item 第2次迭代:
    \begin{itemize}
        \item 评估:计算 $v_{\pi_1}$
        \item 改进:得到 $\pi_2$
    \end{itemize}
    \item 继续迭代,直到 $\pi_k = \pi_{k+1}$(策略不再改变)
    \item 结果:得到最优策略 $\pi_*$
    \item 时间:执行多次迭代
\end{itemize}

\subsection{类比理解}

\textbf{策略改进} 类似于:
\begin{itemize}
    \item 给定一张地图(价值函数),找到从当前位置到目的地的最佳下一步(改进策略)
    \item 这是一个\textbf{单步决策}
\end{itemize}

\textbf{策略迭代} 类似于:
\begin{itemize}
    \item 从起点开始,不断更新地图并选择最佳路径,直到找到最优路径
    \item 这是一个\textbf{完整过程}
\end{itemize}

\subsection{为什么需要策略迭代?}

\textbf{问题}:为什么不能只执行一次策略改进?

\textbf{答案}:
\begin{itemize}
    \item 策略改进需要知道当前策略的价值函数
    \item 但价值函数依赖于策略本身
    \item 因此需要:
    \begin{enumerate}
        \item 先评估策略,得到价值函数
        \item 再改进策略
        \item 新策略可能更好,需要重新评估
        \item 重复这个过程直到收敛
    \end{enumerate}
\end{itemize}

\textbf{策略迭代的必要性}:
\begin{itemize}
    \item 策略和价值函数相互依赖
    \item 需要迭代求解
    \item 策略迭代提供了系统化的方法
\end{itemize}

\section{总结}

\subsection{策略迭代的核心要点}

\begin{enumerate}
    \item \textbf{两个步骤}:
    \begin{itemize}
        \item 策略评估:计算当前策略的价值函数
        \item 策略改进:基于价值函数改进策略
    \end{itemize}
    
    \item \textbf{收敛性}:
    \begin{itemize}
        \item 每次迭代都严格改进策略
        \item 有限MDP保证在有限次迭代后收敛
        \item 收敛到最优策略和最优价值函数
    \end{itemize}
    
    \item \textbf{两种形式}:
    \begin{itemize}
        \item 基于状态价值函数 $v_\pi$
        \item 基于动作价值函数 $q_\pi$
    \end{itemize}
\end{enumerate}

\subsection{练习4.5的答案}

\textbf{基于动作价值的策略迭代算法}:

\begin{enumerate}
    \item \textbf{初始化}:$Q(s, a)$ 和 $\pi(s)$ 任意初始化
    
    \item \textbf{策略评估}:
    \begin{equation}
    Q(s, a) \gets \sum_{s', r} p(s', r | s, a) \left[r + \gamma \sum_{a'} \pi(a' | s') Q(s', a')\right]
    \end{equation}
    迭代直到收敛
    
    \item \textbf{策略改进}:
    \begin{equation}
    \pi(s) \gets \arg\max_{a} Q(s, a)
    \end{equation}
    
    \item \textbf{重复}步骤2和3,直到策略不再改变
\end{enumerate}

\textbf{关键区别}:
\begin{itemize}
    \item 评估的是动作价值函数 $q_\pi$ 而不是状态价值函数 $v_\pi$
    \item 策略改进直接使用 $Q(s, a)$,不需要计算动作价值
    \item 需要存储 $|\mathcal{S}| \times |\mathcal{A}|$ 个值
\end{itemize}

\subsection{重要性}

\begin{quote}
\textbf{练习4.5特别重要},因为其中涉及的思想将在本书的其余部分中广泛使用。基于动作价值的强化学习算法(如Q学习、SARSA等)都建立在这些思想之上。
\end{quote}

\vspace{1cm}

\textbf{参考文献}:
\begin{itemize}
    \item Sutton, R. S., \& Barto, A. G. (2018). \textit{Reinforcement Learning: An Introduction} (2nd Edition). MIT Press, Chapter 4, Exercise 4.5.
\end{itemize}

\end{document}


