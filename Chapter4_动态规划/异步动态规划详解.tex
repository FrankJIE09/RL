\documentclass[12pt,a4paper]{article}
\usepackage[UTF8]{ctex}
\usepackage{amsmath}
\usepackage{amssymb}
\usepackage{amsthm}
\usepackage{geometry}
\usepackage{hyperref}
\usepackage{enumitem}
\usepackage{xcolor}
\usepackage{bm}
\usepackage{tikz}
\usepackage{booktabs}
\usepackage{array}
\usepackage{algorithm}
\usepackage{algorithmic}

\geometry{margin=2.5cm}

\title{异步动态规划详解}
\subtitle{4.5节:异步更新与同步更新的对比}
\author{}
\date{}

\newtheorem{definition}{定义}
\newtheorem{theorem}{定理}
\newtheorem{proposition}{命题}
\newtheorem{example}{示例}
\newtheorem{remark}{注记}

\begin{document}

\maketitle

\tableofcontents
\newpage

\section{引言}

\subsection{同步动态规划的局限性}

到目前为止讨论的动态规划方法(策略迭代、价值迭代)都有一个主要缺点:

\textbf{问题}:
\begin{itemize}
    \item 它们涉及对整个状态集的操作
    \item 需要\textbf{系统性地扫描}(sweep)整个状态集
    \item 如果状态集非常大,即使单次扫描也可能非常昂贵
\end{itemize}

\textbf{例子}:
\begin{itemize}
    \item 西洋双陆棋(backgammon)有超过 $10^{20}$ 个状态
    \item 即使每秒能更新100万个状态,完成单次扫描也需要超过1000年
\end{itemize}

\subsection{异步动态规划的提出}

\textbf{关键思想}:
\begin{quote}
能否避免系统性地扫描整个状态集?能否以任意顺序更新状态,使用当前可用的任何值?
\end{quote}

\textbf{异步动态规划}:
\begin{itemize}
    \item 不需要系统性地扫描状态集
    \item 可以以任意顺序更新状态
    \item 使用当前可用的任何状态值
    \item 某些状态可能被更新多次,而其他状态只更新一次
\end{itemize}

\section{什么是"异步"?}

\subsection{基本概念}

\begin{definition}[异步(Asynchronous)]
\textbf{异步}指的是操作的执行顺序不是预先确定的、系统性的,而是可以以任意顺序、任意时机执行。
\end{definition}

\textbf{在计算机科学中}:
\begin{itemize}
    \item \textbf{同步}:操作按固定顺序、固定时机执行
    \item \textbf{异步}:操作可以以任意顺序、任意时机执行
\end{itemize}

\textbf{在动态规划中}:
\begin{itemize}
    \item \textbf{同步}:按固定顺序(如从左到右、从上到下)更新所有状态
    \item \textbf{异步}:以任意顺序更新状态,可能跳过某些状态,某些状态可能被多次更新
\end{itemize}

\subsection{生活中的类比}

\textbf{同步的例子}:
\begin{itemize}
    \item 工厂流水线:每个工位按固定顺序处理产品
    \item 学生按学号顺序交作业
\end{itemize}

\textbf{异步的例子}:
\begin{itemize}
    \item 餐厅点餐:顾客可以任意顺序点餐,不需要排队
    \item 学生可以任意顺序交作业,不需要按学号
\end{itemize}

\section{同步动态规划 vs 异步动态规划}

\subsection{同步动态规划(Synchronous DP)}

\textbf{特点}:
\begin{itemize}
    \item \textbf{系统性扫描}:按固定顺序遍历所有状态
    \item \textbf{批量更新}:在一次迭代中,所有状态都使用\textbf{上一次迭代}的值进行更新
    \item \textbf{需要存储两个数组}:$V_k$ 和 $V_{k+1}$
\end{itemize}

\textbf{价值迭代的同步版本}:

\begin{algorithm}[H]
\caption{同步价值迭代}
\begin{algorithmic}[1]
\REQUIRE 环境动态 $p(s', r | s, a)$,折扣因子 $\gamma$,收敛阈值 $\theta > 0$
\ENSURE 最优价值函数 $v_*$
\STATE \textbf{初始化}:$V_{\text{old}}(s) = 0$,对所有 $s \in \mathcal{S}$
\REPEAT
    \STATE $V_{\text{new}}(s) \gets 0$,对所有 $s \in \mathcal{S}$ \COMMENT{创建新数组}
    \STATE $\Delta \gets 0$
    \FOR{每个状态 $s \in \mathcal{S}$ \textbf{按固定顺序}}
        \STATE $V_{\text{new}}(s) \gets \max_{a} \sum_{s', r} p(s', r | s, a) [r + \gamma V_{\text{old}}(s')]$
        \STATE $\Delta \gets \max(\Delta, |V_{\text{new}}(s) - V_{\text{old}}(s)|)$
    \ENDFOR
    \STATE $V_{\text{old}} \gets V_{\text{new}}$ \COMMENT{批量替换}
\UNTIL{$\Delta < \theta$}
\RETURN $v_* = V_{\text{old}}$
\end{algorithmic}
\end{algorithm}

\textbf{关键特征}:
\begin{enumerate}
    \item 使用 $V_{\text{old}}$ 计算所有 $V_{\text{new}}$ 的值
    \item 所有状态更新完成后,才将 $V_{\text{new}}$ 赋值给 $V_{\text{old}}$
    \item 需要存储两个完整的价值函数数组
\end{enumerate}

\subsection{异步动态规划(Asynchronous DP)}

\begin{definition}[异步动态规划]
\textbf{异步动态规划}是一种就地(in-place)迭代DP算法,它不按系统扫描的方式组织。这些算法以任意顺序更新状态值,使用当前可用的任何其他状态的值。
\end{definition}

\textbf{特点}:
\begin{itemize}
    \item \textbf{就地更新}:直接修改当前价值函数,不需要两个数组
    \item \textbf{任意顺序}:可以以任意顺序更新状态
    \item \textbf{使用最新值}:更新状态 $s$ 时,使用当前最新的其他状态值(可能已经更新过)
    \item \textbf{灵活选择}:可以选择更新哪些状态,跳过哪些状态
\end{itemize}

\textbf{价值迭代的异步版本}:

\begin{algorithm}[H]
\caption{异步价值迭代}
\begin{algorithmic}[1]
\REQUIRE 环境动态 $p(s', r | s, a)$,折扣因子 $\gamma$,收敛阈值 $\theta > 0$
\ENSURE 最优价值函数 $v_*$
\STATE \textbf{初始化}:$V(s) = 0$,对所有 $s \in \mathcal{S}$
\REPEAT
    \STATE 选择状态 $s$(任意顺序,可能重复选择)
    \STATE $v_{\text{old}} \gets V(s)$
    \STATE $V(s) \gets \max_{a} \sum_{s', r} p(s', r | s, a) [r + \gamma V(s')]$ \COMMENT{使用当前最新的 $V(s')$}
    \STATE $\Delta \gets |V(s) - v_{\text{old}}|$
\UNTIL{所有状态都满足 $\Delta < \theta$ 且每个状态都被更新过}
\RETURN $v_* = V$
\end{algorithmic}
\end{algorithm}

\textbf{关键特征}:
\begin{enumerate}
    \item 只使用一个数组 $V$
    \item 更新状态 $s$ 时,使用当前 $V$ 中的值(可能已经更新过)
    \item 某些状态可能被更新多次,其他状态可能还没更新
\end{enumerate}

\section{具体例子对比}

\subsection{Gridworld例子}

考虑一个3×3的Gridworld:

\begin{center}
\begin{tabular}{|c|c|c|}
\hline
$s_1$ & $s_2$ & $s_3$ \\
\hline
$s_4$ & $s_5$ & $s_6$ \\
\hline
$s_7$ & $s_8$ & $s_9$ \\
\hline
\end{tabular}
\end{center}

\textbf{初始价值函数}:$V_0(s_i) = 0$,对所有 $i$

\subsection{同步价值迭代(第1次迭代)}

\textbf{更新顺序}:$s_1, s_2, s_3, s_4, s_5, s_6, s_7, s_8, s_9$

\textbf{所有状态都使用 $V_0$ 的值}:

\begin{align}
V_1(s_1) &= \max_{a} \sum_{s', r} p(s', r | s_1, a) [r + \gamma V_0(s')] \\
V_1(s_2) &= \max_{a} \sum_{s', r} p(s', r | s_2, a) [r + \gamma V_0(s')] \\
&\vdots \\
V_1(s_9) &= \max_{a} \sum_{s', r} p(s', r | s_9, a) [r + \gamma V_0(s')]
\end{align}

\textbf{关键点}:
\begin{itemize}
    \item 计算 $V_1(s_2)$ 时,使用 $V_0(s_1)$,而不是 $V_1(s_1)$
    \item 所有状态都基于"旧"的值更新
    \item 更新完成后,$V_1$ 替换 $V_0$
\end{itemize}

\subsection{异步价值迭代(第1次迭代)}

\textbf{更新顺序}:$s_5, s_2, s_5, s_8, s_1, s_5, s_3, \ldots$(任意顺序,可能重复)

\textbf{使用当前最新的值}:

\begin{enumerate}
    \item \textbf{更新 $s_5$}:
    \begin{equation}
    V(s_5) \gets \max_{a} \sum_{s', r} p(s', r | s_5, a) [r + \gamma V(s')]
    \end{equation}
    使用 $V(s_2) = 0$,$V(s_4) = 0$,$V(s_6) = 0$,$V(s_8) = 0$(都是初始值)
    
    \item \textbf{更新 $s_2$}:
    \begin{equation}
    V(s_2) \gets \max_{a} \sum_{s', r} p(s', r | s_2, a) [r + \gamma V(s')]
    \end{equation}
    使用 $V(s_5)$(已经更新过!),$V(s_1) = 0$,$V(s_3) = 0$
    
    \item \textbf{再次更新 $s_5$}:
    \begin{equation}
    V(s_5) \gets \max_{a} \sum_{s', r} p(s', r | s_5, a) [r + \gamma V(s')]
    \end{equation}
    使用 $V(s_2)$(已经更新过!),$V(s_4) = 0$,$V(s_6) = 0$,$V(s_8) = 0$
    
    \item \textbf{更新 $s_8$}:
    \begin{equation}
    V(s_8) \gets \max_{a} \sum_{s', r} p(s', r | s_8, a) [r + \gamma V(s')]
    \end{equation}
    使用 $V(s_5)$(已经更新过两次!),$V(s_7) = 0$,$V(s_9) = 0$
\end{enumerate}

\textbf{关键点}:
\begin{itemize}
    \item 计算 $V(s_2)$ 时,使用 $V(s_5)$(已经更新过)
    \item 再次更新 $s_5$ 时,使用 $V(s_2)$(已经更新过)
    \item 信息传播更快:更新的值立即被后续更新使用
    \item 某些状态(如 $s_5$)可能被更新多次
\end{itemize}

\section{异步动态规划的优势}

\subsection{1. 内存效率}

\textbf{同步DP}:
\begin{itemize}
    \item 需要存储两个数组:$V_{\text{old}}$ 和 $V_{\text{new}}$
    \item 内存需求:$2|\mathcal{S}|$
\end{itemize}

\textbf{异步DP}:
\begin{itemize}
    \item 只需要一个数组:$V$
    \item 内存需求:$|\mathcal{S}|$
    \item \textbf{节省50\%的内存}
\end{itemize}

\subsection{2. 信息传播速度}

\textbf{同步DP}:
\begin{itemize}
    \item 信息传播需要等待一次完整迭代
    \item 如果状态 $s_1$ 更新,状态 $s_2$ 要等到下一次迭代才能使用新值
\end{itemize}

\textbf{异步DP}:
\begin{itemize}
    \item 信息传播更快:更新的值立即被后续更新使用
    \item 如果先更新 $s_1$,然后更新 $s_2$,$s_2$ 可以使用 $s_1$ 的新值
    \item \textbf{可能更快收敛}
\end{itemize}

\subsection{3. 灵活性}

\textbf{同步DP}:
\begin{itemize}
    \item 必须更新所有状态
    \item 必须按固定顺序
    \item 不能跳过任何状态
\end{itemize}

\textbf{异步DP}:
\begin{itemize}
    \item 可以选择更新哪些状态
    \item 可以跳过不重要的状态
    \item 可以优先更新重要的状态
    \item 可以按任意顺序更新
\end{itemize}

\subsection{4. 实时交互}

\textbf{异步DP的优势}:
\begin{itemize}
    \item 可以在智能体实际体验MDP的同时运行DP算法
    \item 智能体的经验可以决定更新哪些状态
    \item 可以只更新智能体访问过的状态
    \item 最新的价值函数可以指导智能体的决策
\end{itemize}

\textbf{例子}:
\begin{quote}
智能体在Gridworld中移动,每访问一个状态,就更新该状态的价值函数。这样可以将计算集中在智能体实际访问的状态上。
\end{quote}

\section{异步动态规划的收敛性}

\subsection{收敛条件}

\begin{theorem}[异步价值迭代收敛性]
如果 $0 \leq \gamma < 1$,且序列 $\{s_k\}$ 中所有状态都出现无限次,则异步价值迭代保证收敛到 $v_*$。
\end{theorem}

\textbf{关键条件}:
\begin{enumerate}
    \item \textbf{折扣因子}:$\gamma < 1$(或所有策略都是适当的)
    \item \textbf{所有状态都被更新}:每个状态在序列中出现无限次
    \item \textbf{不能忽略任何状态}:在计算的某个点之后,不能忽略任何状态
\end{enumerate}

\textbf{注意}:
\begin{itemize}
    \item 更新顺序可以是任意的,甚至可以是随机的
    \item 某些状态可能被更新多次,其他状态可能只更新一次
    \item 只要所有状态都被无限次更新,就能保证收敛
\end{itemize}

\subsection{为什么需要所有状态都被更新?}

\textbf{反例}:
\begin{itemize}
    \item 如果某个状态 $s$ 从未被更新,它的值将保持初始值
    \item 其他状态可能依赖于 $s$ 的值
    \item 如果 $s$ 的值不正确,其他状态也无法收敛到正确值
\end{itemize}

\textbf{正确做法}:
\begin{itemize}
    \item 必须继续更新所有状态的值
    \item 不能忽略任何状态
    \item 可以以任意顺序、任意频率更新
\end{itemize}

\section{异步动态规划的类型}

\subsection{1. 异步价值迭代}

\textbf{特点}:
\begin{itemize}
    \item 每次只更新一个状态
    \item 使用价值迭代更新公式
    \item 状态选择可以是任意的
\end{itemize}

\textbf{更新公式}:
\begin{equation}
V(s_k) \gets \max_{a} \sum_{s', r} p(s', r | s_k, a) [r + \gamma V(s')]
\end{equation}

其中 $s_k$ 是第 $k$ 步选择的状态。

\subsection{2. 异步截断策略迭代}

\textbf{特点}:
\begin{itemize}
    \item 混合策略评估和价值迭代更新
    \item 某些状态使用策略评估更新
    \item 某些状态使用价值迭代更新
\end{itemize}

\textbf{策略评估更新}:
\begin{equation}
V(s) \gets \sum_{s', r} p(s', r | s, \pi(s)) [r + \gamma V(s')]
\end{equation}

\textbf{价值迭代更新}:
\begin{equation}
V(s) \gets \max_{a} \sum_{s', r} p(s', r | s, a) [r + \gamma V(s')]
\end{equation}

\subsection{3. 实时动态规划(RTDP)}

\textbf{特点}:
\begin{itemize}
    \item 智能体在环境中实际体验
    \item 只更新智能体访问过的状态
    \item 将计算集中在相关状态上
\end{itemize}

\textbf{算法}:
\begin{enumerate}
    \item 智能体在状态 $s$ 选择动作 $a$
    \item 转移到状态 $s'$,获得奖励 $r$
    \item 更新状态 $s$ 的价值函数:
    \begin{equation}
    V(s) \gets \max_{a'} \sum_{s'', r'} p(s'', r' | s, a') [r' + \gamma V(s'')]
    \end{equation}
    \item 重复
\end{enumerate}

\section{状态选择策略}

\subsection{1. 随机选择}

\textbf{方法}:
\begin{itemize}
    \item 每次随机选择一个状态
    \item 确保所有状态都有被选中的概率
\end{itemize}

\textbf{优点}:
\begin{itemize}
    \item 简单
    \item 保证所有状态都被更新
\end{itemize}

\textbf{缺点}:
\begin{itemize}
    \item 可能更新不重要的状态
    \item 效率可能不高
\end{itemize}

\subsection{2. 优先扫描(Prioritized Sweeping)}

\textbf{方法}:
\begin{itemize}
    \item 根据状态价值的变化量确定优先级
    \item 优先更新价值变化大的状态
\end{itemize}

\textbf{优先级}:
\begin{equation}
\text{priority}(s) = |V_{\text{new}}(s) - V_{\text{old}}(s)|
\end{equation}

\textbf{优点}:
\begin{itemize}
    \item 更快收敛
    \item 将计算集中在重要的状态上
\end{itemize}

\subsection{3. 基于经验的选择}

\textbf{方法}:
\begin{itemize}
    \item 智能体在环境中实际体验
    \item 只更新智能体访问过的状态
\end{itemize}

\textbf{优点}:
\begin{itemize}
    \item 只计算相关状态
    \item 可以与实际交互结合
    \item 适用于大规模状态空间
\end{itemize}

\section{同步 vs 异步:详细对比}

\subsection{对比表格}

\begin{center}
\begin{tabular}{|l|c|c|}
\hline
\textbf{特性} & \textbf{同步DP} & \textbf{异步DP} \\
\hline
\textbf{更新顺序} & 固定顺序(系统性扫描) & 任意顺序 \\
\hline
\textbf{使用的值} & 上一次迭代的值 & 当前最新的值 \\
\hline
\textbf{内存需求} & 两个数组($2|\mathcal{S}|$) & 一个数组($|\mathcal{S}|$) \\
\hline
\textbf{信息传播} & 需要等待一次迭代 & 立即传播 \\
\hline
\textbf{状态选择} & 必须更新所有状态 & 可以选择更新哪些状态 \\
\hline
\textbf{灵活性} & 低 & 高 \\
\hline
\textbf{实现复杂度} & 简单 & 中等 \\
\hline
\textbf{收敛保证} & 强($\gamma < 1$) & 强($\gamma < 1$,所有状态无限次更新) \\
\hline
\textbf{实时交互} & 困难 & 容易 \\
\hline
\textbf{大规模问题} & 不适合 & 适合 \\
\hline
\end{tabular}
\end{center}

\subsection{选择建议}

\textbf{使用同步DP的情况}:
\begin{itemize}
    \item 状态空间较小
    \item 需要简单实现
    \item 不需要实时交互
    \item 内存充足
\end{itemize}

\textbf{使用异步DP的情况}:
\begin{itemize}
    \item 状态空间非常大
    \item 内存受限
    \item 需要实时交互
    \item 可以智能选择更新哪些状态
    \item 需要更快的信息传播
\end{itemize}

\section{具体例子:异步价值迭代}

\subsection{问题设置}

考虑一个简单的3×3 Gridworld:

\begin{center}
\begin{tabular}{|c|c|c|}
\hline
$s_1$ & $s_2$ & $s_3$ \\
\hline
$s_4$ & $s_5$ & $s_6$ \\
\hline
$s_7$ & $s_8$ & \textbf{终止} \\
\hline
\end{tabular}
\end{center}

\textbf{设置}:
\begin{itemize}
    \item 所有转移的奖励都是 $-1$
    \item 折扣因子:$\gamma = 0.9$
    \item 初始价值:$V(s_i) = 0$,对所有 $i$
\end{itemize}

\subsection{同步价值迭代(前3次迭代)}

\textbf{第1次迭代}(使用 $V_0$):

所有状态都使用 $V_0 = 0$:
\begin{align}
V_1(s_1) &= -1.0 \\
V_1(s_2) &= -1.0 \\
V_1(s_5) &= -1.0 \\
&\vdots
\end{align}

\textbf{第2次迭代}(使用 $V_1$):

所有状态都使用 $V_1$:
\begin{align}
V_2(s_1) &= \max\{-1 + 0.9 \times 0, -1 + 0.9 \times (-1.0), \ldots\} = -1.0 \\
V_2(s_5) &= \max\{-1 + 0.9 \times (-1.0), \ldots\} = -1.9 \\
&\vdots
\end{align}

\subsection{异步价值迭代(前几步)}

\textbf{更新序列}:$s_5, s_2, s_5, s_8, s_1, s_5, s_3, \ldots$

\textbf{步骤1:更新 $s_5$}
\begin{align}
V(s_5) &\gets \max\{-1 + 0.9 \times V(s_2), -1 + 0.9 \times V(s_4), \\
     &\quad -1 + 0.9 \times V(s_6), -1 + 0.9 \times V(s_8)\} \\
     &= \max\{-1.0, -1.0, -1.0, -1.0\} = -1.0
\end{align}
当前:$V(s_5) = -1.0$,其他状态仍为 $0$

\textbf{步骤2:更新 $s_2$}
\begin{align}
V(s_2) &\gets \max\{-1 + 0.9 \times V(s_1), -1 + 0.9 \times V(s_5), \\
     &\quad -1 + 0.9 \times V(s_2), -1 + 0.9 \times V(s_3)\} \\
     &= \max\{-1.0, -1 + 0.9 \times (-1.0), -1.0, -1.0\} \\
     &= \max\{-1.0, -1.9, -1.0, -1.0\} = -1.0
\end{align}
当前:$V(s_2) = -1.0$,$V(s_5) = -1.0$,其他状态仍为 $0$

\textbf{关键观察}:更新 $s_2$ 时使用了 $V(s_5) = -1.0$(已经更新过),而不是初始值 $0$。

\textbf{步骤3:再次更新 $s_5$}
\begin{align}
V(s_5) &\gets \max\{-1 + 0.9 \times V(s_2), -1 + 0.9 \times V(s_4), \\
     &\quad -1 + 0.9 \times V(s_6), -1 + 0.9 \times V(s_8)\} \\
     &= \max\{-1 + 0.9 \times (-1.0), -1.0, -1.0, -1.0\} \\
     &= \max\{-1.9, -1.0, -1.0, -1.0\} = -1.0
\end{align}
当前:$V(s_5) = -1.0$(没有变化,因为其他方向的值更差)

\textbf{关键观察}:再次更新 $s_5$ 时使用了 $V(s_2) = -1.0$(已经更新过)。

\textbf{步骤4:更新 $s_8$}
\begin{align}
V(s_8) &\gets \max\{-1 + 0.9 \times V(s_5), -1 + 0.9 \times V(\text{终止}), \\
     &\quad -1 + 0.9 \times V(s_7), -1 + 0.9 \times V(\text{终止})\} \\
     &= \max\{-1 + 0.9 \times (-1.0), -1 + 0.9 \times 0, -1.0, -1 + 0.9 \times 0\} \\
     &= \max\{-1.9, -1.0, -1.0, -1.0\} = -1.0
\end{align}
当前:$V(s_8) = -1.0$,$V(s_5) = -1.0$,$V(s_2) = -1.0$

\textbf{关键观察}:更新 $s_8$ 时使用了 $V(s_5) = -1.0$(已经更新过)。

\subsection{异步 vs 同步的对比}

\textbf{同步方法}:
\begin{itemize}
    \item 第1次迭代:所有状态都使用 $V_0 = 0$
    \item 第2次迭代:所有状态都使用 $V_1$(第1次迭代的结果)
    \item 信息传播需要等待一次完整迭代
\end{itemize}

\textbf{异步方法}:
\begin{itemize}
    \item 更新 $s_2$ 时立即使用 $s_5$ 的新值
    \item 再次更新 $s_5$ 时立即使用 $s_2$ 的新值
    \item 信息传播更快,可能更快收敛
\end{itemize}

\section{异步动态规划的实现细节}

\subsection{就地更新(In-Place Update)}

\textbf{同步方法}:
\begin{verbatim}
V_old = [0, 0, 0, 0, 0, 0, 0, 0]  # 初始值
V_new = [0, 0, 0, 0, 0, 0, 0, 0]  # 新数组

# 第1次迭代
for s in states:
    V_new[s] = update(V_old, s)  # 使用 V_old

V_old = V_new  # 批量替换
\end{verbatim}

\textbf{异步方法}:
\begin{verbatim}
V = [0, 0, 0, 0, 0, 0, 0, 0]  # 只有一个数组

# 第1次迭代
for s in selected_states:  # 任意顺序
    V[s] = update(V, s)  # 直接修改 V,使用当前最新的值
\end{verbatim}

\textbf{关键区别}:
\begin{itemize}
    \item 同步:需要两个数组,批量替换
    \item 异步:只需要一个数组,就地更新
\end{itemize}

\subsection{更新顺序的影响}

\textbf{例子}:考虑状态 $s_1 \to s_2 \to s_3$ 的链

\textbf{同步方法}(顺序:$s_1, s_2, s_3$):
\begin{enumerate}
    \item 更新 $s_1$:使用 $V_0(s_2) = 0$
    \item 更新 $s_2$:使用 $V_0(s_3) = 0$(不能使用 $s_1$ 的新值)
    \item 更新 $s_3$:使用 $V_0(\text{终止}) = 0$
\end{enumerate}

\textbf{异步方法}(顺序:$s_3, s_2, s_1$):
\begin{enumerate}
    \item 更新 $s_3$:使用 $V(\text{终止}) = 0$,得到 $V(s_3) = -1.0$
    \item 更新 $s_2$:使用 $V(s_3) = -1.0$(已经更新!),得到 $V(s_2) = -1.9$
    \item 更新 $s_1$:使用 $V(s_2) = -1.9$(已经更新!),得到 $V(s_1) = -2.71$
\end{enumerate}

\textbf{关键观察}:
\begin{itemize}
    \item 异步方法中,从后向前的顺序可以立即传播信息
    \item 同步方法中,无论什么顺序,都要等待一次迭代
\end{itemize}

\section{异步动态规划的应用场景}

\subsection{1. 大规模状态空间}

\textbf{问题}:
\begin{itemize}
    \item 状态空间非常大(如 $10^{20}$ 个状态)
    \item 无法系统性地扫描所有状态
    \item 只能访问部分状态
\end{itemize}

\textbf{解决方案}:
\begin{itemize}
    \item 使用异步DP,只更新访问过的状态
    \item 智能选择要更新的状态
    \item 将计算集中在相关状态上
\end{itemize}

\subsection{2. 实时交互}

\textbf{问题}:
\begin{itemize}
    \item 智能体在环境中实际体验
    \item 需要在线学习
    \item 不能等待完整的扫描
\end{itemize}

\textbf{解决方案}:
\begin{itemize}
    \item 使用异步DP,智能体访问哪个状态就更新哪个状态
    \item 最新的价值函数可以立即指导决策
    \item 计算和交互可以并行进行
\end{itemize}

\subsection{3. 内存受限}

\textbf{问题}:
\begin{itemize}
    \item 内存有限,无法存储两个完整的状态价值数组
    \item 需要节省内存
\end{itemize}

\textbf{解决方案}:
\begin{itemize}
    \item 使用异步DP,只需要一个数组
    \item 节省50\%的内存
\end{itemize}

\section{异步动态规划的变体}

\subsection{1. 部分更新(Partial Updates)}

\textbf{特点}:
\begin{itemize}
    \item 只更新部分状态
    \item 可以跳过不重要的状态
    \item 将计算集中在关键状态上
\end{itemize}

\textbf{适用场景}:
\begin{itemize}
    \item 状态空间非常大
    \item 某些状态更重要
    \item 计算资源有限
\end{itemize}

\subsection{2. 优先扫描(Prioritized Sweeping)}

\textbf{特点}:
\begin{itemize}
    \item 根据状态价值的变化量确定优先级
    \item 优先更新价值变化大的状态
    \item 更快收敛
\end{itemize}

\textbf{优先级计算}:
\begin{equation}
\text{priority}(s) = |V_{\text{new}}(s) - V_{\text{old}}(s)|
\end{equation}

\textbf{算法}:
\begin{enumerate}
    \item 计算所有状态的优先级
    \item 按优先级排序
    \item 优先更新高优先级的状态
\end{enumerate}

\subsection{3. 实时动态规划(RTDP)}

\textbf{特点}:
\begin{itemize}
    \item 智能体在环境中实际体验
    \item 只更新智能体访问过的状态
    \item 将计算集中在相关状态上
\end{itemize}

\textbf{算法流程}:
\begin{enumerate}
    \item 智能体在状态 $s$ 选择动作 $a$(基于当前价值函数)
    \item 转移到状态 $s'$,获得奖励 $r$
    \item 更新状态 $s$ 的价值函数
    \item 重复
\end{enumerate}

\section{异步动态规划的收敛性分析}

\subsection{收敛条件}

\begin{theorem}[异步价值迭代收敛性(详细版)]
如果满足以下条件,异步价值迭代保证收敛到 $v_*$:
\begin{enumerate}
    \item \textbf{折扣因子}:$0 \leq \gamma < 1$(或所有策略都是适当的)
    \item \textbf{所有状态都被更新}:序列 $\{s_k\}$ 中,每个状态 $s$ 都出现无限次
    \item \textbf{更新公式正确}:使用正确的贝尔曼更新公式
\end{enumerate}
\end{theorem}

\textbf{证明思路}:
\begin{itemize}
    \item 贝尔曼算子 $T_*$ 是一个压缩映射
    \item 即使更新顺序是任意的,只要所有状态都被无限次更新
    \item 价值函数序列会收敛到唯一不动点 $v_*$
\end{itemize}

\subsection{收敛速度}

\textbf{影响因素}:
\begin{itemize}
    \item \textbf{更新顺序}:某些顺序可能更快收敛
    \item \textbf{状态选择策略}:优先更新重要状态可能更快
    \item \textbf{折扣因子}:$\gamma$ 越小,收敛越快
\end{itemize}

\textbf{与同步方法的对比}:
\begin{itemize}
    \item 异步方法可能更快收敛(如果更新顺序好)
    \item 也可能更慢收敛(如果更新顺序差)
    \item 平均情况下,两者收敛速度相近
\end{itemize}

\section{总结}

\subsection{核心要点}

\begin{enumerate}
    \item \textbf{定义}:异步动态规划是一种就地迭代DP算法,不按系统扫描的方式组织,可以以任意顺序更新状态值。
    
    \item \textbf{关键特征}:
    \begin{itemize}
        \item 就地更新:只使用一个数组
        \item 任意顺序:可以以任意顺序更新状态
        \item 使用最新值:更新时使用当前最新的其他状态值
        \item 灵活选择:可以选择更新哪些状态
    \end{itemize}
    
    \item \textbf{优势}:
    \begin{itemize}
        \item 内存效率:节省50\%的内存
        \item 信息传播:信息传播更快
        \item 灵活性:可以选择更新哪些状态
        \item 实时交互:可以与实际交互结合
    \end{itemize}
    
    \item \textbf{收敛性}:
    \begin{itemize}
        \item 在 $\gamma < 1$ 且所有状态都被无限次更新的条件下保证收敛
        \item 更新顺序可以是任意的,甚至可以是随机的
    \end{itemize}
    
    \item \textbf{应用场景}:
    \begin{itemize}
        \item 大规模状态空间
        \item 实时交互
        \item 内存受限
        \item 需要智能选择更新状态
    \end{itemize}
    
    \item \textbf{变体}:
    \begin{itemize}
        \item 异步价值迭代
        \item 异步截断策略迭代
        \item 实时动态规划(RTDP)
        \item 优先扫描
    \end{itemize}
\end{enumerate}

\subsection{关键洞察}

\begin{quote}
\textbf{异步动态规划通过允许以任意顺序更新状态,使用当前最新的值,提供了更大的灵活性。这使得我们可以将计算集中在相关状态上,节省内存,并实现实时交互。虽然更新顺序是任意的,但只要所有状态都被无限次更新,算法仍然保证收敛到最优解。}
\end{quote}

\subsection{与同步方法的对比总结}

\textbf{同步动态规划}:
\begin{itemize}
    \item 简单、直观
    \item 需要两个数组
    \item 适合小规模问题
    \item 信息传播需要等待一次迭代
\end{itemize}

\textbf{异步动态规划}:
\begin{itemize}
    \item 灵活、高效
    \item 只需要一个数组
    \item 适合大规模问题
    \item 信息传播更快
    \item 可以与实际交互结合
\end{itemize}

\vspace{1cm}

\textbf{参考文献}:
\begin{itemize}
    \item Sutton, R. S., \& Barto, A. G. (2018). \textit{Reinforcement Learning: An Introduction} (2nd Edition). MIT Press, Chapter 4.5.
    \item Bertsekas, D. P. (1982). Distributed asynchronous computation of fixed points. \textit{Mathematical Programming}, 27(1), 107-120.
    \item Bertsekas, D. P. (1983). Asynchronous distributed computation of fixed points. \textit{Mathematical Programming}, 27(1), 107-120.
\end{itemize}

\end{document}
