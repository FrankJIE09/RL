\documentclass[12pt,a4paper]{article}
\usepackage[UTF8]{ctex}
\usepackage{amsmath}
\usepackage{amssymb}
\usepackage{amsthm}
\usepackage{geometry}
\usepackage{hyperref}
\usepackage{enumitem}
\usepackage{xcolor}
\usepackage{bm}
\usepackage{tikz}
\usepackage{booktabs}
\usepackage{array}
\usepackage{algorithm}
\usepackage{algorithmic}

\geometry{margin=2.5cm}

\title{价值迭代详解:Gridworld例子}
\subtitle{4.4节:价值迭代算法的完整讲解}
\author{}
\date{}

\newtheorem{definition}{定义}
\newtheorem{theorem}{定理}
\newtheorem{proposition}{命题}
\newtheorem{example}{示例}
\newtheorem{remark}{注记}

\begin{document}

\maketitle

\tableofcontents
\newpage

\section{引言}

\subsection{策略迭代的局限性}

策略迭代算法的一个缺点是:每次迭代都涉及策略评估,而策略评估本身可能需要多次遍历状态空间才能收敛到 $v_\pi$。

\textbf{问题}:
\begin{itemize}
    \item 策略评估需要迭代多次才能收敛
    \item 必须等待策略评估完全收敛吗?
    \item 能否提前停止策略评估?
\end{itemize}

\subsection{价值迭代的提出}

\textbf{关键观察}:
\begin{quote}
在图4.1的例子中,策略评估迭代超过前3次后,对相应的贪婪策略没有影响。这说明可以截断策略评估。
\end{quote}

\textbf{价值迭代}:
\begin{itemize}
    \item 策略评估只执行一次遍历(one sweep)
    \item 将策略评估和策略改进合并为一个更新操作
    \item 更简单、更高效
\end{itemize}

\section{价值迭代算法}

\subsection{算法定义}

\begin{definition}[价值迭代]
\textbf{价值迭代}(Value Iteration)是一种动态规划算法,通过直接迭代最优价值函数来找到最优策略。它将策略评估和策略改进合并为一个更新操作。
\end{definition}

\textbf{更新公式}:

\begin{equation}
v_{k+1}(s) = \max_{a \in \mathcal{A}(s)} \sum_{s', r} p(s', r | s, a) [r + \gamma v_k(s')]
\label{eq:value_iteration}
\end{equation}

对所有 $s \in \mathcal{S}$ 成立。

\subsection{与策略迭代的关系}

\textbf{策略迭代}:
\begin{enumerate}
    \item 策略评估:多次迭代直到收敛到 $v_\pi$
    \item 策略改进:基于 $v_\pi$ 改进策略
\end{enumerate}

\textbf{价值迭代}:
\begin{itemize}
    \item 将策略评估截断为一次遍历
    \item 同时进行策略改进(通过 $\max$ 操作)
    \item 每次迭代都执行一次策略评估和一次策略改进
\end{itemize}

\textbf{关系}:
\begin{quote}
价值迭代本质上是策略迭代的特殊情况,其中策略评估只执行一次遍历。
\end{quote}

\subsection{与贝尔曼最优性方程的关系}

\textbf{贝尔曼最优性方程}:

\begin{equation}
v_*(s) = \max_{a \in \mathcal{A}(s)} \sum_{s', r} p(s', r | s, a) [r + \gamma v_*(s')]
\end{equation}

\textbf{价值迭代}:

\begin{equation}
v_{k+1}(s) = \max_{a \in \mathcal{A}(s)} \sum_{s', r} p(s', r | s, a) [r + \gamma v_k(s')]
\end{equation}

\textbf{关键洞察}:
\begin{quote}
价值迭代就是将贝尔曼最优性方程转化为更新规则。通过迭代,$v_k$ 收敛到 $v_*$。
\end{quote}

\section{价值迭代算法}

\begin{algorithm}[H]
\caption{价值迭代算法(计算 $v_*$ 和 $\pi_*$)}
\begin{algorithmic}[1]
\REQUIRE 环境动态 $p(s', r | s, a)$,折扣因子 $\gamma$,收敛阈值 $\theta > 0$
\ENSURE 最优价值函数 $v_*$ 和最优策略 $\pi_*$
\STATE \textbf{初始化}:$V(s)$ 任意初始化(通常为0),对所有 $s \in \mathcal{S}$,但 $V(\text{终止}) = 0$
\REPEAT
    \STATE $\Delta \gets 0$
    \FOR{每个状态 $s \in \mathcal{S}$}
        \STATE $v \gets V(s)$
        \STATE $V(s) \gets \max_{a} \sum_{s', r} p(s', r | s, a) [r + \gamma V(s')]$
        \STATE $\Delta \gets \max(\Delta, |v - V(s)|)$
    \ENDFOR
\UNTIL{$\Delta < \theta$}
\STATE \textbf{输出策略}:
\FOR{每个状态 $s \in \mathcal{S}$}
    \STATE $\pi_*(s) \gets \arg\max_{a} \sum_{s', r} p(s', r | s, a) [r + \gamma V(s')]$
\ENDFOR
\RETURN $v_* = V$,$\pi_*$
\end{algorithmic}
\end{algorithmic}

\subsection{算法特点}

\textbf{关键特征}:
\begin{itemize}
    \item \textbf{不显式维护策略}:在迭代过程中不存储策略
    \item \textbf{只更新价值函数}:每次迭代只更新 $V(s)$
    \item \textbf{最后提取策略}:算法结束后,从 $v_*$ 提取最优策略
    \item \textbf{合并操作}:每次遍历同时执行策略评估和策略改进
\end{itemize}

\section{Gridworld例子}

\subsection{问题设置}

考虑一个3×3的Gridworld:

\begin{center}
\begin{tabular}{|c|c|c|}
\hline
\textbf{终止} & $s_2$ & $s_3$ \\
\hline
$s_4$ & $s_5$ & $s_6$ \\
\hline
$s_7$ & $s_8$ & \textbf{终止} \\
\hline
\end{tabular}
\end{center}

\textbf{环境设置}:
\begin{itemize}
    \item \textbf{非终止状态}:$\mathcal{S} = \{s_2, s_3, s_4, s_5, s_6, s_7, s_8\}$
    \item \textbf{终止状态}:左上角和右下角
    \item \textbf{动作空间}:每个状态有4个动作:上、下、左、右
    \item \textbf{奖励}:所有转移的奖励都是 $-1$,直到到达终止状态
    \item \textbf{折扣因子}:$\gamma = 0.9$
\end{itemize}

\subsection{初始化}

\textbf{初始价值函数 $V_0$}:
\begin{equation}
V_0(s) = 0 \quad \text{对所有 } s \in \mathcal{S}
\end{equation}

\textbf{终止状态}:
\begin{equation}
V_0(\text{终止}) = 0
\end{equation}

\subsection{第1次迭代($k=1$)}

\textbf{对每个状态更新价值函数}:

\textbf{状态 $s_2$}(第1行第2列):

计算所有动作的动作价值:
\begin{align}
q(s_2, \text{上}) &= -1 + 0.9 \times V_0(\text{终止}) = -1 + 0.9 \times 0 = -1.0 \\
q(s_2, \text{下}) &= -1 + 0.9 \times V_0(s_5) = -1 + 0.9 \times 0 = -1.0 \\
q(s_2, \text{左}) &= -1 + 0.9 \times V_0(s_2) = -1 + 0.9 \times 0 = -1.0 \\
q(s_2, \text{右}) &= -1 + 0.9 \times V_0(s_3) = -1 + 0.9 \times 0 = -1.0
\end{align}

选择最大值:
\begin{equation}
V_1(s_2) = \max\{-1.0, -1.0, -1.0, -1.0\} = -1.0
\end{equation}

\textbf{状态 $s_5$}(中心状态):

计算所有动作的动作价值:
\begin{align}
q(s_5, \text{上}) &= -1 + 0.9 \times V_0(s_2) = -1.0 \\
q(s_5, \text{下}) &= -1 + 0.9 \times V_0(s_8) = -1.0 \\
q(s_5, \text{左}) &= -1 + 0.9 \times V_0(s_4) = -1.0 \\
q(s_5, \text{右}) &= -1 + 0.9 \times V_0(s_6) = -1.0
\end{align}

选择最大值:
\begin{equation}
V_1(s_5) = \max\{-1.0, -1.0, -1.0, -1.0\} = -1.0
\end{equation}

\textbf{状态 $s_8$}(第3行第2列):

计算所有动作的动作价值:
\begin{align}
q(s_8, \text{上}) &= -1 + 0.9 \times V_0(s_5) = -1.0 \\
q(s_8, \text{下}) &= -1 + 0.9 \times V_0(\text{终止}) = -1.0 \\
q(s_8, \text{左}) &= -1 + 0.9 \times V_0(s_7) = -1.0 \\
q(s_8, \text{右}) &= -1 + 0.9 \times V_0(\text{终止}) = -1.0
\end{align}

选择最大值:
\begin{equation}
V_1(s_8) = \max\{-1.0, -1.0, -1.0, -1.0\} = -1.0
\end{equation}

\textbf{第1次迭代后的价值函数}:

\begin{center}
\begin{tabular}{|c|c|c|}
\hline
0 & -1.0 & -1.0 \\
\hline
-1.0 & -1.0 & -1.0 \\
\hline
-1.0 & -1.0 & 0 \\
\hline
\end{tabular}
\end{center}

\subsection{第2次迭代($k=2$)}

\textbf{状态 $s_2$}:

计算所有动作的动作价值:
\begin{align}
q(s_2, \text{上}) &= -1 + 0.9 \times V_1(\text{终止}) = -1 + 0.9 \times 0 = -1.0 \\
q(s_2, \text{下}) &= -1 + 0.9 \times V_1(s_5) = -1 + 0.9 \times (-1.0) = -1.9 \\
q(s_2, \text{左}) &= -1 + 0.9 \times V_1(s_2) = -1 + 0.9 \times (-1.0) = -1.9 \\
q(s_2, \text{右}) &= -1 + 0.9 \times V_1(s_3) = -1 + 0.9 \times (-1.0) = -1.9
\end{align}

选择最大值:
\begin{equation}
V_2(s_2) = \max\{-1.0, -1.9, -1.9, -1.9\} = -1.0
\end{equation}

\textbf{状态 $s_5$}:

计算所有动作的动作价值:
\begin{align}
q(s_5, \text{上}) &= -1 + 0.9 \times V_1(s_2) = -1 + 0.9 \times (-1.0) = -1.9 \\
q(s_5, \text{下}) &= -1 + 0.9 \times V_1(s_8) = -1 + 0.9 \times (-1.0) = -1.9 \\
q(s_5, \text{左}) &= -1 + 0.9 \times V_1(s_4) = -1 + 0.9 \times (-1.0) = -1.9 \\
q(s_5, \text{右}) &= -1 + 0.9 \times V_1(s_6) = -1 + 0.9 \times (-1.0) = -1.9
\end{align}

选择最大值:
\begin{equation}
V_2(s_5) = \max\{-1.9, -1.9, -1.9, -1.9\} = -1.9
\end{equation}

\textbf{状态 $s_8$}:

计算所有动作的动作价值:
\begin{align}
q(s_8, \text{上}) &= -1 + 0.9 \times V_1(s_5) = -1 + 0.9 \times (-1.0) = -1.9 \\
q(s_8, \text{下}) &= -1 + 0.9 \times V_1(\text{终止}) = -1 + 0.9 \times 0 = -1.0 \\
q(s_8, \text{左}) &= -1 + 0.9 \times V_1(s_7) = -1 + 0.9 \times (-1.0) = -1.9 \\
q(s_8, \text{右}) &= -1 + 0.9 \times V_1(\text{终止}) = -1 + 0.9 \times 0 = -1.0
\end{align}

选择最大值:
\begin{equation}
V_2(s_8) = \max\{-1.9, -1.0, -1.9, -1.0\} = -1.0
\end{equation}

\textbf{第2次迭代后的价值函数}:

\begin{center}
\begin{tabular}{|c|c|c|}
\hline
0 & -1.0 & -1.0 \\
\hline
-1.9 & -1.9 & -1.0 \\
\hline
-1.9 & -1.0 & 0 \\
\hline
\end{tabular}
\end{center}

\subsection{第3次迭代($k=3$)}

\textbf{状态 $s_2$}:

计算所有动作的动作价值:
\begin{align}
q(s_2, \text{上}) &= -1 + 0.9 \times 0 = -1.0 \\
q(s_2, \text{下}) &= -1 + 0.9 \times (-1.9) = -2.71 \\
q(s_2, \text{左}) &= -1 + 0.9 \times (-1.0) = -1.9 \\
q(s_2, \text{右}) &= -1 + 0.9 \times (-1.0) = -1.9
\end{align}

选择最大值:
\begin{equation}
V_3(s_2) = \max\{-1.0, -2.71, -1.9, -1.9\} = -1.0
\end{equation}

\textbf{状态 $s_5$}:

计算所有动作的动作价值:
\begin{align}
q(s_5, \text{上}) &= -1 + 0.9 \times (-1.0) = -1.9 \\
q(s_5, \text{下}) &= -1 + 0.9 \times (-1.0) = -1.9 \\
q(s_5, \text{左}) &= -1 + 0.9 \times (-1.9) = -2.71 \\
q(s_5, \text{右}) &= -1 + 0.9 \times (-1.0) = -1.9
\end{align}

选择最大值:
\begin{equation}
V_3(s_5) = \max\{-1.9, -1.9, -2.71, -1.9\} = -1.9
\end{equation}

\textbf{状态 $s_8$}:

计算所有动作的动作价值:
\begin{align}
q(s_8, \text{上}) &= -1 + 0.9 \times (-1.9) = -2.71 \\
q(s_8, \text{下}) &= -1 + 0.9 \times 0 = -1.0 \\
q(s_8, \text{左}) &= -1 + 0.9 \times (-1.9) = -2.71 \\
q(s_8, \text{右}) &= -1 + 0.9 \times 0 = -1.0
\end{align}

选择最大值:
\begin{equation}
V_3(s_8) = \max\{-2.71, -1.0, -2.71, -1.0\} = -1.0
\end{equation}

\textbf{第3次迭代后的价值函数}:

\begin{center}
\begin{tabular}{|c|c|c|}
\hline
0 & -1.0 & -1.0 \\
\hline
-1.9 & -1.9 & -1.0 \\
\hline
-1.9 & -1.0 & 0 \\
\hline
\end{tabular}
\end{center}

\subsection{继续迭代}

\textbf{第4次迭代}:

价值函数继续更新,但变化逐渐减小。

\textbf{收敛}:

假设经过多次迭代后,价值函数收敛到最优价值函数 $v_*$。

\subsection{提取最优策略}

\textbf{算法结束后,从 $v_*$ 提取最优策略}:

对每个状态 $s$:
\begin{equation}
\pi_*(s) = \arg\max_{a} \sum_{s', r} p(s', r | s, a) [r + \gamma v_*(s')]
\end{equation}

\textbf{状态 $s_2$}:

假设 $v_*(s_2) = -1.0$,$v_*(s_5) = -1.9$,$v_*(s_3) = -1.0$:

\begin{align}
q_*(s_2, \text{上}) &= -1 + 0.9 \times 0 = -1.0 \quad \text{(最大)} \\
q_*(s_2, \text{下}) &= -1 + 0.9 \times (-1.9) = -2.71 \\
q_*(s_2, \text{左}) &= -1 + 0.9 \times (-1.0) = -1.9 \\
q_*(s_2, \text{右}) &= -1 + 0.9 \times (-1.0) = -1.9
\end{align}

因此:$\pi_*(s_2) = \text{上}$

\textbf{状态 $s_5$}:

假设 $v_*(s_2) = -1.0$,$v_*(s_8) = -1.0$,$v_*(s_4) = -1.9$,$v_*(s_6) = -1.0$:

\begin{align}
q_*(s_5, \text{上}) &= -1 + 0.9 \times (-1.0) = -1.9 \\
q_*(s_5, \text{下}) &= -1 + 0.9 \times (-1.0) = -1.9 \\
q_*(s_5, \text{左}) &= -1 + 0.9 \times (-1.9) = -2.71 \\
q_*(s_5, \text{右}) &= -1 + 0.9 \times (-1.0) = -1.9
\end{align}

因此:$\pi_*(s_5) = \text{上}$(或下、右)

\textbf{最优策略}:

\begin{center}
\begin{tabular}{|c|c|c|}
\hline
\textbf{终止} & $\uparrow$ & $\uparrow$ \\
\hline
$\uparrow$ & $\uparrow$ & $\rightarrow$ \\
\hline
$\rightarrow$ & $\downarrow$ & \textbf{终止} \\
\hline
\end{tabular}
\end{center}

\section{价值迭代 vs 策略迭代}

\subsection{对比表格}

\begin{center}
\begin{tabular}{|l|c|c|}
\hline
\textbf{特性} & \textbf{策略迭代} & \textbf{价值迭代} \\
\hline
\textbf{策略评估} & 多次迭代直到收敛 & 只执行一次遍历 \\
\hline
\textbf{策略维护} & 显式维护策略 & 不显式维护策略 \\
\hline
\textbf{更新公式} & $v_\pi^{k+1} = \sum_a \pi(a|s) \sum_{s',r} p(\cdot)[r + \gamma v_\pi^k(s')]$ & $v_{k+1} = \max_a \sum_{s',r} p(\cdot)[r + \gamma v_k(s')]$ \\
\hline
\textbf{每次迭代} & 策略评估 + 策略改进 & 合并为一个更新 \\
\hline
\textbf{迭代次数} & 通常较少 & 可能较多 \\
\hline
\textbf{每次迭代计算量} & 较大(多次评估) & 较小(一次更新) \\
\hline
\textbf{策略提取} & 迭代过程中 & 算法结束后 \\
\hline
\end{tabular}
\end{center}

\subsection{关键区别}

\textbf{策略迭代}:
\begin{itemize}
    \item 先完全评估策略,再改进策略
    \item 每次迭代需要多次子迭代来评估策略
    \item 迭代次数通常较少
\end{itemize}

\textbf{价值迭代}:
\begin{itemize}
    \item 每次迭代只更新一次价值函数
    \item 将策略评估和策略改进合并
    \item 迭代次数可能较多,但每次迭代更快
\end{itemize}

\section{价值迭代的收敛性}

\subsection{收敛定理}

\begin{theorem}[价值迭代收敛性]
对于任意初始价值函数 $v_0$,价值迭代算法在 $\gamma < 1$ 或所有策略都是适当的条件下,序列 $\{v_k\}$ 收敛到唯一的最优价值函数 $v_*$。
\end{theorem}

\textbf{证明思路}:
\begin{itemize}
    \item 贝尔曼最优性算子 $T_*$ 是一个压缩映射
    \item 压缩映射有唯一不动点
    \item 价值迭代收敛到该不动点 $v_*$
\end{itemize}

\subsection{收敛速度}

\textbf{收敛速度取决于}:
\begin{itemize}
    \item \textbf{折扣因子} $\gamma$:$\gamma$ 越小,收敛越快
    \item \textbf{初始值}:通常初始化为0
    \item \textbf{状态空间大小}:状态越多,可能需要更多迭代
\end{itemize}

\section{截断策略迭代}

\subsection{一般形式}

价值迭代是截断策略迭代的一个极端情况。

\textbf{截断策略迭代}:
\begin{itemize}
    \item 策略评估执行 $k$ 次遍历($k \geq 1$)
    \item $k = 1$:价值迭代
    \item $k = \infty$:策略迭代
\end{itemize}

\textbf{一般算法}:
\begin{enumerate}
    \item 策略评估:执行 $k$ 次遍历
    \item 策略改进:基于当前价值函数改进策略
    \item 重复
\end{enumerate}

\section{价值迭代的本质理解}

\subsection{你的理解}

\begin{quote}
\textbf{价值迭代可以理解为:把策略评估的方程改成了贝尔曼最优方程,然后不需要策略改进部分。}
\end{quote}

\textbf{这个理解基本正确,但需要一些澄清}:

\subsection{正确的理解}

\textbf{1. 策略评估方程 vs 贝尔曼最优方程}

\textbf{策略评估方程}(策略 $\pi$ 的价值函数):
\begin{equation}
v_\pi(s) = \sum_{a} \pi(a | s) \sum_{s', r} p(s', r | s, a) [r + \gamma v_\pi(s')]
\end{equation}

\textbf{贝尔曼最优方程}(最优价值函数):
\begin{equation}
v_*(s) = \max_{a} \sum_{s', r} p(s', r | s, a) [r + \gamma v_*(s')]
\end{equation}

\textbf{关键区别}:
\begin{itemize}
    \item 策略评估:对动作求\textbf{加权平均}(按策略 $\pi$)
    \item 贝尔曼最优方程:对动作求\textbf{最大值}
\end{itemize}

\textbf{2. 价值迭代的更新公式}

\begin{equation}
v_{k+1}(s) = \max_{a} \sum_{s', r} p(s', r | s, a) [r + \gamma v_k(s')]
\end{equation}

\textbf{这确实是贝尔曼最优方程的迭代形式}!

\subsection{策略改进部分去哪了?}

\textbf{关键洞察}:

策略改进并没有"消失",而是\textbf{合并}到了更新公式中!

\textbf{策略迭代}(分离的):
\begin{enumerate}
    \item \textbf{策略评估}:
    \begin{equation}
    v_\pi^{k+1}(s) = \sum_{a} \pi(a | s) \sum_{s', r} p(s', r | s, a) [r + \gamma v_\pi^k(s')]
    \end{equation}
    
    \item \textbf{策略改进}:
    \begin{equation}
    \pi'(s) = \arg\max_{a} \sum_{s', r} p(s', r | s, a) [r + \gamma v_\pi(s')]
    \end{equation}
\end{enumerate}

\textbf{价值迭代}(合并的):
\begin{equation}
v_{k+1}(s) = \max_{a} \sum_{s', r} p(s', r | s, a) [r + \gamma v_k(s')]
\end{equation}

\textbf{关键观察}:
\begin{itemize}
    \item 价值迭代的 $\max$ 操作\textbf{同时执行了策略评估和策略改进}
    \item 策略评估:计算每个动作的价值
    \item 策略改进:选择最大值(隐含地选择了最优动作)
    \item 两者合并为一个更新操作
\end{itemize}

\subsection{详细对比}

\textbf{策略迭代的两步}:

\textbf{步骤1:策略评估}
\begin{align}
v_\pi^{k+1}(s) &= \sum_{a} \pi(a | s) \sum_{s', r} p(s', r | s, a) [r + \gamma v_\pi^k(s')] \\
               &= \sum_{a} \pi(a | s) q_\pi^k(s, a)
\end{align}

\textbf{步骤2:策略改进}
\begin{equation}
\pi'(s) = \arg\max_{a} q_\pi(s, a)
\end{equation}

\textbf{价值迭代的一步}:

\begin{align}
v_{k+1}(s) &= \max_{a} \sum_{s', r} p(s', r | s, a) [r + \gamma v_k(s')] \\
           &= \max_{a} q_k(s, a)
\end{align}

\textbf{关键点}:
\begin{itemize}
    \item 价值迭代直接计算 $\max_{a} q_k(s, a)$
    \item 这\textbf{隐含地}选择了最优动作
    \item 策略改进被"嵌入"到了 $\max$ 操作中
\end{itemize}

\subsection{为什么不需要显式的策略改进?}

\textbf{原因}:

\begin{enumerate}
    \item \textbf{$\max$ 操作已经选择了最优动作}:
    \begin{itemize}
        \item 计算 $\max_{a} q_k(s, a)$ 时,已经找到了最优动作
        \item 不需要单独的策略改进步骤
    \end{itemize}
    
    \item \textbf{不显式维护策略}:
    \begin{itemize}
        \item 价值迭代不存储策略 $\pi$
        \item 只更新价值函数 $v_k$
        \item 算法结束后,从 $v_*$ 提取策略
    \end{itemize}
    
    \item \textbf{策略改进是"隐式的"}:
    \begin{itemize}
        \item 每次更新都选择最优动作(通过 $\max$)
        \item 策略改进的效果体现在价值函数的更新中
    \end{itemize}
\end{enumerate}

\subsection{数学上的等价性}

\textbf{策略迭代}(完整形式):
\begin{align}
v_{\pi_{k+1}}(s) &= \max_{a} \sum_{s', r} p(s', r | s, a) [r + \gamma v_{\pi_k}(s')] \\
                  &= \max_{a} q_{\pi_k}(s, a)
\end{align}

\textbf{价值迭代}:
\begin{align}
v_{k+1}(s) &= \max_{a} \sum_{s', r} p(s', r | s, a) [r + \gamma v_k(s')] \\
           &= \max_{a} q_k(s, a)
\end{align}

\textbf{关键区别}:
\begin{itemize}
    \item 策略迭代:$v_{\pi_k}$ 是策略 $\pi_k$ 的价值函数(完全评估后)
    \item 价值迭代:$v_k$ 是迭代过程中的价值函数(只更新一次)
\end{itemize}

\textbf{等价性}:
\begin{quote}
价值迭代可以看作是策略迭代的截断版本,其中策略评估只执行一次遍历,策略改进通过 $\max$ 操作隐式执行。
\end{quote}

\subsection{具体例子说明}

\textbf{状态 $s_2$ 的更新}:

\textbf{策略迭代方式}:
\begin{enumerate}
    \item \textbf{策略评估}(假设当前策略是随机策略):
    \begin{align}
    v_\pi(s_2) &= \frac{1}{4} \times q(s_2, \text{上}) + \frac{1}{4} \times q(s_2, \text{下}) \\
               &\quad + \frac{1}{4} \times q(s_2, \text{左}) + \frac{1}{4} \times q(s_2, \text{右}) \\
               &= \frac{1}{4} \times (-1.0 - 1.9 - 1.9 - 1.9) = -1.675
    \end{align}
    
    \item \textbf{策略改进}:
    \begin{align}
    \pi'(s_2) &= \arg\max_{a} q(s_2, a) = \text{上} \quad \text{(因为 } q(s_2, \text{上}) = -1.0 \text{ 最大)}
    \end{align}
\end{enumerate}

\textbf{价值迭代方式}:
\begin{align}
v_{k+1}(s_2) &= \max_{a} q_k(s_2, a) \\
             &= \max\{-1.0, -1.9, -1.9, -1.9\} \\
             &= -1.0
\end{align}

\textbf{观察}:
\begin{itemize}
    \item 价值迭代直接得到 $-1.0$(最优动作的价值)
    \item 策略迭代先得到 $-1.675$(随机策略的平均),然后改进策略
    \item 价值迭代"跳过"了中间的平均步骤,直接选择最优
\end{itemize}

\subsection{总结:你的理解修正}

\textbf{你的理解}:
\begin{quote}
价值迭代 = 把策略评估方程改成贝尔曼最优方程 + 不需要策略改进部分
\end{quote}

\textbf{更准确的表述}:
\begin{quote}
\textbf{价值迭代 = 将策略评估方程改为贝尔曼最优方程 + 策略改进通过 $\max$ 操作隐式执行}
\end{quote}

\textbf{关键点}:
\begin{enumerate}
    \item \textbf{是的}:将策略评估的加权平均改为最大值(贝尔曼最优方程)
    
    \item \textbf{但是}:策略改进并没有"不需要",而是\textbf{合并}到了更新公式中
    
    \item \textbf{$\max$ 操作的作用}:
    \begin{itemize}
        \item 计算每个动作的价值(策略评估的一部分)
        \item 选择最优动作(策略改进)
        \item 两者合并为一个操作
    \end{itemize}
    
    \item \textbf{不显式维护策略}:
    \begin{itemize}
        \item 不需要存储策略 $\pi$
        \item 策略改进的效果体现在价值函数的更新中
        \item 算法结束后从 $v_*$ 提取策略
    \end{itemize}
\end{enumerate}

\section{总结}

\subsection{价值迭代的核心要点}

\begin{enumerate}
    \item \textbf{定义}:将策略评估截断为一次遍历,同时进行策略改进
    
    \item \textbf{更新公式}:
    \begin{equation}
    v_{k+1}(s) = \max_{a} \sum_{s', r} p(s', r | s, a) [r + \gamma v_k(s')]
    \end{equation}
    
    \item \textbf{特点}:
    \begin{itemize}
        \item 不显式维护策略
        \item 每次迭代只更新一次价值函数
        \item 算法结束后提取策略
    \end{itemize}
    
    \item \textbf{优势}:
    \begin{itemize}
        \item 每次迭代计算量小
        \item 实现简单
        \item 不需要等待策略评估完全收敛
    \end{itemize}
    
    \item \textbf{与策略迭代的关系}:
    \begin{itemize}
        \item 价值迭代是策略迭代的特殊情况
        \item 策略评估只执行一次遍历
        \item 将策略评估和策略改进合并
    \end{itemize}
\end{enumerate}

\subsection{关键洞察}

\begin{quote}
\textbf{价值迭代有效地将策略评估和策略改进合并。在每次遍历中,它同时执行一次策略评估和一次策略改进。通过迭代,价值函数收敛到最优价值函数 $v_*$,然后可以提取最优策略 $\pi_*$。}
\end{quote}

\vspace{1cm}

\textbf{参考文献}:
\begin{itemize}
    \item Sutton, R. S., \& Barto, A. G. (2018). \textit{Reinforcement Learning: An Introduction} (2nd Edition). MIT Press, Chapter 4.4.
\end{itemize}

\end{document}

